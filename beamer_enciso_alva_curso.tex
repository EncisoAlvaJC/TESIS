%%%%%%%%%%%%%%%%%%%%%%%%%%%%%%%%%%%%%%%%%%%%%%%%%%%%%%%%%%%%%%%%%%%%%%%%%%%%%%%%%%%%%%%%%%%%%%%%%%%
\documentclass[11pt]{beamer}
\usetheme{metropolis}
\usepackage[utf8]{inputenc}
\usepackage[spanish]{babel}
\usepackage{amsmath}
\usepackage{amsfonts}
\usepackage{amssymb}
\usepackage{graphicx}

% para la bibliografia
\usepackage[style=verbose,backend=bibtex]{biblatex}
\bibstyle{abbrv}

% codigo de R
\usepackage{listings}
\usepackage{color}

%%%%%%%%%%%%%%%%%%%%%%%%%%%%%%%%%%%%%%%%%%%%%%%%%%%%%%%%%%%%%%%%%%%%%%%%%%%%%%%%%%%%%%%%%%%%%%%%%%%

\addbibresource{referencias_estacionariedad.bib}
\addbibresource{referencias_fisiologia.bib}
\addbibresource{referencias_otros.bib}
\addbibresource{referencias_mixto.bib}

\renewcommand{\footnotesize}{\tiny}

\newtheorem{defn}{Definici\'on}
\newtheorem{thrm}{Teorema}
\newtheorem{demostracion}{Demostraci\'on}
\newtheorem{prop}{Proposici\'on}

\newcommand{\R}{\mathbb{R}}
\newcommand{\intR}{\int_{-\infty}^{\infty}}
\newcommand{\intZ}{\int_{-\infty}^{0}}
\newcommand{\intPI}{\int_{-\pi}^{\pi}}
\newcommand{\simint}[1]{\int_{- #1 }^{ #1 }}
\newcommand{\prima}{^{\prime}}

\newcommand{\ddd}{$\delta$}
\newcommand{\dirac}{$\delta$  de Dirac}

\newcommand{\aste}[1]{\widehat{ #1 }^{\star}}
\newcommand{\est}[1]{\widehat{ #1 }}

\newcommand{\COS}[1]{\mathrm{cos}\left( #1 \right)}
\newcommand{\SEN}[1]{\mathrm{sen}\left( #1 \right)}

\newcommand{\E}[1]{\mathrm{E}\left[ #1 \right]}
\newcommand{\Var}[1]{\mathrm{Var}\left( #1 \right)}
\newcommand{\Cov}[1]{\mathrm{Cov}\left( #1 \right)}
\newcommand{\abso}[1]{\left| #1 \right|}

%%%%%%%%%%%%%%%%%%%%%%%%%%%%%%%%%%%%%%%%%%%%%%%%%%%%%%%%%%%%%%%%%%%%%%%%%%%%%%%%%%%%%%%%%%%%%%%%%%%

\author{Julio Cesar Enciso Alva}
\title{Estacionariedad d\'ebil}
\subtitle{Detecci\'on en series electrofisiol\'ogicas}
%\setbeamercovered{transparent} 
\setbeamertemplate{navigation symbols}{} 
%\logo{} 
\institute{Instituto de Ciencias B\'asicas e Ingenier\'ia\\ 
Universidad Aut\'onoma del Estado de Hidalgo} 
\date{6 de julio de 2017} 
%\subject{} 

%%%%%%%%%%%%%%%%%%%%%%%%%%%%%%%%%%%%%%%%%%%%%%%%%%%%%%%%%%%%%%%%%%%%%%%%%%%%%%%%%%%%%%%%%%%%%%%%%%%

\begin{document}

\begin{frame}
\titlepage
\end{frame}

%\begin{frame}
%\tableofcontents
%\end{frame}

\section{Introducci\'on}

%\subsection{Antecedentes}

%%%%%%%%%%%%%%%%%%%%%%%%%%%%%%%%%%%%%%%%%%%%%%%%%

%\begin{frame}\frametitle{Antecedentes}
%\begin{itemize}
%\item Encuesta Intercensal 2015 (INEGI): 12,500,000 adultos mayores, 10.4 \%  de la poblaci\'on 
%\footcite{Intercensal15}
%
%\item Posible relaci\'on trastornos del sue\~no y DC en la vejez \footcite{Miyata13}
%
%\item Epidemiolog\'ia del DC en Hidalgo: eficiencia del sue\~no \footcite{VazquezTagle16}
%
%\item DFA en registros de PSG \footcite{Valeria}: exponente de Hurst diferente en sujetos con y 
%sin DC 
%
%\item Se buscan marcadores cl\'inicos para el diagn\'ostico de DC
%\end{itemize}
%\end{frame}

\begin{frame}\frametitle{Motivaci\'on}
El estudio y diagnóstico de una gran cantidad de enfermedades depende de nuestra habilidad para
registrar y analizar se\~nales electrofisiol\'ogicas. 

\vspace{3em}

Se suele asumir que estas se\~nales son complejas: no lineales, no estacionarias y sin equilibrio 
por naturaleza. Pero usualmente no se comprueban formalmente estas propiedades.
\end{frame}

\subsection{Matem\'aticas}

\begin{frame}\frametitle{Conceptos}
\begin{defn}[Estacionariedad d\'ebil]
Un proceso estoc\'astico es d\'ebilmente estacionario si y s\'olo si para cualesquiera tiempos 
admisibles $t$, $s$ se tiene que
\begin{itemize}
\item $\E{X(t)} = \mu_X$
\item $\Var{X(t)} = \sigma^{2}_X$
\item $\Cov{X(t),X(s)} = \rho_X (s-t)$
\end{itemize}
Con $\mu_X$, $\sigma^{2}_X$ constantes, $\rho_X(\tau)$ \'unicamente depende de $\tau$
\end{defn}
\end{frame}

\begin{frame}\frametitle{Conceptos}
\begin{defn}[Funci\'on de densidad espectral (SDF)]
Sea $\{X(t)\}$ un proceso estoc\'astico a tiempo continuo, d\'ebilmente estacionario
\begin{equation*}
h(\omega) = \lim_{T\rightarrow \infty} \E{ \frac{ \left| G_T(\omega) \right|^{2}}{2 T} }
\end{equation*}
Donde $\displaystyle G_T (\omega) = \frac{1}{\sqrt{2 \pi}} \int_{-T}^{T} X(t) e^{-i \omega t} dt$
\end{defn}
\end{frame}

\begin{frame}%\frametitle{}
\begin{thrm}[Wiener-Khinchin]
Una condici\'on suficiente y necesaria para que $\rho$ sea funci\'on de autocorrelaci\'on para 
alg\'un proceso a tiempo continuo d\'ebilmente estacionario y estoc\'asticamente continuo, 
$\{X(t)\}$,  es que exista una funci\'on $F$ tal que
\begin{itemize}
\item Es mon\'otonamente creciente
\item $F(-\infty) = 0$
\item $F(+\infty) = 1$
\item Para todo $\tau \in \R$ se cumple que
\begin{equation*}
\rho(\tau) = \intR e^{i \omega \tau} dF(\omega)
\end{equation*}
\end{itemize}
\end{thrm}
\end{frame}

\begin{frame}\frametitle{Espectro evolutivo}
Se consideran procesos no-estacionarios, estoc\'asticamente continuos, de media cero y varianza 
finita, y que admitan una representaci\'on de la forma
\begin{equation*}
X(t) = \intPI A(t,\omega) e^{i t \omega} dZ(\omega)
\end{equation*}
tal que 
\begin{itemize}
\item $\Cov{dZ(\omega),dZ(\lambda)} = 0 \Leftrightarrow \omega \neq \lambda$
\item $\E{\abso{dZ(\omega)}^{2}} = \mu(\omega)$
\end{itemize}

El \textbf{espectro evolutivo} fue definido por Priestley \footcite{Priestley65} como
\begin{equation*}
f(t,\omega) = \abso{A(t,\omega)}^{2}
\end{equation*}
\end{frame}

\begin{frame}%\frametitle{Estimador de doble ventana}
\begin{defn}[Estimador de doble ventana]
Se define a $\est{f}$, estimador para la $f$, como
\begin{equation*}
\widehat{f}(t,\omega) = \int_{t-T}^{t} w_{T'}(u) \lvert U(t-u,\omega) \lvert^{2} du
\end{equation*}

\begin{itemize}
\item $U(t,\omega) = \int_{t-T}^{t} g(u) X({t-u}) e^{i \omega (t-u)} du$

\item $2\pi \int_{-\infty}^{\infty} \lvert g(u) \lvert^{2} du = 
\int_{-\infty}^{\infty} \lvert \Gamma(\omega) \lvert^{2} d\omega = 1$
\item $w_{\tau}(t) \geq 0$ para cualesquiera $t$, $\tau$
\item $w_{\tau}(t) \rightarrow 0$ cuando $\lvert t \lvert \rightarrow \infty$, para todo $\tau$
\item $\int_{-\infty}^{\infty} w_{\tau}(t) dt = 1$ para todo $\tau$
\item $ \int_{-\infty}^{\infty} \left( w_{\tau}(t) \right)^{2} dt < \infty$ para todo $\tau$
\item $\exists C$ tal que  
$ \lim_{\tau\rightarrow\infty} \tau \int_{-\infty}^{t} \abso{ W_{\tau}(\lambda) }^{2} d\lambda = C$
\end{itemize}
\end{defn}
\end{frame}

\begin{lrbox}{\mybox}%
\begin{lstlisting}[caption={}]
Priestley-Subba Rao stationarity Test for datos
-----------------------------------------------
Samples used              : 3072 
Samples available         : 3069 
Sampling interval         : 1 
SDF estimator             : Multitaper 
  Number of (sine) tapers : 5 
  Centered                : TRUE 
  Recentered              : FALSE 
Number of blocks          : 11 
Block size                : 279 
Number of blocks          : 11 
p-value for T             : 0.4130131 
p-value for I+R           : 0.1787949 
p-value for T+I+R         : 0.1801353 
\end{lstlisting}
\end{lrbox}%

\begin{frame}[fragile]
\begin{figure}
\scalebox{0.8}{\usebox{\mybox}}
\caption{La prueba de Priestley-Subba Rao se encuentra implementada en R como la funci\'on 
\texttt{stationarity()}, del paquete \texttt{fractal}}
\end{figure}
\end{frame}

%%%%%%%%%%%%%%%%%%%%%%%%%%%%%%%%%%%%%%%%%%%%%%%%%

%\begin{frame}{Algo}
%Texto de ejemplo
%\end{frame}

%\begin{frame}{•}
%
%\end{frame}

%%%%%%%%%%%%%%%%%%%%%%%%%%%%%%%%%%%%%%%%%%%%%%%%%%%%%%%%%%%%%%%%%%%%%

\end{document}

%%%%%%%%%%%%%%%%%%%%%%%%%%%%%%%%%%%%%%%%%%%%%%%%%%%%%%%%%%%%%%%%%%%%%%%%%%%%%%%%%%%%%%%%%%%%%%%%%%%