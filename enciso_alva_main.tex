%\documentclass[12pt,a4paper]{article}
\documentclass[12pt,a4paper]{mitthesis}
\usepackage[utf8]{inputenc}
\usepackage[spanish]{babel}

\usepackage{cite}

\usepackage{amsmath}
\usepackage{amsfonts}
\usepackage{amssymb}
\usepackage{graphicx}

\usepackage{cmap}
\pagestyle{plain}

\usepackage{pdfpages}

\usepackage{listings}
\usepackage{color}

\usepackage{tcolorbox}

\usepackage{nicefrac}

%%%%%%%%%%%%%%%%%%%%%%%%%%%%%%%%%%%%%%%%%%%%%%%%%%%%%%%%%%%%%%%%%%%%%%%%%%%%%%%%%%%%%%%%%%%%%%%%%%%
%%%%%%%%%%%%%%%%%%%%%%%%%%%%%%%%%%%%%%%%%%%%%%%%%%%%%%%%%%%%%%%%%%%%%%%%%%%%%%%%%%%%%%%%%%%%%%%%%%%

\begin{document}
\pagestyle{plain}

\newtheorem{defn}{Definici\'on}

\newcommand{\R}{\mathbb{R}}
\newcommand{\Var}[1]{\mathrm{Var}\left( #1 \right)}
\newcommand{\Cov}[1]{\mathrm{Cov}\left( #1 \right)}
\newcommand{\abso}[1]{\lvert #1 \rvert}

%%%%%%%%%%%%%%%%%%%%%%%%%%%%%%%%%%%%%%%%%%%%%%%%%%%%%%%%%%%%%%%%%%%%%%%%%%%%%%%%%%%%%%%%%%%%%%%%%%%

\tableofcontents
\newpage
%\listoffigures
%\newpage
%\listoftables

%%%%%%%%%%%%%%%%%%%%%%%%%%%%%%%%%%%%%%%%%%%%%%%%%%%%%%%%%%%%%%%%%%%%%%%%%%%%%%%%%%%%%%%%%%%%%%%%%%%

%%%%%%%%%%%%%%%%%%%%%%%%%%%%%%%%%%%%%%%%%%%%%%%%%%%%%%%%%%%%%%%%%%%%%%%%%%%%%%%%%%%%%%%%%%%%%%%%%%%
%%%%%%%%%%%%%%%%%%%%%%%%%%%%%%%%%%%%%%%%%%%%%%%%%%%%%%%%%%%%%%%%%%%%%%%%%%%%%%%%%%%%%%%%%%%%%%%%%%%
\chapter{Introducci\'on}

%La poblaci\'on mundial est\'a envejeciendo aceleradamente, lo que se debe en gran parte a la 
%mejor\'ia en la atenci\'on de la salud durante el \'ultimo siglo, traducida en vidas m\'as 
%largas y saludables. Sin embargo, este logro tambi\'en ha tenido como resultado un aumento en el 
%n\'umero de personas con enfermedades no transmisibles, incluida la demencia.
%Por otro lado, todav\'ia son incipientes las investigaciones para identificar 
%factores de riesgo modificables de la demencia.
%\cite{PlanAlzheimer04}
%
%%La demencia es un síndrome de naturaleza cr\'onica y progresiva, caracterizado 
%%por el deterioro de las funciones cognoscitivas y de la conducta, lo que ocasiona 
%%discapacidad y dependencia.
%%\cite{PlanAlzheimer04}
%
%%%%%%%%%%%%%%%%%%%%%%%%%%%%%%%%%%%%%%%%%%%%%%%%%%%%%%%%%%%%%%%%%%%%%%%%%%%%%%%%%%%%%%%%%%%%%%%%%%%%
%
%El objetivo de este trabajo es explorar la hip\'otesis de estacionariedad en registros
%de PSG en adultos mayores con Deterioro Cognitivo y de
%un grupo control.
%
%A tyrav\'es de una metodolog\'ia de estudio de casos, se describen posibles diferencias entre 
%registros 
%de PSG en sujetos de ambos grupos, lo que sugieren su utilizaci\'on como
%como marcadores de uso cl\'inico en el diagn\'ostico del DC en adultos mayores.

%
%El estudio y diagnóstico de una gran cantidad de enfermedades depende de nuestra habilidad para
%registrar y analizar se\~nales electrofisiol\'ogicas. 
%
%Se suele asumir que estas se\~nales son complejas: no lineales, no estacionarias y sin equilibrio 
%por naturaleza. Pero usualmente no se comprueban formalmente estas propiedades.
%
%Correlaci\'on inter-hemisf\'erica durante el sueño MOR del Adulto Mayor con Deterioro Cognitivo.

%\begin{figure}[h]
%\centering
%\includegraphics[width=.8\linewidth]{graficaintro.pdf}
%\caption{Adaptado de V\'azquez-Tagle y colaboradores (2016)}
%\end{figure}

%%%%%%%%%%%%%%%%%%%%%%%%%%%%%%%%%%%%%%%%%%%%%%%%%%%%%%%%%%%%%%%%%%%%%%%%%%%%%%%%%%%%%%%%%%%%%%%%%%%
%%%%%%%%%%%%%%%%%%%%%%%%%%%%%%%%%%%%%%%%%%%%%%%%%%%%%%%%%%%%%%%%%%%%%%%%%%%%%%%%%%%%%%%%%%%%%%%%%%%

%%%%%%%%%%%%%%%%%%%%%%%%%%%%%%%%%%%%%%%%%%%%%%%%%%%%%%%%%%%%%%%%%%%%%%%%%%%%%%%%%%%%%%%%%%%%%%%%%%%
%%%%%%%%%%%%%%%%%%%%%%%%%%%%%%%%%%%%%%%%%%%%%%%%%%%%%%%%%%%%%%%%%%%%%%%%%%%%%%%%%%%%%%%%%%%%%%%%%%%
\section{Conceptos, fisiolog\'ia}

En esta secci\'on se exponen conceptos propios de la biolog\'ia y que ayudar\'an a definir
al sujeto de estudio: registros de PSG en adultos mayores con y sin PDC.
Se pretende que la exposici\'on sea accesible a\'un sin una preparaci\'on 
especializada en fisiolog\'ia, especialmente considerando que el autor pertenece a tal grupo.

%%%%%%%%%%%%%%%%%%%%%%%%%%%%%%%%%%%%%%%%%%%%%%%%%%%%%%%%%%%%%%%%%%%%%%%%%%%%%%%%%%%%%%%%%%%%%%%%%%%

\subsection{Adulto mayor}

Se define como adulto mayor a un individuo de 60 a\~nos o m\'as que habite un pa\'is en v\'ias de
desarrollo, o 65 a\~nos en pa\'ises desarrollados \cite{Hita14}.
En esta etapa 
%ocurren cambios fisiol\'ogicos, psicol\'ogicos y sociales
el organismo sufre cambios fisiolo\'ogicos y psicol\'ogicos que dificultan la capacidad de
adaptaci\'on al ambiente, teniendo como consecuencia una mayor suceptibilidad a padecer enfermedades
y morir en consecuencia \cite{Hita14}.


%El envejecimiento es un proceso biológico que se caracteriza por la disminución de las funciones 
%que hacen más susceptible al ser humano de padecer enfermedades y morir a consecuencia de 
%ellas [1]. 
%Durante esta etapa ocurren cambios biol\'ogicos, psicol\'ogicos y sociales, normales e 
%inherentes a todo individuo debido a que el organismo va perdiendo la habilidad para responder 
%ante el estr\'es y mantener la regulación homeost\'atica y metabólica, teniendo como consecuencia 
%la disminuci\'on de las capacidades cognitivas y de sobrevivencia, reflejándose en la imposibilidad 
%de adaptarse a situaciones de restricci\'on o sobrecarga de cualquier tipo [4,9,10]

El envejecimiento considerado normal viene determinado por una serie de procesos moleculares, 
celulares, fisiol\'ogicos y psicol\'ogicos que conducen directamente al deterioro de funciones 
cognitivas, específicamente en la atenci\'on y memoria \cite{Navarrete03,Park09}.
%Aunque el envejecimiento es un proceso normal, la 
La
funcionalidad durante la vejez no es homog\'enea
en general; se relaciona con el estilo de vida, los factores de riesgo, el acceso a la
educaci\'on y las acciones de promoci\'on a la salud realizadas en edades m\'as tempranas
\cite{Ohayon04,Sanhueza14}.

%A pesar de ser un proceso natural, no todos los individuos envejecen de la misma forma debido a 
%que la calidad de vida y funcionalidad durante la vejez est\'a relacionada con los aprendizajes 
%adquiridos durante la infancia, adolescencia y edad adulta \cite{Ohayon04}. 
%Los estilos de vida, los factores de riesgo, la accesibilidad a la educaci\'on y la promoci\'on 
%de la salud adoptados a lo largo de la vida son fundamentales al momento de llegar a esta etapa 
%para que en el presente de esta se logre autonom\'ia, a pesar de la edad y los padecimientos que 
%se tengan \cite{Sanhueza14}.

%%%%%%%%%%%%%%%%%%%%%%%%%%%%%%%%%%%%%%%%%%%%%%%%%%%%%%%%%%%%%%%%%%%%%%%%%%%%%%%%%%%%%%%%%%%%%%%%%%%

%\subsubsection{Cambios en la anatom\'ia cerebral con la vejez}

En un principio se consideraba que el envejecimiento cerebral ocurr\'ia fundamentalmente por una 
muerte neuronal programada \cite{Coleman87}, sin embargo, estudios realizados con tejido cerebral 
post mortem de adultos mayores que en vida fueron sanos, mostraron que dicha muerte neuronal no 
alcanza un 10\% en su totalidad \cite{Esiri07}. 
En este sentido, los cambios morfol\'gicos que 
sufren las neuronas durante el envejecimiento son abundantes, observ\'andose una importante 
disminuci\'on de la arborizaci\'n dendr\'itica as\'i como en la densidad y volumen \cite{Hita14}. 
%La disminución en la arborización dendr\'itica y de las espinas dendr\'iticas de las neuronas 
%piramidales de la corteza prefrontal, temporal superior, pre central y occipital \cite{Hita14}. 
%Dichas alteraciones morfol\'ogicas conducen durante el envejecimiento a una disminuci\'on de 
%la densidad sin\'aptica y a una desmielinizaci\'on ax\'onica en neuronas de la 
%neocorteza \cite{Terry}.
Con el paso del tiempo, la organizaci\'on an\'atomo-funcional del cerebro sufre modificaciones 
que traen como consecuencia la afectaci\'on de diferentes capacidades cognitivas, 
sin embargo, 
la vulnerabilidad de los circuitos neuronales ante los procesos que ocurren durante el 
envejecimiento no suceden de forma homog\'enea en todo el cerebro \cite{Hita14}.

%Por otro lado, la relevancia del estudio de los cambios anat\'omicos asociados al envejecimiento 
%fisiol\'ogico ha ido aumentando al permitir evaluar como dichos cambios se correlacionan con 
%el deterioro funcional y cognitivo que caracteriza a las personas mayores, facilita la 
%identificaci\'on de estadios tempranos de diferentes patolog\'ias neurodegenerativas estableciendo 
%diferencias entre estas y los cambios asociados al envejecimiento fisiol\'ogico \cite{Hita14}.

%%%%%%%%%%%%%%%%%%%%%%%%%%%%%%%%%%%%%%%%%%%%%%%%%%%%%%%%%%%%%%%%%%%%%%%%%%%%%%%%%%%%%%%%%%%%%%%%%%%

\subsection{El sue\~no}

%(Esta secci\'on tambi\'en es copiada, por el momento)

El sue\~no se define como ''un proceso vital c\'iclico complejo y activo, compuesto por varias 
fases y que posee una estructura interna caracter\'istica, con diversas interrelaciones en los 
sistemas hormonales y nerviosos'' \cite{FernandezConde07}. 
%Una suspensi\'on f\'acilmente reversible 
%de la interacci\'on sensoriomotriz con el medio ambiente, por lo general asociados con el 
%dec\'ubito y la inmovilidad.
El sue\~no en el ser humano se puede caracterizar por las siguientes propiedades\cite{CarrilloMora}
%Las caracter\'isticas conductuales que se asocian con el sue\~no en el ser humano pueden 
%enumerarse de la siguiente forma\cite{CarrilloMora} 
\begin{enumerate}
\item Disminuci\'on de conciencia y reactividad a est\'imulos externos
\item F\'acilmente reversible\footnote{Lo cual lo diferencia de otros estados 
patol\'ogicos como el estupor y el coma}
\item Inmovilidad y relajaci\'on muscular
\item Periodicidad t\'ipica circadiana (diaria)
\item Los individuos adquieren una postura estereotipada
\item La privaci\'on induce alteraciones conductuales y 
fisiol\'ogicas, adem\'as de que genera una ''deuda'' acumulativa
\end{enumerate}

%%%%%%%%%%%%%%%%%%%%%%%%%%%%%%%%%%%%%%%%%%%%%%%%%%%%%%%%%%%%%%%%%%%%%%%%%%%%%%%%%%%%%%%%%%%%%%%%%%%

%El sustrato neurol\'ogico relacionado con la ritmicidad del sueño se encuentra en el hipot\'alamo, 
%estructura que tiene diversidad de conexiones en el Sistema Nervioso Central, con el fin de 
%ejercer una funci\'on o funciones capaces de sincronizar el organismo 
%\cite{FernandezConde07,Cabrera14}.
%Las estructuras l\'imbicas, tales como la am\'igdala y el hipot\'alamo, tambi\'en estar\'ian 
%activadas, lo que explicar\'ia los fen\'omenos emotivos durante la fase de sue\~no REM ya que 
%las emociones est\'an directamente vinculadas con estas zonas cerebrales \cite{Bonet08}.

%%%%%%%%%%%%%%%%%%%%%%%%%%%%%%%%%%%%%%%%%%%%%%%%%%%%%%%%%%%%%%%%%%%%%%%%%%%%%%%%%%%%%%%%%%%%%%%%%%%
%%%%%%%%%%%%%%%%%%%%%%%%%%%%%%%%%%%%%%%%%%%%%%%%%%%%%%%%%%%%%%%%%%%%%%%%%%%%%%%%%%%%%%%%%%%%%%%%%%%

\subsection{Electroencefalograma}

%Esta secci\'on est\'a fuermente basada\footnote{A este punto, ya
%no es una copia, pero quiz\'a ser\'ia mejor diversificar las fuentes} 
%en el cap\'itulo de libro 
%''The origin of biopotentials'' de John William Clark \cite{clark98}.

La actividad el\'ectrica en el cerebro de animales ya hab\'ia sido descrita desde finales del
siglo XIX, pero se le atribuye al psiquiatra alem\'an Hans Berger ser el primero en analizar
este fen\'omeno sistem\'aticamente adem\'as de acu\~nar el t\'ermino ''electroencefalograma'' (EEG)
para referirse a las fluctuaciones en los potenciales de acci\'on registradas en el cerebro.
De manera convencional, la actividad el\'ectrica del cerebro 
se registra en tres locaciones diferentes: en la corteza cerebral
expuesta (electrocorticograma, ECoG), a trav\'es de agujas incrustadas en el tejido nervioso
(registro profundo), o el cuero cabelludo (EEG).

%La actividad el\'ectrica en el cerebro de animales con y sin anestesia ya hab\'ia sido descrito
%cualitativamente desde el siglo XIX, pero el primero en analizarla sistem\'aticamente fue el
%psiquiatra alem\'an Hans Berger, quien introdujo el t\'ermino electroencefalograma (EEG) para
%denotar las fluctuaciones en los potenciales registrados en el cerebro. [tengo citas sobre ello]
%De manera convencional, la actividad el\'ectrica del cerebro se registra usando tres tipos de
%electrodos: 'scalp', cortical y electrodos de profundidad.
%Cuando los electrodos se colocan en la superficie expuesta del cerebro (cortex) el
%registro se llama electrocorticograma (ECoG). Tambi\'en se pueden usar electrodos de aguja
%aislada fina --de varios dise\~nos-- incrustados en el tejido nervioso del cerebro, en
%cuyo caso el registro se conoce como 'registro profundo'. Seg\'un se reporta, el da\~no al
%tejido cerebral es sorprendentemente peque\~no si se usan agujas con un tama\~no
%y dise\~no adecuados.

As\'i la actividad el\'ectrica cerebral se mida
en el cuero cabelludo, la corteza cerebral o las profundidades 
del mismo,
las fluctuaciones de potenciales registrados representan una superposici\'on
de potenciales de campo producidos por una amplia variedad de 
generadores de corriente dentro de un medio conductor volum\'etrico --es decir, los elementos
neuronales activos generan, cada cual, corrientes que son conducidas y disipada a 
trav\'es del espacio.
A su vez, estos generadores de campos el\'ectricos corresponden a agregados de elementos
neuronales con interconexiones complejas: dendritas, somas y axones.
%M\'as a\'un, la arquitectura del tejido cerebral no es uniforme
%espacialmente.
A ello hay que adicionar que la arquitectura cerebral es altamente no homog\'enea.

%%%%%%%%%%%%%%%%%%%%%%%%%%%%%%%%%%%%%%%%%%%%%%%%%%%%%%%%%%%%%%%%%%%%%%%%%%%%%%%%%%%%%%%%%%%%%%%%%%%

%\subsubsection{Potenciales bioel\'ectricos del cerebro}

%Los registros unipolares de potencial en la superficie cortical, relativos a un potencial de
%referencia remoto, pueden ser vistos como una medida del potencial de campo integrado sobre la
%frontera de un volumen conductor 'grande' que contiene un conjunto de fuentes de potenciales de 
%campo.

%Bajo condiciones normales los potenciales de campo producidos por axones, en 
%la corteza cerebral, aportan muy poco al potencial registrado;
%%al potencial de superficie integrado; 
%esto d
Debido a que los axones en la corteza cerebral tienen orientaciones muy diversas --con
%orientaciones muy diversas 
respecto a la superficie-- y a que disparan de manera as\'incrona, el aporte neto de estos campos
al potencial registrado es negligible bajo condiciones normales.
%En consecuencia, su influencia neta sobre el potencial de campo en la superficie es negligible.
Una excepci\'on, muy importante, ocurre en caso de una respuesta evocada por un est\'imulo
simult\'aneo (s\'incronizado) del del n\'ucleo 
tal\'amico o de las aferentes nerviosas.
%, que se proyecta en el cortex v\'ia axones corticotal\'amicos).
Estas respuestas sincronizadas suelen tener una amplitud relativamente alta, y
son referidas como 'potenciales evocados'.
%, y su amplitud es
%relativamente alta.

%%%%%%%%%%%%%%%%%%%%%%%%%%%%%%%%%%%%%%%%%%%%%%%%%%%%%%%%%%%%%%%%%%%%%%%%%%%%%%%%%%%%%%%%%%%%%%%%%%%

\subsubsection{EEG cl\'inico}

El sistema m\'as usado para la colocaci\'on de los electrodos en el EEG con fines cl\'inicos es el 
'International Federation 10--20 system' \cite{Jasper58,AASM07} mostrado en la figura \ref{img1020}. 
Este sistema usa varios
referentes anat\'omicos estandarizados para la ubicaci\'on de los electrodos.

La representaci\'on de los canales de EEG es referida como un \textbf{montaje}
En un montaje bipolar, cada canal mide la diferencia entre dos electrodos adyacentes.
En un monntaje referencial, cada canal mide la diferencia entre un electrodo y un electrodo
de referencia, usualmente la oreja.
%\footnote{Existen otros tipos de montaje, como el promedio (promedio sobre electrodos adyacentes,
%como los sistemas 5\% y 10\%) o el Laplaciano (parecido al promedio, pero usando un filtrado
%basado en pesos relacionados con la distancia entre electrodos)
%. S\'olo hablo del referencial
%y diferencial porque uno se usa en el estudio y otro es importante hist\'oricamente.}
Aunque los mismos eventos el\'ectricos se registran en cada uno de los montajes,
aparecen en un diferente formato seg\'un el caso. Los potenciales cambiantes
son amplificados por amplificadores diferenciales acoplados de alta ganancia.
La se\~nal resultante es grabada y graficada.

\begin{figure}
\centering
\includegraphics[width=0.8\linewidth]{figura_6.png} 
\caption{El sistema 10--20, recomendado por la
International Federation of EEG Societies. 
%\cite{Jasper58,AASM07}
}
\label{img1020}
\end{figure}

%The system most often used to place electrodes for monitoring the clinical
%EEG is the International Federation 10–20 system shown in Figure 4.28. This
%system uses certain anatomical landmarks to standardize placement of EEG
%electrodes. The representation of the EEG channels is referred to as a
%montage. 
%In the bipolar montage, each channel measures the difference
%between two adjacent electrodes. 
%In the referential montage, each channel
%measures the diffference between one electrode and a reference electrode,
%such as on the ear. In the average reference montage, each channel measures
%the difference between one electrode and the average of all other electrodes.
%In the Laplacian montage, each channel measures the difference between one
%electrode and a weighted average of the surrounding electrodes. The differ-
%ential amplifier requires a separate ground electrode plus differential inputs to
%the electrode connections. The advantage of using a differential recording
%between closely spaced electrodes (between successive pairs in the standard
%system, for example) is cancellation of far-field activity common to both
%electrodes; one thereby obtains sharp localization of the response. 
%Although
%the same electric events are recorded in each of the ways, they appear in a
%different format in each case. The potential changes that occur are amplified by
%high-gain, differential, capacitively coupled amplifiers. The output signals are
%recorded and displayed.

En el EEG cl\'inico de rutina, los electrodos 
%son un problema: 
deben ser peque\~nos, deben estar
fijados al cuero cabelludo de manera sencilla con una distorsi\'on m\'inima debido al cuero cabelludo,
deben ser c\'omodos, y deben permanecer en el mismo sitio por largos periodos de tiempo.
El [encargado del registro] prepara la superficie del cuero cabelludo 
desengrasando el \'area de registro
limpi\'andola con alcohol, aplica una pasta conductora, pega los electrodos no-polarizables (Ag/AgCl)
al cuero cabelludo con pegamento (coloid\'on), y los sostiene en el sitio con cintas de caucho
--o se usa una gorra de caucho que contiene todos los electrodos.

%In the routine recording of clinical EEGs, the input electrodes are a
%problem. They must be small, they must be easily affixed to the scalp with
%minimal disturbance of the hair, they must cause no discomfort, and they must
%remain in place for extended periods of time. Technicians prepare the surface
%of the scalp, degrease the recording area by cleaning it with alcohol, apply a
%conducting paste, and glue nonpolarizable Ag/AgCl electrodes to the scalp
%with a glue (collodion) and hold them in place with rubber straps, or use a
%rubber cap that contains all electrodes.

El EEG usualmente se registra con el sujeto despierto pero relajado, descansando en una cama con 
los ojos cerrados; la posici\'on debe ser tal que los artefactos debidos al movimiento de 
electrodos en el 'scalp' sean m\'inimas. 
La actividad muscular, de la cara, cuello, orejas, etc., es quiz\'a la forma m\'as sutil de 
contaminaci\'on de los registros de EEG cuando se busca actividad espont\'anea del cerebro
durante una actividad, o la actividad evocada por est\'imulos sensoriales.
Por ejemplo, el espectro de frecuencias del potencial de campo producido por m\'usculos faciales
medianamente contra\'idos, incluye componentes de frecuencia que bien cuadran en el rango
usual del EEG (0.5--100 Hz).
Una vez se ha conseguido el estado de reposo en un adulto normal, sus registros de scalp
muestran un ritmo alfa dominante en el \'area parietal-occipital, mientras que en \'area
frontal ha un ritmo beta con baja amplitud y alta frecuencia --adem\'as del ritmo alfa.
En un sujeto normal hay cierta simetr\'ia entre los registros de los hemisferios derecho e 
izquierdo. 
La variedad de artefactos conocidos es muy basta.

%The EEG is usually recorded with the subject awake but resting recumbent
%on a bed with eyes closed. With the patient relaxed in such a manner, artifacts
%from electrode-lead movement are significantly reduced, as are contaminating
%signals from the scalp. 
%Muscle activity from the face, neck, ears, and so on is
%perhaps the most subtle contaminant of EEG records in the recording of both
%spontaneous ongoing activity in the brain and activity evoked by a sensory
%stimulus (evoked response). 
%For example, the frequency spectrum of the field
%produced by mildly contracted facial muscles contains frequency components
%well within the nominal EEG range (0.5 to 100 Hz). After technicians have
%achieved resting, quiescent conditions in the normal adult subject, the subject’s
%scalp recordings show a dominant alpha rhythm in the parietal-occipital areas,
%whereas in the frontal areas, there is a low-amplitude, higher-frequency beta
%rhythm in addition to the alpha rhythm. In the normal subject there is
%symmetry between the recordings of the right and left hemispheres. There
%can be a wide range of EEG measurement artifacts.

En general hay una relaci\'on entre el grado de actividad cerebral
y la 'frecuencia promedio' del EEG: la frecuancia incrementa progresivamente cuando hay altos
grados de actividad. Por ejemplo, las ondas delta se encuentran frecuentemente durante el
estupor, anestesia quir\'urgica, y sue\~no; las ondas theta son comunes en infantes;
las ondas alfa ocurren en estado de relajaci\'on; las ondas beta aparecen durante actividad
mental intensa.
Sin embargo, durante periodos de actividad mental las ondas se vuelven m\'as as\'incronas
que sincronizadas, de modo que la magnitud del potencial integrado de superficie
decrece a pesar de la alta actividad cortical.

%In general there is a relationship between the degree of cerebral activity
%and the average frequency of the EEG rhythm: The frequency increases
%progressively with higher and higher degrees of activity. For example, delta
%waves are frequently found in stupor, surgical anesthesia, and sleep; theta
%waves in infants; alpha waves during relaxed states; and beta waves during
%intense mental activity. However, during periods of mental activity, the waves
%usually become asynchronous rather than synchronous, so that the magnitude
%of the summed surface potential recording decreases despite increased cortical
%activity.

%%%%%%%%%%%%%%%%%%%%%%%%%%%%%%%%%%%%%%%%%%%%%%%%%%%%%%%%%%%%%%%%%%%%%%%%%%%%%%%%%%%%%%%%%%%%%%%%%%%

\subsection{Ritmos de sue\~no en el EEG}

Los registros de EEG desde el cuero cabelludo muestran una actividad el\'ectrica oscilatoria 
continua y cambiante. Tanto la intensidad como los patrones de esta actividad est\'an determinados
por los eventos de excitaci\'on conjunta del cerebro, resultante de las funciones en el
sistema reticular de activaci\'on del tallo cerebral \cite{Clark98}.
Estas 'ondas' observadas en los registros de potenciales el\'ectricos en el cerebro 
(\ref{ritmos}) son referidas como
\textbf{ondas cerebrales}, mientras que %el registro \textit{per se} recibe el nombre de EEG.

La frecuencia de las ondas cerebrales var\'ia entre 0.5 y 100 Hz, 
se ha identificado que su composici\'on est\'a fuertemente
relacionada con el grado de actividad cerebral, habiendo --por ejemplo-- diferencias claras
entre registros durante vigilia y sue\~no.
Aunque la mayor parte del tiempo el EEG es irregul y no muestra patrones claros,
es relativamente com\'un que muestre ondas cerebrales relativamente organizadas; para su estudio,
estas se clasifican en cuatro grandes grupos: alfa, beta, gamma, delta.


\begin{figure}
\centering
\includegraphics[width=0.7\linewidth]{figura_4.png} 
\caption{\textbf{(a)} Diferentes tipos de ondas normales en el EEG.
\textbf{(b)} Supresi\'on del ritmo alfa debido a una descarga dessincronizada cuando
el paciente abre los ojos.
%(From A. C. Guyton, Structure and Function of the Nervous System, 2nd ed.,
%Philadelphia: W.B. Saunders, 1972; used with permission.)
[Estos gr\'aficos ser\'an reconstruidos]
}
\label{ritmos}
\end{figure}


Las ondas alfa ocurren en frecuencias entre 8 y 13 Hz. 
Se encuentran en
los EEG --bajo condiciones normales--
de sujetos despiertos en un estado de quietud del pensamiento.
Estas ondas ocurren m\'as intensamente en la regi\'on occipital, pero tambi\'en pueden ser
registradas en las regiones frontal y parietal. Su voltaje aproximado est\'a entre 20 y 200 mV.
Cuando el sujeto duerme, las ondas alfa desaparecen completamente. 
Cuando el sujeto est\'a
despierto y su atenci\'on se dirige a una actividad mental espec\'ifica, las ondas alfa
son reemplaxadas por ondas dessincronizadas de mayor frecuencia y menor voltaje.
%En la figura \ref{ritmos} se muestra este efecto con la acci\'on de cerrar los ojos.

Las ondas beta ocurren en el rango de frecuencias de 14 a 30 Hz.
%, y usualmente
%--en especial durante actividad mental intensa-- hasta 50 Hz.
Normalmente se registran en las regiones parietal y frontal. A veces se les divide
 en dos tipos: beta I y beta II. Las ondas beta I tienen una frecuencia de cerca del doble a
 las ondas alfa, y son afectadas de manera similar por la actividad mental --desaparecen
y son reemplazadas por ondas dessincronizadas de menor amplitud.
Las ondas beta II, en cambio, aparecen durante una activaci\'on intensa del sistema nervioso
central y durante tensi\'on.

%Beta waves normally occur in the frequency range of 14 to 30 Hz, and
%sometimes—particularly during intense mental activity—as high as 50 Hz.
%These are most frequently recorded from the parietal and frontal regions of the
%scalp. They can be divided into two major types: beta I and beta II. The beta I
%waves have a frequency about twice that of the alpha waves. They are affected
%by mental activity in much the same way as the alpha waves (they disappear
%and in their place appears an asynchronous, low-voltage wave). The beta II
%waves, on the other hand, appear during intense activation of the central
%nervous system and during tension. Thus one type of beta activity is elicited by
%mental activity, whereas the other is inhibited by it.

Las ondas theta tienen frecuencias entre 4 y 7 Hz. Ocurren principalmente en las regiones
parietal y temporal en ni\~nos, pero pueden aparecer en algunos adultos durante 
estr\'es emocional, sobre todo durante periodos de decepsi\'on y frustraci\'on.
%Por ejemplo, pueden ser inducidos en el EEG de una persona frustrada una vez se le permite
%disfrutar una experiencia placentera, la cual es removida s\'ubitamente. Esto causa
%aproximadamente 20 s de ondas theta.

%Theta waves have frequencies between 4 and 7 Hz. These occur mainly in
%the parietal and temporal regions in children, but they also occur during
%emotional stress in some adults, particularly during periods of disappointment
%and frustration. For example, they can often be brought about in the EEG of a
%frustrated person by allowing the person to enjoy some pleasant experience
%and then suddenly removing the element of pleasure. This causes approxi-
%mately 20 s of theta waves.

Las ondas delta incluyen todas las ondas del EEG 'abajo de' 3.5 Hz. 
%A veces, estas ondas s\'olo
%aparecen cada 2 o 3 seg. 
Ocurren generalmente en el sue\~no profundo en infantes,
y despu\'es de enfermedades org\'anicas serias del cerebro.
Tambi\'en pueden ser registradas en cerebros de animales a los cuales se ha hecho 
transsecci\'on subcortical, produciendo una separaci\'on funcional entre la corteza
cerebral y el 
sistema reticular de activaci\'on del tallo cerebral. 
%Entonces, las ondas delta s\'olo pueden ocurrir en la cortreza cerebral,
%independientemente de la actividad en las regiones m\'as bajas del cerebro.

%%%%%%%%%%%%%%%%%%%%%%%%%%%%%%%%%%%%%%%%%%%%%%%%%%%%%%%%%%%%%%%%%%%%%%%%%%%%%%%%%%%%%%%%%%%%%%%%%%%

%\subsection{Patrones en el sue\~no}

%When an individual in a relaxed, inattentive state becomes drowsy and falls
%asleep, the alpha rhythm is replaced by slower, larger waves 
%(\ref{ritmosEEG},\cite{Jasper42}). 

En el sue\~no profundo se observan ondas delta muy irregulares. Junto con ellas, durante el sue\~no
medianamente profundo, ocurren trenes cortos de ondas parecidas a las alfa y que son referidas como
\textit{husos de sue\~no} (sleep spindles). El ritmo alfa y los husos de sue\~no 
est\'a sincronizados en
el sue\~no y la somnolencia, en contraste con la actividad irregular, desincronizada y de bajo 
voltaje registrada en estado de alerta.

%In
%deep sleep, very large, somewhat irregular delta waves are observed. Inter-
%spersed with these waves—during moderately deep sleep—are bursts of alpha-
%like activity called sleep spindles. The alpha rhythm and the patterns of the
%drowsy and sleeping subject are synchronized, in contrast with the low-voltage
%desynchronized, irregular activity seen in the subject who is in an alert state.
%The high-amplitude, slow waves seen in the EEG of a subject who is asleep
%are sometimes replaced by rapid, low-voltage irregular activity resembling that
%obtained in alert subjects. 
%However, the sleep of a subject with this irregular
%pattern is not interrupted; in fact, the threshold for arousal by sensory stimuli is
%elevated. 
%This condition has therefore come to be called paradoxical sleep.
%During paradoxical sleep, the subject exhibits rapid, roving eye movements.

A veces,
las ondas lentas de amplitud alta son reemplazadas durante el sue\~no por ondas r\'apidas de
bajo voltaje, irregulares, que recuerdan la actividad en el EEG durante el estado de alerta.
La presencia de estos patrones irregulares no interrumpen el sue\~no, sino que 
incluso se incrementa el umbral
para que los est\'imulos externos despierten al paciente; 
este comportamiento es referido como ''sue\~no parad\'ogico''.
Durante este sue\~no parad\'ogico, el sujeto exhibe movimientos oculares r\'apidos,
raz\'on por la cual se le conoce como ''sue\~no de movimientos oculares r\'apidos'' (MOR).
%o sue\~no parad\'ojico debido a que se presentan simult\'aneamente
%una reducci\'on marcada en el tono
%muscular y los movimientos oculares r\'apidos.
%La etapa del sue\~no donde se encuentran los husos de sue\~no y la actividad
%sincronizada 
La etapa fuera del sue\~no
es referida como sue\~no no-MOR (NMOR) o sue\~no de ondas lentas.
Los sujetos humanos que despiertan durante la fase de sue\~no MOR suelen reportar que
ten\'ian enso\~naciones, a diferencia de aquellos que despiertan durante la fase NREM.

%For this reason, it is also called rapid-eye-movement sleep, or REM sleep.
%Conversely, spindle or synchronized sleep is frequently called nonrapid-eye-
%movement (NREM), or slow-wave sleep. Human subjects aroused at a time
%when their EEG exhibits a paradoxical (REM) sleep pattern generally report
%that they were dreaming, whereas individuals wakened from spindle sleep do
%not. 

%Estas observaciones sugieren que el sue\~no MOR y las enso\~naciones est\'an fuermente
%asociadas. De manera parad\'ojica, durante el sue\~no REM hay una reducci\'on marcada en el tono
%muscular a pesar de los movimientos oculares r\'apidos.

%This observation and other evidence indicate that REM sleep and
%dreaming are closely associated. It is interesting that during REM sleep, there
%is a marked reduction in muscle tone, despite the rapid eye movements.

\begin{figure}
\centering
\includegraphics[width=0.7\linewidth]{figura_7.png} 
\caption{Los cambios en el EEG que ocurren durante el sue\~no en un sujeto.
Las marcas de calibraci\'on corresponden a 50 mV.
%H. Jasper, ‘‘Electrocephalography.’’ In Epilepsy and Cerebral Localization, W.
%G. Penfield and T. C. Erickson (eds.). Springfield, IL: Charles C. Thomas,
%1941.)
[Estos gr\'aficos ser\'an redibujados]
}
\label{ritmosEEG}
\end{figure}

%%%%%%%%%%%%%%%%%%%%%%%%%%%%%%%%%%%%%%%%%%%%%%%%%%%%%%%%%%%%%%%%%%%%%%%%%%%%%%%%%%%%%%%%%%%%%%%%%%%

\subsection{Etapas del sue\~no}

El sue\~no normal se divide en dos etapas: sue\~no  MOR (fase R)
%\footnote{Tambi\'en conocido como
%REM (Rapid-eyemovement)} (movimiento-ocular-r\'apido) 
y sue\~o no-MOR (fase N), los cuales se diferencian 
fundamentalmente por sus rasgos electroencefalogr\'aficos y una serie de caracter\'isticas 
fisiol\'ogicas --de los cuales surgen sus nombres.
Cabe mencionar que
la nomenclatura acerca de las fases del sue\~no ha sido 
recientemente modificada por la Academia Americana de Medicina del Sueño en 2007, 
de modo que en este trabajo se usar\'an ambas nomenclaturas siempre que sea posible, por fines
de compatibilidad con la terminolog\'ia usual.

%Una herramienta tecnol\'ogica que ha sido de vital importancia para el estudio de la fisiolog\'ia 
%del sue\~no es el electroencefalograma (EEG). De forma muy simplificada, el EEG es el la 
%representaci\'on gr\'afica y digital de las oscilaciones que muestra la actividad el\'ectrica 
%del cerebro, al ser registrada mediante electrodos colocados encima de la piel cabelluda en 
%distintas regiones de la cabeza \cite{Hita14}.

%El sue\~no MOR se caracteriza por la presencia de ondas de bajo voltaje y alta frecuencia en el 
%EEG, aton\'ia muscular y movimientos oculares r\'apidos, adem\'as es donde se presentan 
%la mayor\'ia de los sue\~nos. 
%El sue\~no no-MOR se compone de cuatro fases, 1 y 2, que son de sue\~no ligero, y 3 y 4 de 
%sue\~no profundo, las mismas que transcurren de manera secuencial desde la primera hasta la 
%cuarta fase, que es la fase reparadora del sue\~no, aquella que produce en la persona la 
%sensaci\'on de haber descansado cuando se levanta 13,22,43.

%Las características de las fases del sueño no-MOR incluyen cuatro etapas, la primera que 
%corresponde a la transición de la vigilia al sueño; la etapa 2 es la intermedia (mayor porcentaje 
%del tiempo de sueño) y en el EEG aparecen husos de sueño y los complejos K. La etapa 3 es la del 
%sueño relativamente profundo, representado en el electroencefalograma por ondas lentas de gran 
%amplitud, y la etapa 4o de sueño profundo con más del 50\% de ondas lentas de gran amplitud13.

%Durante el estado de alerta, mientras se mantienen los ojos cerrados, en el EEG se observan 
%oscilaciones de la actividad eléctrica que suelen encontrarse entre 8-13 ciclos por segundo (Hz), 
%principalmente a nivel de las regiones occipitales (ritmo alfa). Durante el sueño ocurren cambios 
%característicos de la actividad eléctrica cerebral que son la base para dividir el sueño en varias 
%fases. Como ya se mencionó, el sueño suele dividirse en dos grandes fases que, de forma normal, 
%ocurren siempre en la misma sucesión: todo episodio de sueño comienza con el llamado sueño sin 
%movimientos oculares rápidos (No MOR), que tiene varias fases, y después pasa al sueño con 
%movimientos oculares rápidos (MOR). 

%%%%%%%%%%%%%%%%%%%%%%%%%%%%%%%%%%%%%%%%%%%%%%%%%%%%%%%%%%%%%%%%%%%%%%%%%%%%%%%%%%%%%%%%%%%%%%%%%%%

\subsubsection{Sueño no-MOR (N)}

\begin{description}
\item[Fase 1 (N1)] Corresponde con la somnolencia o el inicio del sue\~no ligero, en ella es muy 
f\'acil despertarse. La actividad muscular disminuye paulatinamente y pueden observarse algunas 
breves sacudidas musculares s\'ubitas que a veces coinciden con una sensación de ca\'ida 
(mioclon\'ias h\'ipnicas). En el EEG se observa actividad de frecuencias mezcladas, pero de bajo 
voltaje y algunas ondas agudas (ondas agudas del v\'ertex). 

\item[Fase 2 (N2)] En el EEG se caracteriza por que aparecen patrones espec\'ificos de actividad 
cerebral llamados \textbf{husos de sue\~no} y \textbf{complejos K}. F\'isicamente la 
temperatura, la frecuencia cardiaca y respiratoria comienzan a disminuir paulatinamente. 

\item[Fases 3 y 4 (N3)] Sue\~no de ondas lentas, es la fase de sue\~no no-MOR m\'as profunda, 
y en el EEG se observa actividad de frecuencia muy lenta (< 2 Hz).
\end{description}

%%%%%%%%%%%%%%%%%%%%%%%%%%%%%%%%%%%%%%%%%%%%%%%%%%%%%%%%%%%%%%%%%%%%%%%%%%%%%%%%%%%%%%%%%%%%%%%%%%%

\subsubsection{Sueño MOR (R)}

Se caracteriza por la presencia de movimientos oculares r\'apidos. F\'isicamente el tono de todos 
los m\'usculos disminuye (con excepción de los m\'usculos respiratorios y los esf\'interes vesical 
y anal), as\'i mismo la frecuencia cardiaca y respiratoria se vuelve irregular.%,
%e incluso puede 
%incrementarse y 
%existe erección espontánea del pene o del clítoris. 
Durante el sue\~no MOR se producen la mayor\'ia de las enso\~naciones (lo que conocemos 
coloquialmente como sue\~nos), y la mayor\'ia de los pacientes que despiertan durante esta fase 
suelen recordar v\'ividamente el contenido de sus enso\~naciones \cite{Chokroverty09}.

%Por otro lado, las necesidades de sueño son muy variables según la edad y las circunstancias 
%individuales 43,54:
%Un niño recién nacido duerme casi todo el día, con una proporción próxima al 50\% del denominado 
%sueño «activo», que es el equivalente del sueño MOR. A lo largo de la lactancia los períodos de 
%vigilia son progresivamente más prolongados y se consolida el sueño de la noche; además, la 
%proporción de sueño MOR desciende al 25-30 \%, que se mantendrá durante toda la vida. Entre el 1er 
%y 3er año de vida el niño ya sólo duerme una o dos siestas. Entre los 4 y 5 años y la adolescencia 
%los niños son hipervigilantes, muy pocos duermen siesta, pero tienen un sueño nocturno de 9-10 
%horas bien estructurado en 5 ciclos o más. Por lo que se refiere a los individuos jóvenes, en 
%ellos reaparece en muchos casos la necesidad fisiológica de una siesta a mitad del día43,55.

%%%%%%%%%%%%%%%%%%%%%%%

%La necesidad de sue\~no en un adulto puede oscilar entre 5 y 9 horas. Asimismo, var\'ia 
%notablemente el horario de sueño entre noct\'ambulos y madrugadores. En \'epocas de mucha actividad 
%intelectual o de crecimiento o durante los meses del embarazo, puede aumentar la necesidad de 
%sue\~no, mientras que el estr\'es, la ansiedad o el ejercicio f\'isico practicado por la tarde 
%pueden reducir la cantidad de sue\~no. 

%%%%%%%%%%%%%%%%%%%%%%%

%Los estudios efectuados en individuos aislados de influencias exteriores han mostrado que la 
%tendencia fisiológica general es a retrasar ligeramente la fase de sueño con respecto al ciclo 
%convencional de 24 horas y a dormir una corta siesta “de mediodía” 43,56. 
Un adulto j\'oven pasa aproximadamente entre 70--100 min en el sue\~no no-MOR para despu\'es entrar 
al sue\~no MOR, el cual puede durar entre 5--30 min; este ciclo se repite cada hora y media.
%durante toda la noche de sue\~no. 
A lo largo de la noche pueden presentarse normalmente entre 4 y 6 ciclos de 
sue\~no MOR.
En los ancianos se va fragmentando el sue\~no nocturno con frecuentes episodios de despertar, y se 
reduce mucho el porcentaje de sue\~no en fase 4 y no tanto el de sue\~no MOR, que se mantiene 
m\'as constante.% a lo largo de la vida. 
%Las personas de edad avanzada tienen tendencia a aumentar el tiempo de permanencia en la cama. 
%Muchas de ellas dormitan 
Adicionalmente, muchos adultos mayores dormitan
%f\'acilmente 
durante el d\'ia varias siestas cortas\cite{CarrilloMora}.

%%%%%%%%%%%%%%%%%%%%%%%%%%%%%%%%%%%%%%%%%%%%%%%%%%%%%%%%%%%%%%%%%%%%%%%%%%%%%%%%%%%%%%%%%%%%%%%%%%%

%\subsection{Alteraciones del ciclo vigilia-sueño}
%
%La relevancia que tiene el sueño para para la supervivencia de un individuo es la cantidad de horas que este duerme a lo largo de su vida, mismas que depende fundamentalmente de sus necesidades fisiológicas y de las demandas del ambiente al que está expuesto 4,57
% En el caso de los humanos, es posible establecer una clasificación de patrones de sueño en función de su duración (corta, intermedia y larga) 4. Las personas que muestran un patrón de sueño intermedio, es decir, duración aproximada de entre 7-8 horas, presentan un mejor estado de salud a lo largo de su vida, comparado con los individuos de duración de sueño corta o excesivamente larga que frecuentemente tienen de problemas de salud y/o laborales 42,45,46 . 
 
%La estabilidad del sue\~no nocturno es otro factor a tener en cuenta debido a que es razonable 
%pensar que un sue\~no muy fragmentado no cumplir\'a con sus funciones fisiol\'ogicas de igual forma 
%que un patr\'on de sue\~no estable a lo largo de la noche.

%%%%%%%%%%%%%%%%
%%%%%%%%%%%%%%%%
%
%Los adultos mayores informan que duermen menos durante la noche, y se acuestan y se despiertan 
%m\'as temprano de lo habitual. Adem\'as, tardan m\'as tiempo en conciliar el sue\~no, se 
%despiertan con m\'as frecuencia durante la noche y la duraci\'on de estos despertares es 
%m\'as prolongada 58,59.
%
%%%%%%%%%%%%%%%%
%%%%%%%%%%%%%%%%

%La disminución del tiempo de sueño asociada a un incremento de la somnolencia diurna incide negativamente en la función cerebral del día siguiente 60
%Por otro lado, existen diversas formas de pérdida de sueño13,25,46: a) la privación de sueño, que quiere decir la suspensión total del sueño por un periodo ($>$ 24 h), b) la restricción del sueño, que significa una disminución del tiempo habitual de sueño, generalmente de forma crónica, y c) la fragmentación del sueño, que significa la interrupción repetida (despertares) de la continuidad del sueño14. Todos estos tipos de alteraciones del sueño han demostrado afectar distintas funciones cognitivas y variedades de memoria en mayor o menor grado.

%Las alteraciones de sueño específicamente en personas mayores se han asociado con la presencia de enfermedades crónicas, problemas físicos y de salud mental 3

%%%%%%%%%%%%%%%%%%%%%%%%%%%%%%%%%%%%%%%%%%%%%%%%%%%%%%%%%%%%%%%%%%%%%%%%%%%%%%%%%%%%%%%%%%%%%%%%%%%
%%%%%%%%%%%%%%%%%%%%%%%%%%%%%%%%%%%%%%%%%%%%%%%%%%%%%%%%%%%%%%%%%%%%%%%%%%%%%%%%%%%%%%%%%%%%%%%%%%%

%%%%%%%%%%%%%%%%%%%%%%%%%%%%%%%%%%%%%%%%%%%%%%%%%%%%%%%%%%%%%%%%%%%%%%%%%%%%%%%%%%%%%%%%%%%%%%%%%%%
%%%%%%%%%%%%%%%%%%%%%%%%%%%%%%%%%%%%%%%%%%%%%%%%%%%%%%%%%%%%%%%%%%%%%%%%%%%%%%%%%%%%%%%%%%%%%%%%%%%

\section{Conceptos (matem\'aticas)}

En esta secci\'on se describen los conceptos b\'asicos de la teor\'ia espectral 'cl\'asica' para 
procesos estoc\'aticos no-estacionarios. 
De forma m\'as bien pragm\'atica, la descripci\'on est\'a
 fuertemente inspirada por el libro 'Spectral Analysis and Time Series' 
de M. Priestley \cite{Priestley81}, ua que este est\'a expl\'icitamente dirigida a un p\'ublico 
sin un trasfondo matem\'atico.

%Una duda central en el trabajo es el por qu\'e de usar t\'ecnicas desarrolladas hace m\'as de 
%30 a\~nos, y la respuesta tiene tres puntos principales (que se exponen en otra secci\'on con
%m\'as detallles): 
%\begin{itemize}
%\item Hist\'oricamente existe un desface en los conceptos te\'oricos que maneja la 
%neurobiolog\'ia cuantitativa de EEG, lo que motiva a usar materiales cl\'asicos revisados
%como la simplificaci\'on m\'as natural
%
%\item La teor\'ia m\'as reciente consta en gran medida de generalzaciones de la teor\'ia espectral
%cl\'asica: la representaci\'on espectral de Wold-Cram\'er no es \''unica, y pueden extenderse
%sus resultados para las representaciones de periodograma (cambiante en el tiempo), las 
%representeaciones de Wigner-Ville o de Choi-Williams. En otro sentido, es posible
%hacer pruebas de hip\'otesis m\'ultiples usando representaciones de ondeletas. En un tercer 
%apartado, cabe mencionar el enfoque de estacionariedad local para ubicar y relacionar
%componentes de frecuencia
%espec\'ificos.
%
%\item Como el enfoque prescinde de un an\'alisis m\'as riguroso de la composici\'on espectral de
%las se\~nales, el test PSR tiene la propieda de ser relativamente r\'apido, con un orden de
%$N \log{N}$
%\end{itemize}

%Debo citar los trabajos de Cohen, Nason, Adak, Dahlhaus, Gabor, Fryzelwicz, entre otros.
%En discusiones m\'as modernas, se mencionan temas que aun no se han explorado:
%ciclo-estacionariedad, procesos harmonizables, estacionariedad local y por partes,
%diferencias entre memoria larga y memoria corta, espectros de ondaletas, espectros de
%Wigner-Ville, Wold-Cram\'er, Gabor. 
%Debo mencionarlos, pero no he trabajado en ello y no se suficiente sobre ello.

%La informalidad de la redacci\'on se debe al tiempo: en versiones futuras deber\'ia mejorar.

%Nota: no es prioritario, pero ser\'a una buena idea incluir una discusi\'on sobre por qu\'e
%tiene sentido revisar si los EEG son estacionarios, y es que un proceso estacionario es 
%b\'asicamente un ruido.

%%%%%%%%%%%%%%%%%%%%%%%%%%%%%%%%%%%%%%%%%%%%%%%%%%%%%%%%%%%%%%%%%%%%%%%%%%%%%%%%%%%%%%%%%%%%%%%%%%%

\subsection{Estacionariedad d\'ebil}

El ingrediente b\'asico de las series de tiempo son los procesos estoc\'asticos; para ello, se
supone dada la definci\'on de variables aleatorias, espacios de probabilidad, y espacios $L^{p}$;
si es necesario los defino, y si no me conformar\'e con citar un libro sobre series de tiempo
que cubra estos temas,
como el de Chatfield (The Analysis of Time Series: An Introduction, 2003).

Una muy buena raz\'on para empezar a describir \textbf{desde} procesos estoc\'asticos es tener
las definiciones a la mano, evitar conflictos con la notaci\'on $X(t)$ en lugar de $X_t$, y
enfatizar detalles sobre el tiempo continuo.

\begin{defn}[Proceso estoc\'astico]
Un proeso estoc\'astico $\{ X(t) \}$ es una familia de variables aleatorias indexadas por el 
s\'imbolo $t$ que pertenece a alg\'un conjunto $T \in \R$
\end{defn}

Matem\'aticamente se permitir\'a que $t$, referido como \textbf{tiempo}, tome valores 
en todo $\R$; las observaciones, en cambio,
s\'olo pueden ser tomadas en un conjunto discreto y finito de instantes en el tiempo. 
Adicionalmente, en algunas secciones se considerar\'an procesos estoc\'asticos complejos,
si bien la mayor parte del texto s\'olo usar\'a valores reales.

Esta definici\'on particular de proceso estoc\'astico deber\'ia enfatizar que para cada 
tiempo $t$, $X(t)$ es una variable aleatoria con su funci\'on de densidad de probabilidad,
sus momentos [s\'olo se consideran va's con al menos segundos momentos finitos], etc.

Otro concepto clave de este texto es el de \textbf{estaionareidad d\'ebil}; 
quiz\'a la mejor forma de motivar el adjetivo 'd\'ebil' es como contraposici\'on a 
la \textbf{estacionariedad fuerte o total}. 
Para ello, sea $F(X;\cdot)$ la funci\'on de densidad de probabilidad de $X$, es decir, 
la probabilidad de que $X\leq x$ puede expresarse como 
$
F(X;x) = P(X\leq)
$
bajo el entendido que $X$ y $x$ pueden ser vectores en $\R^{d}$.

\begin{defn}[Estacionariedad fuerte]
Un proceso estoc\'astico $\{ X(t) \}$ es fuertemente estacionario si, para cualquier 
conjunto de tiempos admisibles $t_1,t_2,\dots,t_n$ y cualquier $\tau \in \R$
se cumple que
\begin{equation*}
F\left(X(t_1),X(t_2),\dots,X(t_n);\cdot\right) 
\equiv
F\left(X(t_1+\tau),X(t_2+\tau),\dots,X(t_n+\tau);\cdot\right)
\end{equation*}
\end{defn}

La estacionariedad fuerte depende de las funciones de densidad de probabilidad conjunta para
diferentes tiempos. 
%Entre las consecuencias de que un proceso sea estacionario en el
%sentido fuerte, se encuentran:
%\begin{itemize}
%\item Media y varianzas constantes, todos los momentos constantes; es decir
%\begin{equation*}
%E[X^{n}(t)]
%\end{equation*}
%\item La funci\'on de autocorrelaci\'on s\'olo depende de 
%\end{itemize}
Si un proceso es estacionario en el sentido fuerte, entonces todas las variables $X(t)$ son 
id\'enticamente distribuidas.

%Al modelar eventos como proceso estoc\'asticos, tiene sentido que las variables aleatorias
%interfieran las unas con las otras de diversas

Con viene definir versiones menos fuertes de estacionariedad seg\'un sea posible deducirse de
las mediciones de un fen\'omeno y/o sean relevantes en su modelaci\'on.

\begin{defn}[Estacionariedad de orden $m$]
Un proceso estoc\'astico se dice estacionario de orden $m$ si, para cualquier 
conjunto de tiempos admisibles $t_1,t_2,\dots,t_n$ y cualquier $\tau \in \R$
se cumple que
\begin{equation*}
E\left[ X^{m_1}(t_1)X^{m_2}(t_2)\cdots X^{m_n}(t_n) \right]
=
E\left[ X^{m_1}(t_1+\tau)X^{m_2}(t_2+\tau)\cdots X^{m_n}(t_n+\tau) \right]
\end{equation*}
Para cualesquiera enteros $m_1,m_2,\dots,m_n$ tales que $m_1+m_2+\dots+m_n \leq m$
\end{defn}

Hay una especie de consenso seg\'un el cual la estacionariedad de orden 2, tambi\'en
llamada \textbf{estacionariedad d\'ebil} es suficiente para
que se cumplan los teoremas m\'as comunes sobre medias y varianzas.
Algunas consecuencias que un
proceso sea estacionario debilmente son las siguientes:
\begin{itemize}
\item Para todo $t$, $E[X(t)] = \mu$, una constante
\item Para todo $t$, $\Var{X(t)} = \sigma^{2}$, una constante
\item Para cualesquiera $t$, $\tau$, $\Cov{X(t+\tau),\Cov{X(t)}} = E[X(t+\tau)X(t)] - \mu^{2}$, 
una funci\'on de $\tau$ pero no de $t$
\end{itemize}

El rec\'iproco tambi\'en es cierto: si un proceso cumple las tres condiciones anteriores,
entonces es estacionario de orden 2. A su vez tres condiciones son m\'as usuales en la literatura
y tienen una intepretaci\'on m\'as clara como modelo, pues se exige que el proceso tenga media
y varianza constante, y que la funci\'on de autocorrelaci\'on no dependa de d\'onde se mida --lo
cual simplifica la estimaci\'on de estas cantidades.

Antes de proseguir, cabe mencionar que la estacionariedad fuerte se define
en t\'erminos de las funciones de densidad de probabilidad conjunta, mientras que la 
estacionariedad se define seg\'un los momentos; luego, la estacionariedad d\'ebil excluye 
procesos cuyos momentos no est\'en definidos. Por ejemplo, una colecci\'on de variables
independientes id\'enticamente distribuidos --con distribuci\'on de Cauchy-- ser\'a
fuertemente estacionario, pero no estacionario de orden $m$ para ning\'un $m$. 
%Por el contrario, un proceso estacionario de \textit{orden infinito} siempre es
%fuertemente estacionario.

Por el momento se asumir\'an procesos con segundos momentos finitos 
\textbf{debido a que} hay motivaciones
en el modelo para ello: energ\'ia finita, cambios finitos de energ\'ia, respuestas suaves, etc.

%%%%%%%%%%%%%%%%%%%%%%%%%%%%%%%%%%%%%%%%%%%%%%%%%%%%%%%%%%%%%%%%%%%%%%%%%%%%%%%%%%%%%%%%%%%%%%%%%%%

\subsection{El espectro de una serie de tiempo}

Quiero y me siento obligado a citar la excelente discuci\'on
filos\'ofica
de Loynes \cite{Loynes68}, resaltando la frase ''Los espectros instant\'aneos no existen''.
Tambi\'en quiero citar una discusi\'on m\'as moderna de M\'elard \cite{Melard89}, donde una
frase a favor es ''El supuesto de estacionariedad ha sido v\'alido previamente debido a la corta
duraci\'on de las series y la baja capacidad de c\'omputo''.

Pues la mayor parte de mi trabajo se ha centrado en el concepto de \textbf{espectro} de una serie
de tiempo. La mejor forma de introducir el espectro evolutivo 
--en el sentido que estoy usando-- es
presentar un proceso estacionario de orden 2,
 $\{X(t)\}$, en su representaci\'on de Cram\'er \cite{Priestley81}
[la existencia de esta representacion esta garantizada por el teorema de Khinchin-Wiener --para
procesos a tiempo continuos-- y por una extension del mismo por Wold --para procesos a tiempo
discreto.
por ahora solo cito el resultado, pero quiza sea buena idea escribir la demostracion
como apendice, una demostracion citada ya que es bastante tecnica]

\begin{equation*}
X(t) = \int_{\Lambda} A(\omega) e^{i 2\pi \omega t} dZ(\omega)
\end{equation*}

Donde el proceso $\{ Z(\omega) \}$ tiene incrementos ortogonales, es decir 
\begin{equation*}
\Cov{dZ(\omega_1,dZ(\omega_2))} = \delta(\omega_1,\omega_1) d\omega
\end{equation*}
Con $\delta$ la funci\'on delta de Dirac. Cabe mencionar que es suficiente si los incrementos
son independientes, pero se puede debilitar ese requerimiento; incluso es de notarse que no
se exige que el proceso sea al menos continuo --en el sentido estoc\'astico.

El espectro de potencia de $\{X(t)\}$ se define como

\begin{equation*}
f(\omega) = \abso{A(\omega)}^{2}
\end{equation*}

Citar\'e de Adak \cite{Adak98} una tabla donde compara varias definiciones de espectro, para
procesos no-estacionarios.

\begin{figure}[h]
\centering
\includegraphics[width=0.9\textwidth]{tabla.png} 
\end{figure}

Dos identidades muy importantes para estimar el espectro son la \textit{equivalencia} entre
el espectro y la funci\'on de autocorrelaci\'on

\begin{equation*}
f(\omega ) = \int R_X(\tau ) e^{-i 2\pi \omega t} d\tau
\end{equation*}

Donde funci\'on de autocorrelaci\'on se ha definido como

\begin{equation*}
R_X(\tau) = E\left[ X(t) X(t+\tau) \right] = \int_0^{\infty} X(t)X(t+\tau) dt
\end{equation*}

[la demostracion es corta, batsa con reescribir una composicion de integrales como convolucion,
la incluire mas tarde]

Por otro lado, se tiene la Identidad de Parseval

\begin{equation*}
\int X^{2}(t) dt = \int f(\omega) d\omega
\end{equation*}

[esta demostracion se basa en la convergencia dominada del modulo de la integral de $X^{2}$ por
la integral del modulo (...), la incluire mas tarde]

%%%%%%%%%%%%%%%%%%%%%%%%%%%%%%%%%%%%%%%%%%%%%%%%%%%%%%%%%%%%%%%%%%%%%%%%%%%%%%%%%%%%%%%%%%%%%%%%%%%

\subsection{Test Priestley-Subba Rao (PSR)}

(seccion en proceso de re-redaccion)

A muy grosso modo, el test PSR estima localmente  el espectro evolutivo
 y revisa si estad\'isticamente
cambia en el tiempo.

Para ello, usa un estimador para la funci\'on de densidad espectral
que es aproximadamente (asint\'oticamente) insesgado y cuya varianza est\'a
determinada aproximadamente. La estimaci\'on se lleva a cabo en puntos en el tiempo y
la frecuencia tales que en conjunto son aproximadamente no-correlacionados.
Se aplica logaritmo para que la varianza de todos los estimadores sea aproximadamente
la misma (el logaritmo ayuda a), amen que los errores conjuntos tengan una
distribuci\'on cercana a una multinormal con correlaci\'on cero.
Finalmente se aplica una prueba ANOVA de varianza conocida.

%%%%%%%%%%%%%%%%%%%%%%%%%%%%%%%%%%%%%%%%%%%%%%%%%%%%%%%%%%%%%%%%%%%%%%%%%%%%%%%%%%%%%%%%%%%%%%%%%%%

\subsection{El espectro evolutivo}

Consid\'erese un proceso estoc\'astico a tiempo continuo $\{X(t)\}$, tal que
$E[X(t)]=0$ y $E\left[ X^{2}(t)\right] < \infty$ para todo $t$. Es decir que su media es constante
y sus segundos momentos est\'an bien definidos, aunque 
estos \'ultimos pueden cambiar con el tiempo.

Por el momento se supondr\'a que acepta una representaci\'on de la forma

\begin{equation*}
X(t) = \int_{-\pi}^{\pi} A(t ; \omega) e^{i\omega t} \, d Z(\omega)
\end{equation*}

Con $\{ Z(\omega) \}$ una familia de procesos ortogonales\footnote{De nuevo, esto implica que
$\Cov{dZ(\omega_1,dZ(\omega_2))} = \delta(\omega_1,\omega_1) d\omega$, una condici\'on m\'as
d\'ebil que la independencia} tales que

\begin{itemize}
\item $E \left[\abso{ dZ(\omega)}^{2} \right] = d\omega$
\item Para cada $t$ el m\'aximo de $A(t;\cdot)$ se encuentra en 0
\end{itemize}

Esta representaci\'on es an\'aloga a la representaci\'on de Cram\'er para un proceso
estacionario, salvo que se permite que la funci\'on $A$ cambie con el tiempo.
Siguiendo la analog\'ia, se define 
el \textbf{espectro evolutivo} de $\{X(t)\}$, con respecto a la la familia
$\mathcal{F} = \{ e^{i\omega t} A(t; \omega) \}$
 como
 
\begin{equation*}
d F(\omega;t) = \lvert A(t;\omega) \lvert^{2} d\omega
\end{equation*}

Ahora bien, si se supone que $\{X(t)\}$ es estoc\'asticametne diferenciable, entonces
se puede definir una \textbf{funci\'on de densidad espectral}

\begin{equation*}
f(t;\omega) = \lvert A(t;\omega) \lvert^{2}
\end{equation*}

Cabe destaca que si la funci\'on $A(t;\omega)$ fuera constante con respecto a $t$, se obtendr\'ia
un proceso estacionario de orden dos tal cual fue descrito en la secci\'on anterior.

%%%%%%%%%%%%%%%%%%%%%%%%%%%%%%%%%%%%%%%%%%%%%%%%%%%%%%%%%%%%%%%%%%%%%%%%%%%%%%%%%%%%%%%%%%%%%%%%%%%

\subsection{El estimador de doble ventana}

Esta t\'ecnica fue presentada por Priestley en 1965. Muy a grosso modo, es un estimador de la
funci\'on de densidad espectral con ciertas propiedades y que parte de la idea que un proceso
no-estacionario puede verse localmente como un proceso lineal generalizado.

Como meta-nota, yo empec\'e a estudiar este tipo de estimadores porque es \textit{el qeu ven\'ia
con el m\'etodo} ya que el test esta implementado en R; desde un punto de vista de difusi\'on,
es una ventaja usar un m\'etodo implementado en un software gratuito y de c\'odigo abierto --y
no una mera excusa para no explorar otros m\'etodos. En todo caso, he revisado varios otros test,
pero de momento solo este ha arrojado suficientes resultados para llenar un informe.

%{Estimador de doble ventana (Priestley, 1965 \& 1966)}
Para construir el estimador se reuieren dos funciones, $g$ y $w_T$, que servir\'an como ventanas
para extraer informaci\'on local de los datos. Debido a que sus propiedades tienen una interpretaci\'on
f\'isica desde la teor\'ia de circuitos, absorben su terminolog\'ia

\textit{
nota al pie: deberia incluir una motivacion de estos nombres,
que en parte tiene relevancia en la interpretacion. Los 
Linear Invariant Systems (LIS) suponen dependencia lineal
--constante-- respecto a todos los tiempos anteriores; 
a tiempo continuo son equivalentes a una ecuacion diferencial ordinaria lineal,
y a su vez a modelos AR. Un modelo fisico para ello son los circuitos RC, que
fueron usables en radios, y para los cuales las palabras 'filtro' y 'frecuencia'
tienen una interpretacion clara. Esta terminologia de circuitos electricos tiene sentido
para mi ya que todos los modelos de neuronas y poblaciones de neuronas que he visto hasta ahora,
por ejemplo de Ermentrout (falta citar), {Clark98,Priestley81}, PARTEN de considerar
circuitos equivalentes a los componentes neuronales, lo cual me hace pensar que es buena idea
mantener esta vision conjunta.
}

Primeramente se toma una funci\'on $g(u)$ normalizada, que en conjunto a su
transformada inversa de Fourier\footnote{Esta funci\'on 
$\Gamma(u) = \int_{-\infty}^{\infty} g(u) e^{i u \omega} du$
es referida como
\textbf{frequency-response function}, nombre tiene un poco de encanto cuando
$g$ adopta ciertas formas particulares (senos y cosenos).} 
$\Gamma$ tiene las siguientes propiedades

\begin{equation*}
2\pi \int_{-\infty}^{\infty} \lvert g(u) \lvert^{2} du 
= 
\int_{-\infty}^{\infty} \lvert \Gamma(\omega) \lvert^{2} d\omega
= 1
\end{equation*}


A partir de $g$ y $\Gamma$ se define el filtro $U$ como una convoluci\'on
con las realizaciones del proceso

\begin{equation*}
U(t,\omega) = \int_{t-T}^{t} g(u) X({t-u}) e^{i \omega (t-u)} du
\end{equation*}

Un ejemplo que est\'a en el libro de Priestley es tomar funciones del tipo

\begin{equation*}
g_h(u) = 
\begin{cases}
{1 \big{/} 2\sqrt{\pi h}} & \text{ , } \abso{u} \leq h
\\
0 & \text{ , } \abso{u} > 0
\end{cases}
\end{equation*}

Su correspondiente funci\'on de respuesta de frecuencia es complicada [me falta 
escribirla]. Es referida como la \textbf{ventana de Bartlett} y
est\'a totalmente caracterizada la siguiente propiedad

\begin{equation*}
\abso{\Gamma_h(\omega)}^{2} = \frac{1}{\pi h} \left( \frac{\text{sen} (h \omega)}{\omega} \right)^{2}
\end{equation*}

Cabe mencionar que puede entenderse al par $g$ y $\Gamma$ como ventanas en el tiempo
y las frecuencias para la serie.

---

Ahora bien, se toma una segunda ventana $W_\tau$ con las siguientes
restricciones para
su funci\'on de respuesta ante frecuencia $w_\tau$

\begin{itemize}
\item $w_{\tau}(t) \geq 0$ para cualesquiera $t$, $\tau$
\item $w_{\tau}(t) \rightarrow 0$ cuando $\lvert t \lvert \rightarrow \infty$, para todo $\tau$
\item $\displaystyle \int_{-\infty}^{\infty} w_{\tau}(t) dt = 1$ para todo $\tau$
\item $\displaystyle \int_{-\infty}^{\infty} \left( w_{\tau}(t) \right)^{2} dt < \infty$ para todo $\tau$
\item Existe una constante $C$ tal que  [T est\'a relacionado con el 'tiempo 0', pero para
tiempos de muestreo grandes se puede reemplazar por $-\infty$ EXCEPTO cerca del inicio y el final dle muestreo]
$$\lim_{\tau\rightarrow\infty} \left[ \tau \int_{t-T}^{t} \lvert W_{\tau}(\lambda) \lvert^{2} d\lambda \right] = C$$
\end{itemize}

%Ahora, si se define 
%$\displaystyle W_{T'}(\lambda) = \int_{-\infty}^{\infty} e^{-i\lambda t}w_{T'}(t) dt $

[posteriormente annadire mas detalles sobre el papel que juega el par $w_\tau$, $W_\tau$]

Como ejemplo, se puede tomar la siguiente funci\'on llamada \textbf{ventana de Daniell}

\begin{equation*}
W_\tau (t) = 
\begin{cases}
{1 \big{/} \tau} & \text{ , } -\nicefrac{1}{2} \tau \leq t \leq \nicefrac{1}{2} \tau
\\
0 & \text{ , otro caso}
\end{cases}
\end{equation*}

La cual se puede demostrar [tengo en algun lado esa demostracion]

$$\lim_{\tau\rightarrow\infty} \left[ \tau \int_{t-T}^{t} \lvert W_{\tau}(\lambda) \lvert^{2} d\lambda \right] = 2\pi$$

-----

Se define el estimador para $f_t$, con $0 \leq t \leq T$
\begin{equation*}
\widehat{f_t}(\omega) = \int_{t-T}^{t} w_{T'}(u) \lvert U(t-u,\omega) \lvert^{2} du
\end{equation*}

Fue demostrado por Priestley (1965, falta citar) que 

[aqui van las expresiones para el valor esperado y la varianza de $\widehat{f_t}$, me falta
escribirlas]

Pero, bajo varios supuesto adicionales [que me falta trascribir] se puede aproximar

\begin{equation*}
E\left[ \widehat{f_t}(\omega) \right] \sim f_t(\omega)
\end{equation*}

\begin{equation*}
\Var{\widehat{f_t}(\omega)} 
\sim 
\frac{C}{\tau} \left(f_t(\omega)\right)^{2} \int_{-\infty}^{\infty} \abso{\Gamma(\theta)}^{4} d\theta
\end{equation*}

Se advierte claramente que $\widehat{f_t}$ es unnestimados aproximadamente insesgado.
Para las ventanas de Bartlett y Daniell usadas como ejemplo, se tiene

\begin{equation*}
\Var{\widehat{f_t}} 
\sim 
\frac{4h}{3\tau} \left(f_t(\omega)\right)^{2}
\end{equation*}

Cabe mencionar que hay una expresi\'on expl\'icita para la covarianza de $\widehat{f_t}$
en para diferentes puntos en el tiempo y las frecuencias. Lamentablemente,
aun me falta escribirlas, son complicadas, y se describen situaciones bajo las
cuales estas covarianzas son negligibles; cabe destacar que TODAS las condiciones 
que se usan para aproximar son b\'asicamente las mismas, y dependen de que la distancia
entre los tiempos y las frecuencias sean tan grandes como sea posible.

------------

El \'ultimo ingrediente del test PSR es una transformaci\'on logar\'itmica
para regular la varianza, y quiza para cortar los bordes de las aproxiamciones.
Se define $Y_{i,j} = \log \left( \widehat{f_{t_i}}(\omega_j) \right)$, con las siguientes propiedades

\begin{equation*}
E\left[ Y_{i,j} \right] \thicksim \log \left( f_{t_i}(\omega_j) \right)
\hspace{4em}
\text{Var}\left( {Y\left(t,\omega\right)}\right) \thicksim \sigma^{2}
\end{equation*}

Luego as\'i, puede escribirse aproximadamente que

$$Y_{i,j} = \log \left( f_{t_i}(\omega_j) \right) + \varepsilon_{i,j}$$

donde $\varepsilon_{i,j}$ va iid tales que

$
E\left[ \varepsilon_{i,j} \right] = 0
\hspace{4em}
\text{Var}\left( \varepsilon_{i,j} \right) \sigma^{2}
$

Priestley cita que con esta informaci\'on incluso se puede considerar que los $\varepsilon_{i,j}$
siguen una distribuci\'on normal cada uno; Nason (2015, falta citar) comenta que
este supuesto no tiene por que cumplirse, y que es una popsible fuente de falsos positivos
para el test. Yo he hecho pruebas de normalidad a los datos, que incluire como anexos
mas tarde.

El test PSR \textit{per se} son tres test ANOVA --en su versi\'on en la que la varianza es conocida--
sobre si los $\varepsilon_{i,j}$ son estad\'isticamente negligibles en total, sobre el tiempo y sobre
las frecuencias. Para el fin de estudiar la estacionariedad, basta con que sean estad\'iticamente
no-negligibles sobre el tiempo.

[Por supuesto que los otros dos test tienen interpretacion: la negigibilidad total da informacion
sobre las marginales, y si estas pueden ser estimadas adecuadamente usando el estimador, si se
combina con negativo para no-estacionariedad es \textbf{efectivamente positivo} para estacionariedad
y toma una forma muy particular (proceso uniformemente modulado). Si sobre las frecuencias resulta
significativo (no-negligible) da informacion sobre la 'aeatoridad total' del proceso.
De tener tiempo, lo incluire como anexo, ya que ninguna de estas caracteristicas es estudiada :( ]

Lo detalles de la implementaci\'on en R estar\'an en la secci\'on de resultados.

%%%%%%%%%%%%%%%%%%%%%%%%%%%%%%%%%%%%%%%%%%%%%%%%%%%%%%%%%%%%%%%%%%%%%%%%%%%%%%%%%%%%%%%%%%%%%%%%%%%
%%%%%%%%%%%%%%%%%%%%%%%%%%%%%%%%%%%%%%%%%%%%%%%%%%%%%%%%%%%%%%%%%%%%%%%%%%%%%%%%%%%%%%%%%%%%%%%%%%%

%%%%%%%%%%%%%%%%%%%%%%%%%%%%%%%%%%%%%%%%%%%%%%%%%%%%%%%%%%%%%%%%%%%%%%%%%%%%%%%%%%%%%%%%%%%%%%%%%%%
%%%%%%%%%%%%%%%%%%%%%%%%%%%%%%%%%%%%%%%%%%%%%%%%%%%%%%%%%%%%%%%%%%%%%%%%%%%%%%%%%%%%%%%%%%%%%%%%%%%
\chapter{Resultados (no re-redactados)}

\textbf{Falta redactar los resultados de manera adecuada}

Como se mencion\'o previamente, este trabajo se ha basado en los registros de PSG de 6 adultos
mayores con deterioro cognitivo (DC) y 3 sin este padecimiento. La calidad de deterioro
cognitivo fue medida a trav\'es de una bater\'ia de pruebas neuropsicol\'ogicas;
adicionalmente, se midi\'o su posible depresi\'on geri\'atrica. Las caracter\'isticas
de cada sujeto se resumen en la tabla \ref{sujetos}

\begin{table}
\centering
\begin{tabular}{l|cc}
Sujeto & Deterioro cogn. & Depresi\'on
\\
\hline
\\
MJNN &   &   \\
RLMN & X &   \\
JANA &   & X \\
CLMN & X &   \\
JGMN & X &   \\
%& & \\
%& & \\
\end{tabular}
\caption{Caracter\'isticas de los adultos mayores considerados en el estudio, seg\'un la bater\'ia
de pruebas neuropsicol\'ogicas que se les aplicaron. 
Las dos primeras letras fueron dadas por sus nombre mientras que las \'ultimas dos son
mnemotecnias de sus caracter\'isticas: \textbf{M}= mean, \textbf{N}=normal, \textbf{A}=anormal.
Para m\'as detalles consultar \cite{VazquezTagle16}}
\label{sujetos}
\end{table}


||
Este anexo parece impresionante porque eleva el numero de p\'aginas, pero s\'olo son im\'agenes 
cuyo an\'alisis puede ser --y ser\'a-- reducido a la secci\'on [Discusi\'on]. 
Estas im\'agenes son muy importantes
porque muestran una la distribuci\'ion temporal y pseudo-espacial 
de algunas caracter\'isticas de la se\~nal. Pero m\'as que eso, esta distribuci\'on gr\'afica
puede extenderse a otros an\'alisis.


Me siento particularmente orgulloso
de haber dise\~nado este tipo de gr\'aficos, ya que  organizan datos que ya se ten\'ian
y dejan la sensaci\'on de portar nueva informaci\'on.

%\includepdf[pages={1-},scale=.85]{reporte_de_estacionariedad_170120.pdf}
%%%%%%%%%%%%%%%%%%%%%%%%%%%%%%%%%%%%%%%%%%%%%%%%%%%%%%%%%%%%%%%%%%%%%%%%%%%%%%%%%%%%%%%%%%%%%%%%%%%
%%%%%%%%%%%%%%%%%%%%%%%%%%%%%%%%%%%%%%%%%%%%%%%%%%%%%%%%%%%%%%%%%%%%%%%%%%%%%%%%%%%%%%%%%%%%%%%%%%%

%%%%%%%%%%%%%%%%%%%%%%%%%%%%%%%%%%%%%%%%%%%%%%%%%%%%%%%%%%%%%%%%%%%%%%%%%%%%%%%%%%%%%%%%%%%%%%%%%%%
%%%%%%%%%%%%%%%%%%%%%%%%%%%%%%%%%%%%%%%%%%%%%%%%%%%%%%%%%%%%%%%%%%%%%%%%%%%%%%%%%%%%%%%%%%%%%%%%%%%
\section{Discusi\'on}

Como se mencion\'o en la secci\'on de hip\'otesis, este trabajo pare del supuesto en que los
sujetos con PDC presentan con mayor probabilidad estacionariedad d\'ebil en sus registros de EEG.
Esta idea fue sugerida por Cohen \cite{Cohen77}, quien a su vez se refiere a trabajos anteriores
sobre regularidad estad\'stica --estacionariedad y normalidad-- sobre registros de 
EEG \cite{McEwen75,Sugimoto78,Kawabata73}. 
Si bien en estos primeros estudios se palpa la posibilidad de que los registros de EEG fueran
ruido de alg\'un tipo, esta idea se ha probado \'erronea en estudios m\'as recientes 
\cite{Klonowski09}.

Cabe entonces mencionar una segunda justificaci\'on, un poco m\'as arbitraria y personal, sobre
las hip\'otesis de este trabajo: en el trabajo de Valeria [no se como citarlo] se describen
diferencias significativas entre los registros de PSG en adultos mayores con y sin PDC,
refiri\'endose al exponente de Hurst ($H_\alpha$) estimado.
La cantidad $H_\alpha$, tambi\'en referido como el ''color'' de la se\~nal,
mide la ''fractalidad''\footnote{Este concepto no se
describir\'a en este trabajo, para m\'as informaci\'on ver el trabajo de Valeria} 
de un proceso estoc\'astico y es estimado a trav\'es del
algoritmo Detrended Fluctuation Analysis (DFA); se reporta
que el exponente $H_\alpha$ es menor para registros de PSG en adultos mayores con PSG, y que es
cercano a aqu\'el en el movimiento browniano. 
Luego entonces, cabe preguntarse sobre la naturaleza exacta de las diferencias detectadas en 
el trabajo de Valeria: ¿la se\~nal es ''menos compleja'' o 
s\'olo ''tiene otro color''?
De manera concreta, en este trabajo se ha hipotetizado sobre la primera opci\'on.

En cierto modo, se ha aportado evidencias suficientes para decir que no hay cambios significativos
en la porci\'on de tiempo durante la cual el registro de PSG se comporta de manera ''simple''
--es PE. Esto puede interpretarse como que --quiz\'a-- los mecanismos afectados durante el PDC no 
provocan que la se\~nal se vuelva m\'as simple desde el punto de vista estad\'istico

Cabe un comentario sobre c\'omo la evidencia exhibe al PSG como se\~nales no-estacionarias
por una porci\'on muy prque\~na de tiempo; luego, no es adecuado analizarla con m\'etodos que
supongan estacionariedad. M\'as a\'un este comentario aplica para individuos con y sin PDC, y
se acent\'ua m\'as en individuos con problemas adicionales.

\subsubsection{La inclusi\'on de sujetos}

%Con respecto a los sujetos con problemas adicionales, cabe mencionar el caso de FGH

Durante el trabajo se menciona constantemente a tres sujetos (FGH,MGG,EMT) que fueron considerados
pero que no son considerados dentro de las estad\'isticas; 
como se mencion\'o anteriormente,
cada uno de ellos fue exclu\'ido del
trabajo original por diversos motivos, pero dieron su consentimiento informado para la etapa
de registro de sue\~no debido a lo cual se decidi\'o analizar el efecto de su inclusi\'on dentro 
de los estad\'isticas.

El caso m\'as notorio es el sujeto FGH, quien padece de par\'alisis facial, problemas 
no especificados en la 
hipotiroides, en la columna y tiene cataratas. Seg\'un se reporta en el trabajo original,
el sujeto no inform\'o de la par\'alsis facial sino hasta despu\'es del registro de PSG, por lo
que su exlusi\'on se efectu\'o a posteriori.
Si bien la metodolog\'ia presentada aqu\'i no tiene como objetivo el diagn\'ostico
de tal padecimiento --y bajo el etendido que hay m\'etodos menos invasivos para ello--, los
registros confirman picos inusuales 


uashdflhasjkdfhlkasjdhflkajdhlkadfkasld+


\begin{comment}
\subsubsection{Otros estimadores espectrales}

Cabe mencionar que una motivaci\'on muy fuerte para utilizar el test PSR para detectar 
estacionariedad d\'ebil, tiene su origen en el objetivo informal de
''usar un m\'etodo previamente validado, f\'acil de
usar e interppretar, y que se encuentre implementado en software de f\'acil acceso''.
Este anhelo parece cumplido usando la functi\'on \texttt{stationarity} del paquete
\texttt{fractal}, en el software estad\'istico multiplataforma, gratuito y de c\'odigo abierto 
\texttt{R} --al menos en el \'ultimo punto.

Sin embargo, dado que la prueba PSR fue mostrada por primera vez en 1969 \cite{Priestley69}, es
intuitivo que debieran existir enfoques ''m\'as actuales''. En ese sentido
se puede hablar, por ejemplo, de la estimaci\'on del espectro que en este trabajo se realiz\'o a
trav\'es del estimador de doble ventana; un enfoque m\'as moderno que
cabe destacar
con mucho \'enfasis es
la familia de estimadores que satisfacen una serie depropiedades descritas por Cohen 
\cite{Cohen89} y que son referidos
como \textbf{la clase de Cohen}.
Esta clase puede ser interpretada como una ''suavizaci\'on'' del espectrograma, de forma similar al
uso de la ventana espectral; su uso parece m\'as adecuado para 
funciones 
deterministas cuyo espectro cambia en el tiempo, y se ha generalizado su uso para procesos 
d\'ebilmente estacionarios. 
El autor desconoce si existe alguna generalizaci\'on
para espectros de procesos estoc\'asticos.
%, pero se pueden exhibir trabajos donde se usan para
%calcular el espectro de realizaciones de procesos que presuponene como aleatorios [citar].
M\'as a\'un, un enfoque m\'as reciente se basa en la definici\'on de estacionariedad local

\end{comment}

\subsubsection{Otros usos para las t\'ecnicas utilizadas}

Como una etapa exploratoria de este trabajo, se dio un
%Un primer 
tratamiento cualitativo a los resultados obtenidos del test PSR
% es su
%disposici\'on gr\'afica.
grafic\'andolos de varias formas distintas; cabe destacar una de ellas que pudiera resultar
\'util pero que no hubo aportado suficiente informaci\'on clara sobre la hip\'otesis
principal de este trabajo.
%Una vez se hubo realizado el test para todas las \'epocas consideradas, se dispuso de los 
%resultados de manera gr\'afica  
%como se muestra en la figura \ref{ejemplo1}.
En esta disposici\'on gr\'afica,
se coloc\'o en l\'inea horizontal un cuadro blanco por cada
\'epoca PE (negro para \'epocas no-estacionarias);
% seg\'un el 
%el segmento en cuesti\'on halla sido clasi
%segmento referido haya sido clasificado como
%no-estacionario o posiblemente estacionario; 
posteriormente se colocaron verticalmente las
l\'ineas as\'i obtenidas.
% de todos los canales.
%Esta disposici\'on gr\'afica pretende ser consistente con las representaciones gr\'aficas
%usuales de EEG.
%, tomando en cuanta una escala m\'as amplia de tiempo gracias a que por
%cada \'epoca s\'olo se ha obtenido un dato.
Puede verse en la figura \ref{ejemplo1} un ejemplo de esta disposici\'on gr\'afica, mientras
que el resto de estos gr\'aficos se incluye como anexo.
%Los gr\'aficos as\'i obtenidos se incluyen como anexo.
%como se muestra en la figura \ref{ejemplo1}.

\begin{figure}
\includegraphics[width=\textwidth]{MJNNVIGILOS_127_mor127_tot1032_esttotal.pdf} 
\caption{Disposici\'on gr\'afica para los resultados del test PSR en el sujeto MJH, 
para 1032 \'epocas de sue\~no y 22 canales. 
En el eje horizontal se muestra el tiempo desde el inicio de registro, en el eje vertical se muestra al 
nombre del canal. 
Se han resaltado con color verde las \'epocas clasificadas como sue\~no MOR (ver texto), que son 127.
Para este gr\'afico se consider\'o con un p-valor cr\'itico de 0.01 para la hip\'otesis
de estacionariedad. Ver texto para m\'as detalles.}
\label{ejemplo1}
\end{figure}

Una debilidad importante de los gr\'aficos as\'i obtenidos es que, si bien muestran patrones 
claros en el tiempo, \'estos no se pueden cuantificar de una manera obvia y se dificulta la
comparaci\'on entre sujetos, raz\'on por la cual se omiti\'on del cuerpo principal del trabajo. 
%Se han inclu\'ido estos resultados porque sus caracter\'isticas 
%sugieren una posible utilizaci\'on para otros fines --en alg\'un trabajo futuro.
Sin embargo, dentro de un mismo sujeto, parecen visibles diferencias cualitativas
entre el sue\~no MOR y el resto del sue\~no nocturno.

Conviene mencionar que el origen de esta representaci\'on gr\'afica es un intento preeliminar de
fragmentar los an\'alsis de sue\~no MOR en grupos de \'epocas consecutivas en sue\~no MOR ya que,
en general, el sue\~no MOR aparece fragmetnado durante el sue\~no nocturno \cite{CarrilloMora}.
La hip\'otesis de que se podr\'ian definir diferencias que involucraran la componente espacial,
sin embargo, se vio opacada por la dificultad de definir formalmente tales diferencias a modo
que pudieran compararse entre sujetos.
Una sugerencia recibida consiste en seguir explorando estos patrones en el tiempo, pero quiz\'a
no con la intenci\'on de detectar deterioro cognitivo sino como apoyo para la identificaci\'on de
diferentes etapas de sue\~no.

%\begin{figure}
%\includegraphics[width=\textwidth]{est02.png} 
%\caption{En este gráfico sólo se ilustran épocas MOR. Las líneas punteadas separan bloques continuos.
%Total de épocas: 1032 , Épocas MOR: 127}
%\label{ejemplo2}
%\end{figure}

%Me siento particularmente orgulloso
%de haber dise\~nado este tipo de gr\'aficos, ya que  organizan datos que ya se ten\'ian
%y dejan la sensaci\'on de portar nueva informaci\'on.

%\includepdf[pages={1-},scale=.85]{reporte_de_estacionariedad_170120.pdf}
%
%\afterpage{%
%    \clearpage% Flush earlier floats (otherwise order might not be correct)
%    \thispagestyle{empty}% empty page style (?)
%    \begin{landscape}% Landscape page
%        \centering % Center table
%        \begin{figure}
%            \includegraphics[width=\textwidth]{MJNNVIGILOS_127_mor127_tot1032_esttotal.pdf} 
%            \caption{Total de \'epocas: 1032, \'epocas MOR: 127}
%            %\label{ejemplo1}
%        \end{figure}
%    \end{landscape}
%    \clearpage% Flush page
%}

%%%%%%%%%%%%%%%%%%%%%%%%%%%%%%%%%%%%%%%%%%%%%%%%%%%%%%%%%%%%%%%%%%%%%%%%%%%%%%%%%%%%%%%%%%%%%%%%%%%
%%%%%%%%%%%%%%%%%%%%%%%%%%%%%%%%%%%%%%%%%%%%%%%%%%%%%%%%%%%%%%%%%%%%%%%%%%%%%%%%%%%%%%%%%%%%%%%%%%%

\section{Conclusiones}

Se aportan evidencias sobre que la presencia proporcional de estacionariedad d\'ebil en registros 
de PSG para adultos mayores, 
no presenta diferencias significativas entre sujetos con y sin PDC diagnosticado.
Luego entonces, esta caracter\'istica no es un indicador fiable 

%%%%%%%%%%%%%%%%%%%%%%%%%%%%%%%%%%%%%%%%%%%%%%%%%%%%%%%%%%%%%%%%%%%%%%%%%%%%%%%%%%%%%%%%%%%%%%%%%%%
%%%%%%%%%%%%%%%%%%%%%%%%%%%%%%%%%%%%%%%%%%%%%%%%%%%%%%%%%%%%%%%%%%%%%%%%%%%%%%%%%%%%%%%%%%%%%%%%%%%

\section{Trabajo a futuro}

Como se ha sugerido, los bloques de estacionariedad pueden tener un uso como 
caracter\'isticas auxiliares
para la detecci\'on autom\'atica de \'epocas MOR en registros de PSG: el hecho que la proporci\'on
de \'epocas PE no se vea afectada --estad\'isticamente-- por el PDC del paciente, sugiere que es
posible obtener resultados independientes de ello. Para ello cabe recordar, como se mencion\'o 
en la secci\'on de discusi\'on, que sujetos fuera del rango de los grupos considerados puede
que fallen respecto a esta conclusi\'on: hace falta m\'as indagaci\'on al respecto. 

Por otro lado, el uso de estimadores espectrales de ventana pueden explorarse de manera m\'as
puntual para detectar estacionariedad sobre componentes de frecuencia espec\'ificas, de modo
que es en principio posible separar las ondas cerebrales.

%%%%%%%%%%%%%%%%%%%%%%%%%%%%%%%%%%%%%%%%%%%%%%%%%%%%%%%%%%%%%%%%%%%%%%%%%%%%%%%%%%%%%%%%%%%%%%%%%%%
%%%%%%%%%%%%%%%%%%%%%%%%%%%%%%%%%%%%%%%%%%%%%%%%%%%%%%%%%%%%%%%%%%%%%%%%%%%%%%%%%%%%%%%%%%%%%%%%%%%

\appendix

\chapter{Resultados no re-redactados}

Estos reultados ya fueron presentados a la Dra Rosales Lagarde de ICSA, con quien se
trabajo si eran de relevancia fisiologica o no. Aun me falta redactar como texto las conclusiones
encontradas, y que tengo dispersas como notas. 

Cabe mencionar que se incluyen partes de un analisis
sobre somposicion de frecuencias sobre el que no he hablado para nada en el resto del texto,
y es que apenas y se esta trabajando en ello: he priorizado en el tiempo la redaccion
sobre  las bases
formales del test PSR, porque ya esta batsante avanzado y no habia escrito al respecto nada que fuera
suficientemente bueno. Si continuaba con esta actitud, ocurriria lo mismo con el analisis por bandas.

Aunque parece impresionante porque eleva el numero de paginas, son imagenes cuyo analisis puede 
reducirse a menos de 5 p\'aginas. Estas im\'agenes son muy importantes
porque muestran una suerte de distribuci\'ion temporal y --de manera reducida-- espacial 
de algunas caracter\'isticas de la se\~nal. 

Me siento particularmente orgulloso
de haber dise\~nado este tipo de gr\'aficos, ya que 
simlemente organizan
gr\'aficamente los datos que ya se ten\'ian de una forma
totalmente contraria a algo novedoso,
y a\'un as\'i dejan la sensaci\'on de portar nueva informaci\'on.

\includepdf[pages={1-},scale=.85]{reporte_de_estacionariedad_170120.pdf}

%\include{fft}

%%%%%%%%%%%%%%%%%%%%%%%%%%%%%%%%%%%%%%%%%%%%%%%%%%%%%%%%%%%%%%%%%%%%%%%%%%%%%%%%%%%%%%%%%%%%%%%%%%%
%%%%%%%%%%%%%%%%%%%%%%%%%%%%%%%%%%%%%%%%%%%%%%%%%%%%%%%%%%%%%%%%%%%%%%%%%%%%%%%%%%%%%%%%%%%%%%%%%%%

\bibliography{referencias_estacionariedad,referencias_fisiologia,referencias_otros}{}
%\bibliographystyle{apalike-es}
\bibliographystyle{plain}

%%%%%%%%%%%%%%%%%%%%%%%%%%%%%%%%%%%%%%%%%%%%%%%%%%%%%%%%%%%%%%%%%%%%%%%%%%%%%%%%%%%%%%%%%%%%%%%%%%%
%%%%%%%%%%%%%%%%%%%%%%%%%%%%%%%%%%%%%%%%%%%%%%%%%%%%%%%%%%%%%%%%%%%%%%%%%%%%%%%%%%%%%%%%%%%%%%%%%%%

\end{document}