\documentclass[10pt,a4paper]{article}
\usepackage[utf8]{inputenc}
\usepackage[spanish]{babel}

\usepackage{cite}

\usepackage{amsmath}
\usepackage{amsfonts}
\usepackage{amssymb}
\usepackage{graphicx}

%%%%%%%%%%%%%%%%%%%%%%%%%%%%%%%%%%%%%%%%%%%%%%%%%%%%%%%%%%%%%%%%%%%%%%%%%%%%%%%%%%%%%%%%%%%%%%%%%%%
%%%%%%%%%%%%%%%%%%%%%%%%%%%%%%%%%%%%%%%%%%%%%%%%%%%%%%%%%%%%%%%%%%%%%%%%%%%%%%%%%%%%%%%%%%%%%%%%%%%

\begin{document}

%%%%%%%%%%%%%%%%%%%%%%%%%%%%%%%%%%%%%%%%%%%%%%%%%%%%%%%%%%%%%%%%%%%%%%%%%%%%%%%%%%%%%%%%%%%%%%%%%%%

\section{Introducci\'on}

%Visually, Rapid Eye Movement (REM) sleep is characterized by REMs, muscle atonia and desynchronized EEG activity. When quantitative analyses of the signals are carried out, usually, non-linearity and non-stationarity are assumed without an adequate analysis, especially in Old Adults (OA). Among the “weak” stationarity tests, the Priestley-Subba Rao (PSR) test calculates a “local” spectra that is “valid” only for punctual moments in time. A series of “smoothed” frequency filters give information of the time the local spectra is calculated. In here, weak REM sleep stationarity by the PSR test was compared to that from Wakefulness (W) and Non-REM (NREM) sleep. Methods:  8 Old Adults (OA) (age: 67.6 ± 5.7; education: 8.8 ± 2.6) without depression neither anxiety and with intact daily living activities were selected. Also, evaluations with the Mini-Mental State Examination (MMSE, 28.1 ± 1.8) and a one night polysomnography were performed. 30 second epochs were classified according to the AASM and every epoch of W, NREM and REM sleep was subjected to PSR tests. Percentages of stationary epochs were obtained with respect to the total number of epochs of each stage and Student t-tests were used to compare them. Results: The PSR effectively showed different proportions of stationarity according to the classification of stages in each subject. In Figure 1, in one OA, epochs with stationarity are shown in black and the classification of REM sleep is shown in green. Clearly, a lower proportion of stationarity was found in REM sleep vs the other stages. These differences reached significance in F7, Fp2, LOG and ROG (p < 0.05, Figure 2). Conclusions: In OA, REM sleep showed lower proportions of epochs with stationarity vs. W and NREM sleep at anterior areas, a result that could be explained by the tonic and phasic REM sleep. When stationarity measurements are planned, it is recommended to differentiate anterior from lateral and posterior areas.

Visualmente, el sue\~no MOR se caracteriza por movimientos oculares r\'apidos, aton\'ia muscular y 
una actividad electroencefalogr\'afica desincronizada \cite{RosalesLagarde09}.

%%%%%%%%%%%%%%%%%%%%%%%%%%%%%%%%%%%%%%%%%%%%%%%%%%%%%%%%%%%%%%%%%%%%%%%%%%%%%%%%%%%%%%%%%%%%%%%%%%%

\section{Morfolog\'a neuronal y cambios en la anatomía cerebral con la vejez (aka parte fisiol\'ogica)}

(Esta secci\'on entera es citada de G\'enesis)

El envejecimiento considerado normal viene determinado por una serie de procesos moleculares, 
celulares, fisiol\'ogicos y psicol\'ogicos que conducen directamente al deterioro de funciones 
cognitivas, específicamente en la atenci\'on y memoria \cite{Navarrete03,Park09}.

En un principio se consideraba que el envejecimiento cerebral ocurr\'ia fundamentalmente por una 
muerte neuronal programada \cite{Coleman87}, sin embargo, estudios realizados con tejido cerebral 
post mortem de adultos mayores que en vida fueron sanos, mostraron que dicha muerte neuronal no 
alcanza un 10\% en su totalidad \cite{Esiri07}. En este sentido, los cambios morfol\'gicos que 
sufren las neuronas durante el envejecimiento son abundantes, observ\'andose una importante 
disminuci\'on de la arborizaci\'n dendr\'itica as\'i como en la densidad y volumen \cite{Hita14}. 
La disminución en la arborización dendr\'itica y de las espinas dendr\'iticas de las neuronas 
piramidales de la corteza prefrontal, temporal superior, pre central y occipital \cite{Hita14}. 
Dichas alteraciones morfol\'ogicas conducen durante el envejecimiento a una disminuci\'on de 
la densidad sin\'aptica y a una desmielinizaci\'on ax\'onica en neuronas de la 
neocorteza \cite{Terry}.

Con el paso del tiempo, la organizaci\'on an\'atomo-funcional del cerebro sufre modificaciones 
que traen como consecuencia la afectaci\'on de diferentes capacidades cognitivas, sin embargo, 
la vulnerabilidad de los circuitos neuronales ante los procesos que ocurren durante el 
envejecimiento no suceden de forma homogénea en todo el cerebro \cite{Hita14}.

Por otro lado, la relevancia del estudio de los cambios anat\'omicos asociados al envejecimiento 
fisiol\'ogico ha ido aumentando al permitir evaluar como dichos cambios se correlacionan con 
el deterioro funcional y cognitivo que caracteriza a las personas mayores, facilita la 
identificaci\'on de estadios tempranos de diferentes patolog\'ias neurodegenerativas estableciendo 
diferencias entre estas y los cambios asociados al envejecimiento fisiol\'ogico \cite{Hita14}.

Durante el envejecimiento, el cerebro sufre una afectaci\'n progresiva del peso \cite{Dekaban78} 
y volumen \cite{Hubbard81} cambios atribuidos a la reducci\'on de sustancia gris y blanca en
las regiones c\'ortico-subcorticales \cite{Hita14}.

Consecuencia de la disminuci\'on de la sustancia gris cortical, se produce una reducci\'on de 
la girificaci\'on en las circunvoluciones, as\'i como un incremento de la profundidad y 
expansi\'on de los surcos de la corteza, siendo estos fen\'omenos ms acentuados en los l\'obulos 
frontales, temporales y parentales, y mucho menos evidentes en la corteza occipital \cite{Raz05}.

Los cambios de presi\'on externa producidos por la dilataci\'on de las astas frontales y por la 
disminuci\'on de la sustancia blanca peri ventricular durante la etapa del envejecimiento 
provocan tambi\'en un aumento del espacio ventricular que conduce a la expansi\'on del l\'iquido 
cerebroespinal \cite{Hita14,Raz05}.
Durante el envejecimiento se reduce significativamente el volumen de estructuras subcorticales 
como la am\'igdala \cite{Allen05}, el n\'ucleo caudado \cite{Raz05}, el t\'alamo 
\cite{CarrilloMora} y el cerebelo \cite{Hita14}.

%%%%%%%%%%%%%%%%%%%%%%%%%%%%%%%%%%%%%%%%%%%%%%%%%%%%%%%%%%%%%%%%%%%%%%%%%%%%%%%%%%%%%%%%%%%%%%%%%%%

\section{El sue\~no (aka explicaci\'on fisiol\'ogica)}

(Esta secci\'on tambi\'en es copiada, por el momento)

El sue\~no se define como un proceso vital c\'iclico complejo y activo, compuesto por varias 
fases y que posee una estructura interna caracter\'istica, con diversas interrelaciones en los 
sistemas hormonales y nerviosos \cite{FernandezConde07}. Una suspensi\'on f\'acilmente reversible 
de la interacci\'on sensoriomotriz con el medio ambiente, por lo general asociados con el 
dec\'ubito y la inmovilidad.

El sue\~no se determina por cuatro dimensiones diferentes: tiempo circadiano (esto es la hora 
del d\'ia en que se localiza); factores intr\'insecos del organismo (edad, sexo, patrones de 
sue\~o, estado fisiol\'ogico o necesidad de dormir, entre otros); conductas que facilitan o 
inhiben el sue\~no; y el ambiente. Las dos \'ultimas dimensiones se relacionan con la higiene 
del sueño, que incluye las pr\'acticas necesarias para mantener un sue\~no nocturno y una 
vigilancia diurna normales \cite{Sierra02}.

Las caracter\'isticas conductuales que se asocian con el sue\~no en el ser humano pueden 
enumerarse de la siguiente forma\cite{CarrilloMora} 
\begin{enumerate}
\item Disminuci\'on de la conciencia y reactividad a los est\'imulos externos
\item Se trata de proceso f\'acilmente reversibles (lo cual lo diferencia de otros estados 
patol\'ogicos como el estupor y el coma)
\item Se asocia a inmovilidad y relajaci\'on muscular
\item Suele presentarse con una periodicidad circadiana (diaria)
\item Durante el sue\~no los individuos adquieren una postura estereotipada
\item La ausencia de sue\~no (privaci\'on), induce distintas alteraciones conductuales y 
fisiol\'ogicas, adem\'as de que genera una ''deuda'' acumulativa de sueño que eventualmente 
deber\'a recuperarse 
\end{enumerate}

\subsection{Fisiolog\'ia del sueño}

%%%%%%%%%%%%%%%%%%%%%%%%%%%%%%%%%%%%%%%%%%%%%%%%%%%%%%%%%%%%%%%%%%%%%%%%%%%%%%%%%%%%%%%%%%%%%%%%%%%%

Los organismos vivos tienen su propio ritmo de actividad y reposo, mismos que desencadenan en la 
percepci\'on de ciclos naturales tales como la sucesión del d\'ia y la noche. En este sentido, 
el sustrato neurol\'ogico relacionado con la ritmicidad del sueño se encuentra en el hipot\'alamo, 
estructura que tiene diversidad de conexiones en el Sistema Nervioso Central, con el fin de 
ejercer una funci\'on o funciones capaces de sincronizar el organismo \cite{FernandezConde07,Cabrera14}.

%	Diversos y muy importantes procesos fisiológicos y cerebrales, están estrechamente relacionados o determinados por el sueño o la periodicidad del mismo. A este respecto, existen diversas teorías acerca de las funciones del sueño, por ejemplo: 1) restablecimiento o conservación de la energía, 2) eliminación de radicales libres acumulados durante el día, 3) regulación y restauración de la actividad eléctrica cortical, 4) regulación térmica, 5) regulación metabólica y endocrina, 5) homeostasis sináptica, 7) activación inmunológica, 8) consolidación de la memoria, etc
%	Las estructuras límbicas, tales como la amígdala y el hipotálamo, también estarían activadas, lo que explicaría los fenómenos emotivos durante la fase de sueño REM ya que las emociones están directamente vinculadas con estas zonas cerebrales43.
%
%%%%%%%%%%%%%%%%%%%%%%%%%%%%%%%%%%%%%%%%%%%%%%%%%%%%%%%%%%%%%%%%%%%%%%%%%%%%%%%%%%%%%%%%%%%%%%%%%%%%
%
%4.2 Función del sueño
%	Los efectos del sueño no se limitan al propio organismo (necesidad de restauración neurológica y la salud), sino que influyen en el desarrollo y funcionamiento normal de un individuo en la sociedad, afectando el rendimiento laboral o escolar 42,44–46, el bienestar psicosocial47–49, la seguridad vial, entre otras50.  
%Dentro de los factores que se pueden ver afectados por la disminución de las horas de sueño se encuentra la calidad del sueño, la cual no sólo se refiere al hecho de dormir bien durante la noche, sino que incluye también un buen funcionamiento diurno. 
%
%%%%%%%%%%%%%%%%%%%%%%%%%%%%%%%%%%%%%%%%%%%%%%%%%%%%%%%%%%%%%%%%%%%%%%%%%%%%%%%%%%%%%%%%%%%%%%%%%%%%
%
%4.3 Etapas del sueño
%El sueño normal se divide en dos etapas: sueño REM (Rapid-eyemovement) o también conocido como sueño MOR (movimiento-ocular-rápido) y sueño no-REM, los cuales se diferencian fundamentalmente por sus rasgos electroencefalográficos y una serie de características fisiológicas 51
%Una herramienta tecnológica que ha sido de vital importancia para el estudio de la fisiología del sueño es el electroencefalograma (EEG). De forma muy simplificada, el EEG es el la representación gráfica y digital de las oscilaciones que muestra la actividad eléctrica del cerebro, al ser registrada mediante electrodos colocados encima de la piel cabelluda en distintas regiones de la cabeza 4
%El sueño MOR se caracteriza por la presencia de ondas de bajo voltaje y alta frecuencia en el electroencefalograma, atonía muscular y movimientos oculares rápidos, además es donde se presentan la mayoría de los sueños. El sueño no- MOR se compone de cuatro fases, 1 y 2, que son de sueño ligero, y 3 y 4 de sueño profundo, las mismas que transcurren de manera secuencial desde la primera hasta la cuarta fase, que es la fase reparadora del sueño, aquella que produce en la persona la sensación de haber descansado cuando se levanta 13,22,43.
%Las características de las fases del sueño no-MOR incluyen cuatro etapas, la primera que corresponde a la transición de la vigilia al sueño; la etapa 2 es la intermedia (mayor porcentaje del tiempo de sueño) y en el EEG aparecen husos de sueño y los complejos K. La etapa 3 es la del sueño relativamente profundo, representado en el electroencefalograma por ondas lentas de gran amplitud, y la etapa 4o de sueño profundo con más del 50% de ondas lentas de gran amplitud13.
%	Durante el estado de alerta, mientras se mantienen los ojos cerrados, en el EEG se observan oscilaciones de la actividad eléctrica que suelen encontrarse entre 8-13 ciclos por segundo (Hz), principalmente a nivel de las regiones occipitales (ritmo alfa). Durante el sueño ocurren cambios característicos de la actividad eléctrica cerebral que son la base para dividir el sueño en varias fases. Como ya se mencionó, el sueño suele dividirse en dos grandes fases que, de forma normal, ocurren siempre en la misma sucesión: todo episodio de sueño comienza con el llamado sueño sin movimientos oculares rápidos (No MOR), que tiene varias fases, y después pasa al sueño con movimientos oculares rápidos (MOR). La nomenclatura acerca de las fases del sueño ha sido recientemente modificada por la Academia Americana de Medicina del Sueño (2007)52. Quedó de la siguiente manera:
%
%%%%%%%%%%%%%%%%%%%%%%%%%%%%%%%%%%%%%%%%%%%%%%%%%%%%%%%%%%%%%%%%%%%%%%%%%%%%%%%%%%%%%%%%%%%%%%%%%%%%
%
%4.3.1 Sueño No MOR. 
%Fase 1 (ahora denominada N1): esta fase corresponde con la somnolencia o el inicio del sueño ligero, en ella es muy fácil despertarse, la actividad muscular disminuye paulatinamente y pueden observarse algunas breves sacudidas musculares súbitas que a veces coinciden con una sensación de caída (mioclonías hípnicas), en el EEG se observa actividad de frecuencias mezcladas, pero de bajo voltaje y algunas ondas agudas (ondas agudas del vértex). Fase 2 (ahora denominada N2): en el EEG se caracteriza por que aparecen patrones específicos de actividad cerebral llamados husos de sueño y complejos K; físicamente la temperatura, la frecuencia cardiaca y respiratoria comienzan a disminuir paulatinamente. Fases 3 y 4 o sueño de ondas lentas (en conjunto llamadas fase N3): esta es la fase de sueño No MOR más profunda, y en el EEG se observa actividad de frecuencia muy lenta (< 2 Hz)53.
%
%%%%%%%%%%%%%%%%%%%%%%%%%%%%%%%%%%%%%%%%%%%%%%%%%%%%%%%%%%%%%%%%%%%%%%%%%%%%%%%%%%%%%%%%%%%%%%%%%%%%
%
%4.3.2 Sueño MOR. 
%Ahora es llamado fase R y se caracteriza por la presencia de movimientos oculares rápidos; físicamente el tono de todos los músculos disminuye (con excepción de los músculos respiratorios y los esfínteres vesical y anal), así mismo la frecuencia cardiaca y respiratoria se vuelve irregular e incluso puede incrementarse y existe erección espontánea del pene o del clítoris. Durante el sueño MOR se producen la mayoría de las ensoñaciones (lo que conocemos coloquialmente como sueños), y la mayoría de los pacientes que despiertan durante esta fase suelen recordar vívidamente el contenido de sus ensoñaciones53.
%Por otro lado, las necesidades de sueño son muy variables según la edad y las circunstancias individuales 43,54:
%Un niño recién nacido duerme casi todo el día, con una proporción próxima al 50 % del denominado sueño «activo», que es el equivalente del sueño MOR. A lo largo de la lactancia los períodos de vigilia son progresivamente más prolongados y se consolida el sueño de la noche; además, la proporción de sueño MOR desciende al 25-30 %, que se mantendrá durante toda la vida. Entre el 1er y 3er año de vida el niño ya sólo duerme una o dos siestas. Entre los 4 y 5 años y la adolescencia los niños son hipervigilantes, muy pocos duermen siesta, pero tienen un sueño nocturno de 9-10 horas bien estructurado en 5 ciclos o más. Por lo que se refiere a los individuos jóvenes, en ellos reaparece en muchos casos la necesidad fisiológica de una siesta a mitad del día43,55.
%La necesidad de sueño en un adulto puede oscilar entre 5 y 9 horas. Asimismo, varía notablemente el horario de sueño entre noctámbulos y madrugadores. En épocas de mucha actividad intelectual o de crecimiento o durante los meses del embarazo, puede aumentar la necesidad de sueño, mientras que el estrés, la ansiedad o el ejercicio físico practicado por la tarde pueden reducir la cantidad de sueño. Los estudios efectuados en individuos aislados de influencias exteriores han mostrado que la tendencia fisiológica general es a retrasar ligeramente la fase de sueño con respecto al ciclo convencional de 24 horas y a dormir una corta siesta “de mediodía” 43,56. Un adulto joven pasa aproximadamente entre 70-100 min en el sueño no MOR para después entrar al sueño MOR, el cual puede durar entre 5-30 min, y este ciclo se repite cada hora y media durante toda la noche de sueño. Por lo tanto, a lo largo de la noche pueden presentarse normalmente entre 4 y 6 ciclos de sueño MOR 22
%Por otro lado, en los ancianos se va fragmentando el sueño nocturno con frecuentes episodios de despertar y se reduce mucho el porcentaje de sueño en fase IV y no tanto el de sueño MOR, que se mantiene más constante a lo largo de la vida. Las personas de edad avanzada tienen tendencia a aumentar el tiempo de permanencia en la cama. Muchas de ellas dormitan fácilmente durante el día varias siestas cortas43.
%
%%%%%%%%%%%%%%%%%%%%%%%%%%%%%%%%%%%%%%%%%%%%%%%%%%%%%%%%%%%%%%%%%%%%%%%%%%%%%%%%%%%%%%%%%%%%%%%%%%%%
%
%4.4 Alteraciones del ciclo vigilia-sueño
%La relevancia que tiene el sueño para para la supervivencia de un individuo es la cantidad de horas que este duerme a lo largo de su vida, mismas que depende fundamentalmente de sus necesidades fisiológicas y de las demandas del ambiente al que está expuesto 4,57
% En el caso de los humanos, es posible establecer una clasificación de patrones de sueño en función de su duración (corta, intermedia y larga) 4. Las personas que muestran un patrón de sueño intermedio, es decir, duración aproximada de entre 7-8 horas, presentan un mejor estado de salud a lo largo de su vida, comparado con los individuos de duración de sueño corta o excesivamente larga que frecuentemente tienen de problemas de salud y/o laborales 42,45,46 . 
%La estabilidad del sueño nocturno es otro factor a tener en cuenta debido a que es razonable pensar que un sueño muy fragmentado no cumplirá con sus funciones fisiológicas de igual forma que un patrón de sueño estable a lo largo de la noche. Al respecto, los adultos mayores informan que duermen menos durante la noche, y se acuestan y se despiertan más temprano de lo habitual. Además, tardan más tiempo en conciliar el sueño, se despiertan con más frecuencia durante la noche y la duración de estos despertares es más prolongada 58,59.
%La disminución del tiempo de sueño asociada a un incremento de la somnolencia diurna incide negativamente en la función cerebral del día siguiente 60
%Por otro lado, existen diversas formas de pérdida de sueño13,25,46: a) la privación de sueño, que quiere decir la suspensión total del sueño por un periodo (> 24 h), b) la restricción del sueño, que significa una disminución del tiempo habitual de sueño, generalmente de forma crónica, y c) la fragmentación del sueño, que significa la interrupción repetida (despertares) de la continuidad del sueño14. Todos estos tipos de alteraciones del sueño han demostrado afectar distintas funciones cognitivas y variedades de memoria en mayor o menor grado.
%Las alteraciones de sueño específicamente en personas mayores se han asociado con la presencia de enfermedades crónicas, problemas físicos y de salud mental 3

%%%%%%%%%%%%%%%%%%%%%%%%%%%%%%%%%%%%%%%%%%%%%%%%%%%%%%%%%%%%%%%%%%%%%%%%%%%%%%%%%%%%%%%%%%%%%%%%%%%

\section{Estacionariedad d\'ebil}

%\chapter{Metodolog\'ia}

%\chapter{Resultados}

\bibliography{referencias_enciso_alva}{}
%\bibliographystyle{apalike-es}
\bibliographystyle{plain}

\end{document}