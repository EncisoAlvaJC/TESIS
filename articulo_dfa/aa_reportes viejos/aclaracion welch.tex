\documentclass[10pt,a4paper]{article}
\usepackage[utf8]{inputenc}
\usepackage[spanish]{babel}
\usepackage{amsmath}
\usepackage{amsfonts}
\usepackage{amssymb}

% para que escriba 'Figura...' en negritas
\usepackage[labelfont=bf]{caption}

% tablas con lineas gruesas
\usepackage{booktabs}
\usepackage{multirow}

% para usar doble caption
\usepackage{caption}

% numeros con punto decimal (el default es coma decimal)
\decimalpoint

% para centrar numeros sin .0
\newcommand{\pz}{\phantom{.0}}
\newcommand{\pp}{\phantom{0}}

\begin{document}

Aclaración sobre los grados de libertad en el test t de Welch para varianzas desiguales.

\textbf{t de Student}

Para comparar, la prueba t de Student trabaja bajo el supuesto que se tienen dos muestras
$\{ X_1, \dots , X_{N_1} \}$ y $\{ Y_1, \dots , Y_{N_2} \}$ las cuales 
provienen de 
distribuciones normales, $X \sim N(\mu_1,\sigma)$ y $Y \sim N(\mu_2,\sigma)$.
Se compara la hipótesis 
$
H_0 : \mu_1 = \mu_2
$
mediante el estadístico 
$$ t = \frac{\overline{X}-\overline{Y}_2}{S_{X,Y}\sqrt{\frac{1}{N_1}+\frac{1}{N_2}}} $$
donde
$$ S_{X,Y} = \sqrt{\frac{(N_1-1)S_X^2+(N_2-1)S_Y^2}{N_1+N_2-2}} $$

y $\overline{X}$ y $S^2$ son estimadores para los promedios y varianzas muestrales de
$X$, y $Y$ lo mismo para $Y$.
Bajo estas condiciones, $t$ tiene una distribución $\chi^2$ con $N_1+N_2-2$ grados de libertad, es
decir 
\begin{align*}
t &\sim \chi^2(\nu) \\
 \nu &= N_1+N_2-2
\end{align*}
El símbolo $\nu$ para los grados de libertad lo uso aquí porque así aparece en los libros de 
estadística, en las tablas se usa \texttt{dF}.

Claramente $\nu$ cambia radicalmente si se usan los exponentes de Hurst de todas las épocas o
sólo el promedio por sujeto
\begin{align*}
\nu_{\text{todo}} &= 9\times 30 -2 = 268 \\
\nu_{\text{promedio}} &= 9\times 1 -2 = 7
\end{align*}
Estas cantidades cambian, por ejemplo, para el canal Fp2 que para algunos sujetos generó errores
y no participa.

\textbf{t de Welch}

El test de Welch parte de las dos muestras, aunque permite un supuesto más débil de que las
varianzas muestrales no tienen que ser iguales, es decirque $X \sim N(\mu_1,\sigma_1)$ y 
$Y \sim N(\mu_2,\sigma_2)$. La hipótesis $\mu_1=\mu_2$ se compara usando el siguiente estadístico
$$
t = \frac{\overline{X}-\overline{Y}}{S_{X-Y}}
$$
donde
$$
S_{X-Y} = \sqrt{\frac{S_X^2}{N_1}+\frac{S_Y^2}{N_2}}
$$
Un aporte clave por Welch es aproximar a $t$ usando una distribución de Pearson tipo III, que a su
vez sigue aproximadamente una distribución $\chi^2$ con $\nu$ grados de libertad y tal que
$$
\nu \approx \frac{(\gamma_X+\gamma_Y)^2}{\frac{\gamma_X^2}{N_1-1}+\frac{\gamma_Y^2}{N_2-1}}
$$
Con $\gamma_X = \frac{\sigma_X^2}{N_1}$; se ocupa una tercera aproximación 
al usar $S_X$, la varianza muestral, como una aproximación de $\sigma_X^2$. Similarmente para $Y$.

Como ejemplo tomaré el canal F8 porque no generó ningún problema en particular;
usando las 10 épocas MOR  los cálculos hechos en R para la varianza
\begin{align*}
N_{\text{CTRL}} &= 50 \\
N_{\text{PMCI}} &= 40 \\
S_{\text{CTRL}} &= 0.163 \\
S_{\text{PMCI}} &= 0.167
\end{align*}
Con lo cual se calcula que
\begin{align*}
\gamma_{\text{CTRL}} = \frac{0.163^{2}}{50} = 0.000531 \\
\gamma_{\text{PMCI}} = \frac{0.167^{2}}{40} = 0.000697 
\end{align*}
$$
\nu = \frac{(0.000531+0.000697)^{2}}{\frac{0.000531^{2}}{49}+\frac{0.000697^{2}}{39}} = \cdots = 82.81
$$

Este valor es muy cercano al que se reporta ya que se han usado las cantidades truncadas a 3 cifras 
significativas, pero en general sí es esa cantidad de grados de libertad que deberían ser.

Una cosa que hay que tomar muy en cuenta al interpretar los grados de libertad de la prueba de Welch es 
que, a diferencia de la prueba de Student, se ven afectados por la varianza muestral por lo que
incluso cambia entre cada par de comparaciones.

Por otro lado, el efecto del tamaño muestral es evidente, y cabe la opción de realizar el mismo análisis
usando únicamente los promedios.

\begin{table}
\centering
\caption{Comparación Exponetnte de Hurst entre los dos grupos usando las 10 épocas MOR}
\begin{tabular}{llclcccc}
\toprule
 & \multicolumn{1}{l}{CTRL} & \phantom{.} & \multicolumn{1}{l}{PMCI} & \phantom{.} &
 \multicolumn{3}{l}{Welch t test} \\
\cmidrule{2-2} \cmidrule{4-4} \cmidrule{6-8}
& Mean (SD) & & Mean (SD) & & $p$ & $T$ & dF \\
\midrule
Fp2 &	1.227	(	0.160	) & &	1.312	(	0.193	) & &	0.050 &	-2.01 &	52.6 \\
Fp1 &	1.242	(	0.152	) & &	1.268	(	0.210	) & &	0.617 &	-0.51 &	27.3 \\
F8 &	1.251	(	0.163	) & &	1.346	(	0.167	) & &\bf	0.008 &	-2.70 &	82.7 \\
F7 &	1.278	(	0.198	) & &	1.379	(	0.177	) & &\bf	0.013 &	-2.55 &	86.9 \\
F4 &	1.225	(	0.206	) & &	1.325	(	0.209	) & &\bf	0.025 &	-2.28 &	83.1 \\
F3 &	1.238	(	0.172	) & &	1.250	(	0.175	) & &	0.741 &	-0.33 &	83.1 \\
T4 &	1.253	(	0.181	) & &	1.330	(	0.152	) & &\bf	0.029 &	-2.22 &	87.8 \\
T3 &	1.182	(	0.173	) & &	1.265	(	0.225	) & &	0.058 &	-1.93 &	72.0 \\
C4 &	1.210	(	0.144	) & &	1.284	(	0.168	) & &\bf	0.029 &	-2.23 &	77.4 \\
C3 &	1.215	(	0.168	) & &	1.233	(	0.172	) & &	0.610 &	-0.51 &	83.0 \\
T6 &	1.085	(	0.253	) & &	1.206	(	0.531	) & &	0.189 &	-1.33 &	53.0 \\
T5 &	1.211	(	0.154	) & &	1.324	(	0.201	) & &\bf	0.004 &	-2.94 &	71.5 \\
P4 &	1.154	(	0.154	) & &	1.225	(	0.170	) & &\bf	0.044 &	-2.04 &	79.5 \\
P3 &	1.153	(	0.149	) & &	1.229	(	0.189	) & &\bf	0.041 &	-2.08 &	73.3 \\
O2 &	1.185	(	0.109	) & &	1.256	(	0.203	) & &	0.050 &	-2.00 &	56.7 \\
O1 &	1.202	(	0.167	) & &	1.212	(	0.177	) & &	0.792 &	-0.26 &	81.6 \\
FZ &	1.236	(	0.179	) & &	1.246	(	0.168	) & &	0.785 &	-0.27 &	85.8 \\
CZ &	1.194	(	0.192	) & &	1.261	(	0.146	) & &	0.067 &	-1.85 &	87.8 \\
PZ &	1.176	(	0.176	) & &	1.218	(	0.189	) & &	0.286 &	-1.07 &	80.8 \\
LOG &	1.316	(	0.178	) & &	1.523	(	0.212	) & &\bf	0.000 &	-4.92 &	76.2 \\
ROG	&   1.282	(	0.192	) & &	1.473	(	0.225	) & &\bf	0.000 &	-4.26 &	76.9 \\
EMG &   0.792	(	0.382	) & &	0.861	(	0.361	) & &	0.446 &	-0.77 &	64.4 \\
\bottomrule
\end{tabular}
\end{table}

\end{document}