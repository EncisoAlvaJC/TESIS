%%%%%%%%%%%%%%%%%%%%%%%%%%%%%%%%%%%%%%%%%%%%%%%%%%%%%%%%%%%%%%%%%%%%%%%%%%%%%%%%%%%%%%%%%%%%%%%%%%%
%%%%%%%%%%%%%%%%%%%%%%%%%%%%%%%%%%%%%%%%%%%%%%%%%%%%%%%%%%%%%%%%%%%%%%%%%%%%%%%%%%%%%%%%%%%%%%%%%%%

%\documentclass[12pt,a4paper]{article}
\documentclass[12pt,a4paper]{mitthesis}
\usepackage[utf8]{inputenc}
\usepackage[spanish]{babel}
\decimalpoint

% en algun momento queria cambiar el formato de las referencias
\usepackage{cite}

% tienen algunos comandos que ocupo, como align o defn
\usepackage{amsmath}
\usepackage{amsfonts}
\usepackage{amssymb}

% sin este no se pueden incluir imagenes
\usepackage{graphicx}

\usepackage{cmap}
\pagestyle{plain}

% para incluir el reporte en pdf, temporalmente
\usepackage{pdfpages}

% para importar el codigo de R y que se vea bien
\usepackage{listings}
\usepackage{color}

% para hacer cuadros de color
%\usepackage{tcolorbox}

% este paquete pone las fracciones bonitas
\usepackage{nicefrac}

% este paquete
\usepackage{url}

% para el simbolo TM de marca registrada
\usepackage{textcomp}

% para poner imagenes verticales en pagina completa
\usepackage{pdflscape}
\usepackage{afterpage}
\usepackage{everypage}
\usepackage{environ}

% paquete para tablas a hoja completa
\usepackage{tabularx}

% para que las tablas tengan lineas gruesas
%\usepackage{booktabs}

% para poner grandes cantidades de codigo como comentario
\usepackage{verbatim}

% para poner etquetas a un multiplot
\usepackage[caption=false]{subfig}
\usepackage{float}

% para repartir un subfig entre varias paginas
%\usepackage{caption}
%\usepackage{subcaption}

%%%%%%%%%%%%%%%%%%%%%%%%%%%%%%%%%%%%%%%%%%%%%%%%%%%%%%%%%%%%%%%%%%%%%%%%%%%%%%%%%%%%%%%%%%%%%%%%%%%
%%%%%%%%%%%%%%%%%%%%%%%%%%%%%%%%%%%%%%%%%%%%%%%%%%%%%%%%%%%%%%%%%%%%%%%%%%%%%%%%%%%%%%%%%%%%%%%%%%%

\newcounter{abspage}% \thepage not reliab

\makeatletter
\newcommand{\newSFPage}[1]% #1 = \theabspage
  {\global\expandafter\let\csname SFPage@#1\endcsname\null}

\NewEnviron{SidewaysFigure}{
\begin{figure}[p]
\protected@write\@auxout{\let\theabspage=\relax}% delays expansion until shipout
  {\string\newSFPage{\theabspage}}%
\ifdim\textwidth=\textheight
  \rotatebox{90}{\parbox[c][\textwidth][c]{\linewidth}{\BODY}}%
\else
  \rotatebox{90}{\parbox[c][\textwidth][c]{\textheight}{\BODY}}%
\fi
\end{figure}}

%%%%%%%%%%%%%%%%%%%%%%%%%%%%%%%%%%%%%%%%%%%%%%%%%%%%%%%%%%%%%%%%%%%%%%%%%%%%%%%%%%%%%%%%%%%%%%%%%%%
%%%%%%%%%%%%%%%%%%%%%%%%%%%%%%%%%%%%%%%%%%%%%%%%%%%%%%%%%%%%%%%%%%%%%%%%%%%%%%%%%%%%%%%%%%%%%%%%%%%

\newcommand{\abbrlabel}[1]{\makebox[3cm][l]{\textbf{#1}\ \dotfill}}
\newenvironment{abbreviations}{\begin{list}{}{\renewcommand{\makelabel}{\abbrlabel}}}{\end{list}}

\pagestyle{plain}

\newtheorem{defn}{Definici\'on}
\newtheorem{thrm}{Teorema}
\newtheorem{demostracion}{Demostraci\'on}

\newcommand{\R}{\mathbb{R}}
%\newcommand{\el2}{L^2}
\newcommand{\intR}{\int_{-\infty}^{\infty}}
\newcommand{\intZ}{\int_{-\infty}^{0}}
\newcommand{\intPI}{\int_{-\pi}^{\pi}}
\newcommand{\prima}{^{\prime}}

\newcommand{\ddd}{$\delta$}

\newcommand{\aste}[1]{\widehat{ #1 }^{\star}}
\newcommand{\est}[1]{\widehat{ #1 }}

\newcommand{\COS}[1]{\mathrm{cos}\left( #1 \right)}
\newcommand{\SEN}[1]{\mathrm{sen}\left( #1 \right)}

\newcommand{\E}[1]{\mathrm{E}\left[ #1 \right]}
\newcommand{\Var}[1]{\mathrm{Var}\left( #1 \right)}
\newcommand{\Cov}[1]{\mathrm{Cov}\left( #1 \right)}
\newcommand{\abso}[1]{\left| #1 \right|}

%\newcommand{\hline2}{•}

%%%%%%%%%%%%%%%%%%%%%%%%%%%%%%%%%%%%%%%%%%%%%%%%%%%%%%%%%%%%%%%%%%%%%%%%%%%%%%%%%%%%%%%%%%%%%%%%%%%

\definecolor{dkgreen}{rgb}{0,0.6,0}
\definecolor{gray}{rgb}{0.5,0.5,0.5}
\definecolor{mauve}{rgb}{0.58,0,0.82}

\lstset{ %
  language=R,                     % the language of the code
  basicstyle=\footnotesize,       % the size of the fonts that are used for the code
% basicstyle=\tiny,               % the size of the fonts that are used for the code
  numbers=left,                   % where to put the line-numbers
  numberstyle=\tiny\color{gray},  % the style that is used for the line-numbers
  stepnumber=1,                   % the step between two line-numbers. If it's 1, each line
                                  % will be numbered
  numbersep=5pt,                  % how far the line-numbers are from the code
  backgroundcolor=\color{white},  % choose the background color. You must add \usepackage{color}
  showspaces=false,               % show spaces adding particular underscores
  showstringspaces=false,         % underline spaces within strings
  showtabs=false,                 % show tabs within strings adding particular underscores
  frame=single,                   % adds a frame around the code
  rulecolor=\color{black},        % if not set, the frame-color may be changed on line-breaks within not-black text (e.g. commens (green here))
  tabsize=2,                      % sets default tabsize to 2 spaces
  captionpos=b,                   % sets the caption-position to bottom
  breaklines=true,                % sets automatic line breaking
  breakatwhitespace=false,        % sets if automatic breaks should only happen at whitespace
  title=\lstname,                 % show the filename of files included with \lstinputlisting;
                                  % also try caption instead of title
  keywordstyle=\color{blue},      % keyword style
  commentstyle=\color{dkgreen},   % comment style
  stringstyle=\color{mauve},      % string literal style
  escapeinside={\%*}{*)},         % if you want to add a comment within your code
  morekeywords={*,/,.}            % if you want to add more keywords to the set
  deletekeywords={t,_,max,R}      % to remove keywords
} 

%%%%%%%%%%%%%%%%%%%%%%%%%%%%%%%%%%%%%%%%%%%%%%%%%%%%%%%%%%%%%%%%%%%%%%%%%%%%%%%%%%%%%%%%%%%%%%%%%%%
%%%%%%%%%%%%%%%%%%%%%%%%%%%%%%%%%%%%%%%%%%%%%%%%%%%%%%%%%%%%%%%%%%%%%%%%%%%%%%%%%%%%%%%%%%%%%%%%%%%
%%%%%%%%%%%%%%%%%%%%%%%%%%%%%%%%%%%%%%%%%%%%%%%%%%%%%%%%%%%%%%%%%%%%%%%%%%%%%%%%%%%%%%%%%%%%%%%%%%%



%%%%%%%%%%%%%%%%%%%%%%%%%%%%%%%%%%%%%%%%%%%%%%%%%%%%%%%%%%%%%%%%%%%%%%%%%%%%%%%%%%%%%%%%%%%%%%%%%%%
%%%%%%%%%%%%%%%%%%%%%%%%%%%%%%%%%%%%%%%%%%%%%%%%%%%%%%%%%%%%%%%%%%%%%%%%%%%%%%%%%%%%%%%%%%%%%%%%%%%
%%%%%%%%%%%%%%%%%%%%%%%%%%%%%%%%%%%%%%%%%%%%%%%%%%%%%%%%%%%%%%%%%%%%%%%%%%%%%%%%%%%%%%%%%%%%%%%%%%%
%%%%%%%%%%%%%%%%%%%%%%%%%%%%%%%%%%%%%%%%%%%%%%%%%%%%%%%%%%%%%%%%%%%%%%%%%%%%%%%%%%%%%%%%%%%%%%%%%%%

\begin{document}

%%%%%%%%%%%%%%%%%%%%%%%%%%%%%%%%%%%%%%%%%%%%%%%%%%%%%%%%%%%%%%%%%%%%%%%%%%%%%%%%%%%%%%%%%%%%%%%%%%%

\setcounter{page}{0}
\thispagestyle{empty}

\title{Estacionariedad d\'ebil en registros polisomnogr\'aficos de adultos mayores,
como posible marcador de deterioro cognitivo}
%
\author{Julio Cesar Enciso Alva}

\begin{center}
\huge{Estacionariedad d\'ebil en registros polisomnogr\'aficos de adultos mayores durante el
sue\~no MOR, como posible marcador de deterioro cognitivo}


\Large{Julio Cesar Enciso Alva}
\end{center}

\newpage

%%%%%%%%%%%%%%%%%%%%%%%%%%%%%%%%%%%%%%%%%%%%%%%%%%%%%%%%%%%%%%%%%%%%%%%%%%%%%%%%%%%%%%%%%%%%%%%%%%%
%%%%%%%%%%%%%%%%%%%%%%%%%%%%%%%%%%%%%%%%%%%%%%%%%%%%%%%%%%%%%%%%%%%%%%%%%%%%%%%%%%%%%%%%%%%%%%%%%%%
%%%%%%%%%%%%%%%%%%%%%%%%%%%%%%%%%%%%%%%%%%%%%%%%%%%%%%%%%%%%%%%%%%%%%%%%%%%%%%%%%%%%%%%%%%%%%%%%%%%
%%%%%%%%%%%%%%%%%%%%%%%%%%%%%%%%%%%%%%%%%%%%%%%%%%%%%%%%%%%%%%%%%%%%%%%%%%%%%%%%%%%%%%%%%%%%%%%%%%%

\pagenumbering{roman}
\setcounter{page}{1}
%\thispagestyle{empty}

\chapter*{Res\'umen}

{\small

Los avances m\'edicos del \'ultimo siglo se han traducido en un incremento tanto en la esperanza
de vida como en la calidad de la misma; sin embargo, aumenta la presencia de enfermedades 
no-transmisibles asociadas con la edad --entre ellas la demencia.
Recientemente, los trastornos del sue\~no han sido se\~nalados como relacionados con el deterioro 
cognitivo durante la vejez \cite{Amer13,Miyata13,Potvin12}

En la modelaci\'on matem\'atica, usualmente se supone que las se\~nales electrofisiol\'ogicas son 
no-estacionarias, no-lineales y sin-equilibrio por naturaleza, aunque estas propiedades no suelen 
ser confirmadas; se ha sugerido que en casos at\'ipicos esto puede no ser cierto
\cite{McEwen75,Cohen77,Sugimoto78},
lo qu epodr\'ia explicar los resultados reportados en un trabajo anterior [Valeria]. 
En este trabajo se investiga la posibilidad de que los registros de actividad cerebral durante el 
sue\~no (polisomnograma, PSG) en personas con posible deterioro cognitivo (PDC) exhiban una 
estructura 'm\'as simple', en el sentido de tratrse de series d\'ebilmente estacionarias.
Para ello, se utiliza la prueba propuesta por Priestley y Subba Rao \cite{Priestley69}, que consiste
en la estimaci\'on local del espectro de potencias --para procesos estoc\'asticos.

Fueron analizaron los registros de PSG correspondientes a 9 adultos mayores, previamente 
diagnosticados a trav\'es de una bater\'ia de pruebas neuropsicol\'ogicas.
Se prest\'o especial atenci\'on a la etapa de sue\~no MOR, cartacterizada por exhibir aton\'ia
muscular, actividad cerebral de baja amplitud y frecuencias mixtas, y movimientos oculares r\'apidos
(MOR).
Fueron halladas diferencias significativas para el grupo control en las regiones
frontales y posteriores, respecto a
la presencia proporcional de 
estacionariedad d\'ebil 
medida durante sue\~no MOR y no-MOR.
Estos resultados sugieren que durante el DC cambia la din\'amica del PSG al transitar entre 
etapas del sue\~no. 
%Se encontraron diferencis significativas en el grupo Control vs DC en las regiones 
%frontales y posteriores, que son acordes a la literatura.
Adicionalmente, se reporta una serie de patrones visuales respecto a la presencia de 
estacionariedad a trav\'es del tiempo, y que pudieran usarse para se\~nalar el sue\~no
MOR de forma semi-autom\'atica.%, la cual no es detectada eficientemente actualmente.

}

%%%%%%%%%%%%%%%%%%%%%%%%%%%%%%%%%%%%%%%%%%%%%%%%%%%%%%%%%%%%%%%%%%%%%%%%%%%%%%%%%%%%%%%%%%%%%%%%%%%
%%%%%%%%%%%%%%%%%%%%%%%%%%%%%%%%%%%%%%%%%%%%%%%%%%%%%%%%%%%%%%%%%%%%%%%%%%%%%%%%%%%%%%%%%%%%%%%%%%%
%%%%%%%%%%%%%%%%%%%%%%%%%%%%%%%%%%%%%%%%%%%%%%%%%%%%%%%%%%%%%%%%%%%%%%%%%%%%%%%%%%%%%%%%%%%%%%%%%%%
%%%%%%%%%%%%%%%%%%%%%%%%%%%%%%%%%%%%%%%%%%%%%%%%%%%%%%%%%%%%%%%%%%%%%%%%%%%%%%%%%%%%%%%%%%%%%%%%%%%

%\thispagestyle{empty}

\chapter*{Acr\'onimos}

\begin{tabular}{rl}
%\textbf{DCL} & Deterioro Cognitivo Leve
%\\
\textbf{EEG} & Electroencefalograma %/ Electroencefalograf\'ia
\\
\textbf{EMG} & Electromiograma %/ Electromiograf\'ia
\\
\textbf{EOG} & Electrooculograma %/ Electrooculograf\'ia
\\
%\textbf{MCI} & Deterioro Cognitivo Leve (Mild Cognitive Impairment)
%\\
\textbf{MOR} & Movimientos Oculares R\'apidos
\\
\textbf{PSG} & Polisomnograma %/ Polisomnograf\'ia
\\
\textbf{PDC} & Posible Deterioro Cognitivo
\\
\textbf{SDF} & Funci\'on de Densidad Espectral (Spectral Density Function)
\end{tabular}

\newpage

%%%%%%%%%%%%%%%%%%%%%%%%%%%%%%%%%%%%%%%%%%%%%%%%%%%%%%%%%%%%%%%%%%%%%%%%%%%%%%%%%%%%%%%%%%%%%%%%%%%
%%%%%%%%%%%%%%%%%%%%%%%%%%%%%%%%%%%%%%%%%%%%%%%%%%%%%%%%%%%%%%%%%%%%%%%%%%%%%%%%%%%%%%%%%%%%%%%%%%%
%%%%%%%%%%%%%%%%%%%%%%%%%%%%%%%%%%%%%%%%%%%%%%%%%%%%%%%%%%%%%%%%%%%%%%%%%%%%%%%%%%%%%%%%%%%%%%%%%%%
%%%%%%%%%%%%%%%%%%%%%%%%%%%%%%%%%%%%%%%%%%%%%%%%%%%%%%%%%%%%%%%%%%%%%%%%%%%%%%%%%%%%%%%%%%%%%%%%%%%

\thispagestyle{empty}

\tableofcontents
\newpage
%\listoffigures
%\newpage
%\listoftables

%%%%%%%%%%%%%%%%%%%%%%%%%%%%%%%%%%%%%%%%%%%%%%%%%%%%%%%%%%%%%%%%%%%%%%%%%%%%%%%%%%%%%%%%%%%%%%%%%%%
%%%%%%%%%%%%%%%%%%%%%%%%%%%%%%%%%%%%%%%%%%%%%%%%%%%%%%%%%%%%%%%%%%%%%%%%%%%%%%%%%%%%%%%%%%%%%%%%%%%
%%%%%%%%%%%%%%%%%%%%%%%%%%%%%%%%%%%%%%%%%%%%%%%%%%%%%%%%%%%%%%%%%%%%%%%%%%%%%%%%%%%%%%%%%%%%%%%%%%%
%%%%%%%%%%%%%%%%%%%%%%%%%%%%%%%%%%%%%%%%%%%%%%%%%%%%%%%%%%%%%%%%%%%%%%%%%%%%%%%%%%%%%%%%%%%%%%%%%%%

\setcounter{page}{1}
\pagenumbering{arabic}

\chapter{Antecedentes}

En este trabajo se retoma la l\'inea de investigaci\'on trazada por \cite{VazquezTagle16} [y
Valeria]. 
En el primero se estudi\'o la epidemiolog\'ia del deterioro cognitivo dentro del estado de
Hidalgo. Dado que los trastornos del sue\~no han sido se\~nalados como posiblemente relacionados
con el deterioro cognitivo en adultos mayores \cite{Amer13,Miyata13,Potvin12}, en aqu\'el estudio
se efectuaron registros polisomnogr\'aficos (PSG) de los pacientes y se report\'o una relaci\'on
entre una menor eficiancia del sue\~no y la presencia de deterioro cognitivo.

En el segundo trabajo se analizaron posibles cambios en la estructura funcional\footnote{Se suele 
hablar de \textbf{conectividad funcional} cuando las se\~nales registradas en dos elementos/lugares 
est\'an estad\'isticamente ''muy'' correlacionadas; este t\'ermino se contrapone al de
\textbf{conectividad anat\'omica}, que se refiere a conexiones f\'isicas entre los mismos} del 
cerebro para adultos mayores con PDC, con respecto a individuos sanos, durante el sue\~no.
Las diferencias reportadas se refieren al exponente de Hurst ($H_\alpha$) calculado para los 
registros de PSG.
La cantidad $H_\alpha$, a veces referida como el ''color'' de la se\~nal, est\'a asociada con la
''estructura fractal'' de un proceso estoc\'astico. De manera concreta, en el trabajo referido
se reporta que para registros de PSG correspondientes adultos mayores con 
PDC la cantidad $H_\alpha$ es menor. Dado que un menor exponente de Hurst est\'a 
asociado con se\~nales cuya estructura es ''m\'as simple'', cabe preguntarse si 
los registros de PSG en adultos mayores con PDC son efectivamente diferentes --en cuanto a 
complejidad-- a sus
contrapartes para individuos sanos. 

En este trabajo se pretende comprobar la hip\'otesis anterior revisando si los registros de PSG en 
adultos mayores con PDC poseen una caracter\'istica particular que tipifica se\~nales 
''sencillas'': siendo que los registros de PSG son entendidos como generados por procesos 
estoc\'asticos, se investiga si estos pueden considerarse como estacionarios --cuando menos en el 
sentido d\'ebil.
Este supuesto es b\'asico en el estudio de series de tiempo, y usualmente se acepta o rechaza 
sin un tratamiento formal.
La idea de que los sujetos con PDC presentan, en mayor medida, estacionariedad d\'ebil en sus 
registros de EEG fue sugerida por Cohen \cite{Cohen77}, quien a su vez se refiere a trabajos 
anteriores sobre regularidad estad\'stica --estacionariedad y normalidad-- en registros de EEG \cite{McEwen75,Sugimoto78,Kawabata73}. 
Si bien en estos primeros estudios se palpa la posibilidad de que los registros de EEG fueran
ruido de alg\'un tipo, esta idea se ha probado \'erronea en estudios m\'as recientes 
\cite{Klonowski09}; esta posibilidad se retoma a la luz de los resultados reportados por
[Valeria] pues, como se mencion\'o, es en principio posible que los registros de EEG durante el 
PDC tengan una estructura m\'as simple --m\'as aleatoria.

%%%%%%%%%%%%%%%%%%%%%%%%%%%%%%%%%%%%%%%%%%%%%%%%%%%%%%%%%%%%%%%%%%%%%%%%%%%%%%%%%%%%%%%%%%%%%%%%%%%
%%%%%%%%%%%%%%%%%%%%%%%%%%%%%%%%%%%%%%%%%%%%%%%%%%%%%%%%%%%%%%%%%%%%%%%%%%%%%%%%%%%%%%%%%%%%%%%%%%%

\begin{comment}

Hist\'oricamente,
el registro sistem\'atico de la actividad el\'ectrica del cerebro se considera iniciado 
a principios del siglo XX
por Hans Berger \cite{Berger29}, 
quien acu\~n\'o el t\'ermino electroencefalograma (EEG).
Por otro lado, la idea que una se\~nal estoc\'astica puede no ser estacionaria se puede 
rastrear a los a\~nos 50’s \cite{Page52,Silverman57}. 
Sin embargo, las interrogantes sobre la regularidad de se\~nales electrofisiol\'ogicas no 
aparecen plasmadas claramente sino hasta los a\~nos 
70's \cite{Kawabata73,McEwen75,Cohen77,Sugimoto78}.
Esta brecha temporal se debe, quiz\'a, a la aparici\'on de computadoras digitales de bajo costo, 
gracias a las cuales es posible analizar mayores vol\'umenes de datos.

Se puede se\~nalar que estos primeros estudios aparentan estar disconexos del sustento te\'orico 
asociado a las t\'ecnicas  utilizadas en cuanto a la interpretaci\'on de los datos obtenidos. 
Conviene citar, por ejemplo, que la escasa capacidad de c\'omputo promovi\'o, indirectamente, la 
hip\'otesis de que {toda} serie de tiempo suficientemente corta pod\'ia entenderse como generada 
por un proceso estoc\'astico d\'ebilmente estacionario.
Actualmente, la estacionariedad d\'ebil ''garantizada'' para series cortas se considera rebatida\cite{Melard89,Adak98,Klonowski09}.

\end{comment}

%%%%%%%%%%%%%%%%%%%%%%%%%%%%%%%%%%%%%%%%%%%%%%%%%%%%%%%%%%%%%%%%%%%%%%%%%%%%%%%%%%%%%%%%%%%%%%%%%%%
%%%%%%%%%%%%%%%%%%%%%%%%%%%%%%%%%%%%%%%%%%%%%%%%%%%%%%%%%%%%%%%%%%%%%%%%%%%%%%%%%%%%%%%%%%%%%%%%%%%

\section{Justificaci\'on}

Los avances m\'edicos del \'ultimo siglo se han traducido en un incremento tanto en la esperanza de 
vida como en la calidad de la misma. 
De acuerdo a la Encuesta Nacional de Salud y Nutrici\'on (ENSANUT) efectuada en M\'exico 2002, se 
estima que existen 800,000 adultos mayores en el pa\'is \cite{Sosa12}. 
Lamentablemente, tambi\'en se ve incrementada la presencia de enfermedades no-transmisibles --entre 
ellas la demencia. 
El cuidado de enfermedades cr\'onicas en la poblaci\'on de edad avanzada representa un gran peso 
econ\'onomico y de recursos humanos, que recae sobre el sistema de salud y los familiares de los 
afectados; por ello, cobra importancia un diagn\'ostico temprano del deterioro cognitivo que 
disminuya el risgo de su avance irreversible a demencia.

Todav\'ia son incipientes las investigaciones para identificar los factores de riesgo 
modificables asociados a la demencia \cite{PlanAlzheimer04}; recientemente, los trastornos del 
sue\~no han sido se\~nalados como posiblemente relacionados con el deterioro cognitivo durante la 
vejez \cite{Amer13,Miyata13,Potvin12}. Concretamente, una duraci\'on menor del sue\~no nocturno y 
una mala eficiencia del mismo, en personas mayores, se relaciona con una peor ejecuci\'on en tareas 
de memoria \cite{Reid06}. Las afectaciones relativas al sue\~no en personas mayores podr\'ian ser 
m\'as problem\'aticas que para otros grupos de edad, dado que el individuo cursa con deterioro 
cognitivo y empeora con la edad \cite{Potvin12}.

%%%%%%%%%%%%%%%%%%%%%%%%%%%%%%%%%%%%%%%%%%%%%%%%%%%%%%%%%%%%%%%%%%%%%%%%%%%%%%%%%%%%%%%%%%%%%%%%%%%

\section{Pregunta de investigaci\'on}

¿Es posible que la caracterizaci\'on de registros de PSG como series de tiempo d\'ebilmente 
estacionarias, pueda ser usada como un marcador en el diagn\'ostico cl\'inico de PDC en adultos 
mayores?

%¿Es plausible que la clasificaci\'on, para fragmentos de registros de PSG, como ''series de tiempo 
%d\'ebilmente estacionarias'' pueda ser usada como un marcador usable en el diagn\'ostico cl\'inico
%de posible deterioro cognitivo en adultos mayores?

%%%%%%%%%%%%%%%%%%%%%%%%%%%%%%%%%%%%%%%%%%%%%%%%%%%%%%%%%%%%%%%%%%%%%%%%%%%%%%%%%%%%%%%%%%%%%%%%%%%

\subsection{Hip\'otesis}

Existen diferencias en la conectividad funcional del cerebro en adultos mayores con PDC --respecto
a sujetos sanos-- y es posible detectar estas diferencias como una mayor o menor ''presencia'' de 
estacionariedad d\'ebil en registros de PSG durante el sue\~no profundo.

%%%%%%%%%%%%%%%%%%%%%%%%%%%%%%%%%%%%%%%%%%%%%%%%%%%%%%%%%%%%%%%%%%%%%%%%%%%%%%%%%%%%%%%%%%%%%%%%%%%

\subsection{Objetivo general}

Deducir, a partir de pruebas estad\'isticas formales, las presencia de estacionariedad d\'ebil en
registros de PSG para adultos mayores con PDC, as\'i como individuos control.

%%%%%%%%%%%%%%%%%%%%%%%%%%%%%%%%%%%%%%%%%%%%%%%%%%%%%%%%%%%%%%%%%%%%%%%%%%%%%%%%%%%%%%%%%%%%%%%%%%%

\subsection{Objetivos espec\'ificos}

\begin{itemize}
\item Estudiar la definici\'on de estacionariedad para procesos estoc\'asticos y sus posibles 
consecuencias dentro de un modelo para los datos considerados

\item Investigar en la literatura c\'omo detectar si es plausible que una serie de tiempo dada sea 
una realizaci\'on para un proceso estoc\'astico d\'ebilmente estacionario, y bajo qu\'e supuestos 
es v\'alida esta caracterizaci\'on

\item Usando los an\'alisis hallados en la literatura, determinar si las series de tiempo 
obtenidas a partir de los datos considerados provienen de procesos debilmente estacionarios.
Revisar si la informaci\'on obtenida en los diferentes sujetos muestra diferencias entre sujetos 
con y sin PDC
\end{itemize}

%%%%%%%%%%%%%%%%%%%%%%%%%%%%%%%%%%%%%%%%%%%%%%%%%%%%%%%%%%%%%%%%%%%%%%%%%%%%%%%%%%%%%%%%%%%%%%%%%%%
%%%%%%%%%%%%%%%%%%%%%%%%%%%%%%%%%%%%%%%%%%%%%%%%%%%%%%%%%%%%%%%%%%%%%%%%%%%%%%%%%%%%%%%%%%%%%%%%%%%
%%%%%%%%%%%%%%%%%%%%%%%%%%%%%%%%%%%%%%%%%%%%%%%%%%%%%%%%%%%%%%%%%%%%%%%%%%%%%%%%%%%%%%%%%%%%%%%%%%%
%%%%%%%%%%%%%%%%%%%%%%%%%%%%%%%%%%%%%%%%%%%%%%%%%%%%%%%%%%%%%%%%%%%%%%%%%%%%%%%%%%%%%%%%%%%%%%%%%%%

\section{Conceptos, fisiolog\'ia}

A continuaci\'on se exponen definiciones relativas a la componente fisiol\'ogica del problema
estudiado, y que son parte integral del mismo. 
%Se pretende que la exposici\'on sea accesible a\'un 
%sin una preparaci\'on especializada en medicina, especialmente considerando que el autor pertenece 
%a tal grupo.

\subsection{Adulto mayor}

Primeramente se presenta una definici\'on formal de qu\'e se entiende por ''adulto mayor'' en el 
contexto de la psicolog\'ia --y que es usado durante este trabajo.

\begin{description}
\item[Adulto Mayor.] Individuo de 60 a\~nos o m\'as que habite un pa\'is en v\'ias de desarrollo, o 
65 a\~nos en pa\'ises desarrollados \cite{Hita14}.
\end{description}

El envejecimiento considerado normal es determinado por una serie de procesos moleculares, 
celulares, fisiol\'ogicos y psicol\'ogicos que conducen directamente al deterioro de funciones 
cognitivas, específicamente en la atenci\'on y memoria \cite{Navarrete03,Park09}. 
%En esta etapa el organismo sufre cambios fisiolo\'ogicos y psicol\'ogicos que dificultan la capacidad de
%adaptaci\'on al ambiente, teniendo como consecuencia una mayor suceptibilidad a padecer enfermedades
%y morir en consecuencia \cite{Hita14}.
En un principio se consideraba que el envejecimiento cerebral ocurr\'ia fundamentalmente por una 
muerte neuronal programada \cite{Coleman87}, sin embargo, estudios realizados con tejido cerebral 
post mortem de adultos mayores que en vida fueron sanos, mostraron que dicha muerte neuronal no 
alcanza un 10\% en su totalidad \cite{Esiri07}. 

Con el paso del tiempo, la organizaci\'on an\'atomo-funcional del cerebro sufre modificaciones que 
traen como consecuencia la afectaci\'on de diferentes capacidades cognitivas; sin embargo, la 
vulnerabilidad de los circuitos neuronales ante estos cambios no suceden de forma homog\'enea en 
todo el cerebro \cite{Hita14}.
La funcionalidad durante la vejez se relaciona con el estilo de vida, los factores de riesgo, el 
acceso a la educaci\'on y las acciones de promoci\'on a la salud realizadas en edades m\'as 
tempranas \cite{Ohayon04,Sanhueza14}.
En la escala cl\'inica del deterioro cognitivo, en este trabajo se han analizado sujetos que lo
padecen en un grado leve; m\'as a\'un, en el transcurso de este escrito ser\'a referido como 
Posible Deterioro Cognitivo, am\'en de los esfuerzos vertidos para el mejoramiento de los 
individuos afectados.

\begin{description}
\item[Deterioro cognitivo leve.] S\'indrome caracterizado por una alteraci\'on adquirida y 
prolongada de una o varias funciones cognitivas, que no corresponde a un s\'indrome focal y no 
cumple criterios suficientes de gravedad para ser calificada como demencia \cite{Robles02}.
\end{description}

%En este sentido, los cambios morfol\'gicos que sufren las neuronas durante el envejecimiento son 
%abundantes, observ\'andose una importante disminuci\'on de la arborizaci\'n dendr\'itica as\'i como 
%en la densidad y volumen \cite{Hita14}. 

\subsection{Electroencefalograma}

Si bien es perfectamente posible definir el sue\~no sin necesidad de hablar de 
electroencefalogramas, conviene hablar primero de \'este debido a la forma en que son tipificadas
cl\'inicamente las diferentes etapas del sue\~no.

\begin{description}
\item[Electroencefalograma (EEG).] Registro de las fluctuaciones en potenciales de acci\'on en el cerebro.
\end{description}

De manera convencional, la actividad el\'ectrica del cerebro se registra en tres locaciones: en la 
corteza cerebral expuesta (electrocorticograma, ECoG), a trav\'es de agujas incrustadas en el 
tejido nervioso (registro profundo), o el cuero cabelludo (EEG).
En cualquiera de tales sitios, el registro representa una superposici\'on de potenciales de campo 
producidos por una amplia variedad de generadores de corriente dentro de un medio conductor 
volum\'etrico: los elementos neuronales generan, cada cual, corrientes que son conducidas y 
disipadas a trav\'es del espacio en el cerebro.
%A su vez, estos generadores de campos el\'ectricos corresponden a agregados de elementos neuronales 
%con interconexiones complejas: dendritas, somas y axones.
A ello hay que adicionar que la arquitectura cerebral es altamente no homog\'enea.

Debido a que los axones en la corteza cerebral tienen orientaciones muy diversas con respecto a la 
superficie, y a que disparan de manera as\'incrona, el aporte neto de estos campos al potencial 
registrado es negligible bajo condiciones normales.
Una excepci\'on muy importante ocurre en caso de un est\'imulo simult\'aneo (s\'incronizado) del 
del n\'ucleo tal\'amico o de las aferentes nerviosas.
Estas respuestas sincronizadas suelen tener una amplitud relativamente alta, y son referidas como 
'potenciales evocados'.

El registro de los electrodos (los canales) son referidos como un \textbf{montaje}: en un montaje 
bipolar, cada canal mide la diferencia entre dos electrodos adyacentes, mientras que en un montaje 
referencial cada canal mide la diferencia respecto a un electrodo de referencia, usualmente la 
oreja.
Aunque los mismos eventos el\'ectricos se registran en todos los montajes, aparecen en un diferente 
formato seg\'un el caso. 
Los potenciales son amplificados anal\'ogicamente y posteriormente registrados.
El sistema m\'as usado para la colocaci\'on de los electrodos con fines cl\'inicos es el 
\textit{International Federation 10--20 system}, que fue propuesto por la International Federation 
of EEG Societies \cite{Jasper58,AASM07} y es mostrado en la figura \ref{img1020}. 
%Este sistema usa referentes anat\'omicos estandarizados para la ubicaci\'on de los electrodos.

\begin{figure}
\centering
%\includegraphics[width=0.8\linewidth]{figura_6.png} 
\includegraphics[width=0.9\linewidth]{Fig.png} 
\caption{El sistema 10--20, recomendado por la
International Federation of EEG Societies. [Este gr\'afico se volver\'a a dibujar]
%\cite{Jasper58,AASM07}
}
\label{img1020}
\end{figure}

Usualmente el EEG muestra una actividad el\'ectrica oscilatoria continua y cambiante. 
%Tanto la intensidad como los patrones de esta actividad est\'an determinados por los eventos de 
%excitaci\'on conjunta del cerebro, resultante de las funciones en el sistema reticular de 
%activaci\'on del tallo cerebral \cite{Clark98}.
Estas 'ondas' observadas en los registros de potenciales el\'ectricos en el cerebro son referidas 
como \textbf{ondas cerebrales}; la 'frecuencia' de estas ondas var\'ia entre 0.5 y 100 Hz, y se ha 
identificado que su composici\'on est\'a fuertemente relacionada con el grado de actividad cerebral: 
hay diferencias claras entre registros durante vigilia y sue\~no.
En general la frecuencia promedio del EEG incrementa progresivamente cuando hay un altos grados 
de actividad cerebral: las ondas se vuelven m\'as as\'incronas, de modo que la magnitud del 
potencial integrado de superficie decrece a pesar de la alta actividad cortical.
%Por ejemplo, las ondas delta se encuentran frecuentemente durante el estupor, anestesia 
%quir\'urgica, y sue\~no; las ondas theta son comunes en infantes; las ondas alfa ocurren en 
%estado de relajaci\'on; las ondas beta aparecen durante actividad mental intensa.
Aunque la mayor parte del tiempo el EEG es irregular y no muestra patrones claros, es com\'un que 
muestre ondas cerebrales relativamente organizadas que, para su estudio, han sido clasificadas en 
cuatro grandes grupos: alfa, beta, gamma, delta.
Estos grupos son ilustrados en la figura \ref{ritmos}.

\begin{figure}
\centering
\includegraphics[width=0.5\linewidth]{figura_4.png} 
\caption{(a) Diferentes tipos de ondas normales en el EEG. (b) Supresi\'on del ritmo alfa debido a 
una descarga desincronizada cuando el paciente abre los ojos.
%(From A. C. Guyton, Structure and Function of the Nervous System, 2nd ed.,
%Philadelphia: W.B. Saunders, 1972; used with permission.)
[Estos gr\'aficos ser\'an reconstruidos]
}
\label{ritmos}
\end{figure}

\begin{description}
\item[Ondas alfa.] Frecuencias entre 8 y 13 Hz. Ocurren en sujetos despiertos en un estado de 
quietud del pensamiento. Aparecen m\'as frecuentemente en la regi\'on occipital, pero tambi\'en 
pueden ser registradas en las regiones frontal y parietal. Su voltaje aproximado est\'a entre 20 y 
200 mV. Cuando el sujeto duerme, las ondas alfa desaparecen completamente. Si el sujeto est\'a
despierto y su atenci\'on se dirige a una actividad mental espec\'ifica, las ondas alfa
son reemplazadas por ondas dessincronizadas de mayor frecuencia y menor voltaje.

\item[Ondas beta.] Frecuencias de 14 a 30 Hz. Normalmente se registran en las regiones parietal y 
frontal. A veces se les divide en dos tipos: beta I y beta II. Las ondas beta I (14--20 Hz)
son afectadas por la actividad mental de manera similar a las ondas alfa.
Las ondas beta II (20--30 Hz), en cambio, aparecen durante una activaci\'on intensa del sistema 
nervioso central y durante tensi\'on.

\item[Ondas theta.] Frecuencias entre 4 y 7 Hz. Ocurren principalmente en las regiones parietal y 
temporal en ni\~nos, pero pueden aparecer en algunos adultos durante estr\'es emocional, sobre 
todo durante periodos de decepsi\'on y frustraci\'on.

\item[Ondas delta.] Incluye todas las ondas del EEG 'abajo de' 3.5 Hz. Ocurren generalmente en el 
sue\~no profundo en infantes, y despu\'es de enfermedades org\'anicas serias del cerebro.
\end{description}
%Tambi\'en pueden ser registradas en cerebros de animales a los cuales se ha hecho transsecci\'on 
%subcortical, produciendo una separaci\'on funcional entre la corteza cerebral y el sistema 
%reticular de activaci\'on del tallo cerebral. 


Cabe mencionar que el espectro de frecuencias del potencial de campo producido por m\'usculos 
faciales medianamente contra\'idos, incluye componentes de frecuencia que bien cuadran en el rango 
usual del EEG (0.5--100 Hz).
%Una vez se ha conseguido el estado de reposo en un adulto normal, sus registros del cuero cabelludo
%muestran un ritmo alfa (ver m\'as adelante) dominante en el \'area parietal-occipital, mientras que 
%en \'area frontal ha un ritmo beta con baja amplitud y alta frecuencia --adem\'as del ritmo alfa.
%En un sujeto normal hay cierta simetr\'ia entre los registros de los hemisferios derecho e 
%izquierdo. 
La variedad de artefactos conocidos es muy basta, y constituye un tema muy complejo.

\subsection{Sue\~no}

El sue\~no normal se divide en dos etapas principales: MOR (fase R) y NMOR (fase N), que se 
diferenc\'ian por sus rasgos electroencefalogr\'aficos y una serie de caracter\'isticas 
fisiol\'ogicas (de los cuales obtienen sus nombres).
Cabe mencionar que la nomenclatura acerca de las fases del sue\~no ha sido recientemente modificada 
por la American Association of Sleep Medicine en 2007 \cite{AASM07}, de modo que en este trabajo 
se usar\'an ambas nomenclaturas siempre que sea posible, por fines de compatibilidad.

%Durante el estado de alerta, mientras se mantienen los ojos cerrados, en el EEG se observan 
%oscilaciones de la actividad eléctrica que suelen encontrarse entre 8-13 ciclos por segundo (Hz), 
%principalmente a nivel de las regiones occipitales (ritmo alfa). Durante el sueño ocurren cambios 
%característicos de la actividad eléctrica cerebral que son la base para dividir el sueño en varias 
%fases. Como ya se mencionó, el sueño suele dividirse en dos grandes fases que, de forma normal, 
%ocurren siempre en la misma sucesión: todo episodio de sueño comienza con el llamado sueño sin 
%movimientos oculares rápidos (No MOR), que tiene varias fases, y después pasa al sueño con 
%movimientos oculares rápidos (MOR). 

----------------------------------

\begin{description}
\item[Sue\~no] Proceso vital c\'iclico complejo y activo, compuesto por varias fases y que posee 
una estructura interna caracter\'istica, con diversas interrelaciones en los sistemas hormonales y 
nerviosos \cite{FernandezConde07}.
El sue\~no en el ser humano se puede caracterizar por las siguientes propiedades\cite{CarrilloMora}:
\begin{enumerate}
\item Disminuci\'on de conciencia y reactividad a est\'imulos externos
\item F\'acilmente reversible\footnote{Lo cual lo diferencia de otros estados 
patol\'ogicos como el estupor y el coma}
\item Inmovilidad y relajaci\'on muscular
\item Periodicidad t\'ipica circadiana (diaria)
\item Los individuos adquieren una postura estereotipada
\item La privaci\'on induce alteraciones conductuales y 
fisiol\'ogicas, adem\'as de que genera una ''deuda'' acumulativa
\end{enumerate}
\end{description}





En el sue\~no profundo se observan ondas delta muy irregulares. Junto con ellas ocurren trenes 
cortos de ondas, parecidas a las alfa, y que son referidas como \textit{husos de sue\~no} (sleep 
spindles). El ritmo alfa y los husos de sue\~no est\'an sincronizados en el sue\~no y la somnolencia 
--en contraste con la actividad irregular, desincronizada y de bajo voltaje registrada en estado de 
alerta.
A veces, las ondas lentas de amplitud alta son reemplazadas durante el sue\~no por ondas r\'apidas 
de bajo voltaje, irregulares, y que recuerdan la actividad en el EEG durante el estado de alerta.
La presencia de estos patrones irregulares no interrumpen el sue\~no, sino que incrementan el 
umbral para que los est\'imulos externos despierten al paciente; 
este comportamiento es referido como ''sue\~no parad\'ojico''.
Durante este sue\~no parad\'ojico, el sujeto exhibe movimientos oculares r\'apidos, raz\'on por la 
cual esta etpa recibe el nombre de ''sue\~no de movimientos oculares r\'apidos'' (MOR).
La etapa fuera del sue\~no
es referida como sue\~no no-MOR (NMOR) o sue\~no de ondas lentas.
Los sujetos humanos que despiertan durante la fase de sue\~no MOR suelen reportar que
ten\'ian enso\~naciones, a diferencia de aquellos que despiertan durante la fase NREM.

%\begin{figure}
%\centering
%\includegraphics[width=0.5\linewidth]{figura_7.png} 
%\caption{Los cambios en el EEG que ocurren durante el sue\~no en un sujeto.
%Las marcas de calibraci\'on corresponden a 50 mV.
%%H. Jasper, ‘‘Electrocephalography.’’ In Epilepsy and Cerebral Localization, W.
%%G. Penfield and T. C. Erickson (eds.). Springfield, IL: Charles C. Thomas,
%%1941.)
%[Estos gr\'aficos ser\'an redibujados]
%}
%\label{ritmosEEG}
%\end{figure}



%El sue\~no MOR se caracteriza por la presencia de ondas de bajo voltaje y alta frecuencia en el 
%EEG, aton\'ia muscular y movimientos oculares r\'apidos, adem\'as es donde se presentan 
%la mayor\'ia de los sue\~nos. 
%El sue\~no no-MOR se compone de cuatro fases, 1 y 2, que son de sue\~no ligero, y 3 y 4 de 
%sue\~no profundo, las mismas que transcurren de manera secuencial desde la primera hasta la 
%cuarta fase, que es la fase reparadora del sue\~no, aquella que produce en la persona la 
%sensaci\'on de haber descansado cuando se levanta 13,22,43.

%Las características de las fases del sueño no-MOR incluyen cuatro etapas, la primera que 
%corresponde a la transición de la vigilia al sueño; la etapa 2 es la intermedia (mayor porcentaje 
%del tiempo de sueño) y en el EEG aparecen husos de sueño y los complejos K. La etapa 3 es la del 
%sueño relativamente profundo, representado en el electroencefalograma por ondas lentas de gran 
%amplitud, y la etapa 4o de sueño profundo con más del 50\% de ondas lentas de gran amplitud13.

%%%%%%%%%%%%%%%%%%%%%%%%%%%%%%%%%%%%%%%%%%%%%%%%%%%%%%%%%%%%%%%%%%%%%%%%%%%%%%%%%%%%%%%%%%%%%%%%%%%

\begin{description}
\item[Sue\~no NMOR (fase N)]

\begin{description}
\item[Fase 1 (N1)] Corresponde con la somnolencia o el inicio del sue\~no ligero, en ella es muy 
f\'acil despertarse. La actividad muscular disminuye paulatinamente y pueden observarse algunas 
breves sacudidas musculares s\'ubitas que a veces coinciden con una sensación de ca\'ida 
(mioclon\'ias h\'ipnicas). En el EEG se observa actividad de frecuencias mezcladas, pero de bajo 
voltaje y algunas ondas agudas. 

\item[Fase 2 (N2)] Se caracteriza por que aparecen patrones espec\'ificos de actividad 
cerebral (husos de sue\~no y complejos K). La temperatura, la frecuencia card\'iaca y respiratoria 
comienzan a disminuir paulatinamente. 

\item[Fases 3 y 4 (N3)] La fase m\'as profunda del sue\~no NMOR. Se observan ondas con frecuencias 
muy bajas ($<2$ Hz), por lo que es referido como 'sue\~no de ondas lentas'.
\end{description}
\item[Sueño MOR (fase R)] Se caracteriza por la presencia de movimientos oculares r\'apidos. F\'isicamente el tono de todos 
los m\'usculos disminuye (con excepción de los m\'usculos respiratorios y los esf\'interes vesical 
y anal), as\'i mismo la frecuencia cardiaca y respiratoria se vuelve irregular.%,
%e incluso puede 
%incrementarse y 
%existe erección espontánea del pene o del clítoris. 
Durante el sue\~no MOR se producen la mayor\'ia de las enso\~naciones (lo que conocemos 
coloquialmente como sue\~nos), y la mayor\'ia de los pacientes que despiertan durante esta fase 
suelen recordar v\'ividamente el contenido de sus enso\~naciones \cite{Chokroverty09}.
\end{description}

Un adulto j\'oven pasa aproximadamente entre 70--100 minutos en el sue\~no NMOR para despu\'es 
entrar al sue\~no MOR, el cual puede durar entre 5--30 min; este ciclo se repite cada hora y media.
%A lo largo de la noche pueden presentarse normalmente entre 4 y 6 ciclos de sue\~no MOR.
En los ancianos se va fragmentando el sue\~no nocturno con frecuentes episodios de despertar, se 
reduce mucho el porcentaje de sue\~no en fase 4, pero se mantiene constante el porcentaje de 
sue\~no MOR. Adicionalmente, muchos adultos mayores dormitan durante el d\'ia varias siestas 
cortas \cite{CarrilloMora}.

%
%Los adultos mayores informan que duermen menos durante la noche, y se acuestan y se despiertan 
%m\'as temprano de lo habitual. Adem\'as, tardan m\'as tiempo en conciliar el sue\~no, se 
%despiertan con m\'as frecuencia durante la noche y la duraci\'on de estos despertares es 
%m\'as prolongada 58,59.
%

%%%%%%%%%%%%%%%%%%%%%%%%%%%%%%%%%%%%%%%%%%%%%%%%%%%%%%%%%%%%%%%%%%%%%%%%%%%%%%%%%%%%%%%%%%%%%%%%%%%
%%%%%%%%%%%%%%%%%%%%%%%%%%%%%%%%%%%%%%%%%%%%%%%%%%%%%%%%%%%%%%%%%%%%%%%%%%%%%%%%%%%%%%%%%%%%%%%%%%%
%%%%%%%%%%%%%%%%%%%%%%%%%%%%%%%%%%%%%%%%%%%%%%%%%%%%%%%%%%%%%%%%%%%%%%%%%%%%%%%%%%%%%%%%%%%%%%%%%%%
%%%%%%%%%%%%%%%%%%%%%%%%%%%%%%%%%%%%%%%%%%%%%%%%%%%%%%%%%%%%%%%%%%%%%%%%%%%%%%%%%%%%%%%%%%%%%%%%%%%

%%%%%%%%%%%%%%%%%%%%%%%%%%%%%%%%%%%%%%%%%%%%%%%%%%%%%%%%%%%%%%%%%%%%%%%%%%%%%%%%%%%%%%%%%%%%%%%%%%%
%%%%%%%%%%%%%%%%%%%%%%%%%%%%%%%%%%%%%%%%%%%%%%%%%%%%%%%%%%%%%%%%%%%%%%%%%%%%%%%%%%%%%%%%%%%%%%%%%%%

\section{Conceptos (matem\'aticas)}

En esta secci\'on se describen los conceptos b\'asicos de la teor\'ia espectral 'cl\'asica' para 
procesos estoc\'aticos no-estacionarios. 
De forma m\'as bien pragm\'atica, la descripci\'on est\'a
 fuertemente inspirada por el libro 'Spectral Analysis and Time Series' 
de M. Priestley \cite{Priestley81}, ua que este est\'a expl\'icitamente dirigida a un p\'ublico 
sin un trasfondo matem\'atico.

%Una duda central en el trabajo es el por qu\'e de usar t\'ecnicas desarrolladas hace m\'as de 
%30 a\~nos, y la respuesta tiene tres puntos principales (que se exponen en otra secci\'on con
%m\'as detallles): 
%\begin{itemize}
%\item Hist\'oricamente existe un desface en los conceptos te\'oricos que maneja la 
%neurobiolog\'ia cuantitativa de EEG, lo que motiva a usar materiales cl\'asicos revisados
%como la simplificaci\'on m\'as natural
%
%\item La teor\'ia m\'as reciente consta en gran medida de generalzaciones de la teor\'ia espectral
%cl\'asica: la representaci\'on espectral de Wold-Cram\'er no es \''unica, y pueden extenderse
%sus resultados para las representaciones de periodograma (cambiante en el tiempo), las 
%representeaciones de Wigner-Ville o de Choi-Williams. En otro sentido, es posible
%hacer pruebas de hip\'otesis m\'ultiples usando representaciones de ondeletas. En un tercer 
%apartado, cabe mencionar el enfoque de estacionariedad local para ubicar y relacionar
%componentes de frecuencia
%espec\'ificos.
%
%\item Como el enfoque prescinde de un an\'alisis m\'as riguroso de la composici\'on espectral de
%las se\~nales, el test PSR tiene la propieda de ser relativamente r\'apido, con un orden de
%$N \log{N}$
%\end{itemize}

%Debo citar los trabajos de Cohen, Nason, Adak, Dahlhaus, Gabor, Fryzelwicz, entre otros.
%En discusiones m\'as modernas, se mencionan temas que aun no se han explorado:
%ciclo-estacionariedad, procesos harmonizables, estacionariedad local y por partes,
%diferencias entre memoria larga y memoria corta, espectros de ondaletas, espectros de
%Wigner-Ville, Wold-Cram\'er, Gabor. 
%Debo mencionarlos, pero no he trabajado en ello y no se suficiente sobre ello.

%La informalidad de la redacci\'on se debe al tiempo: en versiones futuras deber\'ia mejorar.

%Nota: no es prioritario, pero ser\'a una buena idea incluir una discusi\'on sobre por qu\'e
%tiene sentido revisar si los EEG son estacionarios, y es que un proceso estacionario es 
%b\'asicamente un ruido.

%%%%%%%%%%%%%%%%%%%%%%%%%%%%%%%%%%%%%%%%%%%%%%%%%%%%%%%%%%%%%%%%%%%%%%%%%%%%%%%%%%%%%%%%%%%%%%%%%%%

\subsection{Estacionariedad d\'ebil}

El ingrediente b\'asico de las series de tiempo son los procesos estoc\'asticos; para ello, se
supone dada la definci\'on de variables aleatorias, espacios de probabilidad, y espacios $L^{p}$;
si es necesario los defino, y si no me conformar\'e con citar un libro sobre series de tiempo
que cubra estos temas,
como el de Chatfield (The Analysis of Time Series: An Introduction, 2003).

Una muy buena raz\'on para empezar a describir \textbf{desde} procesos estoc\'asticos es tener
las definiciones a la mano, evitar conflictos con la notaci\'on $X(t)$ en lugar de $X_t$, y
enfatizar detalles sobre el tiempo continuo.

\begin{defn}[Proceso estoc\'astico]
Un proeso estoc\'astico $\{ X(t) \}$ es una familia de variables aleatorias indexadas por el 
s\'imbolo $t$ que pertenece a alg\'un conjunto $T \in \R$
\end{defn}

Matem\'aticamente se permitir\'a que $t$, referido como \textbf{tiempo}, tome valores 
en todo $\R$; las observaciones, en cambio,
s\'olo pueden ser tomadas en un conjunto discreto y finito de instantes en el tiempo. 
Adicionalmente, en algunas secciones se considerar\'an procesos estoc\'asticos complejos,
si bien la mayor parte del texto s\'olo usar\'a valores reales.

Esta definici\'on particular de proceso estoc\'astico deber\'ia enfatizar que para cada 
tiempo $t$, $X(t)$ es una variable aleatoria con su funci\'on de densidad de probabilidad,
sus momentos [s\'olo se consideran va's con al menos segundos momentos finitos], etc.

Otro concepto clave de este texto es el de \textbf{estaionareidad d\'ebil}; 
quiz\'a la mejor forma de motivar el adjetivo 'd\'ebil' es como contraposici\'on a 
la \textbf{estacionariedad fuerte o total}. 
Para ello, sea $F(X;\cdot)$ la funci\'on de densidad de probabilidad de $X$, es decir, 
la probabilidad de que $X\leq x$ puede expresarse como 
$
F(X;x) = P(X\leq)
$
bajo el entendido que $X$ y $x$ pueden ser vectores en $\R^{d}$.

\begin{defn}[Estacionariedad fuerte]
Un proceso estoc\'astico $\{ X(t) \}$ es fuertemente estacionario si, para cualquier 
conjunto de tiempos admisibles $t_1,t_2,\dots,t_n$ y cualquier $\tau \in \R$
se cumple que
\begin{equation*}
F\left(X(t_1),X(t_2),\dots,X(t_n);\cdot\right) 
\equiv
F\left(X(t_1+\tau),X(t_2+\tau),\dots,X(t_n+\tau);\cdot\right)
\end{equation*}
\end{defn}

La estacionariedad fuerte depende de las funciones de densidad de probabilidad conjunta para
diferentes tiempos. 
%Entre las consecuencias de que un proceso sea estacionario en el
%sentido fuerte, se encuentran:
%\begin{itemize}
%\item Media y varianzas constantes, todos los momentos constantes; es decir
%\begin{equation*}
%E[X^{n}(t)]
%\end{equation*}
%\item La funci\'on de autocorrelaci\'on s\'olo depende de 
%\end{itemize}
Si un proceso es estacionario en el sentido fuerte, entonces todas las variables $X(t)$ son 
id\'enticamente distribuidas.

%Al modelar eventos como proceso estoc\'asticos, tiene sentido que las variables aleatorias
%interfieran las unas con las otras de diversas

Con viene definir versiones menos fuertes de estacionariedad seg\'un sea posible deducirse de
las mediciones de un fen\'omeno y/o sean relevantes en su modelaci\'on.

\begin{defn}[Estacionariedad de orden $m$]
Un proceso estoc\'astico se dice estacionario de orden $m$ si, para cualquier 
conjunto de tiempos admisibles $t_1,t_2,\dots,t_n$ y cualquier $\tau \in \R$
se cumple que
\begin{equation*}
E\left[ X^{m_1}(t_1)X^{m_2}(t_2)\cdots X^{m_n}(t_n) \right]
=
E\left[ X^{m_1}(t_1+\tau)X^{m_2}(t_2+\tau)\cdots X^{m_n}(t_n+\tau) \right]
\end{equation*}
Para cualesquiera enteros $m_1,m_2,\dots,m_n$ tales que $m_1+m_2+\dots+m_n \leq m$
\end{defn}

Hay una especie de consenso seg\'un el cual la estacionariedad de orden 2, tambi\'en
llamada \textbf{estacionariedad d\'ebil} es suficiente para
que se cumplan los teoremas m\'as comunes sobre medias y varianzas.
Algunas consecuencias que un
proceso sea estacionario debilmente son las siguientes:
\begin{itemize}
\item Para todo $t$, $E[X(t)] = \mu$, una constante
\item Para todo $t$, $\Var{X(t)} = \sigma^{2}$, una constante
\item Para cualesquiera $t$, $\tau$, $\Cov{X(t+\tau),\Cov{X(t)}} = E[X(t+\tau)X(t)] - \mu^{2}$, 
una funci\'on de $\tau$ pero no de $t$
\end{itemize}

El rec\'iproco tambi\'en es cierto: si un proceso cumple las tres condiciones anteriores,
entonces es estacionario de orden 2. A su vez tres condiciones son m\'as usuales en la literatura
y tienen una intepretaci\'on m\'as clara como modelo, pues se exige que el proceso tenga media
y varianza constante, y que la funci\'on de autocorrelaci\'on no dependa de d\'onde se mida --lo
cual simplifica la estimaci\'on de estas cantidades.

Antes de proseguir, cabe mencionar que la estacionariedad fuerte se define
en t\'erminos de las funciones de densidad de probabilidad conjunta, mientras que la 
estacionariedad se define seg\'un los momentos; luego, la estacionariedad d\'ebil excluye 
procesos cuyos momentos no est\'en definidos. Por ejemplo, una colecci\'on de variables
independientes id\'enticamente distribuidos --con distribuci\'on de Cauchy-- ser\'a
fuertemente estacionario, pero no estacionario de orden $m$ para ning\'un $m$. 
%Por el contrario, un proceso estacionario de \textit{orden infinito} siempre es
%fuertemente estacionario.

Por el momento se asumir\'an procesos con segundos momentos finitos 
\textbf{debido a que} hay motivaciones
en el modelo para ello: energ\'ia finita, cambios finitos de energ\'ia, respuestas suaves, etc.

%%%%%%%%%%%%%%%%%%%%%%%%%%%%%%%%%%%%%%%%%%%%%%%%%%%%%%%%%%%%%%%%%%%%%%%%%%%%%%%%%%%%%%%%%%%%%%%%%%%

\subsection{El espectro de una serie de tiempo}

Quiero y me siento obligado a citar la excelente discuci\'on
filos\'ofica
de Loynes \cite{Loynes68}, resaltando la frase ''Los espectros instant\'aneos no existen''.
Tambi\'en quiero citar una discusi\'on m\'as moderna de M\'elard \cite{Melard89}, donde una
frase a favor es ''El supuesto de estacionariedad ha sido v\'alido previamente debido a la corta
duraci\'on de las series y la baja capacidad de c\'omputo''.

Pues la mayor parte de mi trabajo se ha centrado en el concepto de \textbf{espectro} de una serie
de tiempo. La mejor forma de introducir el espectro evolutivo 
--en el sentido que estoy usando-- es
presentar un proceso estacionario de orden 2,
 $\{X(t)\}$, en su representaci\'on de Cram\'er \cite{Priestley81}
[la existencia de esta representacion esta garantizada por el teorema de Khinchin-Wiener --para
procesos a tiempo continuos-- y por una extension del mismo por Wold --para procesos a tiempo
discreto.
por ahora solo cito el resultado, pero quiza sea buena idea escribir la demostracion
como apendice, una demostracion citada ya que es bastante tecnica]

\begin{equation*}
X(t) = \int_{\Lambda} A(\omega) e^{i 2\pi \omega t} dZ(\omega)
\end{equation*}

Donde el proceso $\{ Z(\omega) \}$ tiene incrementos ortogonales, es decir 
\begin{equation*}
\Cov{dZ(\omega_1,dZ(\omega_2))} = \delta(\omega_1,\omega_1) d\omega
\end{equation*}
Con $\delta$ la funci\'on delta de Dirac. Cabe mencionar que es suficiente si los incrementos
son independientes, pero se puede debilitar ese requerimiento; incluso es de notarse que no
se exige que el proceso sea al menos continuo --en el sentido estoc\'astico.

El espectro de potencia de $\{X(t)\}$ se define como

\begin{equation*}
f(\omega) = \abso{A(\omega)}^{2}
\end{equation*}

Citar\'e de Adak \cite{Adak98} una tabla donde compara varias definiciones de espectro, para
procesos no-estacionarios.

\begin{figure}[h]
\centering
\includegraphics[width=0.9\textwidth]{tabla.png} 
\end{figure}

Dos identidades muy importantes para estimar el espectro son la \textit{equivalencia} entre
el espectro y la funci\'on de autocorrelaci\'on

\begin{equation*}
f(\omega ) = \int R_X(\tau ) e^{-i 2\pi \omega t} d\tau
\end{equation*}

Donde funci\'on de autocorrelaci\'on se ha definido como

\begin{equation*}
R_X(\tau) = E\left[ X(t) X(t+\tau) \right] = \int_0^{\infty} X(t)X(t+\tau) dt
\end{equation*}

[la demostracion es corta, batsa con reescribir una composicion de integrales como convolucion,
la incluire mas tarde]

Por otro lado, se tiene la Identidad de Parseval

\begin{equation*}
\int X^{2}(t) dt = \int f(\omega) d\omega
\end{equation*}

[esta demostracion se basa en la convergencia dominada del modulo de la integral de $X^{2}$ por
la integral del modulo (...), la incluire mas tarde]

%%%%%%%%%%%%%%%%%%%%%%%%%%%%%%%%%%%%%%%%%%%%%%%%%%%%%%%%%%%%%%%%%%%%%%%%%%%%%%%%%%%%%%%%%%%%%%%%%%%

\subsection{Test Priestley-Subba Rao (PSR)}

(seccion en proceso de re-redaccion)

A muy grosso modo, el test PSR estima localmente  el espectro evolutivo
 y revisa si estad\'isticamente
cambia en el tiempo.

Para ello, usa un estimador para la funci\'on de densidad espectral
que es aproximadamente (asint\'oticamente) insesgado y cuya varianza est\'a
determinada aproximadamente. La estimaci\'on se lleva a cabo en puntos en el tiempo y
la frecuencia tales que en conjunto son aproximadamente no-correlacionados.
Se aplica logaritmo para que la varianza de todos los estimadores sea aproximadamente
la misma (el logaritmo ayuda a), amen que los errores conjuntos tengan una
distribuci\'on cercana a una multinormal con correlaci\'on cero.
Finalmente se aplica una prueba ANOVA de varianza conocida.

%%%%%%%%%%%%%%%%%%%%%%%%%%%%%%%%%%%%%%%%%%%%%%%%%%%%%%%%%%%%%%%%%%%%%%%%%%%%%%%%%%%%%%%%%%%%%%%%%%%

\subsection{El espectro evolutivo}

Consid\'erese un proceso estoc\'astico a tiempo continuo $\{X(t)\}$, tal que
$E[X(t)]=0$ y $E\left[ X^{2}(t)\right] < \infty$ para todo $t$. Es decir que su media es constante
y sus segundos momentos est\'an bien definidos, aunque 
estos \'ultimos pueden cambiar con el tiempo.

Por el momento se supondr\'a que acepta una representaci\'on de la forma

\begin{equation*}
X(t) = \int_{-\pi}^{\pi} A(t ; \omega) e^{i\omega t} \, d Z(\omega)
\end{equation*}

Con $\{ Z(\omega) \}$ una familia de procesos ortogonales\footnote{De nuevo, esto implica que
$\Cov{dZ(\omega_1,dZ(\omega_2))} = \delta(\omega_1,\omega_1) d\omega$, una condici\'on m\'as
d\'ebil que la independencia} tales que

\begin{itemize}
\item $E \left[\abso{ dZ(\omega)}^{2} \right] = d\omega$
\item Para cada $t$ el m\'aximo de $A(t;\cdot)$ se encuentra en 0
\end{itemize}

Esta representaci\'on es an\'aloga a la representaci\'on de Cram\'er para un proceso
estacionario, salvo que se permite que la funci\'on $A$ cambie con el tiempo.
Siguiendo la analog\'ia, se define 
el \textbf{espectro evolutivo} de $\{X(t)\}$, con respecto a la la familia
$\mathcal{F} = \{ e^{i\omega t} A(t; \omega) \}$
 como
 
\begin{equation*}
d F(\omega;t) = \lvert A(t;\omega) \lvert^{2} d\omega
\end{equation*}

Ahora bien, si se supone que $\{X(t)\}$ es estoc\'asticametne diferenciable, entonces
se puede definir una \textbf{funci\'on de densidad espectral}

\begin{equation*}
f(t;\omega) = \lvert A(t;\omega) \lvert^{2}
\end{equation*}

Cabe destaca que si la funci\'on $A(t;\omega)$ fuera constante con respecto a $t$, se obtendr\'ia
un proceso estacionario de orden dos tal cual fue descrito en la secci\'on anterior.

%%%%%%%%%%%%%%%%%%%%%%%%%%%%%%%%%%%%%%%%%%%%%%%%%%%%%%%%%%%%%%%%%%%%%%%%%%%%%%%%%%%%%%%%%%%%%%%%%%%

\subsection{El estimador de doble ventana}

Esta t\'ecnica fue presentada por Priestley en 1965. Muy a grosso modo, es un estimador de la
funci\'on de densidad espectral con ciertas propiedades y que parte de la idea que un proceso
no-estacionario puede verse localmente como un proceso lineal generalizado.

Como meta-nota, yo empec\'e a estudiar este tipo de estimadores porque es \textit{el qeu ven\'ia
con el m\'etodo} ya que el test esta implementado en R; desde un punto de vista de difusi\'on,
es una ventaja usar un m\'etodo implementado en un software gratuito y de c\'odigo abierto --y
no una mera excusa para no explorar otros m\'etodos. En todo caso, he revisado varios otros test,
pero de momento solo este ha arrojado suficientes resultados para llenar un informe.

%{Estimador de doble ventana (Priestley, 1965 \& 1966)}
Para construir el estimador se reuieren dos funciones, $g$ y $w_T$, que servir\'an como ventanas
para extraer informaci\'on local de los datos. Debido a que sus propiedades tienen una interpretaci\'on
f\'isica desde la teor\'ia de circuitos, absorben su terminolog\'ia

\textit{
nota al pie: deberia incluir una motivacion de estos nombres,
que en parte tiene relevancia en la interpretacion. Los 
Linear Invariant Systems (LIS) suponen dependencia lineal
--constante-- respecto a todos los tiempos anteriores; 
a tiempo continuo son equivalentes a una ecuacion diferencial ordinaria lineal,
y a su vez a modelos AR. Un modelo fisico para ello son los circuitos RC, que
fueron usables en radios, y para los cuales las palabras 'filtro' y 'frecuencia'
tienen una interpretacion clara. Esta terminologia de circuitos electricos tiene sentido
para mi ya que todos los modelos de neuronas y poblaciones de neuronas que he visto hasta ahora,
por ejemplo de Ermentrout (falta citar), {Clark98,Priestley81}, PARTEN de considerar
circuitos equivalentes a los componentes neuronales, lo cual me hace pensar que es buena idea
mantener esta vision conjunta.
}

Primeramente se toma una funci\'on $g(u)$ normalizada, que en conjunto a su
transformada inversa de Fourier\footnote{Esta funci\'on 
$\Gamma(u) = \int_{-\infty}^{\infty} g(u) e^{i u \omega} du$
es referida como
\textbf{frequency-response function}, nombre tiene un poco de encanto cuando
$g$ adopta ciertas formas particulares (senos y cosenos).} 
$\Gamma$ tiene las siguientes propiedades

\begin{equation*}
2\pi \int_{-\infty}^{\infty} \lvert g(u) \lvert^{2} du 
= 
\int_{-\infty}^{\infty} \lvert \Gamma(\omega) \lvert^{2} d\omega
= 1
\end{equation*}


A partir de $g$ y $\Gamma$ se define el filtro $U$ como una convoluci\'on
con las realizaciones del proceso

\begin{equation*}
U(t,\omega) = \int_{t-T}^{t} g(u) X({t-u}) e^{i \omega (t-u)} du
\end{equation*}

Un ejemplo que est\'a en el libro de Priestley es tomar funciones del tipo

\begin{equation*}
g_h(u) = 
\begin{cases}
{1 \big{/} 2\sqrt{\pi h}} & \text{ , } \abso{u} \leq h
\\
0 & \text{ , } \abso{u} > 0
\end{cases}
\end{equation*}

Su correspondiente funci\'on de respuesta de frecuencia es complicada [me falta 
escribirla]. Es referida como la \textbf{ventana de Bartlett} y
est\'a totalmente caracterizada la siguiente propiedad

\begin{equation*}
\abso{\Gamma_h(\omega)}^{2} = \frac{1}{\pi h} \left( \frac{\text{sen} (h \omega)}{\omega} \right)^{2}
\end{equation*}

Cabe mencionar que puede entenderse al par $g$ y $\Gamma$ como ventanas en el tiempo
y las frecuencias para la serie.

---

Ahora bien, se toma una segunda ventana $W_\tau$ con las siguientes
restricciones para
su funci\'on de respuesta ante frecuencia $w_\tau$

\begin{itemize}
\item $w_{\tau}(t) \geq 0$ para cualesquiera $t$, $\tau$
\item $w_{\tau}(t) \rightarrow 0$ cuando $\lvert t \lvert \rightarrow \infty$, para todo $\tau$
\item $\displaystyle \int_{-\infty}^{\infty} w_{\tau}(t) dt = 1$ para todo $\tau$
\item $\displaystyle \int_{-\infty}^{\infty} \left( w_{\tau}(t) \right)^{2} dt < \infty$ para todo $\tau$
\item Existe una constante $C$ tal que  [T est\'a relacionado con el 'tiempo 0', pero para
tiempos de muestreo grandes se puede reemplazar por $-\infty$ EXCEPTO cerca del inicio y el final dle muestreo]
$$\lim_{\tau\rightarrow\infty} \left[ \tau \int_{t-T}^{t} \lvert W_{\tau}(\lambda) \lvert^{2} d\lambda \right] = C$$
\end{itemize}

%Ahora, si se define 
%$\displaystyle W_{T'}(\lambda) = \int_{-\infty}^{\infty} e^{-i\lambda t}w_{T'}(t) dt $

[posteriormente annadire mas detalles sobre el papel que juega el par $w_\tau$, $W_\tau$]

Como ejemplo, se puede tomar la siguiente funci\'on llamada \textbf{ventana de Daniell}

\begin{equation*}
W_\tau (t) = 
\begin{cases}
{1 \big{/} \tau} & \text{ , } -\nicefrac{1}{2} \tau \leq t \leq \nicefrac{1}{2} \tau
\\
0 & \text{ , otro caso}
\end{cases}
\end{equation*}

La cual se puede demostrar [tengo en algun lado esa demostracion]

$$\lim_{\tau\rightarrow\infty} \left[ \tau \int_{t-T}^{t} \lvert W_{\tau}(\lambda) \lvert^{2} d\lambda \right] = 2\pi$$

-----

Se define el estimador para $f_t$, con $0 \leq t \leq T$
\begin{equation*}
\widehat{f_t}(\omega) = \int_{t-T}^{t} w_{T'}(u) \lvert U(t-u,\omega) \lvert^{2} du
\end{equation*}

Fue demostrado por Priestley (1965, falta citar) que 

[aqui van las expresiones para el valor esperado y la varianza de $\widehat{f_t}$, me falta
escribirlas]

Pero, bajo varios supuesto adicionales [que me falta trascribir] se puede aproximar

\begin{equation*}
E\left[ \widehat{f_t}(\omega) \right] \sim f_t(\omega)
\end{equation*}

\begin{equation*}
\Var{\widehat{f_t}(\omega)} 
\sim 
\frac{C}{\tau} \left(f_t(\omega)\right)^{2} \int_{-\infty}^{\infty} \abso{\Gamma(\theta)}^{4} d\theta
\end{equation*}

Se advierte claramente que $\widehat{f_t}$ es unnestimados aproximadamente insesgado.
Para las ventanas de Bartlett y Daniell usadas como ejemplo, se tiene

\begin{equation*}
\Var{\widehat{f_t}} 
\sim 
\frac{4h}{3\tau} \left(f_t(\omega)\right)^{2}
\end{equation*}

Cabe mencionar que hay una expresi\'on expl\'icita para la covarianza de $\widehat{f_t}$
en para diferentes puntos en el tiempo y las frecuencias. Lamentablemente,
aun me falta escribirlas, son complicadas, y se describen situaciones bajo las
cuales estas covarianzas son negligibles; cabe destacar que TODAS las condiciones 
que se usan para aproximar son b\'asicamente las mismas, y dependen de que la distancia
entre los tiempos y las frecuencias sean tan grandes como sea posible.

------------

El \'ultimo ingrediente del test PSR es una transformaci\'on logar\'itmica
para regular la varianza, y quiza para cortar los bordes de las aproxiamciones.
Se define $Y_{i,j} = \log \left( \widehat{f_{t_i}}(\omega_j) \right)$, con las siguientes propiedades

\begin{equation*}
E\left[ Y_{i,j} \right] \thicksim \log \left( f_{t_i}(\omega_j) \right)
\hspace{4em}
\text{Var}\left( {Y\left(t,\omega\right)}\right) \thicksim \sigma^{2}
\end{equation*}

Luego as\'i, puede escribirse aproximadamente que

$$Y_{i,j} = \log \left( f_{t_i}(\omega_j) \right) + \varepsilon_{i,j}$$

donde $\varepsilon_{i,j}$ va iid tales que

$
E\left[ \varepsilon_{i,j} \right] = 0
\hspace{4em}
\text{Var}\left( \varepsilon_{i,j} \right) \sigma^{2}
$

Priestley cita que con esta informaci\'on incluso se puede considerar que los $\varepsilon_{i,j}$
siguen una distribuci\'on normal cada uno; Nason (2015, falta citar) comenta que
este supuesto no tiene por que cumplirse, y que es una popsible fuente de falsos positivos
para el test. Yo he hecho pruebas de normalidad a los datos, que incluire como anexos
mas tarde.

El test PSR \textit{per se} son tres test ANOVA --en su versi\'on en la que la varianza es conocida--
sobre si los $\varepsilon_{i,j}$ son estad\'isticamente negligibles en total, sobre el tiempo y sobre
las frecuencias. Para el fin de estudiar la estacionariedad, basta con que sean estad\'iticamente
no-negligibles sobre el tiempo.

[Por supuesto que los otros dos test tienen interpretacion: la negigibilidad total da informacion
sobre las marginales, y si estas pueden ser estimadas adecuadamente usando el estimador, si se
combina con negativo para no-estacionariedad es \textbf{efectivamente positivo} para estacionariedad
y toma una forma muy particular (proceso uniformemente modulado). Si sobre las frecuencias resulta
significativo (no-negligible) da informacion sobre la 'aeatoridad total' del proceso.
De tener tiempo, lo incluire como anexo, ya que ninguna de estas caracteristicas es estudiada :( ]

Lo detalles de la implementaci\'on en R estar\'an en la secci\'on de resultados.

%%%%%%%%%%%%%%%%%%%%%%%%%%%%%%%%%%%%%%%%%%%%%%%%%%%%%%%%%%%%%%%%%%%%%%%%%%%%%%%%%%%%%%%%%%%%%%%%%%%
%%%%%%%%%%%%%%%%%%%%%%%%%%%%%%%%%%%%%%%%%%%%%%%%%%%%%%%%%%%%%%%%%%%%%%%%%%%%%%%%%%%%%%%%%%%%%%%%%%%

%%%%%%%%%%%%%%%%%%%%%%%%%%%%%%%%%%%%%

%%%%%%%%%%%%%%%%%%%%%%%%%%%%%%%%%%%%%%%%%%%%%%%%%%%%%%%%%%%%%%%%%%%%%%%%%%%%%%%%%%%%%%%%%%%%%%%%%%%
%%%%%%%%%%%%%%%%%%%%%%%%%%%%%%%%%%%%%%%%%%%%%%%%%%%%%%%%%%%%%%%%%%%%%%%%%%%%%%%%%%%%%%%%%%%%%%%%%%%
%%%%%%%%%%%%%%%%%%%%%%%%%%%%%%%%%%%%%%%%%%%%%%%%%%%%%%%%%%%%%%%%%%%%%%%%%%%%%%%%%%%%%%%%%%%%%%%%%%%
%%%%%%%%%%%%%%%%%%%%%%%%%%%%%%%%%%%%%%%%%%%%%%%%%%%%%%%%%%%%%%%%%%%%%%%%%%%%%%%%%%%%%%%%%%%%%%%%%%%

\chapter{Metodología}

El presente trabajo resulta de una colaboración con el departamento de Gerontología, dependiente 
del Instituto de Ciencias de la Salud (ICSA); parte de esta colaboración incluye el acceso a los 
registros de PSG obtenidos por Vázquez Tagle y colaboradores \cite{VazquezTagle16}. 
A continuación se expone la metodología de aquél estudio.% de la manera más fiel posible.
%Así mismo se describe, a nivel de implementación, el análisis central de este trabajo: la prueba 
%de Priesltey-Subba Rao. 

\section{Participantes}

Los sujetos fueron elegidos usando un muestreo \textit{no probabilístico por 
conveniencia}\footnote{Esto implica que los resultados pueden  no ser interpolables a poblaciones 
más grandes} bajo los siguientes criterios de inclusión:
\begin{itemize}
\item Edad entre 60 y 85 años
\item Diestros (mano derecha dominante)
\item Sin ansiedad, depresión ni síndromes focales
\item No usar medicamentos o sustancias para dormir
\item Firma de consentimiento informado
\item Voluntario para el registro de PSG
\end{itemize}

Un total de 9 participantes cumplieron todos los criterios de inclusión y procedieron al registro 
de PSG; adicionalmente se tomaron registros de otros tres adultos mayores, bajo el consentimiento 
de éstos y de los responsables del proyecto (ver más adelante).
%
Todos los participantes fueron sometidos a una batería de pruebas neuropsicológicas para determinar
su estado cognoscitivo general (Neuropsi, MMSE), así como descartar cuadros depresivos (GDS, SATS) 
y cambios en la vida cotidiana (KATZ).
Usando los resultados obtenidos, los sujetos se dividieron en tres grupos:

\begin{table}[h]
\centering
\begin{tabular}{lcl}
\toprule
Grupo & Sujetos & Características \\
\midrule
Mn & 4 & Posible Deterioro Cognitivo \\
Nn & 5 & Sin PDC \\
ex & 3 & No satisfacen los criterios de inclusión \\
\bottomrule
\end{tabular}
\end{table}
%
%\begin{description}
%\item[Mn] (4 sujetos) Posible Deterioro Cognitivo
%\item[Nn] (5 sujetos) Sin PDC
%\item[ex] (3 sujetos) No satisfacen los criterios de inclusión
%\end{description}

El grupo ex se conforma de sujeto que incumplen al menos uno de los criterios de inclusión: {FGH} 
padece parálisis facial y posiblemente daño cerebral (síndromes focales), MGG padece depresión, 
EMT no califica como adulto mayor por su edad.
Se efectuaron todos los análisis sobre este grupo, con la finalidad de exhibir las capacidades y
limitaciones de las técnicas utilizadas; por ello este grupo es ignorado en la sección de 
resultados pero no en la discusión.

\begin{table}
\centering
\bordes{1.1}
\begin{tabular}{c}
\textbf{Datos generales de los participantes}
\vspace{1em}
\end{tabular}
{\small
\begin{tabular}{llcrrrrrrr}
\toprule
 \phantom{.}&
 & {Sexo} & {Edad} & {Escol.} & {Neuropsi} & {MMSE} & {SATS} & {KATZ} & {Gds} \\
\midrule
\multicolumn{6}{l}{{Grupo Nn}}\\
&VCR    & F    & 59\pz & 12\pz & 107\pz & 29\pz & 21\pz & 0\pz & 3\pz \\
&MJH    & F    & 72\pz & 9\pz  & 113\pz & 30\pz & 18\pz & 0\pz & 0\pz \\
&JAE    & F    & 78\pz & 5\pz  & 102\pz & 28\pz & 19\pz & 0\pz & 5\pz \\
&GHA    & M    & 65\pz & 9\pz  & 107.5  & 30\pz & 23\pz & 0\pz & 7\pz \\
&MFGR   & F    & 67\pz & 11\pz & 110\pz & 30\pz & 18\pz & 0\pz &      \\
\rowcolor{gris}
&\multicolumn{1}{c}{$\widehat{\mu}$} & 
               & 68.2  & 9.2   & 107.9  & 29.4  & 19.8  & 0.0  & 3.0  \\
\rowcolor{gris}
&\multicolumn{1}{c}{$\widehat{\sigma}$} & 
               & 7.2   & 2.7   & 4.1    & 0.9   & 2.2   & 0.0  & 3.0  \\
\midrulec
%\hline
\multicolumn{6}{l}{{Grupo Mn}}\\
&CLO    & F    & 68\pz & 5\pz  & 81\pz & 28\pz & 22\pz & 1\pz & 6\pz \\
&RLO    & F    & 63\pz & 9\pz  & 90\pz & 29\pz & 20\pz & 0\pz & 3\pz \\
&RRU    & M    & 69\pz & 9\pz  & 85\pz & 27\pz & 10\pz & 0\pz & 3\pz \\
&JGZ    & M    & 65\pz & 11\pz & 87\pz & 25\pz & 20\pz & 0\pz & 1\pz \\
\rowcolor{gris}
&\multicolumn{1}{c}{$\widehat{\mu}$} & 
              & 66.3   & 8.5   & 85.8  & 27.3  & 18.0  & 0.3  & 3.3  \\
\rowcolor{gris}
&\multicolumn{1}{c}{$\widehat{\sigma}$} & 
              & 2.8    & 2.5   & 3.8   & 1.7   & 5.4   & 0.5  & 2.1  \\
\midrulec
%\hline
\multicolumn{6}{l}{{Grupo ex}}\\
&FGH    & M    & 71\pz   & 9\pz    & 83.5     & 21\pz   & 23\pz   & 0\pz    & 4\pz  \\
&MGG    & F    & 61\pz   & 9\pz    & 114\pz      & 28\pz   & 29\pz   & 1\pz    & 14\pz \\
&EMT    & M    & 50\pz   & 22\pz   & 106\pz      & 30\pz   & 15\pz   & 0\pz    & 4\pz  \\
\bottomrule
\end{tabular} 
}
\label{tab_sujetos}
\caption{Resultados de las pruebas neuropsicológicas 
}
\end{table}

%%%%%%%%%%%%%%%%%%%%%%%%%%%%%%%%%%%%%%%%%%%%%%%%%%%%%%%%%%%%%%%%%%%%%%%%%%%%%%%%%%%%%%%%%%%%%%%%%%%
%%%%%%%%%%%%%%%%%%%%%%%%%%%%%%%%%%%%%%%%%%%%%%%%%%%%%%%%%%%%%%%%%%%%%%%%%%%%%%%%%%%%%%%%%%%%%%%%%%%

\section{Registro del polisomnograma}

Los adultos mayores participantes fueron invitados a acudir a las instalaciones de la Clínica 
Gerontológica de Sueño (ubicadas dentro del Instituto de Ciencias de la Salud) para llevar a cabo 
el registro. Los participantes recibieron instrucciones de realizar una rutina normal de 
actividades durante la semana que precedió al estudio, y se les recomendó que no ingirieran bebidas 
alcohólicas o energizantes (como café o refresco) durante las 24 horas previas al experimento, ni 
durmieran siesta ese día.

El protocolo de PSG incluye 19 electrodos de EEG, 4 electrodos de EOG para registrar movimientos 
oculares horizontales y verticales, y 2 electrodos de EMG colocados en los músculos submentonianos 
para registrar la actividad muscular. 
La colocación de los electrodos para registrar la actividad EEG se realizó siguiendo las 
coordenadas del Sistema Internacional 10--20.

Las señales fueron amplificadas (amplificador de alta ganancia en cadena), filtradas (filtro paso 
de banda de 0.5--30 Hz) y digitalizadas para su posterior análisis.
En la tabla \ref{frecuencias} se reportan la duración de estos registros para cada sujeto.

Debido a problemas técnicos el registro se efectúo a 512 puntos por segundo (Hz) para algunos
participantes, y a 200 Hz para otros; en ambos casos se satisface la recomendación de la AASM de un 
mínimo de 128 Hz. 

La clasificación del PSG en fases de sueño se realizó \textit{manualmente} sobre épocas de 30 
segundos siguiendo los criterios estandarizados de la AAMS \cite{Hori01}.

%Debido a un cambio en el polisomnógrafo 
%usado, la frecuencia de muestreo (en Hz) cambia entre sujetos.

\begin{table}
\centering
\bordes{1.2}
\begin{tabular}{c}
\textbf{Datos generales sobre los registros de PSG}
\vspace{1em}
\end{tabular}
{\small
\begin{tabular}{llcrrcrrr}
\toprule
    \phantom{.}&
    &\multirow{2}{*}{\bordes{1}\begin{tabular}{l}Frecuencia\\ muestreo\end{tabular}}
    \bordes{1.2}
    & \multicolumn{2}{c}{Total} & \phantom{l}   & \multicolumn{3}{c}{MOR*}\\
    \cmidrule{4-5}  \cmidrule{7-9}
    &&          &Puntos  &  Tiempo   &&Puntos  &  Tiempo   &  \% MOR \\
\midrule
\multicolumn{6}{l}{{Grupo Nn}}\\
&VCR &200       & 5166000&   7:10:30 &&438000  &   0:36:30 & 8.5\% \\
&MJH &512       &15851520&   8:36:00 &&1950720 &   1:03:30 &12.3\% \\
&JAE &512       &13931520&   7:33:30 &&2626560 &   1:25:30 &18.9\% \\
&GHA &200       &6558000 &   9:06:00 &&330000  &   0:27:30 & 5.0\% \\
&MFGR&200       &4932000 &   6:51:00 &&570000  &   0:47:30 &11.6\% \\

\rowcolor{gris}
&\multicolumn{1}{c}{$\widehat{\mu}$}  
              & &        & 7:51:30   &&        &   0:52:06 &11.2\% \\
\rowcolor{gris}
&\multicolumn{1}{c}{$\widehat{\sigma}$} 
              & &        & 0:57:36   &&        &   0:23:00 & 5.1\% \\
\midrulec

\multicolumn{6}{l}{{Grupo Mn}}\\
&CLO &512       &14499840&   7:52:00 &&2027520 &   1:06:00 &14.0\% \\
&RLO &512       &12994560&   7:03:00 &&1520640 &   0:49:30 &11.7\% \\
&RRU &200       &2484000 &   3:27:00 &&228000  &   0:19:00 & 9.2\% \\
&JGZ &512       &18539520&  10:03:30 &&506880  &   0:16:30 & 2.7\% \\

\rowcolor{gris}
&\multicolumn{1}{c}{$\widehat{\mu}$}  
              & &        & 7:06:23   &&        &   0:37:45 &9.4\% \\
\rowcolor{gris}
&\multicolumn{1}{c}{$\widehat{\sigma}$} 
              & &        & 2:44:55   &&        &   0:24:05 &4.9\% \\
\midrulec

\multicolumn{6}{l}{{Grupo ex}}\\
&FGH &512       &6220800 &   3:22:30 &&337920  &   0:11:00 & 5.4\% \\
&MGG &512       &15820800&   8:35:00 &&2549760 &   1:23:00 &16.1\% \\
&EMT &512       &21857280&  11:51:30 &&721920  &   0:23:30 & 3.3\% \\
\bottomrule
\end{tabular}
}
\caption{Cantidad de datos registrados para cada sujeto. *Dado que el sueño MOR aparece fragmentado,
se reporta la suma de tales tiempos.}
\label{frecuencias}
\end{table}

%%%%%%%%%%%%%%%%%%%%%%%%%%%%%%%%%%%%%%%%%%%%%%%%%%%%%%%%%%%%%%%%%%%%%%%%%%%%%%%%%%%%%%%%%%%%%%%%%%%
%%%%%%%%%%%%%%%%%%%%%%%%%%%%%%%%%%%%%%%%%%%%%%%%%%%%%%%%%%%%%%%%%%%%%%%%%%%%%%%%%%%%%%%%%%%%%%%%%%%

\section{Aplicación de la prueba PSR}

Los registros digitalizados de PSG fueron convertidos a formato de texto bajo la codificación 
ASCII, a razón de un archivo por cada canal. 
Las épocas MOR, clasificadas manualmente, fueron indicadas en archivos a parte.

%Como se mencionó en secciones anteriores, la prueba PSR está pensada para series de tiempo con 
%media 0, varianza finita y espectro puramente continuo. Se espera que la segunda condición se 
%cumpla para los registros de PSG; las otras dos condiciones fueron \textit{forzadas}, sustrayendo 
%la media y la componente periódica (estimadas) del proceso.
%Para lo anterior, se usó el algoritmo no-paramétrico STL (Seasonal-Trend decomposition using 
%Loess) \cite{Cleveland1990} y que está implementado en R bajo la función \texttt{stl()}.

%La prueba PSR se encuentra implementado en R bajo la función \texttt{stationarity()} del paquete 
%\texttt{fractal}.    
%Los resultados de la prueba PSR, aplicado a todas las épocas contenidas en los registros de PSG,
%fueron almacenados para su análisis posterior.

El registro de PSG por cada canal fueron analizados por separado, y éstos a su vez fueron divididos
en épocas de 30 segundos de duración (variando el número de puntos según la frecuencia de muestreo);
cada época fue clasificada como \textit{estacionaria} si, no pudo rechazarse la hipótesis de 
estacionariedad usando la prueba PSR ($p < 0.05$).
La cantidad de épocas estacionarias para cada individuo, durante sueño MOR y NMOR, se muestra en 
las tablas \ref{total_gpos_total} y \ref{total_gpos_mor}; debido a la gran variabilidad entre los 
sujetos para la duración del sueño MOR, para el análisis no se consideró el total de épocas sino la 
proporción de éstas en cada etapa de sueño. 

\begin{figure}
\centering
\begin{lstlisting}[caption={}]
Priestley-Subba Rao stationarity Test for datos
-----------------------------------------------
Samples used              : 3072 
Samples available         : 3069 
Sampling interval         : 1 
SDF estimator             : Multitaper 
  Number of (sine) tapers : 5 
  Centered                : TRUE 
  Recentered              : FALSE 
Number of blocks          : 11 
Block size                : 279 
Number of blocks          : 11 
p-value for T             : 0.4130131 
p-value for I+R           : 0.1787949 
p-value for T+I+R         : 0.1801353 
\end{lstlisting}
\caption{Resultado típico para la función \texttt{stationarity}
%El parámetro \texttt{n.blocks} define la cantidad grupos disjuntos para los cuales se calculará 
%el estimador de la FDE.
%Cabe resaltar el antepenúltimo renglón (\texttt{p-value for T}), según el cual se puede
%aceptar o rechazar la hipótesis de estacionariedad débil. 
%La FDE es referida como 'Spectral Density Function' (SDF).
}
\label{res_psr}
\end{figure}

%%%%%%%%%%%%%%%%%%%%%%%%%%%%%%%%%%%%%%%%%%%%%%%%%%%%%%%%%%%%%%%%%%%%%%%%%%%%%%%%%%%%%%%%%%%%%%%%%%%
%%%%%%%%%%%%%%%%%%%%%%%%%%%%%%%%%%%%%%%%%%%%%%%%%%%%%%%%%%%%%%%%%%%%%%%%%%%%%%%%%%%%%%%%%%%%%%%%%%%
%%%%%%%%%%%%%%%%%%%%%%%%%%%%%%%%%%%%%%%%%%%%%%%%%%%%%%%%%%%%%%%%%%%%%%%%%%%%%%%%%%%%%%%%%%%%%%%%%%%
%%%%%%%%%%%%%%%%%%%%%%%%%%%%%%%%%%%%%%%%%%%%%%%%%%%%%%%%%%%%%%%%%%%%%%%%%%%%%%%%%%%%%%%%%%%%%%%%%%%

%%%%%%%%%%%%%%%%%%%%%%%%%%%%%%%%%%%%%

%%%%%%%%%%%%%%%%%%%%%%%%%%%%%%%%%%%%%%%%%%%%%%%%%%%%%%%%%%%%%%%%%%%%%%%%%%%%%%%%%%%%%%%%%%%%%%%%%%%
%%%%%%%%%%%%%%%%%%%%%%%%%%%%%%%%%%%%%%%%%%%%%%%%%%%%%%%%%%%%%%%%%%%%%%%%%%%%%%%%%%%%%%%%%%%%%%%%%%%
%%%%%%%%%%%%%%%%%%%%%%%%%%%%%%%%%%%%%%%%%%%%%%%%%%%%%%%%%%%%%%%%%%%%%%%%%%%%%%%%%%%%%%%%%%%%%%%%%%%
%%%%%%%%%%%%%%%%%%%%%%%%%%%%%%%%%%%%%%%%%%%%%%%%%%%%%%%%%%%%%%%%%%%%%%%%%%%%%%%%%%%%%%%%%%%%%%%%%%%

\chapter{Resultados}

En cada canal que conforma el PSG (EEG, EOG y EMG), cada una de las \'epocas consideradas fue 
clasificada como 'posiblemente estacionaria' (PE) si no pudo ser rechazado la hip\'otesis de 
estacionariedad usando el test PSR ($\alpha < 0.05$), o como 'no--estacionaria en caso contrario'.
Variar el valor cr\'itico para esta clasificaci\'on no pareci\'o generar diferencias significativas.
La cantidad de \'epocas PE en cada individuo durante el sue\~no MOR y NMOR se muestra en las tablas 
\ref{total_gpos_mor}, \ref{total_gpos_nmor} y \ref{total_gpos_total}; debido a que hubo una gran 
variabilidad entre los sujetos para el tiempo que permanecieron en sue\~no MOR, se consider\'o no 
el total de \'epocas PE sino la proporci\'on de estas \'epocas en las respectivas etapas de sue\~no; 
estos resultados se muestran en las tablas \ref{gpos_mor}, \ref{gpos_nmor} y \ref{gpos_total}. 
Adicionalmente se han calculado promedios y desviaciones est\'andar grupales, seg\'un la
clasificaci\'on arrojada por las pruebas neuropsicol\'ogicas.

Como un primer an\'alisis se verific\'o si el sue\~no MOR, entendido como muestra del registro
completo, tiene o no prpiedades estad\'sticas pareceidas a este \'ultimo --y si \'esta similaridad 
pudiera estar relacionada con el PDC. 
Se compar\'o la proporci\'on de \'epocas PE en cada canal durante sue\~no MOR y NMOR usando la 
prueba $\chi^{2}$ para proporciones\footnote{Implementada en R como la funci\'on 
\texttt{prop.test()}}; los resultados se muestran en la tabla \ref{comparacion_mor_vs_total}.

Se encontr\'o que no hay diferencias significativas, consistentes en todos los sujetos, en los 
canales LOG y ROG, lo cual puede ser explicado por la tipificaci\'on del sue\~no MOR. 
Por otro lado, no se encontr\'o una relaci\'on clara entre el estado de salud del sujeto y la 
aparici\'on de diferencias significativas entre estas proporciones.

\begin{SidewaysFigure}
\centering
\begin{tabular}{c||ccccc||cccc||ccc}
&VCR&MJH&JAE&GHA&MFGR&CLO&RLO&RRU&JGZ&FGH&MGG&EMT \\
\hline
C3&**& &*&**& & &**&*& & & &  \\
C4&*& &***&*& & &***& & & &*&  \\
CZ&***& & & & & & & & & &***&  \\
F3&**& & &**& & &***& & & &*&** \\
F4& & & &*& & &***& & & &***&  \\
F7&*& &***&***& & & &*& & &***&*** \\
F8&**& & &**& & &*&*& & &***&  \\
FP1&***& & &*&*& & &*& & &***&  \\
FP2&***& & &*& & & & & & &***&  \\
FZ& & & &***& & &***&*& & &*&  \\
O1&*& & &***& & & & & &*& &  \\
O2& & &*&***& & &***& & & &***& \\ 
P3&*& &*&***& & &**& & & &*&  \\
P4&***& &***&*& & & & & &*&***&  \\
PZ&**& &***&*& & & &*& & &***&  \\
T3& & &**& & & &***& & & & &** \\
T4& & & &*& & &***& & & &*&  \\
T5&*& & &***& & &***& & & & &  \\
T6& & &*& & & & & & & &***&  \\
LOG&***& &***&***&**&***&**&***&*& &***&  \\
ROG&***& &***&***&*&*&***&*&*& &***&  \\
EMG& & & & & &***& &*& & & &  \\
\hline
General& & & & & &***& &*& & & & 
\end{tabular}
\caption{Diferencias significativas para la comparaci\'on entre la proporci\'on de \'epocas PE en 
sue\~no MOR (fase R) y NMOR (fases W y N).
Los asteriscos representan el pvalor con el cual se rechaza la hip\'otesis de que las diferencias 
son significativas: *=0.05 , **=0.01 , ***=0.005}
\label{comparacion_mor_vs_total}
\end{SidewaysFigure}

Posteriormente se busc\'o una diferencia m\'as directa entre los grupos, comparando grupalmente
las proporciones de \'epocas PE (en cada canal y durante las diferentes etapas). Para la 
comparaci\'on per se se us\'o la prueba U de Mann-Whitney\footnote{Implementada en R como la 
funci\'on \texttt{wilcox.test()}}.
No se encontraron diferencias significativas para ninguno de los canales, los resultados se 
muestran en las tablas \ref{gpos_mor}, \ref{gpos_nmor}, \ref{gpos_total}; para una mejor 
vizualizaci\'on, \'estos se han graficado en la figura \ref{comparacion_graf}.

\begin{figure}
\centering
\subfloat[Comparaci\'on entre \'epocas MOR (fase R)]{
\includegraphics[width=0.95\linewidth]
{./new170424/Comparacion_gpos_MOR.pdf} 
}\\
\subfloat[Comparaci\'on entre \'epocas no-MOR (fases W y N)]{
\includegraphics[width=0.95\linewidth]
{./new170424/Comparacion_gpos_NMOR.pdf} 
}\\
\caption{Comparaci\'on sobre las proporciones de \'epocas PE entre los grupos Control (azul) y
PDC (rojo), para diferentes etapas de sue\~no (MOR y NMOR). Se grafica el promedio grupal $\pm$
1 desviac\'on est\'andar $^{\nicefrac{3}{2}}$, como visualizaci\'on aproximada de la varianza.}
\label{comparacion_graf}
\end{figure}

Una segunda variaci\'on del primer an\'alisis es considerar grupalmente a los sujetos como 
'unidades' que transitan entre etapas de sue\~no; se comparan grupalmente las proporciones de 
\'epocas PE --en cada canal-- durante sue\~no MOR y NMOR, usando la prueba U de Mann-Whithney;
en la figura \ref{comparacion_verde} se han representado gr\'aficamente estas diferecias.
Se encontr\'o que hay diferencias significativas ($\alpha<0.1$) para el grupo Control en los 
canales C3, C4, F7, F8, FP1, FP2, O2, P4, LOG y ROG, mientras que en el grupo PDC s\'olo se
observaron diferencias en LOG y ROG.
Descartando los canales LOG y ROG, ya que no son parte del EEG, las diferencias encontradas 
pueden ser relevantes fisiol\'ogicamente, ya que abarcan gran parte de los l\'obulos frontal y 
parietal, y parte de la regi\'on occipital-parietal derecha; en la figura \ref{cabecita} se
indican esquem\'aticamente estas regiones.

\begin{figure}
\centering
\subfloat[Comparaci\'on para el grupo control]{
\includegraphics[width=0.95\linewidth]
{./new170424/comp_etapas_gpos_NORMALMOR_vs_NMOR.pdf} 
}\\
\subfloat[Comparaci\'on para el grupo PDC]{
\includegraphics[width=0.95\linewidth]
{./new170424/comp_etapas_gpos_PDCMOR_vs_NMOR.pdf} 
}\\
\subfloat[Comparaci\'on de los p-valores para aceptar diferencias]{
\includegraphics[width=0.95\linewidth]
{./new170424/Comparacion_pvals_gpos_MOR_vs_NMOR.pdf} 
}\\
\caption{Comparaci\'on sobre las proporciones de \'epocas PE entre las etapas de sue\~no MOR
(verde) y NMOR (negro), para ambos grupos por separado. 
%Se han graficado las proporciones de PE en todos los sujetos de cada grupo, para todo el sue\~no y 
%la etapa MOR.
Se grafica el promedio grupal $\pm$
1 desviac\'on est\'andar $^{\nicefrac{3}{2}}$, como visualizaci\'on aproximada de la varianza.}
\label{comparacion_verde}
\end{figure}

\begin{figure}
\centering
\includegraphics[width=0.4\linewidth]
{cabecita.pdf} 
\caption{Representaci\'on esquem\'atica de los sitios donde se encontraron diferencias 
significaticas en la comparaci\'on entre el porcentaje de \'epocas PE durante sue\~no MOR y NMOR,
para el grupo Control (ver texto)}
\label{cabecita}
\end{figure}

\section{Patrones visuales}

%Se encontrar\'o una serie de patrones visuales que surgen cuando los datos obtenidos son 
%graficados tomando en cuenta el tiempo, y que parecen contener informaci\'on relevante respecto a 
%la aparici\'on del sue\~no MOR.
Como un an\'alisis exploratorio, buscando explicar la variabilidad entre los resultados, se 
graficaron los resultados obtenidos con el test PSR de la siguiente manera: se colocan en 
l\'inea horizontal una serie de cuadros, uno por cada \'epoca analizada seg\'un fue clasificada 
(blanco: PE, negro: no-estacionario), y posteriormente se juntaron verticalmente las l\'ineas
correspondientes a los diferentes canales; en la figura \ref{ejemplo_graf} se muestra un ejemplo de
ello, mientras que en el anexo se muestran los gr\'aficos construidos para cada uno de los sujetos. 

Al construir estos gr\'aficos, se hacen presentes 'bloques' de \'epocas que en su mayor\'ia son
PE --similarmente con \'epocas no-estacionarias. Ha parecido conveniente reportar este hallazgo
ya que los patrones son consistentes en todos los sujetos, y porque parece que estos 'bloques'
aparecen asociados al sue\~no MOR en cierto orden (ilustrado en la figura \ref{patroncito}):
\begin{itemize}
\item Bloque abundante en \'epocas PE, visualmente oscuro
\item Bloque abundante en \'epocas no-estacionarias, visualmente claro
\item Secci\'on que contiene el sue\~o MOR, los canales LOG y ROG muestran son visualmente m\'as
abundante en \'epocas no-estacionarias en esta zona del tiempo
\end{itemize}

%@{.}

\begin{figure}
\includegraphics[width=\textwidth]
{./g170413/MJNNVIGILOS_est.png}
\caption{Disposici\'on gr\'afica para los resultados del test PSR en el sujeto MJH. En el eje 
horizontal se muestra el tiempo desde el inicio de registro, mientras que en el eje vertical se 
encuentra el nombre del canal. 
Se han resaltado con color verde las \'epocas clasificadas como de sue\~no MOR.
}
\label{ejemplo_graf}
\end{figure}

\begin{figure}
\includegraphics[width=\textwidth]
%{./complementario170409/patrones_MJH.png}
{./graphs170427/zoom_MJH.pdf}
\caption{Se ha resaltado el patr\'on visual que, se propone, est\'a asociado con la aparici\'on del
sue\~no MOR: un bloque de \'epocas PE (rojo), un bloque de \'epocas no-estacionarias (azul) y un 
bloque que contiene al sue\~no MOR.}
\label{patroncito}
\end{figure}

%%%%%%%%%%%%%%%%%%%%%%%%%%%%%%%%%%%%%%%%%%%%%%%%%%%%%%%%%%%%%%%%%%%%%%%%%%%%%%%%%%%%%%%%%%%%%%%%%%%
%%%%%%%%%%%%%%%%%%%%%%%%%%%%%%%%%%%%%%%%%%%%%%%%%%%%%%%%%%%%%%%%%%%%%%%%%%%%%%%%%%%%%%%%%%%%%%%%%%%
%%%%%%%%%%%%%%%%%%%%%%%%%%%%%%%%%%%%%%%%%%%%%%%%%%%%%%%%%%%%%%%%%%%%%%%%%%%%%%%%%%%%%%%%%%%%%%%%%%%
%%%%%%%%%%%%%%%%%%%%%%%%%%%%%%%%%%%%%%%%%%%%%%%%%%%%%%%%%%%%%%%%%%%%%%%%%%%%%%%%%%%%%%%%%%%%%%%%%%%

\section{Discusi\'on}

Como se mencion\'o en la secci\'on de hip\'otesis, este trabajo pare del supuesto en que los
sujetos con PDC presentan con mayor probabilidad estacionariedad d\'ebil en sus registros de EEG.
Se ha aportado evidencia para afirmar que no hay cambios significativos en la porci\'on de tiempo 
durante la cual el registro de PSG se comporta de manera 'simple', al comparar sujetos cntrol y con
PDC. Esto puede interpretarse como que --quiz\'a-- los mecanismos afectados durante el PDC no 
provocan que la se\~nal se vuelva m\'as 'simple' en el sentido de volverse estacionaria.

Cabe un comentario sobre c\'omo la evidencia presentada exhibe al PSG como un conjunto de se\~nales 
mayoritariamente no-estacionarias, y que se comportan como estacionarias por una porci\'on m\'as
bien peque\~na del sue~no nocturno; luego entonces, no es adecuado analizar este tipo de se\~nales 
con m\'etodos que supongan estacionariedad --por ejemplo, an\'alisis cl\'asico de Fourier. 
M\'as a\'un este comentario se acent\'ua en individuos con PDC.

\subsection{La inclusi\'on de sujetos}

Durante el trabajo se menciona constantemente a tres sujetos (FGH, MGG, EMT) cuyos registros de PSG 
fueron analizados pero que no son considerados estad\'isticamente; cada uno de ellos fue exclu\'ido 
del trabajo original \cite{VazquezTagle16} por diversos motivos, pero dieron su consentimiento 
informado para el registro de PSG, debido a lo cual se decidi\'o analizar el efecto de su 
inclusi\'on dentro de los estad\'isticas.

El caso m\'as notorio es el sujeto FGH, quien padece de par\'alisis facial, cataratas, y problemas 
no especificados en la hipotiroides y la columna. Seg\'un se reporta, el sujeto no inform\'o de 
estos padecimientos sino hasta despu\'es del registro de PSG, por lo que su exlusi\'on se efectu\'o 
a posteriori.

Dentro del marco de este trabajo, son destacableslas proporciones inusuales de \'epocas PE para 
este sujeto en los canales F4, F7, F8, FP1, FP2, FZ, tanto en sue\~no MOR como no-MOR; estas 
haciendo uso de la representaci\'on gr\'afica mencionada, la estructura de estos datos es m\'as
inusual a\'un (figura \ref{FGH_especial}).
Como comentario, un vistazo a estos resultados inusuales pudiera haber delatado las 
caracter\'isticas de este sujeto, si bien esta metodolog\'ia no se usa expl\'icitamente para tal 
fin.

\begin{figure}
\centering
\includegraphics[width=0.95\linewidth]
{./muypreeliminar170408/FGHSUE_est.png} 
\caption{Compilado gr\'afico para el sujeto FGH; n\'otese la }
\label{FGH_especial}
\end{figure}

%%%%%%%%%%%%%%%%%%%%%%%%%%%%%%%%%%%%%%%%%%%%%%%%%%%%%%%%%%%%%%%%%%%%%%%%%%%%%%%%%%%%%%%%%%%%%%%%%%%
%%%%%%%%%%%%%%%%%%%%%%%%%%%%%%%%%%%%%%%%%%%%%%%%%%%%%%%%%%%%%%%%%%%%%%%%%%%%%%%%%%%%%%%%%%%%%%%%%%%

\subsection{Efecto del tama\~no de las \'epoca}

El uso de \'epocas de 30 segundos est\'a motivado por las recomendaciones de la AAMS para 
clasificar, de manera estandarizada, las etapas de sue\~no a partir de registros de PSG 
\cite{AASM07}. 
No se discutir\'an en este trabajo motivaciones o evidencia para usar esta longitud de \'epoca en 
particular, ni para el caso contrario, sino que se acepta por fines de comparabilidad. 
Sin embargo, debido a un problema t\'ecnico, en alg\'un momento de este trabajo se usaron los 
registros de PSG organizados en \'epocas de 10 segundos de duraci\'on; se realizaron los an\'alisis 
descritos usando esta segmentaci\'on mixta (algunos sujetos con \'epocas de 10 s, otros con 
\'epocas de 30 s) y se obtuvieron resultados seg\'un los cuales no hay diferencias significativas 
en ninguno de los an\'alisis. 
Por otro lado, la representaci\'on gr\'afica construida a partir de los mismos datos, organizados
en \'epocas de 10 s, cambia sustancialmente (ver figura \ref{comp_VCR}).

El hecho de que los resultados fueran afectados de manera contundente por la forma en que se 
organizan los datos, sugiere que ser\'a provechoso prestar mayor atenci\'on a la naturaleza de las 
caracter\'isticas estudiadas y su posible interpretaci\'on en la fisiolog\'ia.

%\begin{figure}
%\centering
%\subfloat[Comparaci\'on entre \'epocas MOR (fase R)]{
%\includegraphics[width=0.95\linewidth]
%{./material170331/Comparacion_gpos_MOR.pdf} 
%}\\
%\subfloat[Comparaci\'on entre \'epocas no-MOR (fases W y N)]{
%\includegraphics[width=0.95\linewidth]
%{./material170331/Comparacion_gpos_NMOR.pdf} 
%}\\
%\caption{Comparaci\'on sobre las proporciones de \'epocas PE entre los grupos, para diferentes
%etapas de sue\~no; se han usado los datos calculados con la segmentaci\'on de los
%registros en \'epocas de 30s y 10 s replicando un detalle t\'ecnico.}
%\label{comparacion_graf_mixto}
%\end{figure}

\begin{figure}
\centering
\subfloat[Usando \'epocas de 10 s]{
\includegraphics[width=0.95\linewidth]
{./grafiquitos170404/VCNNS1_est.png} 
}\\
\subfloat[Usando \'epocas de 30 s]{
\includegraphics[width=0.95\linewidth]
{./muypreeliminar170408/VCNNS1_est.png} 
}\\
\caption{Compilaci\'on gr\'afica de las \'epocas clasificadas como PE, distribuidas en el tiempo
para cada uno de los canales. El registro corresponde al sujeto VCR dado que su registro fuera
segmentado de dos maneras difrentes, variando la duraci\'on de las \'epocas.}
\label{comp_VCR}
\end{figure}

Se propone que los registros de PSG tienen una propiedad referida como 'estacionariedad local'
(definici\'on \ref{est_local}), concepto introducido por Dahlhaus \cite{Dahlhaus97}.
A grosso modo, un proceso localmente estacionario es aqu\'el cuya SDF --que puede depender del 
tiempo-- puede ser aproximada 'a trozos': usando SDF's correspondientes a procesos que poseen una 
representaci\'on espectral de Cram\'er, que est\'an definidos para el intervalo de tiempo $[0,1]$ 
y que est\'an 'correctamente ensamblados'.

\begin{defn}[Estacionariedad local]
Una secuencia de procesos estoc\'asticos de media cero, 
$\{ X_{t,N} \}$ con $t = 1, 2, \dots, N$, se dicen localmente estacionarios en el 
tiempo $t_0 \in [0,1]$ si existe una representaci\'on del tipo
\begin{equation*}
X_{t,N} = \intPI \widetilde{A_{t,N}}(\omega) e^{i \omega t} dZ(\omega)
\end{equation*}
donde se satisface que:
\begin{itemize}
\item $\{ Z(\omega) \}$ es un proceso de media cero con incrementos ortogonales, es decir
\begin{equation*}
\Cov{dZ(\omega_1),dZ(\omega_2)} =
\begin{cases}
d\omega &\text{, } \omega_1 = \omega_1 \\
0 &\text{, otro caso}
\end{cases}
\end{equation*}
\item Existe una constante $K$ y una funci\'on suave 
$A: [0,1]\times [-\pi,\pi] \rightarrow \mathbb{C}$, 
con $A(t_0,\omega) = \overline{A(t_0,-\omega)}$, tal que para toda $N$
\begin{equation*}
\sup_{\nicefrac{t}{N}\in \varepsilon_N(t_0)} 
\sup_{\omega} \abso{ \widetilde{A_{t,N}}(\omega) - A(t_0,\omega) }
\leq \frac{K}{N}
\end{equation*}
con $\varepsilon_N(t_0)$ es una vecindad de $t_0$ 
tal que $\abso{\abso{\nicefrac{t}{N}-t_0}} = O(N^{-\alpha})$,
%cuyo tama\~no es del orden $O(N^{-\alpha})$,
con $0\leq \alpha < 1$
\item $A(t_0,\omega)$ es continuo en $\{ t_0 \} \times [-\pi,\pi]$
\end{itemize}
\label{est_local}
\end{defn}

Se propone que los registros de PSG se comportan como procesos localmente estacionarios; m\'as 
a\'un, esta caracter\'istica podr\'ia cambiar en adultos mayores con y sin PDC. 
Una motivaci\'on fisiol\'ogica para la hip\'otesis anterior es el contenido de los registros de
PSG: un conjunto descoordinado y homog\'eneo de ondas cerebrales, complejos K y husos de sue\~no.
Si bien esta composici\'on sugiere que la no-estacionariedad es la opci\'on m\'as obvia, el
an\'alisis llevado a cabo revela que el contenido de estos eventos no es homog\'eneo durante el
sue\~no; m\'as a\'un, mientras m\'as peque\~no sea el intervalo de tiemo observado, es m\'as
posible encontrar zonas de composici\'on m\'as o menos homog\'enea que puedan ser clasificadas
como PE.
Esta hip\'otesis explicar\'ia el cambio observado al cambiar el tama\~no de la \'epoca; de manera
arriesgada, se podr\'ia concluir que, entre los individuos con PDC, la homogeneidad del PSG es muy
similar durante MOR y NMOR.
Sin embargo, para poder convertir esta hip\'otesis en una conclusi\'on aut\'entica, faltar\'ia
m\'as investigaci\'on al respecto --en particular, una prueba para detectar estacionariedad local.

%La hip\'otesis de la estacionariedad local trae a la discusi\'on un detalle muy particular: tal 
%cual est\'a escrita, esta definici\'on carece de utilidad pr\'actica para ser detectada en una 
%serie de tiempo dada. El enfoque m\'as intuitivo es que como un proceso localmente estacionario 
%puede ser aproximado como 'uni\'on' de procesos d\'ebilmente estacionarios (equivalentemente, 
%como una sucesi\'on de procesos 'd\'ebilmente estacionarios a trozos'), tiene sentido intentar 
%construir esos procesos; sin embargo, el verdadero problema es garantizar la convergencia de 
%estas aproximaciones. En este trabajo no se ha investigado suficiente al respecto, pero parece 
%una alternativa plausible para explicar los fen\'omenos hallados.

%Con respecto a la hip\'otesis de los eventos, bien puede tomarse un enfoque de reconocimiento de
%patrones, ''contando'' las incidencias de eventos y revisando si se relaciona con la
%clasificaci\'on como PE. Ese enfoque va m\'as all\'a de los objetivos de este trabajo.

%%%%%%%%%%%%%%%%%%%%%%%%%%%%%%%%%%%%%%%%%%%%%%%%%%%%%%%%%%%%%%%%%%%%%%%%%%%%%%%%%%%%%%%%%%%%%%%%%%%
%%%%%%%%%%%%%%%%%%%%%%%%%%%%%%%%%%%%%%%%%%%%%%%%%%%%%%%%%%%%%%%%%%%%%%%%%%%%%%%%%%%%%%%%%%%%%%%%%%%

\section{Conclusiones}

Se aportan evidencias sobre que la presencia proporcional de estacionariedad d\'ebil en registros 
de PSG para adultos mayores, segmentado en \'epocas de 30 segundos, 
presenta diferencias significativas al ser medida durante el sue\~no MOR y el resto del sue\~no
nocturno. Estas diferencia se pudieron observar, de manera grupal, 
en sujetos control para los canales 
C3, C4, F7, F8, FP1, FP2, O2, P4, LOG, ROG; en sujetos con PDC s\'olo se detectaron estas 
diferencias para los canales LOG y ROG.
%sujetos con y sin PDC diagnosticado.
%Luego entonces, esta caracter\'istica no es un indicador fiable 
Las diferencias grupales no se expresan claramente al considerar individualmente
a los sujetos excepto en los canales LOG y ROG, siendo consistente este dato con la 
caracterizaci\'on del sue\~no MOR: movimientos oculares r\'apidos y aton\'ia muscular.
Se concluye que esta caracter\'istica no es un indicador claro para ser usado
en el diagn\'ostico de PDC.

Los resultados encontrados sugieren, en cambio, que es --en principio-- posible interpretar los
cambios neurofisiol\'ogicos durante el deterioro cognitivo como un cambio en las 
en el ''mecanismo'' que genera las se\~nales del EEG, y que este cambio es detectable en el 
sue\~no nocturno mientras el sujeto transita entre las diferentes etapas del sue\~no 
(en particular, cuando transita entre el sue\~no MOR y no-MOR).
%Este cambio hipot\'etico explicar\'ia las se\~nales cualitativamente diferentes, en el sentido
%de contener una ''menor'' cantidad de cambios estructurales y eventos; la escasa presencia de 
%estas caracter\'isticas conlleva a clasificar los segmentos de registro como posiblemente
%estacionarios.
Esta interpretaci\'on propuesta es consistente con [los resultados de Valeria].

%Adicionalmente los
En otro \'ambito, los
%el hallazgos incidental de 
patrones cualitativos en el tiempo --vistos al
mostrar gr\'aficamente la distribuci\'on de \'epocas PE-- coinciden parcialmente con las \'epocas
de sue\~no MOR, clasificadas por un experto.
Adicionalmente se han encontrado diferencias significativas entre la proporci\'on de \'epocas PE,
durante sue\~no MOR y el resto del sue\~no registrado, y que son consistentes para todos los
sujetos en los canales LOG y ROG --diferencias calculadas una vez desprovista de la estructura
en el tiempo. En conjunto, se propone que la representaci\'on gr\'afica pudiera ser usado
como auxiliar en la clasificaci\'on de segmentos de registro seg\'un la etapa de sue\~no.

%%%%%%%%%%%%%%%%%%%%%%%%%%%%%%%%%%%%%%%%%%%%%%%%%%%%%%%%%%%%%%%%%%%%%%%%%%%%%%%%%%%%%%%%%%%%%%%%%%%
%%%%%%%%%%%%%%%%%%%%%%%%%%%%%%%%%%%%%%%%%%%%%%%%%%%%%%%%%%%%%%%%%%%%%%%%%%%%%%%%%%%%%%%%%%%%%%%%%%%

\section{Trabajo a futuro}

Como se mencion\'o anteriormente, los patrones visuales en la representaci\'on gr\'afica pueden 
tener un uso como caracter\'isticas auxiliares para la detecci\'on semi-autom\'atica de \'epocas 
MOR en registros de PSG.
Cabe recordar el caso de los sujetos exclu\'idos del estdudio, para los caules estos patrones
parecen no cumplirse; hace falta m\'as indagaci\'on al respecto. 

As\'i tambi\'en se mantiene a nivel de 'posiblidad' varias de las conlcusiones mencionadas, en 
cuanto a que la cantidad de sujetos analizados en este trabajo es relativamente baja. Dada la
automatizaci\'on adecuada del proceso, y un informe debidamente entregado a los sujetos respecto a
los datos obtenidos gracas a su apoyo, se espera poder aplicar estos an\'alisis a un p\'ublico 
mayor.

%%%%%%%%%%%%%%%%%%%%%%%%%%%%%%%%%%%%%%%%%%%%%%%%%%%%%%%%%%%%%%%%%%%%%%%%%%%%%%%%%%%%%%%%%%%%%%%%%%%
%%%%%%%%%%%%%%%%%%%%%%%%%%%%%%%%%%%%%%%%%%%%%%%%%%%%%%%%%%%%%%%%%%%%%%%%%%%%%%%%%%%%%%%%%%%%%%%%%%%
%%%%%%%%%%%%%%%%%%%%%%%%%%%%%%%%%%%%%%%%%%%%%%%%%%%%%%%%%%%%%%%%%%%%%%%%%%%%%%%%%%%%%%%%%%%%%%%%%%%
%%%%%%%%%%%%%%%%%%%%%%%%%%%%%%%%%%%%%%%%%%%%%%%%%%%%%%%%%%%%%%%%%%%%%%%%%%%%%%%%%%%%%%%%%%%%%%%%%%%

\appendix

\chapter{Tablas y gr\'aficos}

En este ap\'endice se incluyen las gr\'aficas y tablas obtenidas durante el trabajo; todos ellos 
son referidos en la secci\'on de Resultados, pero son presentados como ap\'endice a fin de resaltar 
en el texto las conclusiones obtenidas.

Se analizaron los registros de PSG correspondientes a 12 adultos mayores previamente diagnosticados
a trav\'es de varias pruebas neuropsicol\'ogicas: 5 se presentaron como Control, 4 se 
diagnosticaron con PDC y 3 fueron retirados a posetiori.
Los registros fueron segmentados en ''\'epocas'' de 30 segundos y fueron clasificados como sue\~no 
MOR (fase R) o NMOR (fases N y W) siguiendo las reglas de la AASM\cite{AASM07}; adicionalmente, se 
aplic\'o el test PSR a cada una de estas \'epocas para clasificarlas como posiblemente estacionarias 
(PE) si no era posible rechazar la hip\'otesis de estacionariedad ($\alpha < 0.05$).

En las primeras dos tablas se muestra el n\'umero total de \'epocas clasificadas como PE para cada 
sujeto y cada canal --y en total-- para las diferentes etapas de sue\~no. En las siguientes dos 
tablas se calcula la misma informaci\'on pero como proporciones, a modo de normalizaci\'on entre 
los diferentes sujetos. Se muestran los promedios y varianzas por grupo.

Posteriormente se exhiben los gr\'aficos mencionados en la parte de resultados y que exhiben una 
distribuci\'on temporal y pseudo-espacial de las ocurrencia de \'epocas PSG dentro de los registros 
de cada paciente. M\'as adelante se presentan los mismos gr\'aficos por segunda vez, remarcando los 
'patrones visuales' mencionados en la discusi\'on.

%%%%%%%%%%%%%%%%%%%%%%%%%%%%%%%%%%%%%%%%%%%%%%%%%

\begin{SidewaysFigure}
\centering
\begin{tabular}{c|ccccc|cccc|ccc}
& VCR & MJH & JAE & GHA & MFGR
& CLO & RLO & RRU & JGZ
& FGH & MGG & EMT \\
\hline
C3&6&18&10&1&12&6&35&16&1&2&28&22 \\
C4&7&16&4&2&10&4&40&5&0&1&23&26 \\
CZ&2&16&13&2&8&5&22&4&1&1&13&19 \\
F3&5&23&10&0&3&7&43&3&3&6&14&20 \\
F4&11&23&5&1&1&6&36&5&0&0&4&24 \\
F7&5&15&2&0&4&1&18&0&0&0&2&24 \\
F8&4&11&6&1&3&4&23&1&0&0&2&20 \\
FP1&2&7&1&0&1&0&0&1&0&22&0&22 \\
FP2&1&6&3&0&2&1&15&1&0&0&1&18 \\
FZ&11&18&19&0&6&7&38&2&2&0&20&23 \\
O1&10&20&5&3&23&2&25&9&2&5&18&19 \\
O2&13&23&3&3&21&3&34&9&1&1&12&16 \\
P3&6&17&2&2&26&5&33&8&0&1&24&17 \\
P4&4&19&4&5&18&4&27&5&1&4&15&21 \\
PZ&4&15&5&3&22&4&32&4&0&1&8&20 \\
T3&10&29&1&8&26&10&34&4&0&2&29&31 \\
T4&12&20&2&3&21&3&35&6&1&0&10&17 \\
T5&10&26&0&3&27&5&34&5&2&2&31&19 \\
T6&15&18&3&15&20&3&24&4&2&0&9&19 \\
LOG&6&20&8&0&9&5&11&2&0&1&8&30 \\
ROG&6&21&17&2&11&9&7&4&1&0&19&33 \\
EMG&14&11&0&0&17&14&16&4&0&0&3&7 \\
\hline
Total&73&127&171&55&95&132&99&38&33&22&166&47
\end{tabular}
\caption{Total de \'epocas PE clasificadas como sue\~no MOR 
(fase R) para cada
canal. %Se incluyen las medias y desviaciones est\'andar estimadas para los grupos 
%Control (izquierda) y PDC (centro).
}
\label{total_gpos_mor}
\end{SidewaysFigure}

\begin{SidewaysFigure}
\centering
\begin{tabular}{c|ccccc|cccc|ccc}
& VCR & MJH & JAE & GHA & MFGR
& CLO & RLO & RRU & JGZ
& FGH & MGG & EMT \\
\hline
C3&187&135&100&175&112&55&153&76&56&16&201&478 \\
C4&168&129&89&156&87&36&135&94&47&7&207&598 \\
CZ&167&131&88&107&77&54&145&69&62&8&180&518 \\
F3&168&134&83&150&73&57&175&79&68&107&143&331 \\
F4&180&132&55&146&24&41&135&80&49&0&137&549 \\
F7&158&137&77&213&87&45&112&68&58&0&152&262 \\
F8&157&123&30&168&36&41&96&86&48&0&128&574 \\
FP1&163&75&23&128&65&34&0&71&44&381&169&518 \\
FP2&156&82&44&116&21&33&99&26&44&0&146&449 \\
FZ&170&134&78&156&51&55&163&91&65&0&177&533 \\
O1&202&174&51&295&175&48&150&92&96&20&140&675 \\
O2&166&165&63&247&173&32&136&70&106&22&161&573 \\
P3&175&122&53&288&132&72&147&108&95&29&212&490 \\
P4&180&136&108&252&140&56&135&110&73&18&206&495 \\
PZ&156&131&90&216&112&57&167&112&59&15&177&497 \\
T3&181&140&52&230&171&81&112&80&102&27&115&603 \\
T4&181&121&35&182&128&26&110&112&87&10&122&531 \\
T5&218&146&16&265&199&78&137&104&61&19&208&621 \\
T6&218&148&49&194&181&38&118&98&84&18&209&558 \\
LOG&236&224&214&287&170&144&185&128&225&50&437&820 \\
ROG&236&205&212&334&159&126&179&110&225&67&455&873 \\
EMG&94&62&16&1&157&20&82&110&10&1&55&266 \\
\hline
Total&788&905&736&1038&727&812&747&376&1174&383&864&1376
\end{tabular}
\caption{Total  de \'epocas PE dentro del registro pero que no fueron clasificadas
como MOR (fases W y N) para cada
canal. %Se incluyen las medias y desviaciones est\'andar estimadas para los grupos 
%Control (izquierda) y PDC (centro).
}
\label{total_gpos_nmor}
\end{SidewaysFigure}

\begin{SidewaysFigure}
\centering
\begin{tabular}{c|ccccc|cccc|ccc}
& VCR & MJH & JAE & GHA & MFGR
& CLO & RLO & RRU & JGZ
& FGH & MGG & EMT \\
\hline
C3&193&153&110&176&124&61&188&92&57&18&229&500 \\
C4&175&145&93&158&97&40&175&99&47&8&230&624 \\
CZ&169&147&101&109&85&59&167&73&63&9&193&537 \\
F3&173&157&93&150&76&64&218&82&71&113&157&351 \\
F4&191&155&60&147&25&47&171&85&49&0&141&573 \\
F7&163&152&79&213&91&46&130&68&58&0&154&286 \\
F8&161&134&36&169&39&45&119&87&48&0&130&594 \\
FP1&165&82&24&128&66&34&0&72&44&403&169&540 \\
FP2&157&88&47&116&23&34&114&27&44&0&147&467 \\
FZ&181&152&97&156&57&62&201&93&67&0&197&556 \\
O1&212&194&56&298&198&50&175&101&98&25&158&694 \\
O2&179&188&66&250&194&35&170&79&107&23&173&589 \\
P3&181&139&55&290&158&77&180&116&95&30&236&507 \\
P4&184&155&112&257&158&60&162&115&74&22&221&516 \\
PZ&160&146&95&219&134&61&199&116&59&16&185&517 \\
T3&191&169&53&238&197&91&146&84&102&29&144&634 \\
T4&193&141&37&185&149&29&145&118&88&10&132&548 \\
T5&228&172&16&268&226&83&171&109&63&21&239&640 \\
T6&233&166&52&209&201&41&142&102&86&18&218&577 \\
LOG&242&244&222&287&179&149&196&130&225&51&445&850 \\
ROG&242&226&229&336&170&135&186&114&226&67&474&906 \\
EMG&108&73&16&1&174&34&98&114&10&1&58&273 \\
\hline
Total&861&1032&907&1093&822&944&846&414&1207&405&1030&1423
\end{tabular}
\caption{Total de \'epocas PE registradas
(todas las fases) para cada
canal. 
%Se incluyen las medias y desviaciones est\'andar estimadas para los grupos 
%Control (izquierda) y PDC (centro).
}
\label{total_gpos_total}
\end{SidewaysFigure}






%\begin{comment}
\begin{SidewaysFigure}
\centering
\begin{tabular}{c||ccccc|cc||cccc|cc||ccc}
& VCR & MJH & JAE & GHA & MFGR &$\widehat{\mu}$ & $\widehat{\sigma}$
& CLO & RLO & RRU & JGZ &$\widehat{\mu}$ & $\widehat{\sigma}$
& FGH & MGG & EMT \\
\hline
C3&0.082&0.142&0.058&0.018&0.126&0.085&0.050&0.045&0.354&0.421&0.030&0.213&0.204&0.091&0.169&0.468 \\
C4&0.096&0.126&0.023&0.036&0.105&0.077&0.045&0.030&0.404&0.132&0.000&0.141&0.184&0.045&0.139&0.553 \\
CZ&0.027&0.126&0.076&0.036&0.084&0.070&0.040&0.038&0.222&0.105&0.030&0.099&0.089&0.045&0.078&0.404 \\
F3&0.068&0.181&0.058&0.000&0.032&0.068&0.069&0.053&0.434&0.079&0.091&0.164&0.181&0.273&0.084&0.426 \\
F4&0.151&0.181&0.029&0.018&0.011&0.078&0.081&0.045&0.364&0.132&0.000&0.135&0.162&0.000&0.024&0.511 \\
F7&0.068&0.118&0.012&0.000&0.042&0.048&0.047&0.008&0.182&0.000&0.000&0.047&0.090&0.000&0.012&0.511 \\
F8&0.055&0.087&0.035&0.018&0.032&0.045&0.027&0.030&0.232&0.026&0.000&0.072&0.108&0.000&0.012&0.426 \\
FP1&0.027&0.055&0.006&0.000&0.011&0.020&0.022&0.000&0.000&0.026&0.000&0.007&0.013&1.000&0.000&0.468 \\
FP2&0.014&0.047&0.018&0.000&0.021&0.020&0.017&0.008&0.152&0.026&0.000&0.046&0.071&0.000&0.006&0.383 \\
FZ&0.151&0.142&0.111&0.000&0.063&0.093&0.062&0.053&0.384&0.053&0.061&0.138&0.164&0.000&0.120&0.489 \\
O1&0.137&0.157&0.029&0.055&0.242&0.124&0.085&0.015&0.253&0.237&0.061&0.141&0.121&0.227&0.108&0.404 \\
O2&0.178&0.181&0.018&0.055&0.221&0.130&0.089&0.023&0.343&0.237&0.030&0.158&0.158&0.045&0.072&0.340 \\
P3&0.082&0.134&0.012&0.036&0.274&0.108&0.104&0.038&0.333&0.211&0.000&0.145&0.155&0.045&0.145&0.362 \\
P4&0.055&0.150&0.023&0.091&0.189&0.102&0.068&0.030&0.273&0.132&0.030&0.116&0.115&0.182&0.090&0.447 \\
PZ&0.055&0.118&0.029&0.055&0.232&0.098&0.082&0.030&0.323&0.105&0.000&0.115&0.146&0.045&0.048&0.426 \\
T3&0.137&0.228&0.006&0.145&0.274&0.158&0.103&0.076&0.343&0.105&0.000&0.131&0.148&0.091&0.175&0.660 \\
T4&0.164&0.157&0.012&0.055&0.221&0.122&0.086&0.023&0.354&0.158&0.030&0.141&0.155&0.000&0.060&0.362 \\
T5&0.137&0.205&0.000&0.055&0.284&0.136&0.114&0.038&0.343&0.132&0.061&0.143&0.139&0.091&0.187&0.404 \\
T6&0.205&0.142&0.018&0.273&0.211&0.170&0.097&0.023&0.242&0.105&0.061&0.108&0.096&0.000&0.054&0.404 \\
LOG&0.082&0.157&0.047&0.000&0.095&0.076&0.058&0.038&0.111&0.053&0.000&0.050&0.046&0.045&0.048&0.638 \\
ROG&0.082&0.165&0.099&0.036&0.116&0.100&0.047&0.068&0.071&0.105&0.030&0.069&0.031&0.000&0.114&0.702 \\
EMG&0.192&0.087&0.000&0.000&0.179&0.091&0.093&0.106&0.162&0.105&0.000&0.093&0.068&0.000&0.018&0.149
\end{tabular}
\caption{Proporci\'on estimada de \'epocas PE respecto al total de \'epocas MOR 
(fase R) para cada
canal. Se incluyen las medias y desviaciones est\'andar estimadas para los grupos 
Control (izquierda) y PDC (centro).}
\label{gpos_mor}
\end{SidewaysFigure}

\begin{SidewaysFigure}
\centering
\begin{tabular}{c||ccccc|cc||cccc|cc||ccc}
& VCR & MJH & JAE & GHA & MFGR &$\widehat{\mu}$ & $\widehat{\sigma}$
& CLO & RLO & RRU & JGZ &$\widehat{\mu}$ & $\widehat{\sigma}$
& FGH & MGG & EMT \\
\hline
C3&0.237&0.149&0.136&0.169&0.154&0.169&0.040&0.068&0.205&0.202&0.048&0.131&0.085&0.042&0.233&0.347 \\
C4&0.213&0.143&0.121&0.150&0.120&0.149&0.038&0.044&0.181&0.250&0.040&0.129&0.104&0.018&0.240&0.435 \\
CZ&0.212&0.145&0.120&0.103&0.106&0.137&0.045&0.067&0.194&0.184&0.053&0.124&0.075&0.021&0.208&0.376 \\
F3&0.213&0.148&0.113&0.145&0.100&0.144&0.044&0.070&0.234&0.210&0.058&0.143&0.092&0.279&0.166&0.241 \\
F4&0.228&0.146&0.075&0.141&0.033&0.125&0.075&0.050&0.181&0.213&0.042&0.121&0.088&0.000&0.159&0.399 \\
F7&0.201&0.151&0.105&0.205&0.120&0.156&0.046&0.055&0.150&0.181&0.049&0.109&0.066&0.000&0.176&0.190 \\
F8&0.199&0.136&0.041&0.162&0.050&0.117&0.070&0.050&0.129&0.229&0.041&0.112&0.087&0.000&0.148&0.417 \\
FP1&0.207&0.083&0.031&0.123&0.089&0.107&0.065&0.042&0.000&0.189&0.037&0.067&0.083&0.995&0.196&0.376 \\
FP2&0.198&0.091&0.060&0.112&0.029&0.098&0.064&0.041&0.133&0.069&0.037&0.070&0.044&0.000&0.169&0.326 \\
FZ&0.216&0.148&0.106&0.150&0.070&0.138&0.055&0.068&0.218&0.242&0.055&0.146&0.098&0.000&0.205&0.387 \\
O1&0.256&0.192&0.069&0.284&0.241&0.209&0.085&0.059&0.201&0.245&0.082&0.147&0.090&0.052&0.162&0.491 \\
O2&0.211&0.182&0.086&0.238&0.238&0.191&0.063&0.039&0.182&0.186&0.090&0.124&0.072&0.057&0.186&0.416 \\
P3&0.222&0.135&0.072&0.277&0.182&0.178&0.079&0.089&0.197&0.287&0.081&0.163&0.098&0.076&0.245&0.356 \\
P4&0.228&0.150&0.147&0.243&0.193&0.192&0.044&0.069&0.181&0.293&0.062&0.151&0.109&0.047&0.238&0.360 \\
PZ&0.198&0.145&0.122&0.208&0.154&0.165&0.036&0.070&0.224&0.298&0.050&0.160&0.120&0.039&0.205&0.361 \\
T3&0.230&0.155&0.071&0.222&0.235&0.182&0.070&0.100&0.150&0.213&0.087&0.137&0.057&0.070&0.133&0.438 \\
T4&0.230&0.134&0.048&0.175&0.176&0.152&0.068&0.032&0.147&0.298&0.074&0.138&0.117&0.026&0.141&0.386 \\
T5&0.277&0.161&0.022&0.255&0.274&0.198&0.109&0.096&0.183&0.277&0.052&0.152&0.099&0.050&0.241&0.451 \\
T6&0.277&0.164&0.067&0.187&0.249&0.189&0.082&0.047&0.158&0.261&0.072&0.134&0.097&0.047&0.242&0.406 \\
LOG&0.299&0.248&0.291&0.276&0.234&0.270&0.028&0.177&0.248&0.340&0.192&0.239&0.074&0.131&0.506&0.596 \\
ROG&0.299&0.227&0.288&0.322&0.219&0.271&0.046&0.155&0.240&0.293&0.192&0.220&0.060&0.175&0.527&0.634 \\
EMG&0.119&0.069&0.022&0.001&0.216&0.085&0.086&0.025&0.110&0.293&0.009&0.109&0.130&0.003&0.064&0.193
\end{tabular}
\caption{Proporci\'on estimada de \'epocas PE respecto al total de \'epocas no-MOR 
(fases W y N) para cada
canal. Se incluyen las medias y desviaciones est\'andar estimadas para los grupos 
Control (izquierda) y PDC (centro).}
\label{gpos_nmor}
\end{SidewaysFigure}

\begin{SidewaysFigure}
\centering
\begin{tabular}{c||ccccc|cc||cccc|cc||ccc}
& VCR & MJH & JAE & GHA & MFGR &$\widehat{\mu}$ & $\widehat{\sigma}$
& CLO & RLO & RRU & JGZ &$\widehat{\mu}$ & $\widehat{\sigma}$
& FGH & MGG & EMT \\
\hline
C3&0.224&0.148&0.121&0.161&0.151&0.161&0.038&0.065&0.222&0.222&0.047&0.139&0.096&0.044&0.222&0.351 \\
C4&0.203&0.141&0.103&0.145&0.118&0.142&0.038&0.042&0.207&0.239&0.039&0.132&0.106&0.020&0.223&0.439 \\
CZ&0.196&0.142&0.111&0.100&0.103&0.131&0.040&0.063&0.197&0.176&0.052&0.122&0.075&0.022&0.187&0.377 \\
F3&0.201&0.152&0.103&0.137&0.092&0.137&0.043&0.068&0.258&0.198&0.059&0.146&0.098&0.279&0.152&0.247 \\
F4&0.222&0.150&0.066&0.134&0.030&0.121&0.075&0.050&0.202&0.205&0.041&0.124&0.092&0.000&0.137&0.403 \\
F7&0.189&0.147&0.087&0.195&0.111&0.146&0.047&0.049&0.154&0.164&0.048&0.104&0.064&0.000&0.150&0.201 \\
F8&0.187&0.130&0.040&0.155&0.047&0.112&0.065&0.048&0.141&0.210&0.040&0.110&0.081&0.000&0.126&0.417 \\
FP1&0.192&0.079&0.026&0.117&0.080&0.099&0.061&0.036&0.000&0.174&0.036&0.062&0.077&0.995&0.164&0.379 \\
FP2&0.182&0.085&0.052&0.106&0.028&0.091&0.059&0.036&0.135&0.065&0.036&0.068&0.046&0.000&0.143&0.328 \\
FZ&0.210&0.147&0.107&0.143&0.069&0.135&0.052&0.066&0.238&0.225&0.056&0.146&0.099&0.000&0.191&0.391 \\
O1&0.246&0.188&0.062&0.273&0.241&0.202&0.084&0.053&0.207&0.244&0.081&0.146&0.093&0.062&0.153&0.488 \\
O2&0.208&0.182&0.073&0.229&0.236&0.186&0.066&0.037&0.201&0.191&0.089&0.129&0.080&0.057&0.168&0.414 \\
P3&0.210&0.135&0.061&0.265&0.192&0.173&0.078&0.082&0.213&0.280&0.079&0.163&0.100&0.074&0.229&0.356 \\
P4&0.214&0.150&0.123&0.235&0.192&0.183&0.046&0.064&0.191&0.278&0.061&0.149&0.105&0.054&0.215&0.363 \\
PZ&0.186&0.141&0.105&0.200&0.163&0.159&0.038&0.065&0.235&0.280&0.049&0.157&0.118&0.040&0.180&0.363 \\
T3&0.222&0.164&0.058&0.218&0.240&0.180&0.074&0.096&0.173&0.203&0.085&0.139&0.058&0.072&0.140&0.446 \\
T4&0.224&0.137&0.041&0.169&0.181&0.150&0.069&0.031&0.171&0.285&0.073&0.140&0.113&0.025&0.128&0.385 \\
T5&0.265&0.167&0.018&0.245&0.275&0.194&0.107&0.088&0.202&0.263&0.052&0.151&0.098&0.052&0.232&0.450 \\
T6&0.271&0.161&0.057&0.191&0.245&0.185&0.083&0.043&0.168&0.246&0.071&0.132&0.093&0.044&0.212&0.405 \\
LOG&0.281&0.236&0.245&0.263&0.218&0.249&0.024&0.158&0.232&0.314&0.186&0.222&0.068&0.126&0.432&0.597 \\
ROG&0.281&0.219&0.252&0.307&0.207&0.253&0.042&0.143&0.220&0.275&0.187&0.206&0.056&0.165&0.460&0.637 \\
EMG&0.125&0.071&0.018&0.001&0.212&0.085&0.086&0.036&0.116&0.275&0.008&0.109&0.120&0.002&0.056&0.192
\end{tabular}
\caption{Proporci\'on estimada de \'epocas PE respecto al total de \'epocas registradas
(todas las fases) para cada
canal. Se incluyen las medias y desviaciones est\'andar estimadas para los grupos 
Control (izquierda) y PDC (centro).}
\label{gpos_total}
\end{SidewaysFigure}
%\end{comment}

%%%%%%%%%%%%%%%%%%%%%%%%%%%%%%%%%%%%%%%%%%%%%%%%%%%%%%%%%%%%%%%%%%%%%%%%%%%%%%%%%%%%%%%%%%%%%%%%%%%
%%%%%%%%%%%%%%%%%%%%%%%%%%%%%%%%%%%%%%%%%%%%%%%%%%%%%%%%%%%%%%%%%%%%%%%%%%%%%%%%%%%%%%%%%%%%%%%%%%%

\begin{figure}
\centering
\includegraphics[width=0.9\linewidth]
%{./complementario170409/VCNNS1_est.png} 
{./g170413/VCNNS1_est.png} 
%\caption{Sujeto: VCR | Total \'epocas: 861 | \'Epocas MOR: 73
%%| Frecuencia de muestreo: 200 Hz
%}
\label{grf_VCR}
\end{figure}

%%%%%%%%%%%%%%%%%%%%%%%%%%%%%%%%%%%%%%%%%%%%%%%%%

\begin{figure}
\centering
\includegraphics[width=0.9\linewidth]
{./g170413/MJNNVIGILOS_est.png} 
%\caption{Sujeto: MJH | Total \'epocas: 1032 | \'Epocas MOR: 127
%%| Frecuencia de muestreo: 200 Hz
%}
\label{grf_MJH}
\end{figure}

%%%%%%%%%%%%%%%%%%%%%%%%%%%%%%%%%%%%%%%%%%%%%%%%%

\begin{figure}
\centering
\includegraphics[width=0.9\linewidth]
{./g170413/JANASUE_est.png} 
%\caption{Sujeto: JAE | Total \'epocas: 907 | \'Epocas MOR: 171
%%| Frecuencia de muestreo: 200 Hz
%}
\label{grf_JAE}
\end{figure}

%%%%%%%%%%%%%%%%%%%%%%%%%%%%%%%%%%%%%%%%%%%%%%%%%

\begin{figure}
\centering
\includegraphics[width=0.9\linewidth]
{./g170413/GH24031950SUENNO_est.png} 
%\caption{Sujeto: GHA | Total \'epocas: 1093 | \'Epocas MOR: 55
%%| Frecuencia de muestreo: 200 Hz
%}
\label{grf_GHA}
\end{figure}

%%%%%%%%%%%%%%%%%%%%%%%%%%%%%%%%%%%%%%%%%%%%%%%%%

\begin{figure}
\centering
\includegraphics[width=0.9\linewidth]
{./g170413/GURM251148SUE_est.png} 
%\caption{Sujeto: MFGR | Total \'epocas: 822 | \'Epocas MOR: 95
%%| Frecuencia de muestreo: 200 Hz
%}
\label{grf_MFGR}
\end{figure}

%%%%%%%%%%%%%%%%%%%%%%%%%%%%%%%%%%%%%%%%%%%%%%%%%

\begin{figure}
\centering
\includegraphics[width=0.9\linewidth]
{./g170413/CLMN10SUE_est.png} 
%\caption{Sujeto: CLO | Total \'epocas: 944 | \'Epocas MOR: 132
%%| Frecuencia de muestreo: 200 Hz
%}
\label{grf_CLO}
\end{figure}

%%%%%%%%%%%%%%%%%%%%%%%%%%%%%%%%%%%%%%%%%%%%%%%%%

\begin{figure}
\centering
\includegraphics[width=0.9\linewidth]
{./g170413/RLMN10SUE_est.png} 
%\caption{Sujeto: RLO | Total \'epocas: 846 | \'Epocas MOR: 99
%%| Frecuencia de muestreo: 200 Hz
%}
\label{grf_RLO}
\end{figure}

%%%%%%%%%%%%%%%%%%%%%%%%%%%%%%%%%%%%%%%%%%%%%%%%%

\begin{figure}
\centering
\includegraphics[width=0.9\linewidth]
{./g170413/RRMNS_est.png} 
%\caption{Sujeto: RRU | Total \'epocas: 414 | \'Epocas MOR: 38
%%| Frecuencia de muestreo: 200 Hz
%}
\label{grf_RRU}
\end{figure}

%%%%%%%%%%%%%%%%%%%%%%%%%%%%%%%%%%%%%%%%%%%%%%%%%

\begin{figure}
\centering
\includegraphics[width=0.9\linewidth]
{./g170413/JGMN6SUE_est.png} 
%\caption{Sujeto: JGZ | Total \'epocas: 1207 | \'Epocas MOR: 33
%%| Frecuencia de muestreo: 200 Hz
%}
\label{grf_JGZ}
\end{figure}

%%%%%%%%%%%%%%%%%%%%%%%%%%%%%%%%%%%%%%%%%%%%%%%%%

\begin{figure}
\centering
\includegraphics[width=0.9\linewidth]
{./g170413/FGHSUE_est.png} 
%\caption{Sujeto: FGH | Total \'epocas: 405 | \'Epocas MOR: 22
%%| Frecuencia de muestreo: 200 Hz
%}
\label{grf_FGH}
\end{figure}

%%%%%%%%%%%%%%%%%%%%%%%%%%%%%%%%%%%%%%%%%%%%%%%%%

\begin{figure}
\centering
\includegraphics[width=0.9\linewidth]
{./g170413/MGNA5SUE_est.png} 
%\caption{Sujeto: MGG | Total \'epocas: 1030 | \'Epocas MOR: 166
%%| Frecuencia de muestreo: 200 Hz
%}
\label{grf_MGG}
\end{figure}

%%%%%%%%%%%%%%%%%%%%%%%%%%%%%%%%%%%%%%%%%%%%%%%%%

\begin{figure}
\centering
\includegraphics[width=0.9\linewidth]
{./g170413/EMNNS_est.png} 
%\caption{Sujeto: EMT | Total \'epocas: 1423 | \'Epocas MOR: 47
%%| Frecuencia de muestreo: 200 Hz
%}
\label{grf_EMT}
\end{figure}

%%%%%%%%%%%%%%%%%%%%%%%%%%%%%%%%%%%%
%%%%%%%%%%%%%%%%%%%%%%%%%%%%%%%%%%%%

%\begin{figure}
%\centering
%\includegraphics[width=0.9\linewidth]
%{./g170413/azul_VCR.png} 
%\label{azul_VCR}
%\end{figure}
%
%%%%%%%%%%%%%%%%%%%%%%%%%%%%%%%%%%%%%
%
%\begin{figure}
%\centering
%\includegraphics[width=0.9\linewidth]
%{./g170413/azul_MJH.png} 
%\label{azul_MJH}
%\end{figure}
%
%%%%%%%%%%%%%%%%%%%%%%%%%%%%%%%%%%%%%
%
%\begin{figure}
%\centering
%\includegraphics[width=0.9\linewidth]
%{./g170413/azul_JAE.png} 
%\label{azul_JAE}
%\end{figure}
%
%%%%%%%%%%%%%%%%%%%%%%%%%%%%%%%%%%%%%
%
%\begin{figure}
%\centering
%\includegraphics[width=0.9\linewidth]
%{./g170413/azul_GHA.png} 
%\label{azul_GHA}
%\end{figure}

%%%%%%%%%%%%%%%%%%%%%%%%%%%%%%%%%%%%%%%%%%%%%%%%%%%%%%%%%%%%%%%%%%%%%%%%%%%%%%%%%%%%%%%%%%%%%%%%%%%
%%%%%%%%%%%%%%%%%%%%%%%%%%%%%%%%%%%%%%%%%%%%%%%%%%%%%%%%%%%%%%%%%%%%%%%%%%%%%%%%%%%%%%%%%%%%%%%%%%%

%%%%%%%%%%%%%%%%%%%%%%%%%%%%%%%%%%%%%%%%%%%%%%%%%%%%%%%%%%%%%%%%%%%%%%%%%%%%%%%%%%%%%%%%%%%%%%%%%%%
%%%%%%%%%%%%%%%%%%%%%%%%%%%%%%%%%%%%%%%%%%%%%%%%%%%%%%%%%%%%%%%%%%%%%%%%%%%%%%%%%%%%%%%%%%%%%%%%%%%

%\phantomsection
%
%\addcontentsline{toc}{chapter*}{Bibliograf\'ia}

\bibliography{referencias_estacionariedad,referencias_fisiologia,referencias_otros,referencias_mixto}{}
%\bibliographystyle{apalike-es}
\bibliographystyle{abbrv}

%%%%%%%%%%%%%%%%%%%%%%%%%%%%%%%%%%%%%%%%%%%%%%%%%%%%%%%%%%%%%%%%%%%%%%%%%%%%%%%%%%%%%%%%%%%%%%%%%%%
%%%%%%%%%%%%%%%%%%%%%%%%%%%%%%%%%%%%%%%%%%%%%%%%%%%%%%%%%%%%%%%%%%%%%%%%%%%%%%%%%%%%%%%%%%%%%%%%%%%

\end{document}

%%%%%%%%%%%%%%%%%%%%%%%%%%%%%%%%%%%%%%%%%%%%%%%%%%%%%%%%%%%%%%%%%%%%%%%%%%%%%%%%%%%%%%%%%%%%%%%%%%%