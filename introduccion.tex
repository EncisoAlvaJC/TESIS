%%%%%%%%%%%%%%%%%%%%%%%%%%%%%%%%%%%%%%%%%%%%%%%%%%%%%%%%%%%%%%%%%%%%%%%%%%%%%%%%%%%%%%%%%%%%%%%%%%%
%%%%%%%%%%%%%%%%%%%%%%%%%%%%%%%%%%%%%%%%%%%%%%%%%%%%%%%%%%%%%%%%%%%%%%%%%%%%%%%%%%%%%%%%%%%%%%%%%%%
\chapter*{Introducci\'on}

A grosso modo:

El objetivo de este trabajo es explorar la hip\'otesis de estacionariedad en registros
polisomnogr\'aficos (EEG durante el sue\~no) en adultos mayores con Deterioro Cognitivo y de
un grupo control.

Se describen diferencias entre [lo que arrojan los an\'alisis para] registros de ambos grupos, 
que sugieren su posible utilizaci\'on 
como marcadores de uso cl\'inico.

El estudio y diagnóstico de una gran cantidad de enfermedades depende de nuestra habilidad para
registrar y analizar se\~nales electrofisiol\'ogicas. 

Se suele asumir que estas se\~nales son complejas: no lineales, no estacionarias y sin equilibrio 
por naturaleza. Pero usualmente no se comprueban formalmente estas propiedades.

Correlaci\'on inter-hemisf\'erica durante el sueño MOR del Adulto Mayor con Deterioro Cognitivo.

\begin{figure}[h]
\centering
\includegraphics[width=.8\linewidth]{graficaintro.pdf}
\end{figure}

Adaptado de V\'azquez-Tagle y colaboradores (2016)

%Visualmente, el sue\~no MOR se caracteriza por movimientos oculares r\'apidos, aton\'ia muscular y 
%una actividad electroencefalogr\'afica desincronizada \cite{RosalesLagarde09}.

%%%%%%%%%%%%%%%%%%%%%%%%%%%%%%%%%%%%%%%%%%%%%%%%%%%%%%%%%%%%%%%%%%%%%%%%%%%%%%%%%%%%%%%%%%%%%%%%%%%
%%%%%%%%%%%%%%%%%%%%%%%%%%%%%%%%%%%%%%%%%%%%%%%%%%%%%%%%%%%%%%%%%%%%%%%%%%%%%%%%%%%%%%%%%%%%%%%%%%%