\documentclass[10pt,a4paper]{article}
\usepackage[utf8]{inputenc}
\usepackage[english]{babel}
\usepackage{amsmath}
\usepackage{amsfonts}
\usepackage{amssymb}
\usepackage{graphicx}

\title{Pre-proyect}

\author{Enciso Alva}
\date{}

\begin{document}

\maketitle

\section{Abstract}

\section{Antecedents}

Dementia is a non-communicable disease associated with age, and Mild Cognitive Impairment (MCI) is considered an early stage of dementia in Older Adults. 
%    
There are a variety of tracers for MCI using polysomnographic (PSG) records, many of them derived from the power spectrum; among them, the importance of the sleep stage referred to as Rapid Eye Movement (REM) sleep stage is in here highlighted. 
%
In this study in particular, we considered PSG registers obtained with 19 electrodes for electrical brain activity (electroencephalography, EEG), 2 for eye movements (electrooculography, EOG) and 2 for muscle activity (electromyography, EMG). 
%
In this work, diagnostic markers of MCI were searched using the power spectrum homogeneity for the PSG registers, which is identified using weak stationarity tests. 
%
Stationarity is associated with the complexity of brain activity, but is usually discarded beacause of the complications implied for using it. 
%
PSG registers at REM sleep stage were fragmented, and the amount of such stationary fragments was used; statistical differences were found for this quantity on the EEG at both hemispheres at frontal regions, and also for the EOG, both between subjects with and without DCL and between different sleep stages. 
%
These results are consistent both with the characterisics of REM sleep and the function of frontal regions of the brain in decision making and memory consolidation. The results presented, using weak stationarity, confirm that this characteristic contains relevant information about sleep structure and its deterioration during MCI; a further study of the associated phenomena will make possible to use it as an effective tracer for MCI.

\section{Objectives}

\section{Goals}

\section{Cronogram}

\end{document}