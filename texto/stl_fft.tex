%%%%%%%%%%%%%%%%%%%%%%%%%%%%%%%%%%%%%%%%%%%%%%%%%%%%%%%%%%%%%%%%%%%%%%%%%%%%%%%%%%%%%%%%%%%%%%%%%%%
%%%%%%%%%%%%%%%%%%%%%%%%%%%%%%%%%%%%%%%%%%%%%%%%%%%%%%%%%%%%%%%%%%%%%%%%%%%%%%%%%%%%%%%%%%%%%%%%%%%
%%%%%%%%%%%%%%%%%%%%%%%%%%%%%%%%%%%%%%%%%%%%%%%%%%%%%%%%%%%%%%%%%%%%%%%%%%%%%%%%%%%%%%%%%%%%%%%%%%%
%%%%%%%%%%%%%%%%%%%%%%%%%%%%%%%%%%%%%%%%%%%%%%%%%%%%%%%%%%%%%%%%%%%%%%%%%%%%%%%%%%%%%%%%%%%%%%%%%%%
\chapter{Detalles}

Se exponen varios de los conceptos expuestos como preeliminares pero con las formalidades
pertinentes; cabe mencionar que todos estos resultados no son originales del presente trabajo,
razón por la cual fueron omitidos anteriormente.
%presentados como preeliminares, incluyendo demostraciones. 
%Esta informaci\'on fue cortada del 
%cuerpo principal del trabajo porque no son resultados originales, y si bien son sumamaente
%importantess e interesantes, se omitieron a favor de los resultados originales generados en el 
%trabajo.

%%%%%%%%%%%%%%%%%%%%%%%%%%%%%%%%%%%%%%%%%%%%%%%%%%%%%%%%%%%%%%%%%%%%%%%%%%%%%%%%%%%%%%%%%%%%%%%%%%%
%%%%%%%%%%%%%%%%%%%%%%%%%%%%%%%%%%%%%%%%%%%%%%%%%%%%%%%%%%%%%%%%%%%%%%%%%%%%%%%%%%%%%%%%%%%%%%%%%%%

\section{Variables aleatorias}

Un primer motivo para esta sección es enfatizar que formalmente una variable aleatoria se concibe 
no como un recuento de eventos sino como una función de medida. Esta segunda 
caracterización es relevante para varios resultados utilizados, por lo que
conviene no omitirla.

Antes de poder definir formalmente las variable aleatoria, debe definirse las medidas.

\begin{defn}[$\boldsymbol{\sigma}$-álgebra]
Sea $U$ un conjunto y $\mathcal{U}$ una colección de subconjuntos de $U$. Se dice que $\mathcal{U}$
es una $\sigma$-álgebra si comple que
\begin{itemize}
\item $U \in \mathcal{U}$
\item $A \in \mathcal{U}$ implica que $A^{C} \in \mathcal{U}$
\item Si $\{ A_n \}_{n\in \mathbb{N}}$ son conjuntos tales que $A_i \in \mathcal{U}$, entonces
$\displaystyle \cup_{n\in \mathbb{N}} A_n \in \mathcal{U}$
\end{itemize}
Donde $A^{C}$ es el complemento $\{ u \in U | u \notin A \} $
\end{defn}

Por simplicidad, en este trabajo sólo se usarán medidas para conjuntos de números reales derivadas 
de la $\sigma$-álgebra de Borel, que es definida como la $\sigma$-álgebra más pequeña que contiene a 
los intervalos abiertos abiertos\footnote{Si una $\sigma$-álgebra contiene a todos los
intervalos abiertos, entonces debe contener a todos los elementos de la $\sigma$-álgebra de Borel}.

\begin{defn}[Medida]
Sea $U$ un conjunto y $\mathcal{U}$ una $\sigma$-álgebra definida en $U$. Se dice que una función
$\mu : \mathcal{U} \rightarrow \R^{*}$ es una medida si cumple que
\begin{itemize}
\item $\mu(\emptyset) = 0$
\item $\mu(A) \geq 0$ para cualquier $A \in \mathcal{U}$
\item Si $\{ A_n \}_{n\in \mathbb{N}}$ son conjuntos disjuntos a pares y tales que 
$A_i \in \mathcal{U}$, entonces 
$\displaystyle \mu\left( \cup_{n\in \mathbb{N}} A_n \right) = \sum_{n\in \mathbb{N}} \mu(A_n)$
\end{itemize}
Donde $\R^{*}$ designa a los 'reales extendidos' $\R \cup \{-\infty,\infty \}$
\end{defn}

\begin{defn}[Medida de probabilidad en $\boldsymbol{\R}$]
Sea $\mathcal{B}$ la sigma álgebra de Borel definida para $\R$, se dice que una función
$P : \mathcal{B} \rightarrow [0.1]$ es una \textbf{medida de probabilidad} si cumple que
\begin{itemize}
\item $P(\emptyset) = 0$
\item $0 \leq P(A) \leq 1$ para cualquier $A \in \mathcal{B}$
\item Si $A, B \in \mathcal{B}$ y $A\cap B = \emptyset$, entonces $P(A \cup B) = P(A) + P(B)$ 
\item $P(\R) = 1$
\end{itemize}
\label{variable_aleatoria}
\end{defn}

%Cabe mencionar que cuando se usa una variable aleatoria para modelar un fenómeno, existe un paso
%intermedio en que los eventos relevantes se asocian con números reales

Una forma de entender mejor una variables aleatoria es a partir de su función de probabilidad
acumulada (FPA), que a su vez caracteriza a la variable: es equivalente referirse
a una variable aleatoria o a su FPA.

\begin{defn}[Función de Probabilidad Acumulada]
Sea 
\begin{equation*}
F_X (x) = P\left( (-\infty,x] \right)
\end{equation*}
\end{defn}

%%%%%%%%%%%%%%%%%%%%%%%%%%%%%%%%%%%%%%%%%%%%%%%%%%%%%%%%%%%%%%%%%%%%%%%%%%%%%%%%%%%%%%%%%%%%%%%%%%%
%%%%%%%%%%%%%%%%%%%%%%%%%%%%%%%%%%%%%%%%%%%%%%%%%%%%%%%%%%%%%%%%%%%%%%%%%%%%%%%%%%%%%%%%%%%%%%%%%%%

\section{Algunas consecuencias de la estacionariedad}

\begin{defn}[Estacionariedad de orden $\boldsymbol{m}$]
Un proceso estoc\'astico $\{ X(t) \}$ se dice estacionario de orden $m$ si, para cualquier conjunto 
de tiempos admisibles $t_1,t_2,\dots,t_n$ y cualquier $\tau \in \R$ se cumple que
\begin{equation*}
\E{ X^{m_1}(t_1)X^{m_2}(t_2)\cdots X^{m_n}(t_n) }
=
\E{ X^{m_1}(t_1+\tau)X^{m_2}(t_2+\tau)\cdots X^{m_n}(t_n+\tau) }
\end{equation*}
Para cualesquiera enteros $m_1,m_2,\dots,m_n$ tales que $m_1+m_2+\dots+m_n \leq m$
\label{est_orden_m}
\end{defn}

Para entender mejor la definici\'on \ref{est_orden_m} y sus limitaciones frente a la 
estacionariedad fuerte, consid\'erense tres procesos: $\{X(t)\}$ fuertemente estacionario, 
$\{Y_1(t)\}$ estacionario de orden 1, y $\{Y_2(t)\}$ estacionario de orden 2. Luego
\begin{itemize}
\item Las medias\footnote{La media de una variable aleatoria $V$ se define como $ \mu_V := \E{V}$} 
$ \mu_{X(t)}$, $ \mu_{Y_1(t)}$ y $ \mu_{Y_2(t)}$ no dependen de $t$

\item Las varianzas\footnote{La varianza de una variable aleatoria $V$ se define como 
$ \Var{V} := \E{\left(V - \mu_V \right)^{2}}$} $ \Var{Y_1(t)}$ y $ \Var{Y_2(t)}$ no dependen de 
$t$, pero no se puede garantizar lo mismo para $\Var{X(t)}$

\item El coeficiente de asimetr\'ia\footnote{El coeficiente de asimetr\'ia para una variable 
aleatoria $V$ se define como 
$\gamma_V = \frac{\E{\left(V-\mu_V\right)^{3}}}{\Var{V}^{\nicefrac{3}{2}}}$}
$ \gamma_{X(t)}$ no depende de $t$, pero no se puede garantizar lo mismo para $ \gamma_{Y_1(t)}$ ni 
para $ \gamma_{Y_2(t)}$
\end{itemize}

Cabe mencionar que hay una relaci\'on de contenci\'on clara en familia de los conjuntos de procesos 
estacionarios de orden finito (si un proceso es estacionario de orden $m$, entonces es estacionario 
de orden $n$ para todo $n \leq m$); es posible definir procesos estacionarios de orden 'infinito' 
seg\'un \ref{est_orden_m}, que intuitivamente ser\'ian fuertemente estacionarios. 
De manera pragm\'atica, en este trabajo no se discuten tales interrogantes, sino que se usar\'a 
\'unicamente la definici\'on correspondiente al caso $m=2$, referida como estacionariedad d\'ebil o 
de orden 2, y repetida en la definici\'on \ref{est_orden_2}.

-----------------

Cabe comentar  sobre la existencia de procesos que son fuertemente estacionarios pero que no son 
estacionarios de ning\'un orden: por ejemplo, un proceso de variables aleatorias independientes con 
distribuci\'on de Cauchy\footnote{Una variable aleatoria tiene distribuci\'on de Cauchy si su 
funci\'on de probabilidad acumulada es de la forma 
$\displaystyle F(x) = \frac{1}{\pi} \int_{-\infty}^{x} \frac{1}{1+y^{2}} dy$}.
Una condici\'on suficiente para que un proceso fuertemente estacionario sea estacionario de orden 
$m$ es que tenga sus primeros $m$ momentos bien definidos.
Con respecto a las se\~nales registradas en el EEG, entendidas como procesos estoc\'asticos, se 
espera que tengan (cuando menos) segundos momentos bien definidos; m\'as adelante se presentan 
argumentos, desde una interpretaci\'on f\'isica, sobre por qu\'e se espera que ocurra lo anterior.



-------------------

Como ejemplos, un proceso ruido blanco (definici\'on \ref{r_blanco}) no es estoc\'asticamente 
continuo, mientras que un proceso de Wiener (definici\'on \ref{r_wiener}) s\'i lo es.

\begin{defn}[Proceso ruido blanco]
Se dice de un proceso estoc\'astico $\{ R(t) \}$ que cumple, para cualesquiera tiempos admisibles
$t$ y $s$, las siguientes propiedades:
\begin{itemize}
\item $\E{R(t)}=0$
\item $\Cov{R(t),R(s)}=0 \Leftrightarrow t=s$ 
\end{itemize}
\label{r_blanco}
\end{defn}

\begin{defn}[Proceso de Wiener]
Se dice de un proceso estoc\'astico $\{ W(t) \}$ que cumple, para cualesquiera tiempos admisibles
$t$ y $s$ (con $s>t$) las siguientes propiedades:
\begin{itemize}
\item $W(0) = 0$ ($W(0)$ es constante)
\item $W(s)-W(t)$ es independiente de $W(u)$, para todo $u<t$ admisible
\item $W(s)-W(t) \sim N(0,\abso{t-s})$  (los incrementos tienen distribuci\'on normal)
\end{itemize}
\label{r_wiener}
\end{defn}

--------------------------

En la definici\'on \ref{fourier_stieltjes}, si $F$ es derivable en todas partes entonces $F\prima$ 
cumple el mismo papel que la integral de Fourier; en cambio, si $k$ es una funci\'on peri\'odica 
entonces $F$ toma una forma escalonada cuyos aumentos coinciden con la serie de Fourier para $k$. 
M\'as a\'un, existen funciones que no son ni peri\'odicas ni absolutamente sumables pero poseen 
una transformada de Fourier-Stieltjes, como $k(x)=\SEN{x}+\SEN{\sqrt{2}x}$, 
$k(x)=\COS{x} + (1+x^{2})^{-1}$.

%\begin{comment}
\begin{thrm}[Descomposici\'on de Lebesgue]
Sea $f:I\rightarrow \R$ una funci\'on de variaci\'on acotada, con $I$ un intervalo. Entonces pueden 
hallarse funciones $f_j, f_c, f_a :I\rightarrow \R$ tales que
\begin{itemize}
\item $f = f_j+ f_c+ f_a$
\item $f_j = \sum_{y \leq x} f(x-0) + f(x+0)$
\item $f_a$ es absolutamente continua\footnote{Para que una funci\'on sea absolutamente continua,
basta que sea de variaci\'on acotada y que mapee conjuntos de medida cero en conjuntos de medida
cero} en $I$
\item $f_c$ es una funci\'on singular\footnote{Una funci\'on es singular si es continua, de 
variaci\'on acotada y no-constante, y se cumple que tiene derivada cero casi en todas partes} en 
$I$
\end{itemize}
Estas funciones son \'unicas excepto por constantes, y en conjunto son llamados la 
\textit{descomposici\'on de Lebesgue} de $f$
\label{Lebesgue_decomp}
\end{thrm}
%\end{comment}

%%%%%%%%%%%%%%%%%%%%%%%%%%%%%%%%%%%%%%%%%%%%%%%%%%%%%%%%%%%%%%%%%%%%%%%%%%%%%%%%%%%%%%%%%%%%%%%%%%%
%%%%%%%%%%%%%%%%%%%%%%%%%%%%%%%%%%%%%%%%%%%%%%%%%%%%%%%%%%%%%%%%%%%%%%%%%%%%%%%%%%%%%%%%%%%%%%%%%%%

\section{Transformada R\'apida de Fourier}

Como se mostró en el texto, la transformada de Fourier es un operador clave para la definición y el
estudio del \textit{dominio de las frecuencias}. Sin embargo, su aplicación a series de tiempo grandes se ve
dificultada porque es un proceso lento: si se toma una serie de tiempo 
$\{s_n\}_{n=0,\dots,N}$ y se calcula su transformada finita de Fourier según su definición
\begin{equation*}
\mathfrak{F}_s(\omega) = \sum_{n=0}^{N} s_n e^{i \omega n}
\end{equation*}
entonces para cada frecuencia $\omega$ se requerirán $N$ multiplicaciones y $N-1$ sumas, siendo que
usualmente se analizan las frecuencias de la forma $\omega_k = \nicefrac{2 \pi k}{N}$ 
con $k = 0, 1, \dots, \nicefrac{N}{2}$.
Usando la notación de Landau (definición \ref{orden}) se deduce que obtener la transformada discreta
de Fourier de una serie de tiempo de longitud $N$, usando este método, 
ocupa un tiempo de orden $\mathcal{O}(N^{2})$.

\begin{defn}[Orden $\mathcal{O}$]
Sean $f, g$ dos funciones en $\R$ con $g(x)\neq 0$ para $x\in \R$. Se dice que $f = \mathcal{O}(g)$,
que $f$ tiene orden $g$, si existe una constante $k\in \R$ tal que
\begin{equation*}
\lim_{x\rightarrow \infty} \frac{f(x)}{g(x)} = k
\end{equation*}
\label{orden}
\end{defn}

El algoritmo presentado por  transformada Rápida de Fourier (TRF) 

%%%%%%%%%%%%%%%%%%%%%%%%%%%%%%%%%%%%%%%%%%%%%%%%%%%%%%%%%%%%%%%%%%%%%%%%%%%%%%%%%%%%%%%%%%%%%%%%%%%
%%%%%%%%%%%%%%%%%%%%%%%%%%%%%%%%%%%%%%%%%%%%%%%%%%%%%%%%%%%%%%%%%%%%%%%%%%%%%%%%%%%%%%%%%%%%%%%%%%%

\section{Efecto del filtro STL}

En el texto se menciona al filtro STL como un algoritmo para eliminar los efectos de tendencias
deterministas sobre los registros, con lo cual se puede argumentar que los registros tienen
valor esperado cero y que admiten una representaci\'on de Wold-Cram\'er. Sin embargo, se omiti\'o
una descripci\'on m\'as adecuada por motivos narrativos.

El algoritmo, introducido por Cleveland y colaboradores en 1990 \cite{Cleveland1990} toma sus siglas
del ingl\'es \textit{Seasonal-Trend decomposition based on Loess} (descomposici\'on en tendencia y
periodicidad basada en loess)

%%%%%%%%%%%%%%%%%%%%%%%%%%%%%%%%%%%%%%%%%%%%%%%%%%%%%%%%%%%%%%%%%%%%%%%%%%%%%%%%%%%%%%%%%%%%%%%%%%%
%%%%%%%%%%%%%%%%%%%%%%%%%%%%%%%%%%%%%%%%%%%%%%%%%%%%%%%%%%%%%%%%%%%%%%%%%%%%%%%%%%%%%%%%%%%%%%%%%%%
%%%%%%%%%%%%%%%%%%%%%%%%%%%%%%%%%%%%%%%%%%%%%%%%%%%%%%%%%%%%%%%%%%%%%%%%%%%%%%%%%%%%%%%%%%%%%%%%%%%
%%%%%%%%%%%%%%%%%%%%%%%%%%%%%%%%%%%%%%%%%%%%%%%%%%%%%%%%%%%%%%%%%%%%%%%%%%%%%%%%%%%%%%%%%%%%%%%%%%%