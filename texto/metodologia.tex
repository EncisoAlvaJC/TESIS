%%%%%%%%%%%%%%%%%%%%%%%%%%%%%%%%%%%%%%%%%%%%%%%%%%%%%%%%%%%%%%%%%%%%%%%%%%%%%%%%%%%%%%%%%%%%%%%%%%%
%%%%%%%%%%%%%%%%%%%%%%%%%%%%%%%%%%%%%%%%%%%%%%%%%%%%%%%%%%%%%%%%%%%%%%%%%%%%%%%%%%%%%%%%%%%%%%%%%%%
%%%%%%%%%%%%%%%%%%%%%%%%%%%%%%%%%%%%%%%%%%%%%%%%%%%%%%%%%%%%%%%%%%%%%%%%%%%%%%%%%%%%%%%%%%%%%%%%%%%
%%%%%%%%%%%%%%%%%%%%%%%%%%%%%%%%%%%%%%%%%%%%%%%%%%%%%%%%%%%%%%%%%%%%%%%%%%%%%%%%%%%%%%%%%%%%%%%%%%%
\chapter{Planteamiento del problema}

\section{Antecedentes}

En algunos estudios de gran escala se han hallado correlaciones entre diferentes trastornos del 
sueño y algún grado de deterioro cognitivo objetivo en adultos mayores 
\cite{Amer13,Miyata13,Reid06,Potvin12}; entendiendo por ello una ejecuciones más pobres en tareas
cognitivas, pero que no impiden llevar a cabo actividades cotidianas.

En 2016 Vázquez-Tagle y colaboradores estudiaron la epidemiología del deterioro cognitivo en 
adultos mayores dentro del estado de Hidalgo y su posible relación con trastornos de sueño, 
encontrando una correlación entre una menor eficiencia del sueño\footnote{Porcentaje de tiempo
de sueño respecto al tiempo en cama} y la presencia de deterioro cognitivo \cite{VazquezTagle16}.

En un segundo trabajo por García-Muñoz y colaboradores \cite{Valeria} se analizaron 
registros de polisomnograma (PSG) 
%datos de PSG 
para detectar posibles cambios en la conectividad funcional del cerebro\footnote{La 
\textbf{conectividad funcional} se refiere una \textit{fuerte} relación (cuantificada) entre dos 
señales, y usualmente se contrasta con la \textbf{conectividad anatómica}, entendida como conexiones 
físicas entre los generadores de señales} en adultos mayores con posible deterioro 
cognitivo (PDC), reportando un mayor exponente de Hurst para registros de PSG en adultos mayores 
con PDC \cite{Valeria}.
El exponente de Hurst, calculado a través del algoritmo \textit{Detrended Fluctuation Analysis}, 
está relacionado con las correlaciones de largo alcance y la estructura fractal de una serie de 
tiempo, siendo que un mayor exponente está asociado con señales cuya función de 
autocorrelación decrece más lentamente \cite{Rodriguez11}.
Con base a que en aquellos trabajos se ha supuesto que los registros de PSG son no-estacionarios, 
en este trabajo se pretende verificar si efectivamente estas señales se pueden considerar con tal
característica.

El supuesto de estacionariedad es básico en el estudio de series de tiempo, y usualmente se 
acepta o rechaza sin un tratamiento formal; es de particular importancia, por ejemplo, para 
calcular el espectro de potencias a partir de registros.
La idea de que sujetos con PDC exhiben estacionariedad débil en sus registros de EEG en mayor 
proporción, respecto a individuos sanos, fue sugerida por Cohen \cite{Cohen77}, quien a su vez se 
refiere a trabajos anteriores sobre estacionariedad y normalidad en registros de EEG 
\cite{McEwen75,Sugimoto78,Kawabata73}.
%Cabe mencionar que en estos primeros estudios se palpa la posibilidad de que los registros de EEG 
%fueran 'ruido' de algún tipo, una idea que se ha probado errónea en estudios más recientes 
%\cite{Klonowski09}; sin embargo, se retoma como hipótesis a la luz de los estudios mencionados. 

%%%%%%%%%%%%%%%%%%%%%%%%%%%%%%%%%%%%%%%%%%%%%%%%%%%%%%%%%%%%%%%%%%%%%%%%%%%%%%%%%%%%%%%%%%%%%%%%%%%
%%%%%%%%%%%%%%%%%%%%%%%%%%%%%%%%%%%%%%%%%%%%%%%%%%%%%%%%%%%%%%%%%%%%%%%%%%%%%%%%%%%%%%%%%%%%%%%%%%%

%\section{Justificación}

%%%%%%%%%%%%%%%%%%%%%%%%%%%%%%%%%%%%%%%%%%%%%%%%%%%%%%%%%%%%%%%%%%%%%%%%%%%%%%%%%%%%%%%%%%%%%%%%%%%
%%%%%%%%%%%%%%%%%%%%%%%%%%%%%%%%%%%%%%%%%%%%%%%%%%%%%%%%%%%%%%%%%%%%%%%%%%%%%%%%%%%%%%%%%%%%%%%%%%%

\section{Pregunta de investigación}

¿Los registros de PSG\footnote{Polisomnograma: actividad eléctrica del cerebro durante el sueño,
además de otros marcadores como la actividad ocular o la respiración} en adultos mayores, pueden
considerarse como series tiempo débilmente estacionarias?
¿Es posible que tal caracterización se relacione con el estado cognoscitivo del adulto mayor?

%%%%%%%%%%%%%%%%%%%%%%%%%%%%%%%%%%%%%%%%%%%%%%%%%%%%%%%%%%%%%%%%%%%%%%%%%%%%%%%%%%%%%%%%%%%%%%%%%%%
%%%%%%%%%%%%%%%%%%%%%%%%%%%%%%%%%%%%%%%%%%%%%%%%%%%%%%%%%%%%%%%%%%%%%%%%%%%%%%%%%%%%%%%%%%%%%%%%%%%

\subsection{Hipótesis}

Existen diferencias en la conectividad funcional del cerebro en adultos mayores con PDC, respecto
a sujetos sanos, y es posible detectar estas diferencias como una mayor o menor 'presencia' de 
estacionariedad débil en registros de PSG durante el sueño profundo.

%%%%%%%%%%%%%%%%%%%%%%%%%%%%%%%%%%%%%%%%%%%%%%%%%%%%%%%%%%%%%%%%%%%%%%%%%%%%%%%%%%%%%%%%%%%%%%%%%%%

\subsection{Objetivo general}

Deducir, a partir de pruebas estadísticas formales, las presencia de estacionariedad débil en
registros de PSG para adultos mayores con PDC, así como individuos control.

%%%%%%%%%%%%%%%%%%%%%%%%%%%%%%%%%%%%%%%%%%%%%%%%%%%%%%%%%%%%%%%%%%%%%%%%%%%%%%%%%%%%%%%%%%%%%%%%%%%

\subsection{Objetivos específicos}

\begin{itemize}
\item Estudiar la definición de estacionariedad para procesos estocásticos y sus posibles 
consecuencias dentro de un modelo para los datos considerados

\item Investigar en la literatura cómo detectar si es plausible que una serie de tiempo dada sea 
una realización para un proceso estocástico débilmente estacionario, y bajo qué supuestos 
es válida esta caracterización

\item Usando los análisis hallados en la literatura, determinar si las series de tiempo 
obtenidas a partir de los datos considerados provienen de procesos débilmente estacionarios.
Revisar si la información obtenida en los diferentes sujetos muestra diferencias entre sujetos 
con y sin PDC
\end{itemize}

%\chapter{Metodología}
\section{Metodología}

El presente trabajo resulta de una colaboración con el departamento de Gerontología, dependiente 
del Instituto de Ciencias de la Salud (ICSA); parte de esta colaboración incluye el acceso a los 
registros de PSG obtenidos por Vázquez Tagle y colaboradores \cite{VazquezTagle16}. 
A continuación se expone la metodología de aquél estudio.% de la manera más fiel posible.
%Así mismo se describe, a nivel de implementación, el análisis central de este trabajo: la prueba 
%de Priesltey-Subba Rao. 

\section{Participantes}

Los sujetos fueron elegidos usando un muestreo \textit{no probabilístico por 
conveniencia}\footnote{Esto implica que los resultados pueden  no ser interpolables a poblaciones 
más grandes} bajo los siguientes criterios de inclusión:
\begin{itemize}
\item Edad entre 60 y 85 años
\item Diestros (mano derecha dominante)
\item Sin ansiedad, depresión ni síndromes focales
\item No usar medicamentos o sustancias para dormir
\item Firma de consentimiento informado
\item Voluntario para el registro de PSG
\end{itemize}

Un total de 9 participantes cumplieron todos los criterios de inclusión y procedieron al registro 
de PSG; adicionalmente se tomaron registros de otros tres adultos mayores, bajo el consentimiento 
de éstos y de los responsables del proyecto (ver más adelante).
%
Todos los participantes fueron sometidos a una batería de pruebas neuropsicológicas para determinar
su estado cognoscitivo general (Neuropsi, MMSE), así como descartar cuadros depresivos (GDS, SATS) 
y cambios en la vida cotidiana (KATZ).
Usando los resultados obtenidos, los sujetos se dividieron en tres grupos:

\begin{table}[h]
\centering
\begin{tabular}{lcl}
\toprule
Grupo & Sujetos & Características \\
\midrule
Mn & 4 & Posible Deterioro Cognitivo \\
Nn & 5 & Sin PDC \\
ex & 3 & No satisfacen los criterios de inclusión \\
\bottomrule
\end{tabular}
\end{table}
%
%\begin{description}
%\item[Mn] (4 sujetos) Posible Deterioro Cognitivo
%\item[Nn] (5 sujetos) Sin PDC
%\item[ex] (3 sujetos) No satisfacen los criterios de inclusión
%\end{description}

El grupo ex se conforma de sujeto que incumplen al menos uno de los criterios de inclusión: {FGH} 
padece parálisis facial y posiblemente daño cerebral (síndromes focales), MGG padece depresión, 
EMT no califica como adulto mayor por su edad.
Se efectuaron todos los análisis sobre este grupo, con la finalidad de exhibir las capacidades y
limitaciones de las técnicas utilizadas; por ello este grupo es ignorado en la sección de 
resultados pero no en la discusión.

\begin{table}
\centering
\bordes{1.1}
\begin{tabular}{c}
\textbf{Datos generales de los participantes}
\vspace{1em}
\end{tabular}
{\small
\begin{tabular}{llcrrrrrrr}
\toprule
 \phantom{.}&
 & {Sexo} & {Edad} & {Escol.} & {Neuropsi} & {MMSE} & {SATS} & {KATZ} & {Gds} \\
\midrule
\multicolumn{6}{l}{{Grupo Nn}}\\
&VCR    & F    & 59\pz & 12\pz & 107\pz & 29\pz & 21\pz & 0\pz & 3\pz \\
&MJH    & F    & 72\pz & 9\pz  & 113\pz & 30\pz & 18\pz & 0\pz & 0\pz \\
&JAE    & F    & 78\pz & 5\pz  & 102\pz & 28\pz & 19\pz & 0\pz & 5\pz \\
&GHA    & M    & 65\pz & 9\pz  & 107.5  & 30\pz & 23\pz & 0\pz & 7\pz \\
&MFGR   & F    & 67\pz & 11\pz & 110\pz & 30\pz & 18\pz & 0\pz &      \\
\rowcolor{gris}
&\multicolumn{1}{c}{$\widehat{\mu}$} & 
               & 68.2  & 9.2   & 107.9  & 29.4  & 19.8  & 0.0  & 3.0  \\
\rowcolor{gris}
&\multicolumn{1}{c}{$\widehat{\sigma}$} & 
               & 7.2   & 2.7   & 4.1    & 0.9   & 2.2   & 0.0  & 3.0  \\
\midrulec
%\hline
\multicolumn{6}{l}{{Grupo Mn}}\\
&CLO    & F    & 68\pz & 5\pz  & 81\pz & 28\pz & 22\pz & 1\pz & 6\pz \\
&RLO    & F    & 63\pz & 9\pz  & 90\pz & 29\pz & 20\pz & 0\pz & 3\pz \\
&RRU    & M    & 69\pz & 9\pz  & 85\pz & 27\pz & 10\pz & 0\pz & 3\pz \\
&JGZ    & M    & 65\pz & 11\pz & 87\pz & 25\pz & 20\pz & 0\pz & 1\pz \\
\rowcolor{gris}
&\multicolumn{1}{c}{$\widehat{\mu}$} & 
              & 66.3   & 8.5   & 85.8  & 27.3  & 18.0  & 0.3  & 3.3  \\
\rowcolor{gris}
&\multicolumn{1}{c}{$\widehat{\sigma}$} & 
              & 2.8    & 2.5   & 3.8   & 1.7   & 5.4   & 0.5  & 2.1  \\
\midrulec
%\hline
\multicolumn{6}{l}{{Grupo ex}}\\
&FGH    & M    & 71\pz   & 9\pz    & 83.5     & 21\pz   & 23\pz   & 0\pz    & 4\pz  \\
&MGG    & F    & 61\pz   & 9\pz    & 114\pz      & 28\pz   & 29\pz   & 1\pz    & 14\pz \\
&EMT    & M    & 50\pz   & 22\pz   & 106\pz      & 30\pz   & 15\pz   & 0\pz    & 4\pz  \\
\bottomrule
\end{tabular} 
}
\label{tab_sujetos}
\caption{Resultados de las pruebas neuropsicológicas 
}
\end{table}

%%%%%%%%%%%%%%%%%%%%%%%%%%%%%%%%%%%%%%%%%%%%%%%%%%%%%%%%%%%%%%%%%%%%%%%%%%%%%%%%%%%%%%%%%%%%%%%%%%%
%%%%%%%%%%%%%%%%%%%%%%%%%%%%%%%%%%%%%%%%%%%%%%%%%%%%%%%%%%%%%%%%%%%%%%%%%%%%%%%%%%%%%%%%%%%%%%%%%%%

\section{Registro del polisomnograma}

Los adultos mayores participantes fueron invitados a acudir a las instalaciones de la Clínica 
Gerontológica de Sueño (ubicadas dentro del Instituto de Ciencias de la Salud) para llevar a cabo 
el registro. Los participantes recibieron instrucciones de realizar una rutina normal de 
actividades durante la semana que precedió al estudio, y se les recomendó que no ingirieran bebidas 
alcohólicas o energizantes (como café o refresco) durante las 24 horas previas al experimento, ni 
durmieran siesta ese día.

El protocolo de PSG incluye 19 electrodos de EEG, 4 electrodos de EOG para registrar movimientos 
oculares horizontales y verticales, y 2 electrodos de EMG colocados en los músculos submentonianos 
para registrar la actividad muscular. 
La colocación de los electrodos para registrar la actividad EEG se realizó siguiendo las 
coordenadas del Sistema Internacional 10--20.

Las señales fueron amplificadas (amplificador de alta ganancia en cadena), filtradas (filtro paso 
de banda de 0.5--30 \hz) y digitalizadas para su posterior análisis.
En la tabla \ref{frecuencias} se reportan la duración de estos registros para cada sujeto.

Debido a problemas técnicos el registro se efectúo a 512 puntos por segundo (\hz) para algunos
participantes, y a 200 \hz para otros; en ambos casos se satisface la recomendación de la AASM de un 
mínimo de 128 \hz. 

La clasificación del PSG en fases de sueño se realizó \textit{manualmente} sobre épocas de 30 
segundos siguiendo los criterios estandarizados de la AAMS \cite{Hori01}.

%Debido a un cambio en el polisomnógrafo 
%usado, la frecuencia de muestreo (en Hz) cambia entre sujetos.

\begin{table}
\centering
\bordes{1.2}
\begin{tabular}{c}
\textbf{Datos generales sobre los registros de PSG}
\vspace{1em}
\end{tabular}
{\small
\begin{tabular}{llcrrcrrr}
\toprule
    \phantom{.}&
    &\multirow{2}{*}{\bordes{1}\begin{tabular}{l}Frecuencia\\ muestreo\end{tabular}}
    \bordes{1.2}
    & \multicolumn{2}{c}{Total} & \phantom{l}   & \multicolumn{3}{c}{MOR*}\\
    \cmidrule{4-5}  \cmidrule{7-9}
    &&          &Puntos  &  Tiempo   &&Puntos  &  Tiempo   &  \% MOR \\
\midrule
\multicolumn{6}{l}{{Grupo Nn}}\\
&VCR &200       & 5166000&   7:10:30 &&438000  &   0:36:30 & 8.5\% \\
&MJH &512       &15851520&   8:36:00 &&1950720 &   1:03:30 &12.3\% \\
&JAE &512       &13931520&   7:33:30 &&2626560 &   1:25:30 &18.9\% \\
&GHA &200       &6558000 &   9:06:00 &&330000  &   0:27:30 & 5.0\% \\
&MFGR&200       &4932000 &   6:51:00 &&570000  &   0:47:30 &11.6\% \\

\rowcolor{gris}
&\multicolumn{1}{c}{$\widehat{\mu}$}  
              & &        & 7:51:30   &&        &   0:52:06 &11.2\% \\
\rowcolor{gris}
&\multicolumn{1}{c}{$\widehat{\sigma}$} 
              & &        & 0:57:36   &&        &   0:23:00 & 5.1\% \\
\midrulec

\multicolumn{6}{l}{{Grupo Mn}}\\
&CLO &512       &14499840&   7:52:00 &&2027520 &   1:06:00 &14.0\% \\
&RLO &512       &12994560&   7:03:00 &&1520640 &   0:49:30 &11.7\% \\
&RRU &200       &2484000 &   3:27:00 &&228000  &   0:19:00 & 9.2\% \\
&JGZ &512       &18539520&  10:03:30 &&506880  &   0:16:30 & 2.7\% \\

\rowcolor{gris}
&\multicolumn{1}{c}{$\widehat{\mu}$}  
              & &        & 7:06:23   &&        &   0:37:45 &9.4\% \\
\rowcolor{gris}
&\multicolumn{1}{c}{$\widehat{\sigma}$} 
              & &        & 2:44:55   &&        &   0:24:05 &4.9\% \\
\midrulec

\multicolumn{6}{l}{{Grupo ex}}\\
&FGH &512       &6220800 &   3:22:30 &&337920  &   0:11:00 & 5.4\% \\
&MGG &512       &15820800&   8:35:00 &&2549760 &   1:23:00 &16.1\% \\
&EMT &512       &21857280&  11:51:30 &&721920  &   0:23:30 & 3.3\% \\
\bottomrule
\end{tabular}
}
\caption{Cantidad de datos registrados para cada sujeto. *Dado que el sueño MOR aparece fragmentado,
se reporta la suma de tales tiempos.}
\label{frecuencias}
\end{table}

%%%%%%%%%%%%%%%%%%%%%%%%%%%%%%%%%%%%%%%%%%%%%%%%%%%%%%%%%%%%%%%%%%%%%%%%%%%%%%%%%%%%%%%%%%%%%%%%%%%
%%%%%%%%%%%%%%%%%%%%%%%%%%%%%%%%%%%%%%%%%%%%%%%%%%%%%%%%%%%%%%%%%%%%%%%%%%%%%%%%%%%%%%%%%%%%%%%%%%%

\section{Aplicación de la prueba PSR}

Los registros digitalizados de PSG fueron convertidos a formato de texto bajo la codificación 
ASCII, a razón de un archivo por cada canal. 
Las épocas MOR, clasificadas manualmente, fueron indicadas en archivos a parte.

%Como se mencionó en secciones anteriores, la prueba PSR está pensada para series de tiempo con 
%media 0, varianza finita y espectro puramente continuo. Se espera que la segunda condición se 
%cumpla para los registros de PSG; las otras dos condiciones fueron \textit{forzadas}, sustrayendo 
%la media y la componente periódica (estimadas) del proceso.
%Para lo anterior, se usó el algoritmo no-paramétrico STL (Seasonal-Trend decomposition using 
%Loess) \cite{Cleveland1990} y que está implementado en R bajo la función \texttt{stl()}.

%La prueba PSR se encuentra implementado en R bajo la función \texttt{stationarity()} del paquete 
%\texttt{fractal}.    
%Los resultados de la prueba PSR, aplicado a todas las épocas contenidas en los registros de PSG,
%fueron almacenados para su análisis posterior.

El registro de PSG por cada canal fueron analizados por separado, y éstos a su vez fueron divididos
en épocas de 30 segundos de duración (variando el número de puntos según la frecuencia de muestreo);
cada época fue clasificada como \textit{estacionaria} si, no pudo rechazarse la hipótesis de 
estacionariedad usando la prueba PSR ($p < 0.05$).
La cantidad de épocas estacionarias para cada individuo, durante sueño MOR y NMOR, se muestra en 
las tablas \ref{total_gpos_total} y \ref{total_gpos_mor}; debido a la gran variabilidad entre los 
sujetos para la duración del sueño MOR, para el análisis no se consideró el total de épocas sino la 
proporción de éstas en cada etapa de sueño. 

\begin{figure}
\centering
\begin{lstlisting}[caption={}]
Priestley-Subba Rao stationarity Test for datos
-----------------------------------------------
Samples used              : 3072 
Samples available         : 3069 
Sampling interval         : 1 
SDF estimator             : Multitaper 
  Number of (sine) tapers : 5 
  Centered                : TRUE 
  Recentered              : FALSE 
Number of blocks          : 11 
Block size                : 279 
Number of blocks          : 11 
p-value for T             : 0.4130131 
p-value for I+R           : 0.1787949 
p-value for T+I+R         : 0.1801353 
\end{lstlisting}
\caption{Resultado típico para la función \texttt{stationarity}
%El parámetro \texttt{n.blocks} define la cantidad grupos disjuntos para los cuales se calculará 
%el estimador de la FDE.
%Cabe resaltar el antepenúltimo renglón (\texttt{p-value for T}), según el cual se puede
%aceptar o rechazar la hipótesis de estacionariedad débil. 
%La FDE es referida como 'Spectral Density Function' (SDF).
}
\label{res_psr}
\end{figure}

%%%%%%%%%%%%%%%%%%%%%%%%%%%%%%%%%%%%%%%%%%%%%%%%%%%%%%%%%%%%%%%%%%%%%%%%%%%%%%%%%%%%%%%%%%%%%%%%%%%
%%%%%%%%%%%%%%%%%%%%%%%%%%%%%%%%%%%%%%%%%%%%%%%%%%%%%%%%%%%%%%%%%%%%%%%%%%%%%%%%%%%%%%%%%%%%%%%%%%%
%%%%%%%%%%%%%%%%%%%%%%%%%%%%%%%%%%%%%%%%%%%%%%%%%%%%%%%%%%%%%%%%%%%%%%%%%%%%%%%%%%%%%%%%%%%%%%%%%%%
%%%%%%%%%%%%%%%%%%%%%%%%%%%%%%%%%%%%%%%%%%%%%%%%%%%%%%%%%%%%%%%%%%%%%%%%%%%%%%%%%%%%%%%%%%%%%%%%%%%