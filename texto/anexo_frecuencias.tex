%%%%%%%%%%%%%%%%%%%%%%%%%%%%%%%%%%%%%%%%%%%%%%%%%%%%%%%%%%%%%%%%%%%%%%%%%%%%%%%%%%%%%%%%%%%%%%%%%%%
%%%%%%%%%%%%%%%%%%%%%%%%%%%%%%%%%%%%%%%%%%%%%%%%%%%%%%%%%%%%%%%%%%%%%%%%%%%%%%%%%%%%%%%%%%%%%%%%%%%
%%%%%%%%%%%%%%%%%%%%%%%%%%%%%%%%%%%%%%%%%%%%%%%%%%%%%%%%%%%%%%%%%%%%%%%%%%%%%%%%%%%%%%%%%%%%%%%%%%%
%%%%%%%%%%%%%%%%%%%%%%%%%%%%%%%%%%%%%%%%%%%%%%%%%%%%%%%%%%%%%%%%%%%%%%%%%%%%%%%%%%%%%%%%%%%%%%%%%%%

\chapter{Ondas y frecuencias}

\section{Transformada de Fourier como operador}

La exposición inicia con los espacios de las \textbf{series $\boldsymbol{p}$-sumables}
($\lp$), y las  \textbf{funciones $\boldsymbol{p}$-integrables} sobre un intervalo 
$I \subseteq \R$ ($\llp$).
%; en el presente trabajo sólo se usarán los casos $p=1,2$.
%
%\begin{align}
%\ell^{p} &:= \left\{ s: \Z\rightarrow\C \talque \sum_{n=-\infty}^{\infty} \abso{s(n)}^{p} < \infty \right\}
%\label{lpdef} \\
%L^{p}[I] &:= \left\{ S: I\rightarrow\C \talque \int_I \abso{S(t)}^{p} dt < \infty \right\}
%\label{llpdef}
%\end{align}
\begin{align*}
\ell^{p} &:= \left\{ s: \Z\rightarrow\C \talque \sum_{n=-\infty}^{\infty} \abso{s(n)}^{p} < \infty \right\}
\\
L^{p}[I] &:= \left\{ S: I\rightarrow\C \talque \int_I \abso{S(t)}^{p} dt < \infty \right\}
\end{align*}

Estos espacios admiten las operaciones $+$, $\cdot$ y multiplicación por escalares complejos de la 
manera usual.%, es decir

%\begin{align*}
%s, z \in \lp, c \in \C \Rightarrow 
%[s+z](n) &= s(n) + z(n) \\
%[s\cdot z](n) &= s(n) z(n) \\
%[c \cdot s](n) &= s(n)c \\
%S, Z \in \llp, c \in \C \Rightarrow 
%[S+Z](t) &= S(t) + Z(t) \\
%[S\cdot Z](t) &= S(t)  Z(t) \\
%[c \cdot S](t) &= S(t)c
%\end{align*}
%
%En las próximas líneas se seguirán usando $s, z, S, Z, c$.

Para el caso particular $p=2$, los conjuntos $\ldos$ y $\lldos$ admiten los siguientes productos 
internos:
%
\begin{align*}
\left\langle s,z \right\rangle &= \sum_{n=-\infty}^{\infty} s(n) \overline{z(n)}\\
\left\langle S,Z \right\rangle &= \int_I S(t) \overline{Z(t)} dt
\end{align*}

Usando dichos productos internos, junto con las normas y métricas que inducen, los conjuntos 
$\ldos$ y $\lldos$ tienen estructura de \textbf{espacio de Hilbert}.

Con las definiciones anteriores, que muestran que $\ldos$ y $\lldos$ son \textit{muy}
parecidos, se puede formular unas definición para la transformada de Fourier como una equivalencia
entre estos espacios.

%{De manera pragm\'atica, en el presente trabajo la 
%palabra  'frecuencia' se usar\'a para referirse a la cantidad $q$ en expresiones del tipo 
%$e^{i q t}$}

\begin{definicion}[Serie de Fourier]
Sea $S: \R \rightarrow \C$ una función periódica con periodo $2T$ y tal que 
$S \in L^{2}\left[[-T,T]\right]$. Se dice que $A$ es la serie de Fourier para $S$ si cumple que
\begin{equation*}
A(n) = \frac{1}{2 T} \simint{T} S(t) e^{-\nicefrac{ i \abso{n} t}{2T}} dt
\end{equation*}
%Adicionalmente, la función $\mathcal{F} : \lldos \rightarrow \ldos : S \mapsto A$  recibe el nombre
%de \textbf{Transformada de Fourier}
\label{FourierClasico}
\end{definicion}

\begin{definicion}[Transformada de Fourier]
Sean $S$ y $A$ como en la definición \ref{FourierClasico}. Se le llama transformada de Fourier a la
función $\mathcal{F}_T : L^{2}\left[[-T,T]\right] \rightarrow \ldos : S \mapsto A$
\end{definicion}

Puede interpretarse a $A$ como las \textit{coordenadas} de $S$ en $L^{2}\left[[-T,T]\right]$, 
usando una base de funciones $\left\{ e^{\nicefrac{i \abso{n} t}{2 T}} \right\}_{n\in \Z}$, las
cuales resultan ser ortonormales; esta base en particular es conocida como la \textbf{base de 
Fourier}.
Se demuestra en el anexo A que $\mathcal{F_T}$ está bien definida en el sentido de 
tener efectivamente el dominio y codominio indicados. Así mismo, cabe mencionar las siguientes 
propiedades de $\mathcal{F}_T$
\begin{itemize}
\item Es lineal, es decir, $\mathcal{F}_T[cS + Z] = c\mathcal{F}_T[S] + \mathcal{F}_T[Z]$

\item \textbf{No} es invertible, aunque se le suele definir una
pseudoinversa\footnote{$\mathcal{F}_T^{\text{inv}}$ es \textit{exacta} salvo por la suma
de alguna función $S_0$ tal que $\int_{-T}^{T}\abso{S_0(t)}dt = 0$} como
\begin{equation*}
\mathcal{F}_{T}^{\text{inv}} : \ldos \rightarrow L^{2}\left[[-T,T]\right] :
A \mapsto \sum_{n -\infty}^{\infty} A(n) e^{\nicefrac{i \abso{n} t}{2 T}}
\end{equation*}
\end{itemize}

Con esta terminología se define, de manera pragmática, la \textbf{energía disipada} y la 
\textbf{potencia} de una función $S$ en un intervalo $[a,b]$ como 
\begin{align*}
\text{energía}[S]_{[a,b]} &= \int_a^{b} \abso{S(t)}^{2} dt \\
\text{potencia}[S]_{[a,b]} &= \frac{1}{b-a} \int_a^{b} \abso{S(t)}^{2} dt
\end{align*}
%
%Estas últimas definiciones cobran importancia a la luz del teorema \ref{parseval}: la energía de 
%una función equivale a su norma.

Una consecuencia interesante de este concepto de energía frente al teorema \ref{parseval} es que la 
energía disipada por una función equivale a la suma de la energía disipada por sus 
\textit{componentes} en la base de Fourier.
Conviene, entonces, definir una función que \textit{desglose} estos \textit{aportes}.

\begin{teorema}[Parseval]
Sea $S \in L^{2}\left[[-T,T]\right]$, y sea $A = \mathcal{F}[S]$. Se cumple que
\begin{equation*}
\int_{-T}^{T} \abso{S(t)}^{2} dt = \sum_{n=-\infty}^{\infty} \abso{A(n)}^{2}
\end{equation*}
\label{parseval}
\end{teorema}

\begin{definicion}[Espectro de potencias]
Sea $S \in L^{2}\left[[-T,T]\right]$, y sea $A = \mathcal{F}[S]$. Se llama espectro de potencias 
para $S$ a la función $h_S : \R \rightarrow \R $, definida como
\begin{equation*}
h_S(\omega) = 
\begin{cases}
\abso{A(n)}^{2} & \text{ , si } \omega = \nicefrac{n}{2T}, \text{   con } n\in \mathbb{Z} \\
0 & \text{ ,  otro caso}
\end{cases}
\end{equation*}
\label{espec}
\end{definicion}

Un elemento que será de crucial importancia en el desarrollo posterior es la \textbf{convolución}, 
$\ast$, una tercera operación binaria definida en estos espacios como
%
\begin{align*}
[s \ast z] (\tau) &= \sum_{n=-\infty}^{\infty} s(n) \overline{z(\tau-n)} \\
[S \ast Z] (\tau) &= \int_I S(t) \overline{Z(\tau-t)}
\end{align*}
%
donde $\overline{c}$ es el conjugado complejo de $c$. 
%La convolución es conmutativa y asociativa con la suma. 
Esta operación cobra importancia por la forma en que se relaciona con $\mathcal{F}_T$
%
\begin{teorema}%[de la convolución]
Sean $S,Z \in L^{2}\left[[-T,T]\right]$, entonces se satisface que
\begin{align*}
\mathcal{F}_T[S\ast Z]  &= \mathcal{F}_T[S] \cdot \mathcal{F}_T[Z] \\
\mathcal{F}_T[S\cdot Z] &= \mathcal{F}_T[S] \ast  \mathcal{F}_T[Z] \\
\end{align*}
\label{t_convolucion}
\end{teorema}

\subsubsection{Generalizaciones}

La primera gran generalización sobre la transformada de Fourier es para el conjunto de funciones 
$L^{1}\left[ \R \right]$, definido como en la sección anterior; éste también es un espacio de 
Hilbert usando el producto interno descrito.
La generalización propuesta, teorema \ref{intFourier}, sólo se diferencia en que no se exige que 
la función sea periódica y en el espacio que actúa; 
es quizá más llamativo el que el codominio de $\mathcal{F}_\infty$ no sea $\ldos$ sino $\lldos$,
%es decir que su codominio no son series sino funciones integrables
lo cual afecta cómo deben interpretarse los \textit{componentes de frecuencia generalizados}. La 
discusión pertinente se efectúa en el anexo B.

\begin{definicion}[Integral de Fourier]
Sea $S \in L^{1}\left[\R\right]$. Se dice que $A$ es la integral de Fourier para $S$ si cumple que
\begin{equation*}
A(\omega) = \intR S(t) e^{- i \omega t} dt
\end{equation*}
%Adicionalmente, la función $\mathcal{F} : \lldos \rightarrow \ldos : S \mapsto A$  recibe el nombre
%de \textbf{Transformada de Fourier}
\label{intFourier}
\end{definicion}

\begin{definicion}[Transformada de Fourier]
Sean $S$ y $A$ como en la definición \ref{intFourier}. Se le llama transformada de Fourier a la
función $\mathcal{F}_\infty : L^{1}\left[\R\right] \rightarrow \ldos : S \mapsto A$
\end{definicion}

Una forma de \textit{relacionar} a los $\mathcal{F}_T$ con $\mathcal{F}_\infty$ es tomar una
función $S \in L^{1}\left[\R\right]$ y para cada $T$ definir una continuación periódica de $S$
%, $S_T$
%\begin{equation*}
\begin{center}
$\displaystyle S_T(t) = S(t_0) $, 
%\end{equation*}
con $\displaystyle -T\leq t_0 \leq T$ y $\displaystyle \frac{t-t_0}{2T} \in \Z$
\end{center} 
%
Posteriormente puede hacerse que
%\begin{equation*}
$\lim_{T\rightarrow \infty} \mathcal{F}_T[S_T] = \mathcal{F}_{\infty}$.
%\end{equation*}
%Este proceso suele interpretarse como que $\mathcal{F}_\infty$ es una versión de la transformada
%de Fourier donde se permite que la función sea periódica con periodo infinito,
%%el periodo de la función ($\nicefrac{1}{\text{frecuencia}}$) sea
%%infinitamente grande, 
%o que la distancia entre frecuancias sea infinitamente pequeño.
%
Dado que las funciones definidas en \ref{FourierClasico} y en \ref{intFourier} serán importantes 
en lo que prosigue, conviene introducir una segunda generalización que abarque a ambas, para lo que
se acude al concepto de integrales en el sentido de Lebesgue-Stieltjes (en un anexo)

\begin{definicion}[Integral de Fourier-Stieltjes]
Sea $S: \R\rightarrow \C$. Se dice que $F$ es la integral de Fourier-Stieltjes para $S$ si ésta 
puede escribirse como
\begin{equation*}
S(x) = \intR e^{- i \omega t} dF(\omega)
\end{equation*}
donde la integral está definida en el sentido de Lebesgue-Stieltjes, y la igualdad se cumple
casi en todas partes
%Adicionalmente, la función $\mathcal{F} : \lldos \rightarrow \ldos : S \mapsto A$  recibe el nombre
%de \textbf{Transformada de Fourier}
\label{intFourierStieltjes}
\end{definicion}

\section{Transformada Rápida de Fourier}

Como se mostró en el texto, la transformada de Fourier es un operador clave para la definición y el
estudio del \textit{dominio de las frecuencias}. Sin embargo, su aplicación a series de tiempo grandes se ve
dificultada porque es un proceso lento: si se toma una serie de tiempo 
$\{s_n\}_{n=0,\dots,N}$ y se calcula su transformada finita de Fourier según su definición
\begin{equation*}
\mathfrak{F}_s(\omega) = \sum_{n=0}^{N} s_n e^{i \omega n}
\end{equation*}
entonces para cada frecuencia $\omega$ se requerirán $N$ multiplicaciones y $N-1$ sumas, siendo que
usualmente se analizan las frecuencias de la forma $\omega_k = \nicefrac{2 \pi k}{N}$ 
con $k = 0, 1, \dots, \nicefrac{N}{2}$.
Usando la notación de Landau (definición \ref{orden}) se deduce que obtener la transformada discreta
de Fourier de una serie de tiempo de longitud $N$, usando este método, 
ocupa un tiempo de orden $\mathcal{O}(N^{2})$.

\begin{definicion}[Orden $\mathcal{O}$]
Sean $f, g$ dos funciones en $\R$ con $g(x)\neq 0$ para $x\in \R$. Se dice que $f = \mathcal{O}(g)$,
que $f$ tiene orden $g$, si existe una constante $k\in \R$ tal que
\begin{equation*}
\lim_{x\rightarrow \infty} \frac{f(x)}{g(x)} = k
\end{equation*}
\label{orden}
\end{definicion}

El algoritmo presentado por Cooley y Tukey en 1965, la Transformada Rápida de Fourier 
(TRF), usa menos operaciones si el número de observaciones es \textit{altamente 
compuesto}\footnote{Se dice que un número entero es \textit{compuesto} si no tiene dos o más
divisores propios mayores a 1; se dice que es más compuesto si tiene más de estos factores}.

Considerando la serie de tiempo $\{s_n\}_{n=0,\dots,N}$ con $N$ de la forma $N= p q$ para 
$p, q$ enteros mayores a 1, entonces su TRF se puede reescribir como

\begin{align*}
\mathfrak{F}_s(\omega) &= \sum_{m = 0}^{p} \sum_{k=0}^{q} s_n e^{i \omega (mp + q)} 
\\
&= 
\end{align*}

\section{Ondas cerebrales y su frecuencia}

%%%%%%%%%%%%%%%%%%%%%%%%%%%%%%%%%%%%%%%%%%%%%%%%%%%%%%%%%%%%%%%%%%%%%%%%%%%%%%%%%%%%%%%%%%%%%%%%%%%
%%%%%%%%%%%%%%%%%%%%%%%%%%%%%%%%%%%%%%%%%%%%%%%%%%%%%%%%%%%%%%%%%%%%%%%%%%%%%%%%%%%%%%%%%%%%%%%%%%%
%%%%%%%%%%%%%%%%%%%%%%%%%%%%%%%%%%%%%%%%%%%%%%%%%%%%%%%%%%%%%%%%%%%%%%%%%%%%%%%%%%%%%%%%%%%%%%%%%%%
%%%%%%%%%%%%%%%%%%%%%%%%%%%%%%%%%%%%%%%%%%%%%%%%%%%%%%%%%%%%%%%%%%%%%%%%%%%%%%%%%%%%%%%%%%%%%%%%%%%