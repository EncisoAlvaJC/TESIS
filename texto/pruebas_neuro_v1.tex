%%%%%%%%%%%%%%%%%%%%%%%%%%%%%%%%%%%%%%%%%%%%%%%%%%%%%%%%%%%%%%%%%%%%%%%%%%%%%%%%%%%%%%%%%%%%%%%%%%%
%%%%%%%%%%%%%%%%%%%%%%%%%%%%%%%%%%%%%%%%%%%%%%%%%%%%%%%%%%%%%%%%%%%%%%%%%%%%%%%%%%%%%%%%%%%%%%%%%%%
%%%%%%%%%%%%%%%%%%%%%%%%%%%%%%%%%%%%%%%%%%%%%%%%%%%%%%%%%%%%%%%%%%%%%%%%%%%%%%%%%%%%%%%%%%%%%%%%%%%

\chapter{Puntajes para pruebas neuropsicológicas}
\label{apendice_pruebas}

%En el contexto del presente trabajo fueron usadas varias pruebas neuropsicológicas con el fin de identificar el PDCL (posible deterioro cognitivo leve) en adultos mayores.
%%
%Cabe mencionar que este tipo de pruebas funcionan a nivel de comportamiento, es decir, no pueden detectar cambios en el tejido cerebral o su actividad.

En psicología los instrumentos de medición comunes son las pruebas neuropsicológicas, entendidas como muestras de alguna conducta de interés a las que se asignan puntajes para comparar cuantitativamente a los sujetos \cite{Ardila12}. 
%
%Es a través de estas herramientas que se declaran formalmente las deficiencias cognitivas, así como su severidad y subclasificaciones posteriores.
%
Para fines del presente trabajo, fueron usadas varias pruebas neuropsicológicas con el fin de identificar el PDCL (posible deterioro cognitivo leve) en adultos mayores, además de otras afecciones que relacionadas al diagnostico del PDCL.
%
Concretamente, fueron usadas las siguientes pruebas:
\begin{itemize}
\item {Short Anxiety Screening Test (SAST)} 
\item {Geriatric Depression Scale (GDS)}
\item {Mini--Mental State Examination (MMSE)}
\item {Evaluación Neuropsicológica (Neuropsi)}
\item {Escala sobre las actividades cotidianas de la vida diaria (KATZ)}
\end{itemize}

Para más información, ver sección \ref{seccion:dcl}.
%
A continuación se presentan únicamente los \textit{puntajes de corte}, puntajes para los cuales la evidencia aportada por las pruebas es indicativa de alguna característica.
%: depende de otros para transportarse, presenta problemas en la memoria de corto plazo, etc.
%
Este material fue retirado del texto principal para facilitar su lectura.

\begin{table}
\centering
\caption{Puntajes de corte para las pruebas SAST y GDS}
\begin{tabular}{lcl}
\toprule
Prueba & Puntaje & Indicación \\
\midrule
SAST
& $>24$ & Positivo para ansiedad \\
& 22 -- 24 & No es conclusivo \\
& $<22$ & Negativo para ansiedad \\
\midrule
GDS
& 0 -- 4 & Normal \\
& 5 -- 8 & Depresión leve \\
& 9 -- 11 & Depresión moderada \\
& 12 -- 15 & Depresión severa \\
\bottomrule
\multicolumn{3}{l}{Fuente: Yesavage \cite{Yesavage82}, Sinoff \cite{sinoff99} }
\end{tabular}
\label{anexo:sast_gds}
\end{table}

\begin{table}
\centering
\caption{Puntuación para la prueba KATZ}
\begin{tabular}{lll}
\toprule
\multicolumn{2}{l}{Actividad} & Descripción \\
\midrule
1 & Baño           & Se baña completamente, o necesita ayuda sólo para jabonarse\\
                  && ciertas regiones (espalda, o una extremidad dañada). \\
2 & Vestido & Saca la ropa del closet, se viste y desviste. Se excluye el\\
                  && anudar los cordones. \\
3 & Uso del toilet & Llega al baño, se sienta y para del toilet, se arregla la\\
                  && ropa y se limpia (puede usar su propia chata en la noche y\\
                  && usar soportes mecánicos).\\
4 & Movilidad      & Entra y sale de la cama independientemente, se sienta y\\
                  && para de la silla (puede usar soporte mecánico). \\
5 & Continencia    & Controla totalmente esfinter anal y vesical.\\
6 & Alimentación   & Lleva la comida del plato a la boca (se excluye el cortar\\
                  && la carne o preparar la comida). \\
\bottomrule
\multicolumn{3}{l}{Fuente: Katz \cite{katz70}. Se puntúa la dependencia para cada actividad.}
\end{tabular}
\label{anexo:katz}
\end{table}
% http://medicina.uc.cl/programa-geriatria/indice-katz-independencia-en-actividades-diarias

\begin{table}
\centering
\caption{Puntajes de corte para la prueba Neuropsi}
\begin{tabular}{llccrccc}
\toprule
&& \multicolumn{2}{l}{Sano} & \phantom{.} & \multicolumn{3}{l}{Deterioro cognitivo} \\
\cmidrule{3-4} \cmidrule{6-8} 
Edad & Escolaridad & Alto & Normal && Leve & Moderado & Severo\\
\midrule
31 -- 50  
& Nula     & 95  & 68  &  & 54  & 41 & 28 \\
& 1 -- 4   & 105 & 81  &  & 69  & 58 & 46 \\
& 5 -- 9   & 118 & 106 &  & 101 & 90 & 79 \\
& 10 -- 24 & 113 & 102 &  & 97  & 88 & 78 \\
\midrule
51 -- 65  
& Nula     & 91  & 59  &  & 44  & 28 & 13 \\
& 1 -- 4   & 98  & 77  &  & 67  & 57 & 47 \\
& 5 -- 9   & 111 & 98  &  & 91  & 79 & 67 \\
& 10 -- 24 & 102 & 93  &  & 88  & 80 & 72 \\
\midrule
66 -- 85  
& Nula     & 76  & 48  &  & 34  & 20 & 6 \\
& 1 -- 4   & 90  & 61  &  & 46  & 32 & 18 \\
& 5 -- 9   & 97  & 80  &  & 72  & 56 & 39 \\
& 10 -- 24 & 92  & 78  &  & 72  & 59 & 46 \\
\bottomrule
\multicolumn{7}{l}{Fuente: Ardila y Ostrosky \cite{Ardila12}}
\end{tabular}
\label{anexo:neuropsi}
\end{table}

\begin{table}
\centering
\caption{Puntajes de corte para la prueba MMSE}
\begin{tabular}{llcc}
\toprule
Edad & Nivel de estudios & Máximo & Deterioro \\
\midrule
45 -- 49&Elemental&23&18 \\
&Primario&26&20 \\
&Medio&28&22 \\
&Superior&29&23 \\
\midrule
50 -- 54&Elemental&23&18 \\
&Primario&27&21 \\
&Medio&28&22 \\
&Superior&29&23 \\
\midrule
55 -- 59&Elemental&22&17 \\
&Primario&26&20 \\
&Medio&28&22 \\
&Superior&29&23 \\
\midrule
60 -- 64&Elemental&23&18 \\
&Primario&26&20 \\
&Medio&28&22 \\
&Superior&29&23 \\
\midrule
65 -- 69&Elemental&22&17 \\
&Primario&26&20 \\
&Medio&28&22 \\
&Superior&29&23 \\
\midrule
70 -- 74&Elemental&22&17 \\
&Primario&25&20 \\
&Medio&27&21 \\
&Superior&28&22 \\
\midrule
75 -- 79&Elemental&21&16 \\
&Primario&25&20 \\
&Medio&27&21 \\
&Superior&28&22 \\
\midrule
80 -- 84&Elemental&20&16 \\
&Primario&25&20 \\
&Medio&25&20 \\
&Superior&27&21 \\
%\midrule
%$>84$&Elemental&19&15 \\
%&Primario&23&18 \\
%&Medio&26&20 \\
%&Superior&27&21 \\
\bottomrule
\multicolumn{2}{l}{Fuente: Folstein \cite{crum93}}
\end{tabular}
\label{anexo:mmse}
\end{table}
%https://www.hipocampo.org/folstein.asp

%%%%%%%%%%%%%%%%%%%%%%%%%%%%%%%%%%%%%%%%%%%%%%%%%%%%%%%%%%%%%%%%%%%%%%%%%%%%%%%%%%%%%%%%%%%%%%%%%%%
%%%%%%%%%%%%%%%%%%%%%%%%%%%%%%%%%%%%%%%%%%%%%%%%%%%%%%%%%%%%%%%%%%%%%%%%%%%%%%%%%%%%%%%%%%%%%%%%%%%
%%%%%%%%%%%%%%%%%%%%%%%%%%%%%%%%%%%%%%%%%%%%%%%%%%%%%%%%%%%%%%%%%%%%%%%%%%%%%%%%%%%%%%%%%%%%%%%%%%%
%%%%%%%%%%%%%%%%%%%%%%%%%%%%%%%%%%%%%%%%%%%%%%%%%%%%%%%%%%%%%%%%%%%%%%%%%%%%%%%%%%%%%%%%%%%%%%%%%%%