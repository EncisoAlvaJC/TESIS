%%%%%%%%%%%%%%%%%%%%%%%%%%%%%%%%%%%%%%%%%%%%%%%%%%%%%%%%%%%%%%%%%%%%%%%%%%%%%%%%%%%%%%%%%%%%%%%%%%%
%%%%%%%%%%%%%%%%%%%%%%%%%%%%%%%%%%%%%%%%%%%%%%%%%%%%%%%%%%%%%%%%%%%%%%%%%%%%%%%%%%%%%%%%%%%%%%%%%%%
%%%%%%%%%%%%%%%%%%%%%%%%%%%%%%%%%%%%%%%%%%%%%%%%%%%%%%%%%%%%%%%%%%%%%%%%%%%%%%%%%%%%%%%%%%%%%%%%%%%
%%%%%%%%%%%%%%%%%%%%%%%%%%%%%%%%%%%%%%%%%%%%%%%%%%%%%%%%%%%%%%%%%%%%%%%%%%%%%%%%%%%%%%%%%%%%%%%%%%%

\chapter{Introducción}

Gracias a los avances médicos del último siglo, ha incrementado la esperanza de vida y la calidad
de vida; desafortunadamente también ha aumentado la presencia de enfermedades no-transmisibles 
asociadas con la edad, de las cuales no se han identificado factores causales 
\cite{PlanAlzheimer04}.
%
Particularmente en México, el sector de la población con más de 60 años fue de 10 millones de
personas en 2010, aunque se estima que ha aumentado a 12 millones y medio en 2015 
\cite{Censo10,Intercensal15}.

%En México el sector de la población en riesgo\footnote{En el presente trabajo se consideran a los
%\textbf{adultos mayores}, personas mayores de 70 años} era de 10 millones de personas en 2010 
%\cite{Censo10}, aunque se estima que ha aumentado a 12 millones y medio en 2015 
%\cite{Intercensal15}.

De entre las enfermedades ante las cuales este grupo de edad es vulnerable, en este trabajo se 
destaca la demencia. 
La demencia consiste en el desarrollo de déficit cognoscitivos suficientemente graves como para 
interferir en las actividades laborales y sociales.
%
El deterioro cognitivo característico de la demencia se considera irreverible, debido a lo cual 
ha surgido un gran interés en definir y diagnosticar etapas tempranas de este padecimiento con el 
fin de evitar en lo posible dicho síntoma.

%
Como ejemplo Petersen [?]
%Petersen, R. C., Doody, R., Kurz, A., Mohs, R. C., Morris, J. C., Rabins, P.
%V., ... & Winblad, B. (2001). Current concepts in mild cognitive impairment.
%Archives of neurology, 58(12), 1985-1992.
define una muy detallada escala de subtipos para el deterioro cognitivo, basándose principalmente
en criterios psicológicos orientados a la funcionalidad del paciente;
%
en contraparte Montplaisir \cite{Brayet16} reporta asociaciones entre ejecuciones pobres en pruebas
neuropsicológicas y atrofias en el hipocampo, detectadas usando resonancia magnética funcional.
En el presente trabajo se otorga cierta preferencia a los criterios basados en registros 
electrofisiológicos, ya que son más fáciles de cuantificar; aún así se mantiene una conexión 
estrecha con los criterios neuropsicológicos.

Cabe destacar que 
la convergencia de enfoques entre distintos niveles del sujeto (conductual, sistémico) no sería
posible sin un cuerpo común de modelos y métodos que permitan entender los datos recabados dentro
de un mismo marco conceptual centrado en el sujeto de estudio (el deterioro cognitivo);
es bajo estas circunstancia que emerge el estudio de los métodos de análisis \textit{per se}
como una necesidad ante la innovación tecnológica acelerada, que en fechas recientes ha hecho
de uso común el registro y procesamiento de conjuntos masivos datos.

El principal problema que se aborda en el presente trabajo comienza en que las señales
electrofisiológicas típicamente son complejas, y representan procesos no--lineales y 
no--estacionarios; sin embargo, las herramientas para analizar tales datos a menudo siguen 
asumiendo linealidad y estacionariedad.
A consecuencia de lo anterior, los conjuntos de datos pueden contener
información \textit{oculta} y que puede ser de valor clínico; o información que no es extraída y 
manejada de forma apropiada, y por tanto puede no ser representativa del fenómeno que se estudia. 
%En este trabajo de tesis
%se pretende usar herramientas matemáticas que involucren la no linealidad.
%El análisis fractal es uno de los nuevos enfoques más prometedores para
%extraer información oculta. La cual ayudará a tener una idea más clara de
%lo que puede ocurrir a largo plazo, o bien de los cambios que se están pre-
%sentando en los diferentes contextos de manera más clara y sencilla, con
%resultados concretos y exactos. Proporcionando ası́ una herramienta, que
%permita mostrar cambios en las distintas maniobras experimentales y/o en
%aplicaciones clı́nicas con distintas patologı́as.

En el presente trabajo se exponen herramientas relativas a verificar (en el sentido estadístico)
la estacionariedad débil en series de tiempo, un supuesto básico para calcular el espectro de potencias; esta cantidad,
calculada para el polisomnograma,
ha sido reportada como relacionada con el deterioro cognitivo [?,?,?].

%En algunos estudios de gran escala se han hallado correlaciones entre diferentes trastornos del 
%sueño y algún grado de deterioro cognitivo objetivo en adultos mayores 
%\cite{Amer13,Miyata13,Reid06,Potvin12}; entendiendo por ello una ejecuciones más pobres en tareas
%cognitivas, pero que no impiden llevar a cabo actividades cotidianas.

%En 2016 Vázquez-Tagle y colaboradores estudiaron la epidemiología del deterioro cognitivo en 
%adultos mayores dentro del estado de Hidalgo y su posible relación con trastornos de sueño, 
%encontrando una correlación entre una menor eficiencia del sueño\footnote{Porcentaje de tiempo
%de sueño respecto al tiempo en cama} y la presencia de deterioro cognitivo \cite{VazquezTagle16}.

%En un segundo trabajo por García-Muñoz y colaboradores \cite{Valeria} se analizaron 
%registros de polisomnograma (PSG) 
%%datos de PSG 
%para detectar posibles cambios en la conectividad funcional del cerebro\footnote{La 
%\textbf{conectividad funcional} se refiere una \textit{fuerte} relación (cuantificada) entre dos 
%señales, y usualmente se contrasta con la \textbf{conectividad anatómica}, entendida como conexiones 
%físicas entre los generadores de señales} en adultos mayores con posible deterioro 
%cognitivo (PDC), reportando un mayor exponente de Hurst para registros de PSG en adultos mayores 
%con PDC \cite{Valeria}.
%El exponente de Hurst, calculado a través del algoritmo \textit{Detrended Fluctuation Analysis}, 
%está relacionado con las correlaciones de largo alcance y la estructura fractal de una serie de 
%tiempo, siendo que un mayor exponente está asociado con señales cuya función de 
%autocorrelación decrece más lentamente \cite{Rodriguez11}.
%Con base a que en aquellos trabajos se ha supuesto que los registros de PSG son no-estacionarios, 
%en este trabajo se pretende verificar si efectivamente estas señales se pueden considerar con tal
%característica.

%El supuesto de estacionariedad es básico en el estudio de series de tiempo, y usualmente se 
%acepta o rechaza sin un tratamiento formal; es de particular importancia, por ejemplo, para 
%calcular el espectro de potencias a partir de registros.
%La idea de que sujetos con PDC exhiben estacionariedad débil en sus registros de EEG en mayor 
%proporción, respecto a individuos sanos, fue sugerida por Cohen \cite{Cohen77}, quien a su vez se 
%refiere a trabajos anteriores sobre estacionariedad y normalidad en registros de EEG 
%\cite{McEwen75,Sugimoto78,Kawabata73}.


%%%%%%%%%%%%%%%%%%%%%%%%%%%%%%%%%%%%%%%%%

%El presente trabajo resulta de una colaboración con el departamento de Geron-
%tologı́a, dependiente del Instituto de Ciencias de la Salud (ICSA); parte de esta
%colaboración incluye el acceso a los registros de PSG obtenidos por Vázquez Tagle y
%colaboradores [40].

\section{Antecedentes}

En algunos estudios de gran escala se han hallado correlaciones entre diferentes trastornos del 
sueño y algún grado de deterioro cognitivo objetivo en adultos mayores 
\cite{Amer13,Miyata13,Reid06,Potvin12}; entendiendo por ello una ejecuciones más pobres en tareas
cognitivas, pero que no impiden llevar a cabo actividades cotidianas.

En 2016 Vázquez-Tagle y colaboradores estudiaron la epidemiología del deterioro cognitivo en 
adultos mayores dentro del estado de Hidalgo y su posible relación con trastornos de sueño, 
encontrando una correlación entre una menor eficiencia del sueño\footnote{Porcentaje de tiempo
de sueño respecto al tiempo en cama} y la presencia de deterioro cognitivo \cite{VazquezTagle16}.

En un segundo trabajo por García-Muñoz y colaboradores \cite{Valeria} se analizaron 
registros de polisomnograma (PSG) 
%datos de PSG 
para detectar posibles cambios en la conectividad funcional del cerebro\footnote{La 
\textbf{conectividad funcional} se refiere una \textit{fuerte} relación (cuantificada) entre dos 
señales, y usualmente se contrasta con la \textbf{conectividad anatómica}, entendida como conexiones 
físicas entre los generadores de señales} en adultos mayores con posible deterioro 
cognitivo (PDC), reportando un mayor exponente de Hurst para registros de PSG en adultos mayores 
con PDC \cite{Valeria}.
El exponente de Hurst, calculado a través del algoritmo \textit{Detrended Fluctuation Analysis}, 
está relacionado con las correlaciones de largo alcance y la estructura fractal de una serie de 
tiempo, siendo que un mayor exponente está asociado con señales cuya función de 
autocorrelación decrece más lentamente \cite{Rodriguez11}.
Con base a que en aquellos trabajos se ha supuesto que los registros de PSG son no-estacionarios, 
en este trabajo se pretende verificar si efectivamente estas señales se pueden considerar con tal
característica.

El supuesto de estacionariedad es básico en el estudio de series de tiempo, y usualmente se 
acepta o rechaza sin un tratamiento formal; es de particular importancia, por ejemplo, para 
calcular el espectro de potencias a partir de registros.
La idea de que sujetos con PDC exhiben estacionariedad débil en sus registros de EEG en mayor 
proporción, respecto a individuos sanos, fue sugerida por Cohen \cite{Cohen77}, quien a su vez se 
refiere a trabajos anteriores sobre estacionariedad y normalidad en registros de EEG 
\cite{McEwen75,Sugimoto78,Kawabata73}.
%Cabe mencionar que en estos primeros estudios se palpa la posibilidad de que los registros de EEG 
%fueran 'ruido' de algún tipo, una idea que se ha probado errónea en estudios más recientes 
%\cite{Klonowski09}; sin embargo, se retoma como hipótesis a la luz de los estudios mencionados. 

%%%%%%%%%%%%%%%%%%%%%%%%%%%%%%%%%%%%%%%%%%%%%%%%%%%%%%%%%%%%%%%%%%%%%%%%%%%%%%%%%%%%%%%%%%%%%%%%%%%
%%%%%%%%%%%%%%%%%%%%%%%%%%%%%%%%%%%%%%%%%%%%%%%%%%%%%%%%%%%%%%%%%%%%%%%%%%%%%%%%%%%%%%%%%%%%%%%%%%%

%\section{Justificación}

%%%%%%%%%%%%%%%%%%%%%%%%%%%%%%%%%%%%%%%%%%%%%%%%%%%%%%%%%%%%%%%%%%%%%%%%%%%%%%%%%%%%%%%%%%%%%%%%%%%
%%%%%%%%%%%%%%%%%%%%%%%%%%%%%%%%%%%%%%%%%%%%%%%%%%%%%%%%%%%%%%%%%%%%%%%%%%%%%%%%%%%%%%%%%%%%%%%%%%%

\section{Pregunta de investigación}

¿Los registros de PSG\footnote{Polisomnograma: actividad eléctrica del cerebro durante el sueño,
además de otros marcadores como la actividad ocular o la respiración} en adultos mayores, pueden
considerarse como series tiempo débilmente estacionarias?
¿Es posible que tal caracterización se relacione con el estado cognoscitivo del adulto mayor?

%%%%%%%%%%%%%%%%%%%%%%%%%%%%%%%%%%%%%%%%%%%%%%%%%%%%%%%%%%%%%%%%%%%%%%%%%%%%%%%%%%%%%%%%%%%%%%%%%%%
%%%%%%%%%%%%%%%%%%%%%%%%%%%%%%%%%%%%%%%%%%%%%%%%%%%%%%%%%%%%%%%%%%%%%%%%%%%%%%%%%%%%%%%%%%%%%%%%%%%

\subsection{Hipótesis}

Existen diferencias en la conectividad funcional del cerebro en adultos mayores con PDC, respecto
a sujetos sanos, y es posible detectar estas diferencias como una mayor o menor 'presencia' de 
estacionariedad débil en registros de PSG durante el sueño profundo.

%%%%%%%%%%%%%%%%%%%%%%%%%%%%%%%%%%%%%%%%%%%%%%%%%%%%%%%%%%%%%%%%%%%%%%%%%%%%%%%%%%%%%%%%%%%%%%%%%%%

\subsection{Objetivo general}

Deducir, a partir de pruebas estadísticas formales, las presencia de estacionariedad débil en
registros de PSG para adultos mayores con PDC, así como individuos control.

%%%%%%%%%%%%%%%%%%%%%%%%%%%%%%%%%%%%%%%%%%%%%%%%%%%%%%%%%%%%%%%%%%%%%%%%%%%%%%%%%%%%%%%%%%%%%%%%%%%

\subsection{Objetivos específicos}

\begin{itemize}
\item Estudiar la definición de estacionariedad para procesos estocásticos y sus posibles 
consecuencias dentro de un modelo para los datos considerados

\item Investigar en la literatura cómo detectar si es plausible que una serie de tiempo dada sea 
una realización para un proceso estocástico débilmente estacionario, y bajo qué supuestos 
es válida esta caracterización

\item Usando los análisis hallados en la literatura, determinar si las series de tiempo 
obtenidas a partir de los datos considerados provienen de procesos débilmente estacionarios.
Revisar si la información obtenida en los diferentes sujetos muestra diferencias entre sujetos 
con y sin PDC
\end{itemize}

%%%%%%%%%%%%%%%%%%%%%%%%%%%%%%%%%%%%%%%%%%%%%%%%%%%%%%%%%%%%%%%%%%%%%%%%%%%%%%%%%%%%%%%%%%%%%%%%%%%
%%%%%%%%%%%%%%%%%%%%%%%%%%%%%%%%%%%%%%%%%%%%%%%%%%%%%%%%%%%%%%%%%%%%%%%%%%%%%%%%%%%%%%%%%%%%%%%%%%%
%%%%%%%%%%%%%%%%%%%%%%%%%%%%%%%%%%%%%%%%%%%%%%%%%%%%%%%%%%%%%%%%%%%%%%%%%%%%%%%%%%%%%%%%%%%%%%%%%%%
%%%%%%%%%%%%%%%%%%%%%%%%%%%%%%%%%%%%%%%%%%%%%%%%%%%%%%%%%%%%%%%%%%%%%%%%%%%%%%%%%%%%%%%%%%%%%%%%%%%
