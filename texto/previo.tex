%%%%%%%%%%%%%%%%%%%%%%%%%%%%%%%%%%%%%%%%%%%%%%%%%%%%%%%%%%%%%%%%%%%%%%%%%%%%%%%%%%%%%%%%%%%%%%%%%%%
%%%%%%%%%%%%%%%%%%%%%%%%%%%%%%%%%%%%%%%%%%%%%%%%%%%%%%%%%%%%%%%%%%%%%%%%%%%%%%%%%%%%%%%%%%%%%%%%%%%
%%%%%%%%%%%%%%%%%%%%%%%%%%%%%%%%%%%%%%%%%%%%%%%%%%%%%%%%%%%%%%%%%%%%%%%%%%%%%%%%%%%%%%%%%%%%%%%%%%%
%%%%%%%%%%%%%%%%%%%%%%%%%%%%%%%%%%%%%%%%%%%%%%%%%%%%%%%%%%%%%%%%%%%%%%%%%%%%%%%%%%%%%%%%%%%%%%%%%%%

\chapter{Antecedentes}

En 2016, V\'azquez-Tagle y colaboradores \cite{VazquezTagle16} estudiaron la epidemiolog\'ia del 
deterioro cognitivo en adultos mayores dentro del estado de Hidalgo; en aqu\'el estudio se 
efectuaron registros polisomnogr\'aficos (PSG) y se confirm\'o una relaci\'on entre una menor 
eficiencia del sue\~no\footnote{Se espera que el sue\~no cumpla sus funciones habituales si se 
desarrolla con normalidad, luego entonces se puede medir si \'este exhibe una estructura t\'ipica} 
y la presencia de deterioro cognitivo.
En un segundo trabajo por Garc\'ia-Mu\~noz y colaboradores \cite{Valeria} se analizaron datos de 
PSG para detectar posibles cambios en la conectividad funcional del cerebro\footnote{Se suele 
hablar de \textbf{conectividad funcional} cuando las se\~nales registradas en dos lugares est\'an 
estad\'isticamente 'muy' interrelacionadas; este t\'ermino se contrapone al de \textbf{conectividad 
anat\'omica}, que se refiere a conexiones f\'isicas} en adultos mayores con posible deterioro 
cognitivo (PDC), reportando un mayor exponente de Hurst para registros de PSG en adultos mayores 
con PDC.
El exponente de Hurst, calculado a trav\'es del algoritmo \textit{Detrended Fluctuation Analysis}, 
est\'a relacionado con la 'estructura fractal' de una serie de tiempo (realizaci\'on de un proceso 
estoc\'astico) de modo que un mayor exponente est\'a asociado con se\~nales cuya funci\'on de 
autocorrelaci\'on decrece m\'as lentamente \cite{Rodriguez11}.
En base a que en \cite{Valeria} se ha supuesto que los registros de PSG son no-estacionarios, en 
este trabajo se pretende verificar si efectivamente estas se\~nales se pueden considerar con tal
caracter\'istica.

El supuesto de estacionariedad es b\'asico en el estudio de series de tiempo, y usualmente se 
acepta o rechaza sin un tratamiento formal; es de particular importancia, por ejemplo, para 
calcular el espectro de potencias.
La idea de que sujetos con PDC exhiben estacionariedad d\'ebil en sus registros de EEG en mayor 
proporci\'on, respecto a individuos sanos, fue sugerida por Cohen \cite{Cohen77}, quien a su vez se 
refiere a trabajos anteriores sobre estacionariedad y normalidad en registros de EEG 
\cite{McEwen75,Sugimoto78,Kawabata73}.
Cabe mencionar que en estos primeros estudios se palpa la posibilidad de que los registros de EEG 
fueran 'ruido' de alg\'un tipo, una idea que se ha probado err\'onea en estudios m\'as recientes 
\cite{Klonowski09}; sin embargo, se retoma como hip\'otesis a la luz de los estudios mencionados. 

%%%%%%%%%%%%%%%%%%%%%%%%%%%%%%%%%%%%%%%%%%%%%%%%%%%%%%%%%%%%%%%%%%%%%%%%%%%%%%%%%%%%%%%%%%%%%%%%%%%
%%%%%%%%%%%%%%%%%%%%%%%%%%%%%%%%%%%%%%%%%%%%%%%%%%%%%%%%%%%%%%%%%%%%%%%%%%%%%%%%%%%%%%%%%%%%%%%%%%%

\section{Justificaci\'on}

Los avances m\'edicos del \'ultimo siglo se han traducido en un incremento tanto en la esperanza de 
vida como en la calidad de la misma. 
De acuerdo a la Encuesta Intercensal 2015 realizada por INEGI \cite{Intercensal15}, se estima que
en M\'exico habitan cerca de 12,500,000 adultos mayores, lo que representa un 10.4 \%  de la 
poblaci\'on.
Lamentablemente, tambi\'en se ve incrementada la presencia de enfermedades no-transmisibles, de 
entre las cuales destacamos la demencia.
El cuidado de enfermedades cr\'onicas en la poblaci\'on de edad avanzada representa un gran peso 
econ\'omico y de recursos humanos, que recae sobre el sistema de salud y los familiares de los 
afectados; por ello, cobra importancia un diagn\'ostico temprano del deterioro cognitivo que 
disminuya el riesgo de su avance irreversible a demencia.

Todav\'ia son incipientes las investigaciones para identificar los factores de riesgo modificables 
asociados a la demencia \cite{PlanAlzheimer04}; recientemente, los trastornos del sue\~no han sido 
se\~nalados como posiblemente relacionados con el deterioro cognitivo durante la vejez 
\cite{Amer13,Miyata13,Potvin12}. Concretamente, una duraci\'on menor del sue\~no nocturno y una 
mala eficiencia del mismo, en personas mayores, se relaciona con una peor ejecuci\'on en tareas de 
memoria \cite{Reid06}. Las afectaciones relativas al sue\~no en personas mayores podr\'ian ser 
m\'as problem\'aticas que para otros grupos de edad \cite{Potvin12}.

%%%%%%%%%%%%%%%%%%%%%%%%%%%%%%%%%%%%%%%%%%%%%%%%%%%%%%%%%%%%%%%%%%%%%%%%%%%%%%%%%%%%%%%%%%%%%%%%%%%
%%%%%%%%%%%%%%%%%%%%%%%%%%%%%%%%%%%%%%%%%%%%%%%%%%%%%%%%%%%%%%%%%%%%%%%%%%%%%%%%%%%%%%%%%%%%%%%%%%%

\section{Pregunta de investigaci\'on}

¿Es posible que la caracterizaci\'on de registros de PSG como series de tiempo d\'ebilmente 
estacionarias, pueda ser usada como un marcador en el diagn\'ostico cl\'inico de PDC en adultos 
mayores?

%%%%%%%%%%%%%%%%%%%%%%%%%%%%%%%%%%%%%%%%%%%%%%%%%%%%%%%%%%%%%%%%%%%%%%%%%%%%%%%%%%%%%%%%%%%%%%%%%%%
%%%%%%%%%%%%%%%%%%%%%%%%%%%%%%%%%%%%%%%%%%%%%%%%%%%%%%%%%%%%%%%%%%%%%%%%%%%%%%%%%%%%%%%%%%%%%%%%%%%

\subsection{Hip\'otesis}

Existen diferencias en la conectividad funcional del cerebro en adultos mayores con PDC, respecto
a sujetos sanos, y es posible detectar estas diferencias como una mayor o menor 'presencia' de 
estacionariedad d\'ebil en registros de PSG durante el sue\~no profundo.

%%%%%%%%%%%%%%%%%%%%%%%%%%%%%%%%%%%%%%%%%%%%%%%%%%%%%%%%%%%%%%%%%%%%%%%%%%%%%%%%%%%%%%%%%%%%%%%%%%%

\subsection{Objetivo general}

Deducir, a partir de pruebas estad\'isticas formales, las presencia de estacionariedad d\'ebil en
registros de PSG para adultos mayores con PDC, as\'i como individuos control.

%%%%%%%%%%%%%%%%%%%%%%%%%%%%%%%%%%%%%%%%%%%%%%%%%%%%%%%%%%%%%%%%%%%%%%%%%%%%%%%%%%%%%%%%%%%%%%%%%%%

\subsection{Objetivos espec\'ificos}

\begin{itemize}
\item Estudiar la definici\'on de estacionariedad para procesos estoc\'asticos y sus posibles 
consecuencias dentro de un modelo para los datos considerados

\item Investigar en la literatura c\'omo detectar si es plausible que una serie de tiempo dada sea 
una realizaci\'on para un proceso estoc\'astico d\'ebilmente estacionario, y bajo qu\'e supuestos 
es v\'alida esta caracterizaci\'on

\item Usando los an\'alisis hallados en la literatura, determinar si las series de tiempo 
obtenidas a partir de los datos considerados provienen de procesos d\'ebilmente estacionarios.
Revisar si la informaci\'on obtenida en los diferentes sujetos muestra diferencias entre sujetos 
con y sin PDC
\end{itemize}

%%%%%%%%%%%%%%%%%%%%%%%%%%%%%%%%%%%%%%%%%%%%%%%%%%%%%%%%%%%%%%%%%%%%%%%%%%%%%%%%%%%%%%%%%%%%%%%%%%%
%%%%%%%%%%%%%%%%%%%%%%%%%%%%%%%%%%%%%%%%%%%%%%%%%%%%%%%%%%%%%%%%%%%%%%%%%%%%%%%%%%%%%%%%%%%%%%%%%%%
%%%%%%%%%%%%%%%%%%%%%%%%%%%%%%%%%%%%%%%%%%%%%%%%%%%%%%%%%%%%%%%%%%%%%%%%%%%%%%%%%%%%%%%%%%%%%%%%%%%
%%%%%%%%%%%%%%%%%%%%%%%%%%%%%%%%%%%%%%%%%%%%%%%%%%%%%%%%%%%%%%%%%%%%%%%%%%%%%%%%%%%%%%%%%%%%%%%%%%%
