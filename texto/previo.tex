%%%%%%%%%%%%%%%%%%%%%%%%%%%%%%%%%%%%%%%%%%%%%%%%%%%%%%%%%%%%%%%%%%%%%%%%%%%%%%%%%%%%%%%%%%%%%%%%%%%
%%%%%%%%%%%%%%%%%%%%%%%%%%%%%%%%%%%%%%%%%%%%%%%%%%%%%%%%%%%%%%%%%%%%%%%%%%%%%%%%%%%%%%%%%%%%%%%%%%%
%%%%%%%%%%%%%%%%%%%%%%%%%%%%%%%%%%%%%%%%%%%%%%%%%%%%%%%%%%%%%%%%%%%%%%%%%%%%%%%%%%%%%%%%%%%%%%%%%%%
%%%%%%%%%%%%%%%%%%%%%%%%%%%%%%%%%%%%%%%%%%%%%%%%%%%%%%%%%%%%%%%%%%%%%%%%%%%%%%%%%%%%%%%%%%%%%%%%%%%

\chapter{Preliminares}

\section{Antecedentes}

En algunos estudios de gran escala se han hallado correlaciones entre diferentes trastornos del 
sueño y algún grado de deterioro cognitivo objetivo en adultos mayores 
\cite{Amer13,Miyata13,Reid06,Potvin12}; entendiendo por ello una ejecuciones más pobres en tareas
cognitivas, pero que no impiden llevar a cabo actividades cotidianas.

En 2016 Vázquez-Tagle y colaboradores estudiaron la epidemiología del deterioro cognitivo en 
adultos mayores dentro del estado de Hidalgo y su posible relación con trastornos de sueño, 
encontrando una correlación entre una menor eficiencia del sueño\footnote{Porcentaje de tiempo
de sueño respecto al tiempo en cama} y la presencia de deterioro cognitivo \cite{VazquezTagle16}.

En un segundo trabajo por García-Muñoz y colaboradores \cite{Valeria} se analizaron 
registros de polisomnograma (PSG) 
%datos de PSG 
para detectar posibles cambios en la conectividad funcional del cerebro\footnote{La 
\textbf{conectividad funcional} se refiere una \textit{fuerte} relación (cuantificada) entre dos 
señales, y usualmente se contrasta con la \textbf{conectividad anatómica}, entendida como conexiones 
físicas entre los generadores de señales} en adultos mayores con posible deterioro 
cognitivo (PDC), reportando un mayor exponente de Hurst para registros de PSG en adultos mayores 
con PDC \cite{Valeria}.
El exponente de Hurst, calculado a través del algoritmo \textit{Detrended Fluctuation Analysis}, 
está relacionado con las correlaciones de largo alcance y la estructura fractal de una serie de 
tiempo, siendo que un mayor exponente está asociado con señales cuya función de 
autocorrelación decrece más lentamente \cite{Rodriguez11}.
Con base a que en aquellos trabajos se ha supuesto que los registros de PSG son no-estacionarios, 
en este trabajo se pretende verificar si efectivamente estas señales se pueden considerar con tal
característica.

El supuesto de estacionariedad es básico en el estudio de series de tiempo, y usualmente se 
acepta o rechaza sin un tratamiento formal; es de particular importancia, por ejemplo, para 
calcular el espectro de potencias a partir de registros.
La idea de que sujetos con PDC exhiben estacionariedad débil en sus registros de EEG en mayor 
proporción, respecto a individuos sanos, fue sugerida por Cohen \cite{Cohen77}, quien a su vez se 
refiere a trabajos anteriores sobre estacionariedad y normalidad en registros de EEG 
\cite{McEwen75,Sugimoto78,Kawabata73}.
%Cabe mencionar que en estos primeros estudios se palpa la posibilidad de que los registros de EEG 
%fueran 'ruido' de algún tipo, una idea que se ha probado errónea en estudios más recientes 
%\cite{Klonowski09}; sin embargo, se retoma como hipótesis a la luz de los estudios mencionados. 

%%%%%%%%%%%%%%%%%%%%%%%%%%%%%%%%%%%%%%%%%%%%%%%%%%%%%%%%%%%%%%%%%%%%%%%%%%%%%%%%%%%%%%%%%%%%%%%%%%%
%%%%%%%%%%%%%%%%%%%%%%%%%%%%%%%%%%%%%%%%%%%%%%%%%%%%%%%%%%%%%%%%%%%%%%%%%%%%%%%%%%%%%%%%%%%%%%%%%%%

%\section{Justificación}

%%%%%%%%%%%%%%%%%%%%%%%%%%%%%%%%%%%%%%%%%%%%%%%%%%%%%%%%%%%%%%%%%%%%%%%%%%%%%%%%%%%%%%%%%%%%%%%%%%%
%%%%%%%%%%%%%%%%%%%%%%%%%%%%%%%%%%%%%%%%%%%%%%%%%%%%%%%%%%%%%%%%%%%%%%%%%%%%%%%%%%%%%%%%%%%%%%%%%%%

\section{Pregunta de investigación}

¿Los registros de PSG\footnote{Polisomnograma: actividad eléctrica del cerebro durante el sueño,
además de otros marcadores como la actividad ocular o la respiración} en adultos mayores, pueden
considerarse como series tiempo débilmente estacionarias?
¿Es posible que tal caracterización se relacione con el estado cognoscitivo del adulto mayor?

%%%%%%%%%%%%%%%%%%%%%%%%%%%%%%%%%%%%%%%%%%%%%%%%%%%%%%%%%%%%%%%%%%%%%%%%%%%%%%%%%%%%%%%%%%%%%%%%%%%
%%%%%%%%%%%%%%%%%%%%%%%%%%%%%%%%%%%%%%%%%%%%%%%%%%%%%%%%%%%%%%%%%%%%%%%%%%%%%%%%%%%%%%%%%%%%%%%%%%%

\subsection{Hipótesis}

Existen diferencias en la conectividad funcional del cerebro en adultos mayores con PDC, respecto
a sujetos sanos, y es posible detectar estas diferencias como una mayor o menor 'presencia' de 
estacionariedad débil en registros de PSG durante el sueño profundo.

%%%%%%%%%%%%%%%%%%%%%%%%%%%%%%%%%%%%%%%%%%%%%%%%%%%%%%%%%%%%%%%%%%%%%%%%%%%%%%%%%%%%%%%%%%%%%%%%%%%

\subsection{Objetivo general}

Deducir, a partir de pruebas estadísticas formales, las presencia de estacionariedad débil en
registros de PSG para adultos mayores con PDC, así como individuos control.

%%%%%%%%%%%%%%%%%%%%%%%%%%%%%%%%%%%%%%%%%%%%%%%%%%%%%%%%%%%%%%%%%%%%%%%%%%%%%%%%%%%%%%%%%%%%%%%%%%%

\subsection{Objetivos específicos}

\begin{itemize}
\item Estudiar la definición de estacionariedad para procesos estocásticos y sus posibles 
consecuencias dentro de un modelo para los datos considerados

\item Investigar en la literatura cómo detectar si es plausible que una serie de tiempo dada sea 
una realización para un proceso estocástico débilmente estacionario, y bajo qué supuestos 
es válida esta caracterización

\item Usando los análisis hallados en la literatura, determinar si las series de tiempo 
obtenidas a partir de los datos considerados provienen de procesos débilmente estacionarios.
Revisar si la información obtenida en los diferentes sujetos muestra diferencias entre sujetos 
con y sin PDC
\end{itemize}

%%%%%%%%%%%%%%%%%%%%%%%%%%%%%%%%%%%%%%%%%%%%%%%%%%%%%%%%%%%%%%%%%%%%%%%%%%%%%%%%%%%%%%%%%%%%%%%%%%%
%%%%%%%%%%%%%%%%%%%%%%%%%%%%%%%%%%%%%%%%%%%%%%%%%%%%%%%%%%%%%%%%%%%%%%%%%%%%%%%%%%%%%%%%%%%%%%%%%%%
%%%%%%%%%%%%%%%%%%%%%%%%%%%%%%%%%%%%%%%%%%%%%%%%%%%%%%%%%%%%%%%%%%%%%%%%%%%%%%%%%%%%%%%%%%%%%%%%%%%
%%%%%%%%%%%%%%%%%%%%%%%%%%%%%%%%%%%%%%%%%%%%%%%%%%%%%%%%%%%%%%%%%%%%%%%%%%%%%%%%%%%%%%%%%%%%%%%%%%%
