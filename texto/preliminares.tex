%%%%%%%%%%%%%%%%%%%%%%%%%%%%%%%%%%%%%%%%%%%%%%%%%%%%%%%%%%%%%%%%%%%%%%%%%%%%%%%%%%%%%%%%%%%%%%%%%%%
%%%%%%%%%%%%%%%%%%%%%%%%%%%%%%%%%%%%%%%%%%%%%%%%%%%%%%%%%%%%%%%%%%%%%%%%%%%%%%%%%%%%%%%%%%%%%%%%%%%
%%%%%%%%%%%%%%%%%%%%%%%%%%%%%%%%%%%%%%%%%%%%%%%%%%%%%%%%%%%%%%%%%%%%%%%%%%%%%%%%%%%%%%%%%%%%%%%%%%%

\chapter{Preliminares}

Redactar una introducción [?]

%%%%%%%%%%%%%%%%%%%%%%%%%%%%%%%%%%%%%%%%%%%%%%%%%%%%%%%%%%%%%%%%%%%%%%%%%%%%%%%%%%%%%%%%%%%%%%%%%%%

\section{Medidas}

\begin{definicion}%[$\boldsymbol{\sigma}$-álgebra]
Sea $\Omega$ un conjunto y sea $\mathcal{U}$ una familia de subconjuntos de $\Omega$. Se dice que 
$\mathcal{U}$ es una \textbf{$\boldsymbol{\sigma}$-álgebra} si cumple
\begin{itemize}
\item $\Omega \in \mathcal{U}$
\item $A \in \mathcal{U} \Rightarrow A^{C} \in \mathcal{U}$
\item 
$ \displaystyle \{ A_n \}_{n\in \mathbb{N}} \subseteq \mathcal{U} 
\Rightarrow \cup_{n\in \mathbb{N}} A_n \in \mathcal{U}$
\end{itemize}
Donde $A^{C}$ es el complemento de $A$ en $U$
\end{definicion}

Los elementos de una $\sigma$-álgebra se denominan \textbf{conjuntos medibles}. 

\begin{definicion}
Sea $\Omega$ un conjunto y $\mathcal{A} \subseteq \Omega$ una familia de subconjuntos. Se define a $\sigma(\mathcal{A})$, la $\sigma$-álgebra generada por $\mathcal{A}$, como la intersección de todas las $\sigma$-álgebras que contienen a $\mathcal{A}$  
\end{definicion}

En el contexto de la probabilidad, es particularmente importante la $\sigma$-álgebra de Borel, definida como
\begin{equation}
\mathcal{B} = \sigma\left( \left\{ \left( -\infty , a \right] \subset \R \lvert a\in \R \right\} \right)
\end{equation}
Este tipo de $\sigma$-álgebras puede definirse sencillamente para algún subconjunto $A \subset \R$
\begin{equation}
\mathcal{B}_A = \sigma\left( \left\{ \left( -\infty , a \right] \cap A \subset \R \lvert a\in \R \right\} \right)
\end{equation}

\begin{definicion}
Sea $\Omega$ un conjunto y $\mathcal{U}$ una $\sigma$-álgebra definida en $\Omega$. El par $(\Omega,\mathcal{U})$ será referido como \textbf{espacio de medida}. Por nomenclatura, $\Omega$ es referido como \textit{espacio muestral} y $\mathcal{U}$ como \textit{$\sigma$-álgebra de sucesos}.
\end{definicion}

\begin{definicion}%[Medida]
Sea $(\Omega, \mathcal{U})$ un espacio de medida. Se dice que una función $\mu : \mathcal{U} \rightarrow \R_+$ es una \textbf{medida} si cumple que
\begin{itemize}
\item $\mu(\emptyset) = 0$
\item Si $\{ A_n \}_{n\in \mathbb{N}} \subseteq \mathcal{U}$ son tales que 
$A_n \cap A_m = \emptyset \Leftrightarrow m\neq n$, entonces
$$ \mu\left( \bigcup_{n\in \mathbb{N}} A_n \right) = \sum_{n\in \mathbb{N}} \mu(A_n)$$
\end{itemize}
Donde $\R_+ = \{ x\in \R | 0 \leq x \} \cup \{ \infty \}$ y $\emptyset$ es el conjunto vacío.
La terna $(\Omega,\mathcal{U},\mu)$ será referida como \textbf{espacio de medida}.
\label{medida}
\end{definicion}

\begin{definicion}%[Medida $\boldsymbol{\sigma}$-finita]
Sea $(\Omega,\mathcal{U},\mu)$ un espacio de medida. Se dice que $\mu$ es 
\textbf{$\boldsymbol{\sigma}$-finita} si existen una familia de conjuntos medibles $\{ A_n \}_{n\in \mathbb{N}}$ tales que
\begin{itemize}
\item $\mu\left( A_n \right) < \infty$
\item $ \bigcup_{n\in \mathbb{N}} A_n = \Omega$
\end{itemize}
\end{definicion}

\begin{definicion}
Considérese el espacio medible $(\R, \mathcal{B})$, con $\mathcal{B}$ la $\sigma$-álgebra de Borel. Se define la medida de Lebesgue, $\mu_L$, la medida en el espacio mencionado tal que 
\begin{equation}
\mu_L([a,b]) = b-a
\end{equation}
para cualesquiera $a,b \in \R$ con $a<b$
\end{definicion}

\begin{proposicion}
Sea $(\Omega,\mathcal{U},P)$ un espacio de probabilidad. Sea $\{ A_n \}_{n\in \mathbb{N}}$ una amilia de conjuntos medibles tales que 
\begin{itemize}
\item $A_n \subset A_{n+1}$ para todo $n \in \N$
\item Existe un conjunto medible $A$ tal que $\cup_{n\in \N} A_n = A$
\end{itemize}
Entonces
\begin{equation}
\lim_{n\rightarrow \infty} P(A_n) = P(A)
\end{equation}
\end{proposicion}
\begin{proof}
Nótese que
\begin{align*}
P(A) &= P\left( \cup_{n\in \N} A_n \right) \\
\end{align*}
\end{proof}

\subsection{Integración en espacios medibles}

\begin{definicion}
Sea $(\Omega, \mathcal{U}, \mu)$ un espacio de medida y sea $f:\omega \rightarrow \R_+$ una función medible no-negativa. Sea $A\in \mathcal{U}$ un conjunto arbitrario y $\mathcal{C}_A \subset \mathcal{U}$ el conjunto de las particiones de $A$ en una cantidad finita de conjuntos medibles.
Se define la \textbf{integral de $f$ respecto a $\mu$ en el conjunto $A$} como
\begin{equation}
\int_A f(x) \mu(x) := \sup_{\mathcal{C}_A} \left[ \sum_{j=1}^{n} f(\lambda) \mu(E_m) \right]
\end{equation}
Donde $\mathcal{C}_A = \{ E_1, E_2, \dots, E_n \}$
\end{definicion}

\begin{definicion}
Sea $(\Omega, \mathcal{U}, \mu)$ un espacio de medida y sea $f:\omega \rightarrow \R_+$ una función medible. Se definen las funciones $f^{+}$ y $f^{-}$ como
\begin{align*}
f^{+}(x) &= \max (f(x), 0 ) \\
f^{-}(x) &= -\min (f(x), 0 )
\end{align*}
Se dice que $f$ es integrable en $A$ respecto a $\mu$ si cumple que $\int_A f^{+}(\lambda) d\mu(\lambda) < \infty$ y $\int_A f^{-}(\lambda) d\mu(\lambda) < \infty$; si así fuere, se define
\begin{equation}
\int_A f(\lambda) d\mu(\lambda) := \int_A f^{+}(\lambda) d\mu(\lambda) - \int_A f^{-}(\lambda) d\mu(\lambda)
\end{equation}
\end{definicion}

\begin{definicion}
Sean $(\Omega_1,\mathcal{U}_1)$ y $(\Omega_2,\mathcal{U}_2)$ espacios medibles y $\mu$ una medida sobre el primero. Una función $g$ es integrable en $A \in \mathcal{U_2}$ respecto a $\mu_f$ si y sólo si $g \circ f$ es integrable en $f^{-1}(A)$ respecto a $\mu$
\end{definicion}
\begin{proof}
[?] pagina 47 del manual
\end{proof}

%\begin{proposicion}
%Sea $(\Omega, \mathcal{U}, \mu)$ un espacio de medida y sea $f:\omega \rightarrow \R_+$ una función medible no-negativa. La función $\mu_f$ definida como
%\begin{equation}
%\mu_f(A) := \int_A f(\lambda) \mu(\lambda)
%\end{equation}
%es una medida en el espacio medible $(\Omega, \mathcal{U})$
%\end{proposicion}
%
%\begin{proof}
%[?] Si se necesita, en la pagina 39 el manual
%\end{proof}

%\begin{proposicion}
%Sean $f$ y $g$ funciones integrables en $A$ con respecto a $\mu$ y sean $\alpha, \beta \in \R$ arbitrarios. Se cumple que la función $(\alpha f + \beta g)$ es integrable y
%\begin{equation}
%\int_A \left[ \alpha f + \beta g \right] (\lambda) \mu(\lambda) = \alpha \int_A f(\lambda) \mu(\lambda) + \beta \int_A 
%\end{equation}
%\end{proposicion}

%%%%%%%%%%%%%%%%%%%%%%%%%%%%%%%%%%%%%%%%%%%%%%%%%%%%%%%%%%%%%%%%%%%%%%%%%%%%%%%%%%%%%%%%%%%%%%%%%%%

\section{Variables aleatorias}

Si una medida $\mu$ es acotada en todo el espacio de eventos se dice que es una \textbf{medida finita} (no confundir con $\sigma$-finita). 
%
Una medida de probabilidad puede entenderse como un caso particular de medida finita sobre los reales.

\begin{definicion}
El espacio de medida $(\Omega,\mathcal{U},P)$ se dice un \textbf{espacio de probabilidad} si satisface que $P(\Omega) = 1$ 
\end{definicion}

\begin{definicion}
Sean $(\Omega_1,\mathcal{U}_1)$ y $(\Omega_2,\mathcal{U}_2)$ dos espacios medibles. Se dice que una función $f: \omega_1 \rightarrow \Omega_2$ es \textbf{medible} si para todo $A\in \mathcal{U}_2$
$f^{-1}(A)\in\mathcal{U}$
\end{definicion}

\begin{definicion}
Sea $(\Omega,\mathcal{U})$ un espacio medible y $(I,\mathcal{B}_I,P)$ un espacio de probabilidad. Una \textbf{variable aleatoria} es una función medible $X: \Omega \rightarrow \mathcal{B}_I$ entre estos espacios
\end{definicion}

Siendo $X$ una variable aleatoria, intuitivamente se puede definir la función de densidad de probabilidad de un conjunto medible $A \in \mathcal{B}$ como
\begin{equation}
P_X(A) = P\left( X^{-1} \left( A \right) \right)
\end{equation}

%Los espacios de probabilidad pueden definirse para una gran variedad de conjuntos; por simplicidad, en el presente texto únicamente se manejan espacios de probabilidad de la forma $(I,\mathcal{B}_I,P)$, con $I\subseteq \R$ un intervalo.

%Comunmente se asocia el resultado de un experimento \textit{aleatorio} con un espacio de probabilidad; el espacio muestral correspondería a los resultados posibles, y la medida representaría la \textit{probabilidad} de que el resultado ocurra en determinado conjunto.
%%
%Es común interpretar que si la probabilidad de un conjunto medible es $p$, entonces se espera que el resultado ocurra en dicho conjunto el $100*p \%$ de las ocasiones que se repita el experimento.

\begin{definicion}%[Función de Probabilidad Acumulada]
Sea $(\R,\mathcal{B},P)$ un espacio de probabilidad. La \textbf{función de probabilidad acumulada}, $F : \R \rightarrow [0,1]$, se define como
\begin{equation*}
F (x) := P\left( \left(-\infty,x \right] \right)
\end{equation*}
\end{definicion}

Por comodidad, se define una notación alterna para la función de probabilidad acumulada de $X$ como
\begin{equation}
P_X(x\leq x) := F_X(x) = P_X\left( \left( -\infty,x \right] \right)
\end{equation}
Si se puede definir una función de densidad de probabilidad para $X$, entonces puede escribirse
\begin{equation}
P(x\in A) = \int_A f_X(\lambda) d\lambda 
\end{equation}

Una función de probabilidad acumulada satisface las siguientes propiedades
\begin{itemize}
\item Para cualesquiera $x,y\in \R$, $x < y \Rightarrow F(x) < F(y)$
\item Pra cualquier $x\in\R$, $F(x) = \lim_{x\rightarrow x^{-}} F(x) + P(\{x\})$
\item $\lim_{x\rightarrow +\infty} F(x) = 1$
\item $\lim_{x\rightarrow -\infty} F(x) = 0$
\end{itemize}

Conviene considerar las funciones que satisface las condiciones anteriores, referidas simplemente como \textbf{función de distribución}, pero que no necesariamente provienen de un espacio de probabilidad. Naturalmente, una función de distribución $F$ puede inducir una medida $\mu$.

\begin{teorema}
Sea $F:\R \rightarrow \R$ una función de distribución; se puede contruir una medida $\mu_F$ sobre el espacio medible $(\R, \mathcal{B})$ tal que la función de probabilidad acumulada asociada al espacio de probabilidad $(\R, \mathcal{B}, \mu_F)$ es exactamente $F$.

La medida $\mu_F$ será referida como la \textbf{medida inducida} por $F$.
\end{teorema}
\begin{proof}
Para cualesquiera $a, b \in \R$, puede escribirse
\begin{equation}
\mu((a,b]) := F(b) - F(a)
\end{equation}
[? pag 18, 25 del libro de teorei ade la medida]
\end{proof}

%%%%%%%%%%%%%%%%%%%%%%%%%%%%%%%%%%%%%%%%%%%%%%%%%%%%%%%%%%%%%%%%%%%%%%%%%%%%%%%%%%%%%%%%%%%%%%%%%%%

\subsection{Variables aleatorias continuas y discretas}

\begin{definicion}
Una función $F: \R \rightarrow \R$ se dice \textbf{absolutamente continua} si para cualquier $\varepsilon>0$ arbitrario existe un $\delta>0$ y una familia de intervalos, $\{ [a_n, b_n]\}_{n\in \N}$, tal que
\begin{equation}
\sum_{n\in \N} \abso{b_n - a_n} < \delta 
\end{equation}
\begin{equation}
\sum_{n\in \N} \abso{F(b_n) - F(a_n)} < \varepsilon 
\end{equation}
\end{definicion}

Se dice que una medida de probabilidad $P$ es \textbf{continua} si su función de probabilidad acumulada es absolutamente continua.

\begin{proposicion}
Si una medida de probabilidad $F$ es absolutamente continua, entonces existe una función $f$ tal que 
\begin{equation}
F(x) = \int_{-\infty}^{x}f(y) dy
\end{equation}
Se dice que $f$ es la \textbf{función de densidad de probabilidad}.
\end{proposicion}

Sea $P$ una medida de probabilidad, se define su \textbf{soporte} como
\begin{equation}
\mathcal{D}_P = \left\{ x\in \R \lvert P\left( \left\{ x\right\} \right)>0 \right\}
\end{equation}
%
Se dice que una medida de probabilidad, $P$, es \textbf{discreta} si su soporte es numerable.

\begin{proposicion}
Si una medida de probabilidad $F$ es discreta, entonces existe una finito o infinito numerable $Q_F=\{q_n\}_{n\in \N}$ tal que 
\begin{equation}
F(x) = \sum_{n\leq x} q_n F({q_n})
\end{equation}
Es posible construir una función de densidad de probabilidad para $F$ como
\begin{equation}
f(x) = \begin{cases}
F({x}) &, x\in Q_F \\
0 &, \text{otro caso}
\end{cases}
\end{equation}
\end{proposicion}

Naturalmente es posible construir medidas de probabilidad que no sean ni continuas ni discretas. Por ejemplo, considérese la función de Cantor $K$ que puede ser definida iterativamente como
\begin{equation}
K_{n+1}(x) =
\begin{cases}
\frac{1}{2} K_n(3 x) &, 0\leq x \leq \frac{1}{3} \\
\frac{1}{2} K_n(3 x-2) + \frac{1}{2} &, 0\leq x \leq \frac{1}{3} \\
0 &, \text{otro caso}
\end{cases}
\end{equation}
con $K_0(x) = x$ y $K := \lim_{n\rightarrow \infty} K_n$

[?] demostracion de que la funcion de cantor esta bien definida
% https://es.wikipedia.org/wiki/Funci%C3%B3n_de_Cantor

\begin{proposicion}
La función de Cantor es continua pero no es absolutamente continua
\end{proposicion}

Luego entonces, puede construirse la siguiente función de distribución
\begin{equation}
F_K = \begin{cases}
K(x) &, 0\leq x \leq 1 \\
0 &, x < 0 \\
1 ,& x > 1
\end{cases}
\end{equation}
la cual no es ni continua ni discreta. Por simplicidad, en el presente trabajo únicamente se considerarán variables aleatorias que son continuas o discretas.

%La distinción entre variables aleatorias continuas y discretas puede verse más notoria en virtud del teorema \ref{Lebesgue_decomp}.

%\subsection{Vectores aleatorios}
% si se necesita, esta en la pagina 30 del manual

%\begin{teorema}[Descomposición de Radon-Nikodym]
%Sea $\mu$ una medida definida sobre la $\sigma$-álgebra $\mathcal{B}$, y sea $\nu$ una medida 
%$\sigma$-finita definida sobre $\mathcal{B}$. Entonces $\mu$ puede descomponerse de manera única como
%$\mu = \mu_A + \mu_S$, donde
%\begin{itemize}
%\item $\mu_A$ es absolutamente continua respecto a $\nu$
%\item Existe un conjunto $A$ tal que $\nu(A)=0$, $\mu_S\left(A^{C}\right) = 0$
%\end{itemize}
%\label{Lebesgue_decomp}
%\end{teorema}

%? pagina 109 del manual

%Dado que la medida de Lebesgue es $\sigma$-finita, cualquier medida de probabilidad puede 
%\textit{descomponerse} como la suma de una medida continua, una medida discreta y un \textit{residuo}.

%\subsection{Vectores aleatorios}
%
%El concepto de variable aleatoria dado en [?] puede extenderse trivialmente a conjuntos más generales como, por ejemplo, $\R^{n}$ (para algún entero $n$); por simplicidad en la interpretación, conviene definirlos en base a la versión descrita.
%
%\begin{definicion}
%Se llama \textbf{vector aleatorio} a una variable aleatoria sobre el espacio de probabilidad $(\R^{n},\mathcal{B}^{n},P)$, donde $\mathcal{B}^{n}$ es la $\sigma$-álgebra de Borel $n$-dimensional
%\begin{equation}
%\mathcal{B}^{n} := \sigma\left(\left\{ \left(-\infty, a_1\right]\times \left(-\infty, a_2\right]\times \cdots \times \left(-\infty, a_n\right] \lvert a_1, \dots, a_n \in \R \right\}\right)
%\end{equation}
%Por notación, un vector aleatorio es referido como
%\begin{equation}
%X = (X_1, X_2, \dots, X_n)
%\end{equation}
%\end{definicion}

\subsection{Valor esperado}

\begin{definicion}
Sea $X$ una variable aleatoria definida sobre el espacio de probabilidad $(\Omega, \mathcal{U}, P)$. Si $P$ es integrable en $\Omega$ respecto a $P$, entonces se define el \textbf{valor esperado} de $X$ como
\begin{equation}
\E{X} := \int_\omega X(\lambda) dP(\lambda)
\end{equation}
\end{definicion}

\begin{proposicion}
Sea $X$ una variable aleatoria y $g$ una función medible en el espacio edible $(\R,\mathcal{B})$. Entonces $g(X)$ es una variable aleatoria cuyo valor esperado es
\begin{equation}
\E{g(x)} = \int_\Omega [g(X)](\lambda) dP(\lambda) = \int_R g(x) dP(x)
\end{equation}
\end{proposicion}

\begin{definicion}
Sea $X$ una variable aleatoria, se definen (si es posible) su media $\mu_X$ y varianza $\sigma_X^{2}$ como
\begin{align}
\mu_X &{:=} \E{X} \\
\sigma_X^{2} &{:=} \E{(X-\mu_X)^{2}}
\end{align}
\end{definicion}

Naturalmente la notación $\mu_X$ únicamente se usa cuando no hay confusión con la notación para medidas. Así mismo, conviene mencionar ejemplos de varaibles aleatorias para las cuales no está bien definida su media o varianza.

? Ejemplos

%De manera más general, se define el $m$-momento de una variable aleatoria $X$ como
%\begin{equation}
%M
%\end{equation}

\begin{definicion}
Sea $X$ una varaible aleatoria. Se define su \textbf{función característica} como
\begin{equation}
\phi_X (\omega) := \E{e^{i \omega X}} = \int_\R e^{i \omega x} dF_X(x)
\end{equation}
donde, para todo $z\in \R$, $e^{i z} := \COS{z} + i \SEN{z}$
\end{definicion}

\begin{definicion}
Sean $X$, $Y$ dos variables aleatorias. Se define su \textbf{covarianza} como
\begin{equation}
\Cov{X,Y} := \E{X Y} = \int_{\R^{2}} x y d P_{(X,Y)}(x,y) = \int_\R \int_\R x y dP_X(x) dP_Y(y)
\end{equation}
\end{definicion}

\begin{proposicion}
Si $X$, $Y$ son independientes, entonces $\Cov{X,Y} = 0$
\end{proposicion}

\begin{definicion}
Sean $X$, $Y$ dos variables aleatorias. Se define su \textbf{coeficiente de correlación de Pearson} como
\begin{equation}
\rho (X,Y) := \sqrt{\frac{\Cov{X,Y}}{\Var{X} \Var{Y}}}
\end{equation}
\end{definicion}

%\begin{proof}
%en la pagina 65 del maual
%\end{proof}

%%%%%%%%%%%%%%%%%%%%%%%%%%%%%%%%%%%%%%%%%%%%%%%%%%%%%%%%%%%%%%%%%%%%%%%%%%%%%%%%%%%%%%%%%%%%%%%%%%%

\section{Estimación de parámetros}

Es común que se conozca cierta información de estos fenómenos que permita suponer que se comportan como variables aleatorias con cierta forma. Por ejemplo, ?.%se suele suponer que entre la población, 
%
Conviene destacar el caso de fenómenos que son \textit{forzados} a seguir una distribución conocida; por ejemplo, la metodología para aplicar la prueba Neuropsi \cite{Ostrosky00} ha sido diseñada de tal forma que los puntajes siguen una distribución normal para cada segmento poblacional.

En este tipo de escenarios se puede hablar de una función de distribución $f(\bullet; \theta)$ que depende de un parámetro $\theta \in \Omega$, donde $\Omega$ se conoce como \textbf{espacio de parámetros}; el objetivo consiste en deducir el valor de $\theta$ a partir de los datos recabados.

\begin{definicion}
Sea $X$ una variable aleatoria. Una \textbf{muestra de $X$ de tamaño $N$} es una colección de variables aleatorias $\{ X_1, X_2, \dots, X_N \}$ tales que son independientes y que comparten la misma distribución de $X$
\end{definicion}

\begin{proposicion}
Sea $X$ una variable aleatoria que admite una función de densidad $f_X$, y sea $\{ X_1, X_2, \dots, X_N \}$ una muestra. La función de densidad de probabilidad conjunta para el vector $[ X_1, X_2, \dots, X_N ]$ es
\begin{equation}
f_{[ X_1, \dots, X_N ]}(x_1, \dots, x_N ) = \prod_{j=1}^{N} f(x_j)
\end{equation}
Mientras no se indique lo contrario, las variables aleatorias en la muestra no están ordenadas.
\end{proposicion}

\begin{proposicion}
Sea $X$ una variable aleatoria, $\{ X_1, X_2, \dots, X_N \}$ una muestra y $\{ x_1, x_2, \dots, x_N \}$ un conjunto de observaciones. Se define la \textbf{función de distribución muestral} como
\begin{equation}
F_{X; N} (x) := \frac{1}{N} \sum_{x_j \leq x } 1
\end{equation}
\end{proposicion}

\begin{proposicion}
Si el tamaño de una muestra de $X$ se vuelve arbitrariamente grande, la función de distribución muestral converge en probabilidad a $F$, la función de probabilidad acumulada de $X$
\begin{equation}
\lim_{N \rightarrow \infty} F_{X; N} = F_X
\end{equation}
\end{proposicion}

%\begin{proof}
%Ver seccion 5.2 C R Rao, Linear Statistic Inference and its applications 2 ed pag 42
%\end{proof}

\begin{definicion}
Un \textbf{estadístico} es una función de las observaciones en una muestra
\end{definicion}

Ejemplo en pagina 112 del libro de lindgren.
Si $X\sim N(\mu,\sigma^{2})$, sea $\overline{X} = \frac{1}{N} \sum X_j$, entonces
\begin{equation}
\overline{X} \sim N(\mu,\frac{\sigma^{2}}{N})
\end{equation}

Ejemplo.
Considérese la variable aleatoria binomial $X \sim B(\theta)$ con $\theta \in [0, 1]$, cuya FDP es
\begin{equation}
f_X(x; \theta) = \begin{cases}
\theta^{x} (1-\theta)^{1-x} &, x\in \{ 0,1 \} \\
0 &, \text{otro caso}
\end{cases}
\end{equation}
La FDP conjunta para una muestra de tamaño $N$ es
\begin{equation}
f_N(x_1, \dots, x_N; \theta) = 
\begin{cases}
\theta^{\sum_i x_i}(1-\theta)^{\sum_i(1-x_i)} &, x_i \in \{ 0,1 \}, i\in \{1, \dots, N\} \\
0 &, \text{otro caso}
\end{cases}
\end{equation}

Se puede entender a $f_N$, evaluada en los datos obtenidos y como función de $\theta$, como la probabilidad de que se hayan obtenido los datos que de hecho se obtuvieron.
%
Esta redundancia sugiere que una elección adecuada para el parámetro $\theta$ sería aquél que maximice a tal función, que que recibe el nombre de \textbf{función de verosimilitud}
\begin{equation}
L(\theta; x_1, \dots, x_N) = \theta^{\sum_i x_i}(1-\theta)^{\sum_i(1-x_i)}
\end{equation}
con $\theta \in [0, 1]$. Por simplicidad técnica, se maximizará a $L$ igualando a cero la derivada de $\log \circ L$ con respecto a $\theta$.
\begin{align*}
\frac{d}{d\theta} \log\left( L(\theta; x_1, \dots, x_N)\right)
&= 
\frac{d}{d\theta} \log\left(\theta^{\sum_1^{N} x_i}(1-\theta)^{\sum_1^{N}(1-x_i)}\right) \\
&=
\frac{d}{d\theta} \left[ \left( \sum_{1}^{N} x_i \right) \log(\theta) + 
\left( \sum_{1}^{N}(1-x_i) \right) \log (1-\theta)\right] \\
&= \left( \sum_{i=1}^{N} x_i \right) \frac{1}{\theta} -
\left( N - \sum_{i=1}^{N}(x_i) \right) \frac{1}{1-\theta}
\end{align*}

Luego entonces, la función de verosimilitud es maximizada usando $\theta = \frac{1}{N} \sum_{1}^{N}(x_i)$.

\begin{definicion}
Sea $X$ una variable aleatoria que depende de un parámetro $\theta$ y $X_1, \dots, X_N$ una muestra de tamaño $N$. Un estimador $\widehat{\theta}$ es \textbf{suficiente} si
la distribución de la variable $X \lvert \widehat{\theta}$ no depende de $\theta$
\end{definicion}

\begin{teorema}
Sea $X$ una varaible aleatoria que depende del paraámetro $\theta$. Un estadístico $\widehat{\theta}$ es suficiente si y sólo si existen funciones $g$ y $h$ tales que
\begin{equation}
f_X(\bullet; \theta ) = g(\widehat{\theta},\theta) h_X(\bullet)
\end{equation}
\end{teorema}

%\begin{proof}
%ver pagina 121 del libro de lindgren
%\end{proof}

%El \textbf{espacio de datos} es 

%Usando la desigualdad de Chebyshev de puede deducir que
%\begin{equation}
%P\left( \abso{\widehat{\theta} - \theta} > \varepsilon \right) \leq 
%\frac{1}{\varepsilon^{2}} \E{\left( \widehat{\theta} - \theta \right)^{2}}
%\end{equation}

\begin{definicion}
El \textbf{error de media cuadrática} para el estimador $\widehat{\theta}$ se define como
\begin{equation}
\text{EMC}(\widehat{\theta}) := \E{\left( \widehat{\theta} - \theta \right)^{2}} =
\Var{\widehat{\theta}} + \left( \E{\theta} - \theta^{2} \right)
\end{equation}
\end{definicion}

\begin{definicion}
Sea $X$ una variable aleatoria que depende de un parámetro $\theta$ y $X_1, \dots, X_N$ una muestra de tamaño $N$. Un estimador $\widehat{\theta}$ es \textbf{insesgado} si cumple que
\begin{equation}
\E{\widehat{\theta}} = \theta
\end{equation}
\end{definicion}

Se puede hablar del \textbf{sesgo} del estimador $\widehat{\theta}$ como $\E{\widehat{\theta}}-\theta$

\begin{definicion}
Sea $X$ una variable aleatoria que depende de un parámetro $\theta$ y $\left\{ \widehat{\theta}_n \right\}_{n\in \N}$ una familia de estimadores definidos para muestras de $X$ de tamaño arbitrario. La familia de estimadores se dice \textbf{consistente} si para cualquier $\varepsilon > 0$
\begin{equation}
\lim_{n\rightarrow\infty} P\left( \abso{\widehat{\theta}_n-\theta} > \varepsilon \right) = 0
\end{equation}
\end{definicion}

\begin{definicion}
Así como en la definición anterior, se dice que a familia de estimadores \textit{converge en media cuadrática} si
\begin{equation}
\lim_{n\rightarrow\infty} \E{\left( \widehat{\theta}_n - \theta \right)^{2}} = 0
\end{equation}
Si esto se cumple, se dice que la familia de estimadores es \textbf{consistente en media cuadrática} \end{definicion}

\begin{teorema}
Si $\widehat{\theta}_n$ es una familia de estimadores consistente en media cuadrática, entonces es consistente
\end{teorema}

\begin{teorema}
Una condición suficiente para para que una familia sea consistente en en media cuadrática es
\begin{align}
\lim_{n\rightarrow\infty} \E{\widehat{\theta}_n} &= \theta \\
\lim_{n\rightarrow\infty} \Var{\widehat{\theta}_n} &= 0
\end{align}
\end{teorema}

\begin{proof}
? pag 141 del libro de lindgren
\end{proof}



%%%%%%%%%%%%%%%%%%%%%%%%%%%%%%%%%%%%%%%%%%%%%%%%%%%%%%%%%%%%%%%%%%%%%%%%%%%%%%%%%%%%%%%%%%%%%%%%%%%

\section{Pruebas de hipótesis}

%? empieza en la pagina 156 del libro de lindgren

Una hipótesis es una afirmación sobre algún aspecto desconocido.
%
Es tarea común en la estadística el decidir si alguna afirmación puede sostenerse a partir de la
información proporcionada por un conjunto de observaciones. 
%
A partir de la aplicación masiva de pruebas neuropsicológicas a un grupo de adultos mayores uno 
puede preguntarse, por ejemplo, si hombres y mujeres tienden a obtener puntajes diferentes en las
pruebas, o si la edad de los participantes está correlacionada con su desempeño en tareas de 
memoria.
%
En la tabla ... se muestran los datos sobre una simulación (artificial) de dicho escenario.

Una herramienta de uso común para producir estas decisiones es la \textbf{pruebas de hipótesis},
la cual consiste en dos afirmaciones complementarias, es decir, tales que exactamente una de ellas es verdadera; tales afirmaciones
son referidas como \textit{hipótesis}, y deben elegirse de forma que sean equivalentes a la 
decisión que se busca. 
%
Usualmente la primera de las hipótesis (hipótesis nula, $H_0$) representa la afirmación más general o que se cree verdadera por omisión, mientras que la segunda hipótesis (hipótesis alternativa, $H_A$) se tomará como verdadera si
existe suficente información para rechazar la veracidad de la primera.
%
%La idea intuitiva detrás de las pruebas estadísticas es que una muestra de la población se comporta de manera similar a la población.
%
El enfoque de prueba de significancia es  tomar un estadístico $\est{\theta}$ y evaluarlo sobre los datos, posteriormente se analiza qué tan diferente es el valor obrevado del típico cuando la hipótesis nula es verdadera.

Los estadísticos de prueba suele ser un estadístico construido para tener una distribución conocida salvo unos pocos parámetros fáciles de estimar.
%
La interpretación usual es que, si $H_0$ es verdadera entonces $\est{\theta}$ puede no tener el valor predicho debido a factores ajenos al fenómeno estudiado, en consecuencia se suele hablar de una región de rechazo en el espacio de estados (ver más adelante).
%
Bajo esta interpretación, un valor de $\widehat{\theta}$ dentro de la región de rechazo significa que los datos representan evidencia para rechazar $H_0$; un no-rechazo no significa precisamente que $H_0$ sea verdadera, sino que las observaciones no representan evidencia suficiente para rechazar $H_0$.

\begin{definicion}
En una prueba de hipótesis, rechazar $H_0$ cuando es verdadero es un \textbf{error del tipo I}. Así msimo, aceptar $H_0$ cuando es falsa es un \textbf{error del tipo II}.
\end{definicion}

La naturaleza e intepretación de los estadísticos de prueba suelen ser muy particulares de las situaciones bajo las cuales son definidos.
%
Una forma típica de normalizar los diferentes estadísticos es a través del $p$-valor, definido como la probabilidad de que ocurra un valor extremo del estadístico de prueba; 
el $p$-valor suele interpretarse como la \textit{fuerza} de la evidencia contra $H_0$.

\begin{definicion}
Sea $\widehat{\theta}$ un estadístico de prueba. El \textbf{$\boldsymbol{p}$-valor} asociado al $\widehat{\theta}=\theta_0$ es la probabilidad $P\left(\widehat{\theta}\right>\theta_0)$
\end{definicion}

Una \textbf{prueba de sinificancia} se entiende como una pruebas de hipótesis para algunos $p$-valores predefinidos, usualmente 0.05, 0.01, 0.005, entre otros.
%
Un error común, pero muy extendido, es interpretar al $p$-valor como la probabilidad de obtener $H_0$.

\begin{definicion}
Dada una muestra poblacional y dos afirmaciones complementarias $H_0$ y $H_A$, una \textbf{prueba de hipótesis} es una regla de decisión que asigna a cada punto del espacio de estados una acción del conjunto Aceptar $H_0$, rechazar $H_A$, Rechazar $H_0$, aceptar $H_A$.

Al conjunto del espacio muestral sonde se rechaza $H_0$ se le denomina \textbf{región crítica}. 
\end{definicion}

%Por comodidad, uno puede redefinir a la región crítica en términos del estimador $\widehat{\theta}$. 
Una propiedad deseable para un estadístico de prueba es poder acotar los errores de tipo I y de tipo II; para ello, para alguna región crítica arbitraria $\mathcal{C}$ se define el \textbf{nivel de significancia} de la prueba como
\begin{equation}
\alpha := \sup_{\theta \in H_0} p(\mathcal{C} \lvert \theta)
\end{equation}

Ejemplo:
Retomando los datos de la tabla ..., considérese la pregunta \textit{¿Los hombres y mujeres tienden
a obtener puntajes diferentes en las pruebas neuropsicológicas?}. 
%
En este ejemplo se supone que los puntajes de los hombres en la prueba siguen una distribución normal con media $\mu_H$ y varianza 1, y similarmente para las mujeres con media $\mu_M$ y varianza 1.
%
Como hipótesis nula se elige la posibilidad de que en promedio ambos grupos (hombre y mujeres) obtengan el mismo puntaje en la prueba, es decir
\begin{equation}
H_0 : \mu_H = \mu_M
\end{equation}
y como hipótesis alternativa está la posibilidad de que los puntajes sean diferentes
\begin{equation}
H_A : \mu_H \neq \mu_M
\end{equation}

%\subsection{Clasificación de dos vías}
%
%Un \textbf{diseño aleatorio por bloques} es un muestreo aleatorio en varias categorías de sujetos 

%%%%%%%%%%%%%%%%%%%%%%%%%%%%%%%%%%%%%%%%%%%%%%%%%%%%%%%%%%%%%%%%%%%%%%%%%%%%%%%%%%%%%%%%%%%%%%%%%%%

\section{Procesos estocásticos}

\begin{definicion}
Un \textbf{proceso estocástico} \xt es una colección de variables aleatorias indexadas en el conjunto $\mathcal{T}$
\begin{equation}
\{X(t)\}_{t\in \mathcal{T}} := \left\{ X(t) \lvert t\in \mathcal{T} \right\}
\end{equation}
La variable $t$ es referida como \textit{tiempo}, mientras que el conjunto $\mathcal{T}$ es referido como el \textit{conjunto de tiempos admisibles}
\end{definicion}

de $\left\{ t \in \R | {t} \cdot {\Delta_t} \in \Z \right\} $  para algún $\Delta_t$ 
(\textbf{tiempo discreto}). 
%
Las \textit{componentes} de un proceso estocástico serán denotadas como:\\

\begin{tabular}{cl}
\xt    & Todo el proceso \\
$X(t)$ & Una de las variable aleatoria que componen al proceso, en el tiempo $t$ \\
$x(t)$ & Una realización de $X(t)$ \\
$F_{X(t)}$ & FPA para $X(t)$ \\
$ {\Delta_t}$ & Frecuencia de muestreo (en tiempo discreto)
\end{tabular}\\

La estacionariedad es un indicativo de la \textit{homogeneidad} de un proceso; un proceso 
\textit{muy} estacionario sería aquél cuyas variable aleatoria que tiene distribuciones conjuntas que no cambian 
con el tiempo. 
%
La definición \ref{est_fuerte} representa con exactitud tales requerimientos, pero se le considera 
\textit{innecesariamente fuerte}; una definición común es \ref{est_m}.

\begin{definicion}[Estacionariedad fuerte]
Un proceso \xt se dice fuertemente estacionario si para cualesquiera 
$t_1, t_2, \dots, t_n \in \mathcal{T}$ y cualquier $\tau$ tal que $t_i + \tau \in \mathcal{T}$,
se cumple que
\begin{equation*}
F_{\left[ X(t_1), X(t_2), \dots, X(t_n) \right]} \equiv
F_{\left[ X(t_1 + \tau), X(t_2 + \tau), \dots, X(t_n + \tau) \right]}
\end{equation*}
Donde $F_{[v_1,v_2,\dots,v_N]}$ es la FPA conjunta para el vector $[v_1,v_2,\dots,v_N]$
\label{est_fuerte}
\end{definicion}

\begin{definicion}[Estacionariedad de orden $m$]
Un proceso \xt se dice estacionario de orden $m$ si, para cualesquiera
$t_1, t_2, \dots, t_n \in \mathcal{T}$ y cualquier $\tau$ tal que $t_i + \tau \in \mathcal{T}$,
se cumple que
\begin{equation*}
\E{X^{m_1}(t_1)X^{m_2}(t_2)\cdots X^{m_n}(t_n)} =
\E{X^{m_1}(t_1+\tau)X^{m_2}(t_2+\tau)\cdots X^{m_n}(t_n+\tau)}
\end{equation*}
para cualesquiera enteros $m_1, m_2, \dots, m_n$ tales que $m_1+m_2+\cdots+m_n \leq m$
\label{est_m}
\end{definicion}

Cabe mencionar que la definición \ref{est_m} no es equivalente a la definición \ref{est_fuerte}, ni
aún cuando $m\rightarrow \infty$; sin embargo permite asegurar que los \textit{momentos} 
($\E{X^{k}}$ para algún $k$) del proceso sean invariantes en el tiempo, y éstos suelen encontrarse
asociados a cantidades físicas.

Como un ejemplo muy particular conviene destacar la energía, que suele ser asociada con el segundo
momento (definición \ref{energia}). 
%
Dicha conexión motiva a escoger una definición de estacionariedad que permita analizar la energía 
del proceso: la estacionariedad débil.

\begin{definicion}[Estacionariedad débil]
Un proceso \xt se dice débilmente estacionario si existen constantes $\mu, \sigma \in \R$ y una 
función $R : T \rightarrow \R \cup \{ \pm \infty \} $ tales que, para cualesquiera $t, s \in T$ se 
cumple
\begin{itemize}
\item $\E{X(t)} = \mu$
\item $\Var{X(t)} = \sigma^{2}$
\item $\Cov{X(t),X(s)} = R(s-t)$
\end{itemize}
\end{definicion}

\begin{proposicion}
Un proceso es débilmente estacionario si y sólo si es estacionario de orden 2
\end{proposicion}

Cabe destacar que la estacionariedad débil no sólo tiene como condición que todas las variables del
proceso tengan la misma media y varianza, sino que también supone que éstas son finitas.
%
Sobre la función de covarianza $R$ (que en un único proceso es referida como \textit{autocovarianza}),
no hay restricciones sobre los valores que pueda tomar, excepto que 
$R(0) = \Var{X(\bullet)} < \infty$. 
%
En el marco del modelo de series electrofisiológicas, conviene suponer que los registros 
corresponden a procesos a tiempo continuo que son continuos de alguna forma; se ha elegido la 
continuidad en media cuadrática.

%\begin{observacion}
%Sea \xt un proceso débilmente estacionario y $T$ su función de autocovarianza. Si $R$ es continua
%en 0 entonces es continua en todos lados
%\end{observacion}

\begin{definicion}[Continuidad estocástica en media cuadrática]
Un proceso a tiempo continuo \xt es estocásticamente continuo, en el sentido de media cuadrática, 
en un tiempo admisible $t_0$ si
\begin{equation*}
\lim_{t \rightarrow t_0} \E{\left( X(t) - X(t_0) \right)^{2}} = 0
\end{equation*}
\label{cont_est}
\end{definicion}

Una forma natural de pensar en la definición \ref{cont_est} es que si $\abso{t-t_0}$ es muy pequeño 
entonces $X(t)$ y $X(t_0)$ difieren muy poco entre sí, como variables aleatorias.
%
Hablando de procesos débilmente estacionarios, la continuidad estocástica de un proceso es 
equivalente a que su función de autocovarianza sea continua en 0.

%%%%%%%%%%%%%%%%%%%%%%%%%%%%%%%%%%%%%%%%%%%%%%%%%%%%%%%%%%%%%%%%%%%%%%%%%%%%%%%%%%%%%%%%%%%%%%%%%%%
%%%%%%%%%%%%%%%%%%%%%%%%%%%%%%%%%%%%%%%%%%%%%%%%%%%%%%%%%%%%%%%%%%%%%%%%%%%%%%%%%%%%%%%%%%%%%%%%%%%
%%%%%%%%%%%%%%%%%%%%%%%%%%%%%%%%%%%%%%%%%%%%%%%%%%%%%%%%%%%%%%%%%%%%%%%%%%%%%%%%%%%%%%%%%%%%%%%%%%%
