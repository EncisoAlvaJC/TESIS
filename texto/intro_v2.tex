%%%%%%%%%%%%%%%%%%%%%%%%%%%%%%%%%%%%%%%%%%%%%%%%%%%%%%%%%%%%%%%%%%%%%%%%%%%%%%%%%%%%%%%%%%%%%%%%%%%
%%%%%%%%%%%%%%%%%%%%%%%%%%%%%%%%%%%%%%%%%%%%%%%%%%%%%%%%%%%%%%%%%%%%%%%%%%%%%%%%%%%%%%%%%%%%%%%%%%%

\chapter{Introducción}

Gracias a los avances médicos del último siglo se ha incrementado la esperanza de vida y la 
calidad de vida; desafortunadamente también ha aumentado la presencia de enfermedades 
no-transmisibles asociadas con la edad. Para muchas de esas enfermedades no se han identificado 
factores causales o curas definitivas \cite{PlanAlzheimer04}.
%
En México el sector de la población con más de 60 años de edad (aquellos con alto riesgo para este
tipo de enfermedades) contempló a 10 millones de personas en 2010; se estima en 2015 esta cifra 
creció a 12 millones \cite{Censo10,Intercensal15}.

De entre las enfermedades ante las cuales este grupo de edad es vulnerable, en este trabajo se 
destaca la demencia. 
La demencia consiste en el desarrollo de déficit cognoscitivos suficientemente graves como para 
interferir en las actividades laborales y sociales.
%
El deterioro cognitivo característico de la demencia se considera irreverible, debido a lo cual 
ha surgido un gran interés en definir y diagnosticar etapas tempranas de este padecimiento con el 
fin de evitar en lo posible dicho síntoma.

%
Como ejemplo Petersen \cite{Petersen01}
%Petersen, R. C., Doody, R., Kurz, A., Mohs, R. C., Morris, J. C., Rabins, P.
%V., ... & Winblad, B. (2001). Current concepts in mild cognitive impairment.
%Archives of neurology, 58(12), 1985-1992.
define una muy detallada escala de subtipos para el deterioro cognitivo, basándose principalmente
en criterios psicológicos orientados a la funcionalidad del paciente;
%
en contraparte Montplaisir \cite{Brayet16} reporta asociaciones entre ejecuciones pobres en pruebas
neuropsicológicas y atrofias en el hipocampo, detectadas usando resonancia magnética funcional.
En el presente trabajo se otorga cierta preferencia a los criterios basados en registros 
electrofisiológicos, ya que son más fáciles de cuantificar; se mantiene una conexión 
estrecha con los criterios neuropsicológicos.





Cabe destacar que la convergencia de enfoques entre distintos niveles del sujeto (conductual, sistémico) no sería
posible sin un cuerpo común de modelos y métodos que permitan entender los datos recabados dentro
de un mismo marco conceptual centrado en el sujeto de estudio (el deterioro cognitivo);
es bajo estas circunstancia que emerge el estudio de los métodos de análisis \textit{per se}
como una necesidad ante la innovación tecnológica acelerada, que en fechas recientes ha hecho
de uso común el registro y procesamiento de conjuntos masivos datos.

El principal problema que se aborda en el presente trabajo comienza en que las señales
electrofisiológicas típicamente son complejas, y representan procesos no--lineales y 
no--estacionarios; sin embargo, las herramientas para analizar tales datos a menudo siguen 
asumiendo linealidad y estacionariedad.
A consecuencia de lo anterior, los conjuntos de datos pueden contener
información \textit{oculta} y que puede ser de valor clínico; o información que no es extraída y 
manejada de forma apropiada, y por tanto puede no ser representativa del fenómeno que se estudia. 

En el presente trabajo se exponen herramientas relativas a verificar (en el sentido estadístico)
la estacionariedad débil en series de tiempo, un supuesto básico para calcular el espectro de potencias; esta cantidad,
calculada para el polisomnograma,
ha sido reportada como relacionada con el deterioro cognitivo\cite{Amer13,Miyata13,Reid06,Potvin12}.

%En algunos estudios de gran escala se han hallado correlaciones entre diferentes trastornos del 
%sueño y algún grado de deterioro cognitivo objetivo en adultos mayores 
%\cite{Amer13,Miyata13,Reid06,Potvin12}; entendiendo por ello una ejecuciones más pobres en tareas
%cognitivas, pero que no impiden llevar a cabo actividades cotidianas.

%En 2016 Vázquez-Tagle y colaboradores estudiaron la epidemiología del deterioro cognitivo en 
%adultos mayores dentro del estado de Hidalgo y su posible relación con trastornos de sueño, 
%encontrando una correlación entre una menor eficiencia del sueño\footnote{Porcentaje de tiempo
%de sueño respecto al tiempo en cama} y la presencia de deterioro cognitivo \cite{VazquezTagle16}.

%En un segundo trabajo por García-Muñoz y colaboradores \cite{Valeria} se analizaron 
%registros de polisomnograma (PSG) 
%%datos de PSG 
%para detectar posibles cambios en la conectividad funcional del cerebro\footnote{La 
%\textbf{conectividad funcional} se refiere una \textit{fuerte} relación (cuantificada) entre dos 
%señales, y usualmente se contrasta con la \textbf{conectividad anatómica}, entendida como conexiones 
%físicas entre los generadores de señales} en adultos mayores con posible deterioro 
%cognitivo (PDC), reportando un mayor exponente de Hurst para registros de PSG en adultos mayores 
%con PDC \cite{Valeria}.
%El exponente de Hurst, calculado a través del algoritmo \textit{Detrended Fluctuation Analysis}, 
%está relacionado con las correlaciones de largo alcance y la estructura fractal de una serie de 
%tiempo, siendo que un mayor exponente está asociado con señales cuya función de 
%autocorrelación decrece más lentamente \cite{Rodriguez11}.
%Con base a que en aquellos trabajos se ha supuesto que los registros de PSG son no-estacionarios, 
%en este trabajo se pretende verificar si efectivamente estas señales se pueden considerar con tal
%característica.

%El supuesto de estacionariedad es básico en el estudio de series de tiempo, y usualmente se 
%acepta o rechaza sin un tratamiento formal; es de particular importancia, por ejemplo, para 
%calcular el espectro de potencias a partir de registros.
%La idea de que sujetos con PDC exhiben estacionariedad débil en sus registros de EEG en mayor 
%proporción, respecto a individuos sanos, fue sugerida por Cohen \cite{Cohen77}, quien a su vez se 
%refiere a trabajos anteriores sobre estacionariedad y normalidad en registros de EEG 
%\cite{McEwen75,Sugimoto78,Kawabata73}.

\section{Antecedentes}

En algunos estudios de gran escala se han hallado correlaciones entre diferentes trastornos del 
sueño y algún grado de deterioro cognitivo objetivo en adultos mayores 
\cite{Amer13,Miyata13,Reid06,Potvin12}; entendiendo por ello una ejecuciones más pobres en tareas
cognitivas, pero que no impiden llevar a cabo actividades cotidianas.

En 2016 Vázquez-Tagle y colaboradores estudiaron la epidemiología del deterioro cognitivo en 
adultos mayores dentro del estado de Hidalgo y su posible relación con trastornos de sueño, 
encontrando una correlación entre una menor eficiencia del sueño (Porcentaje de tiempo
de sueño respecto al tiempo en cama) y la presencia de deterioro cognitivo \cite{VazquezTagle16}.
%
En aqué estudio se efectuaron registros de electroencefalograma durante el sueño (técnica conocida
como polisomnografía, PSG) para algunos de los participantes, se hipotetizó que la estructura
de dichas señales podría variar en la presencia de deterioro cognitivo leve.

%En un segundo trabajo por García-Muñoz y colaboradores \cite{Valeria} se analizaron 
%registros de polisomnograma (PSG) 
%%datos de PSG 
%para detectar posibles cambios en la conectividad funcional del cerebro\footnote{La 
%\textbf{conectividad funcional} se refiere una \textit{fuerte} relación (cuantificada) entre dos 
%señales, y usualmente se contrasta con la \textbf{conectividad anatómica}, entendida como conexiones 
%físicas entre los generadores de señales} en adultos mayores con posible deterioro 
%cognitivo (PDC), reportando un mayor exponente de Hurst para registros de PSG en adultos mayores 
%con PDC \cite{Valeria}.
%El exponente de Hurst, calculado a través del algoritmo \textit{Detrended Fluctuation Analysis}, 
%está relacionado con las correlaciones de largo alcance y la estructura fractal de una serie de 
%tiempo, siendo que un mayor exponente está asociado con señales cuya función de 
%autocorrelación decrece más lentamente \cite{Rodriguez11}.
%Con base a que en aquellos trabajos se ha supuesto que los registros de PSG son no-estacionarios, 
%en este trabajo se pretende verificar si efectivamente estas señales se pueden considerar con tal
%característica.

%El supuesto de estacionariedad es básico en el estudio de series de tiempo, y usualmente se 
%acepta o rechaza sin un tratamiento formal; es de particular importancia, por ejemplo, para 
%calcular el espectro de potencias a partir de registros.
%La idea de que sujetos con PDC exhiben estacionariedad débil en sus registros de EEG en mayor 
%proporción, respecto a individuos sanos, fue sugerida por Cohen \cite{Cohen77}, quien a su vez se 
%refiere a trabajos anteriores sobre estacionariedad y normalidad en registros de EEG 
%\cite{McEwen75,Sugimoto78,Kawabata73}.
%%Cabe mencionar que en estos primeros estudios se palpa la posibilidad de que los registros de EEG 
%%fueran 'ruido' de algún tipo, una idea que se ha probado errónea en estudios más recientes 
%%\cite{Klonowski09}; sin embargo, se retoma como hipótesis a la luz de los estudios mencionados. 

%El presente trabajo resulta de una colaboración con el departamento de Geron-
%tologı́a, dependiente del Instituto de Ciencias de la Salud (ICSA); parte de esta
%colaboración incluye el acceso a los registros de PSG obtenidos por Vázquez Tagle y
%colaboradores [40].

%%%%%%%%%%%%%%%%%%%%%%%%%%%%%%%%%%%%%%%%%%%%%%%%%%%%%%%%%%%%%%%%%%%%%%%%%%%%%%%%%%%%%%%%%%%%%%%%%%%
%%%%%%%%%%%%%%%%%%%%%%%%%%%%%%%%%%%%%%%%%%%%%%%%%%%%%%%%%%%%%%%%%%%%%%%%%%%%%%%%%%%%%%%%%%%%%%%%%%%

%\section{Justificación}

%%%%%%%%%%%%%%%%%%%%%%%%%%%%%%%%%%%%%%%%%%%%%%%%%%%%%%%%%%%%%%%%%%%%%%%%%%%%%%%%%%%%%%%%%%%%%%%%%%%
%%%%%%%%%%%%%%%%%%%%%%%%%%%%%%%%%%%%%%%%%%%%%%%%%%%%%%%%%%%%%%%%%%%%%%%%%%%%%%%%%%%%%%%%%%%%%%%%%%%

\section{Pregunta de investigación}

¿Los registros de polisomnograma en adultos mayores, pueden
considerarse como series tiempo débilmente estacionarias?
¿Es posible que tal caracterización se relacione con el estado cognitivo general del adulto mayor?

%%%%%%%%%%%%%%%%%%%%%%%%%%%%%%%%%%%%%%%%%%%%%%%%%%%%%%%%%%%%%%%%%%%%%%%%%%%%%%%%%%%%%%%%%%%%%%%%%%%
%%%%%%%%%%%%%%%%%%%%%%%%%%%%%%%%%%%%%%%%%%%%%%%%%%%%%%%%%%%%%%%%%%%%%%%%%%%%%%%%%%%%%%%%%%%%%%%%%%%

\subsection{Hipótesis}

Existen diferencias en la actividad eléctrica cerebral en adultos mayores con PDC, respecto
a individuos sanos, y es posible detectar dichas diferencias como una mayor o menor \textit{presencia} 
de estacionariedad débil en registros de PSG durante el sueño profundo.

%%%%%%%%%%%%%%%%%%%%%%%%%%%%%%%%%%%%%%%%%%%%%%%%%%%%%%%%%%%%%%%%%%%%%%%%%%%%%%%%%%%%%%%%%%%%%%%%%%%

\subsection{Objetivo general}

Deducir, a partir de pruebas estadísticas formales, las presencia de estacionariedad débil en
registros de PSG para adultos mayores con PDC, así como individuos control.
Usar estas pruebas para vigilar registros de PSG en sueño profundo.

%%%%%%%%%%%%%%%%%%%%%%%%%%%%%%%%%%%%%%%%%%%%%%%%%%%%%%%%%%%%%%%%%%%%%%%%%%%%%%%%%%%%%%%%%%%%%%%%%%%

\subsection{Objetivos específicos}

\begin{itemize}
\item Estudiar la definición de estacionariedad para procesos estocásticos y sus posibles 
consecuencias para el modelado de los datos considerados

\item Investigar cómo detectar, como prueba de hipótesis, si una serie de tiempo dada provenga
de un proceso estocástico débilmente estacionario, y bajo qué supuestos 
es válida dicha caracterización

\item Analizar si los registros de PSG, durante sueño profundo, son débilmente estacionarios.
Revisar si se pueden encontrar diferencias significativas entre individuos con y sin PDC
\end{itemize}

%%%%%%%%%%%%%%%%%%%%%%%%%%%%%%%%%%%%%%%%%%%%%%%%%%%%%%%%%%%%%%%%%%%%%%%%%%%%%%%%%%%%%%%%%%%%%%%%%%%
%%%%%%%%%%%%%%%%%%%%%%%%%%%%%%%%%%%%%%%%%%%%%%%%%%%%%%%%%%%%%%%%%%%%%%%%%%%%%%%%%%%%%%%%%%%%%%%%%%%