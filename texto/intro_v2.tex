%%%%%%%%%%%%%%%%%%%%%%%%%%%%%%%%%%%%%%%%%%%%%%%%%%%%%%%%%%%%%%%%%%%%%%%%%%%%%%%%%%%%%%%%%%%%%%%%%%%
%%%%%%%%%%%%%%%%%%%%%%%%%%%%%%%%%%%%%%%%%%%%%%%%%%%%%%%%%%%%%%%%%%%%%%%%%%%%%%%%%%%%%%%%%%%%%%%%%%%

\chapter{Introducción}

Gracias a los avances médicos del último siglo se ha incrementado la esperanza de vida y la 
calidad de vida; desafortunadamente también ha aumentado la presencia de enfermedades 
no-transmisibles asociadas con la edad. 
%
Para muchas de esas enfermedades no se han identificado factores causales o curas definitivas 
\cite{PlanAlzheimer04}.
%
En México el sector de la población con más de 60 años de edad (aquellos con alto riesgo para este
tipo de enfermedades) contempló a 10 millones de personas en 2010; se estima en 2015 esta cifra 
creció a 12 millones \cite{Censo10,Intercensal15}.

De entre las enfermedades ante las cuales este grupo de edad es vulnerable, en este trabajo se 
destaca la demencia. 
%
La demencia consiste en el desarrollo de déficit cognoscitivos suficientemente graves como para 
interferir en las actividades laborales y sociales.
%
El deterioro cognitivo característico de la demencia se considera irreversible, debido a lo cual 
ha surgido un gran interés en definir y diagnosticar etapas tempranas de este padecimiento con el 
fin de evitar en lo posible dicho síntoma.

%Como ejemplo Petersen \cite{Petersen01} define una muy detallada escala de subtipos para el 
%deterioro cognitivo, basándose principalmente en criterios psicológicos orientados a la 
%funcionalidad del paciente; en contraparte Montplaisir \cite{Brayet16} reporta asociaciones entre 
%ejecuciones pobres en pruebas neuropsicológicas y atrofias en el hipocampo, detectadas usando 
%resonancia magnética funcional.

El presente trabajo se favorecen los métodos caracterizar y diagnosticar el deterioro cognitivo
basados en mediciones físicas, y en particular los registros de electroencefalograma (EEG).
%
%Dicha preferencia se justifica por las hipótesis que pueden hacerse sobre los registros así 
%obtenidos, mismas que facilitan su modelado y análisis.
%
Sin embargo se mantendrá presente que el sujeto de estudio, el deterioro cognitivo en el adulto 
mayor, no puede reducirse a tales mediciones; las conclusiones obtenidas sobre los registros de PSG 
deben ser contrastadas, por ejemplo, con evaluaciones neuropsicológicas.



De forma particular se aborda la problemática metodológica de que las señales electrofisiológicas 
típicamente representan procesos no--lineales y no--estacionarios, y sin embargo suelen ser 
analizadas usando herramientas que suponen linealidad y estacionariedad.
%
En el caso del espectro de potencias, por ejemplo, es común que sea calculado usando la
transformada de Fourier sobre segmentos cortos para evitar los \textit{efectos} de la 
no--estacionariedad \cite{Kaiser00}.

A consecuencia de lo anterior los datos pueden contener información \textit{oculta}, o incluso 
pueden llegar a no ser representativos del fenómeno que se estudia. 
%
Es por ello que se buscan herramientas para verificar la estacionariedad débil en los registros
electrofisiológicas, y con especial atención en la posibilidad de que puedan usarse como marcadores
de deterioro cognitivo.
%
Adicionalmente, la posibilidad de que sujetos con PDC exhiben estacionariedad débil en sus 
registros de EEG en mayor proporción (respecto a individuos sanos) fue sugerida anteriormente
\cite{Cohen77}.

%%%%%%%%%%%%%%%%%%%%%%%%%%%%%%%%%%%%%%%%%%%%%%%%%%%%%%%%%%%%%%%%%%%%%%%%%%%%%%%%%%%%%%%%%%%%%%%%%%%
%%%%%%%%%%%%%%%%%%%%%%%%%%%%%%%%%%%%%%%%%%%%%%%%%%%%%%%%%%%%%%%%%%%%%%%%%%%%%%%%%%%%%%%%%%%%%%%%%%%

\section{Antecedentes}

El sueño MOR ha sido ampliamente reconocido como parte de la consolidación de la memoria, así como
otras funciones cognitivas 
\cite{Fishbein1971,Fishbein1977,Lucero1970,Pearlman1971,Pearlman1974,Smith1991}.
%
En el caso de adultos mayores, la correlación entre deterioro cognitivo y trastornos del sueño ha 
sido reportada por varios autores a partir de estudios poblacionales 
\cite{Amer13,Miyata13,Reid06,Potvin12}.
%
Tal correlación era de esperarse ya que los proceso de atención y memoria, por ejemplo, dependen de 
los circuitos colinérgicos activados durante el sueño MOR \cite{Braun1997}; estos circuitos son 
propensos a degradación estructural tanto en tanto en el envejecimiento normal como en el 
patológico,  y especialmente en el segundo \cite{Schliebs11}.

En 2016 Vázquez-Tagle y colaboradores estudiaron la epidemiología del deterioro cognitivo en 
adultos mayores dentro del estado de Hidalgo y su posible relación con trastornos de sueño, 
encontrando efectivamente una correlación entre una menor eficiencia del sueño (porcentaje de 
tiempo de sueño respecto al tiempo en cama) y la presencia de deterioro cognitivo 
\cite{VazquezTagle16}.
%
En aquél estudio se efectuaron registros electroencefalográficos durante el sueño (técnica referida
como polisomnografía, PSG) para algunos de los participantes; se hipotetizó que existen
diferencias en los registros correspondientes a personas con y sin deterioro cognitivo leve.

El presente trabajo se enmarca dentro de una colaboración los responsables del estudio mencionado, 
con el objetivo de identificar los posibles cambios en los registros de PSG ocurridos durante el
deterioro cognitivo.

%La idea de que sujetos con PDC exhiben estacionariedad débil en sus registros de EEG en mayor 
%proporción, respecto a individuos sanos, fue sugerida por Cohen \cite{Cohen77}, quien a su vez se 
%refiere a trabajos anteriores sobre estacionariedad y normalidad en registros de EEG 
%\cite{McEwen75,Sugimoto78,Kawabata73}.

%El presente trabajo resulta de una colaboración con el departamento de Geron-
%tologı́a, dependiente del Instituto de Ciencias de la Salud (ICSA); parte de esta
%colaboración incluye el acceso a los registros de PSG obtenidos por Vázquez Tagle y
%colaboradores [40].

%%%%%%%%%%%%%%%%%%%%%%%%%%%%%%%%%%%%%%%%%%%%%%%%%%%%%%%%%%%%%%%%%%%%%%%%%%%%%%%%%%%%%%%%%%%%%%%%%%%
%%%%%%%%%%%%%%%%%%%%%%%%%%%%%%%%%%%%%%%%%%%%%%%%%%%%%%%%%%%%%%%%%%%%%%%%%%%%%%%%%%%%%%%%%%%%%%%%%%%

\section{Pregunta de investigación}

¿Los registros de polisomnograma en adultos mayores pueden considerarse como series tiempo 
débilmente estacionarias?
%
¿Es posible que tal caracterización se veo influida por el estado cognitivo del sujeto?

%%%%%%%%%%%%%%%%%%%%%%%%%%%%%%%%%%%%%%%%%%%%%%%%%%%%%%%%%%%%%%%%%%%%%%%%%%%%%%%%%%%%%%%%%%%%%%%%%%%
%%%%%%%%%%%%%%%%%%%%%%%%%%%%%%%%%%%%%%%%%%%%%%%%%%%%%%%%%%%%%%%%%%%%%%%%%%%%%%%%%%%%%%%%%%%%%%%%%%%

\subsection{Hipótesis}

Existen diferencias en la actividad eléctrica cerebral en adultos mayores con PDC, respecto a 
individuos sanos, y es posible detectar dichas diferencias como una mayor o menor 
\textit{presencia} de estacionariedad débil en registros de PSG durante el sueño profundo.

%%%%%%%%%%%%%%%%%%%%%%%%%%%%%%%%%%%%%%%%%%%%%%%%%%%%%%%%%%%%%%%%%%%%%%%%%%%%%%%%%%%%%%%%%%%%%%%%%%%

\subsection{Objetivo general}

Obtener pruebas estadísticas formales para detectar si una serie de tiempo dada procede de un
proceso estocástico débilmente estacionario.
%
Usar tales pruebas sobre registros de polisomnograma en adultos mayores, para investigar si la 
presencia de segmentos débilmente estacionarios se correlaciona con la condición de probable
deterioro cognitivo.

%%%%%%%%%%%%%%%%%%%%%%%%%%%%%%%%%%%%%%%%%%%%%%%%%%%%%%%%%%%%%%%%%%%%%%%%%%%%%%%%%%%%%%%%%%%%%%%%%%%

\subsection{Objetivos específicos}

\begin{itemize}
\item Estudiar la definición de estacionariedad para procesos estocásticos

\item Investigar cómo detectar, como prueba de hipótesis, si una serie de tiempo dada proviene
de un proceso estocástico débilmente estacionario, y bajo qué supuestos 
es válida dicha caracterización

\item Decidir si los registros de PSG, durante sueño profundo, son débilmente estacionarios

\item Investigar si la presencia de segmentos estacionarios en los registros es diferente si el
PSG corresponde a un individuo con PDC
\end{itemize}

%%%%%%%%%%%%%%%%%%%%%%%%%%%%%%%%%%%%%%%%%%%%%%%%%%%%%%%%%%%%%%%%%%%%%%%%%%%%%%%%%%%%%%%%%%%%%%%%%%%
%%%%%%%%%%%%%%%%%%%%%%%%%%%%%%%%%%%%%%%%%%%%%%%%%%%%%%%%%%%%%%%%%%%%%%%%%%%%%%%%%%%%%%%%%%%%%%%%%%%