%%%%%%%%%%%%%%%%%%%%%%%%%%%%%%%%%%%%%%%%%%%%%%%%%%%%%%%%%%%%%%%%%%%%%%%%%%%%%%%%%%%%%%%%%%%%%%%%%%%
%%%%%%%%%%%%%%%%%%%%%%%%%%%%%%%%%%%%%%%%%%%%%%%%%%%%%%%%%%%%%%%%%%%%%%%%%%%%%%%%%%%%%%%%%%%%%%%%%%%
%%%%%%%%%%%%%%%%%%%%%%%%%%%%%%%%%%%%%%%%%%%%%%%%%%%%%%%%%%%%%%%%%%%%%%%%%%%%%%%%%%%%%%%%%%%%%%%%%%%

\chapter{Metodología y resultados}

\section{Características de los participantes}

Los sujetos fueron elegidos usando un muestreo \textit{no probabilístico por conveniencia} bajo los 
siguientes criterios de inclusión:
\begin{itemize}
\item Edad entre 60 y 85 años
\item Diestros (mano derecha dominante)
\item Sin ansiedad, depresión ni síndromes focales
\item No usar medicamentos o sustancias para dormir
\item Firma de consentimiento informado
\item Voluntario para el registro de PSG
\end{itemize}

Un total de 14 adultos mayores cumplieron los criterios de inclusión. Estos participantes fueron 
sometidos a una batería de pruebas neuropsicológicas para determinar su estado cognoscitivo general 
(Neuropsi, MMSE), descartar cuadros depresivos (GDS, SATS) y cambios en la vida cotidiana (KATZ).
%
En base a las pruebas se determinó que, objetivamente, 10 de los voluntarios no padecen depresión ni 
ansiedad, además de que no presentan afectaciones significativas en la vida diaria.

Para su análisis, los 10 participantes se dividieron en dos grupos en base a su estado cognoscitivo:
control (CTL) y con Probable Deterioro Cognitivo (PDC). 
%
Para esta clasificación se dio mayor atención al puntaje de Neuropsi, estandarizado según edad y 
escolaridad (cuadro \ref{puntajes}). 
%
El puntaje de MMSE se le otorgó menos importancia como clasificador debido a que tiene baja 
sensibilidad para el diagnóstico de deterioro cognitivo leve \cite{Ardila12}, y baja especificidad 
para individuos con escolaridad muy baja o muy alta \cite{Ostrosky00}.
%
Cabe mencionar que se entiende por especificidad a la probabilidad de un verdadero negativo, es 
decir que un individuo sin deterioro cognitivo obtenga un resultado de no-deterioro.

\begin{table}
\centering
\caption{Puntajes de corte para la prueba Neuropsi}
\begin{tabular}{llccrccc}
\toprule
&& \multicolumn{2}{l}{Sano} & \phantom{.} & \multicolumn{3}{l}{Deterioro cognitivo} \\
\cmidrule{3-4} \cmidrule{6-8} 
Escolaridad & Edad & Alto & Normal && Leve & Moderado & Severo\\
\midrule
Nula
& 16 -- 30 &\ppu 92 &\ppu 60 &&\ppu 45 & 30 & 14 \\
& 31 -- 50 &\ppu 95 &\ppu 68 &&\ppu 54 & 41 & 28 \\
& 51 -- 65 &\ppu 91 &\ppu 59 &&\ppu 44 & 28 & 13 \\
& 66 -- 85 &\ppu 76 &\ppu 48 &&\ppu 34 & 20 &\ppu 6 \\
\midrule
1 -- 4 años
& 16 -- 30 &    105 &\ppu 73 &&\ppu 58 & 42 & 27 \\
& 31 -- 50 &    105 &\ppu 81 &&\ppu 69 & 58 & 46 \\
& 51 -- 65 &\ppu 98 &\ppu 77 &&\ppu 67 & 57 & 47 \\
& 66 -- 85 &\ppu 90 &\ppu 61 &&\ppu 46 & 32 & 18 \\
\midrule
5 -- 9 años
& 16 -- 30 &    114 &    102 &&\ppu 97 & 86 & 75 \\
& 31 -- 50 &    118 &    106 &&    101 & 90 & 79 \\
& 51 -- 65 &    111 &\ppu 98 &&\ppu 91 & 79 & 67 \\
& 66 -- 85 &\ppu 97 &\ppu 80 &&\ppu 72 & 56 & 39 \\
\midrule
10 -- 24 años
& 16 -- 30 &    115 &    103 &&\ppu 98 & 87 & 77 \\
& 31 -- 50 &    113 &    102 &&\ppu 97 & 88 & 78 \\
& 51 -- 65 &    102 &\ppu 93 &&\ppu 88 & 80 & 72 \\
& 66 -- 85 &\ppu 92 &\ppu 78 &&\ppu 72 & 59 & 46 \\
\bottomrule
\multicolumn{5}{l}{Fuente: Ardila y Ostrosky \cite{Ardila12}}
\end{tabular}
\label{puntajes}
\end{table}

\begin{table}
\caption{Datos generales de los participantes}
\centering
\bordes{1.1}
{\small
\begin{tabular}{llcrrrrrrr}
\toprule
 \phantom{.}&
 & {Sexo} & {Edad} & {Escol.} & {Neuropsi} & {MMSE} & {SATS} & {KATZ} & {GDS} \\
\midrule
\multicolumn{6}{l}{\textbf{Grupo CTL}}\\
&VCR    & F    & 59\pz & 12\pz & 107\pz & 29\pz & 21\pz & 0\pz & 3\pz \\
&MJH    & F    & 72\pz & 9\pz  & 113\pz & 30\pz & 18\pz & 0\pz & 0\pz \\
&JAE    & F    & 78\pz & 5\pz  & 102\pz & 28\pz & 19\pz & 0\pz & 5\pz \\
&GHA    & M    & 65\pz & 9\pz  & 107.5  & 30\pz & 23\pz & 0\pz & 7\pz \\
&MFGR   & F    & 67\pz & 11\pz & 115\pz & 30\pz & 18\pz & 0\pz &      \\
\rowcolor{gris}
&\multicolumn{1}{c}{$\widehat{\mu}$} & 
               & 68.2  & 9.2   & 108.9  & 29.4  & 19.8  & 0.0  & 3.0  \\
\rowcolor{gris}
&\multicolumn{1}{c}{$\widehat{\sigma}$} & 
               & 7.2   & 2.7   & 5.2    & 0.9   & 2.2   & 0.0  & 3.0  \\
\midrulec
%\hline
\multicolumn{6}{l}{\textbf{Grupo PDC}}\\
&CLO    & F    & 68\pz & 5\pz  & 81\pz & 28\pz & 22\pz & 1\pz & 6\pz \\
&RLO    & F    & 63\pz & 9\pz  & 90\pz & 29\pz & 20\pz & 0\pz & 3\pz \\
&RRU    & M    & 69\pz & 9\pz  & 85\pz & 27\pz & 10\pz & 0\pz & 3\pz \\
&JGZ    & M    & 65\pz & 11\pz & 87\pz & 25\pz & 20\pz & 0\pz & 1\pz \\
&AEFP   & M    & 73\pz &  8\pz & 96\pz & 29\pz &   \pz & 0\pz & 2\pz \\
\rowcolor{gris}
&\multicolumn{1}{c}{$\widehat{\mu}$} & 
              & 67.6   & 8.4   & 87.8  & 27.4  & 18.0  & 0.2  & 3.0  \\
\rowcolor{gris}
&\multicolumn{1}{c}{$\widehat{\sigma}$} & 
              & 3.4    & 2.2   & 5.6   & 1.8   & 5.4   & 0.4  & 1.9  \\
\bottomrulec
\end{tabular} 
}
\label{tab_sujetos}
\end{table}

%%%%%%%%%%%%%%%%%%%%%%%%%%%%%%%%%%%%%%%%%%%%%%%%%%%%%%%%%%%%%%%%%%%%%%%%%%%%%%%%%%%%%%%%%%%%%%%%%%%
%%%%%%%%%%%%%%%%%%%%%%%%%%%%%%%%%%%%%%%%%%%%%%%%%%%%%%%%%%%%%%%%%%%%%%%%%%%%%%%%%%%%%%%%%%%%%%%%%%%

\subsection{Registro del polisomnograma}

Para llevar a cabo el registro, los adultos mayores participantes fueron invitados a acudir a las 
instalaciones del Laboratorio de Sueño, Emoción y Cognición, ubicado dentro del Instituto de Ciencias 
de la Salud (ICSa) dependiente de la Universidad Autónoma del Estado de Hidalgo. Los participantes 
recibieron instrucciones de realizar una rutina normal de actividades durante la semana que 
precedió al estudio, y se les recomendó no ingerir bebidas alcohólicas o energizantes (como café 
o refresco) durante las 24 horas previas al experimento, y que no durmieran siesta ese día.

Para efectuar el registro se usó un polisomnógrafo Medicid 5 (Neuronic Mexicana). El protocolo de 
PSG incluye: 
\begin{itemize}
\item 19 electrodos de EEG, colocadas siguiendo las coordenadas del Sistema Internacional 10--20
\item 2 electrodos de EOG para movimientos oculares horizontales y verticales
\item 2 electrodos de EMG colocados en los músculos submentonianos
\end{itemize}
%
Los electrodos para registro de EEG fueron montados usando los lóbulos auriculares como referencia
común; se mantuvo por debajo de \SI{50}{\micro\ohm}.

Las señales fueron amplificadas analógicamente usando amplificadores de alta ganancia en cadena, 
y adicionalmente fueron filtradas analógicamente usando filtros de paso de banda: 0.1--100 Hz 
para EEG, 3--20 Hz para EOG. 
Debido a dificultades técnicas el registro se efectuó a razón de 512 puntos por segundo (Hz) para 
algunos participantes, mientras que se usó 200 Hz para otros; en ambos casos se cumple la 
recomendación de la AASM de al menos 128 Hz.
%
Los registros digitalizados fueron almacenados en formato de texto bajo la codificación 
ASCII.

Los registros fueron segmentados en ventanas de 30 segundos de duración, referidas como 
\textit{épocas}, para su estudio posterior \textit{fuera de línea}. 
Usando los criterios de la AASM, cada una de las épocas fueron clasificadas según la etapa
de sueño como MOR o NMOR. Dicha clasificación fue llevada a cabo por expertos en sueño de ICSa.

\begin{table}
\centering
\caption{Datos generales sobre los registros de PSG}
\bordes{1.2}
{\small
\begin{tabular}{llcllcllr}
\toprule
    \phantom{.}&
    &\multirow{2}{*}{\bordes{1}\begin{tabular}{l}Frecuencia de\\ muestreo [\hz]\end{tabular}}
    \bordes{1.2}
    & \multicolumn{2}{c}{Total} & \phantom{l}   & \multicolumn{3}{c}{MOR*}\\
    \cmidrule{4-5}  \cmidrule{7-9}
    &&          &Puntos  &  Tiempo   &&Puntos  &  Tiempo   &  \% \\
\midrule
\multicolumn{6}{l}{\textbf{Grupo CTL}}\\
&VCR &200       &\ppu 5166000 & \ppu  7:10:30 &&\ppu 438000 &   0:36:30 & 8.48 \\
&MJH &512       &    15851520 & \ppu  8:36:00 &&    1950720 &   1:03:30 &12.31 \\
&JAE &512       &    13931520 & \ppu  7:33:30 &&    2626560 &   1:25:30 &18.85 \\
&GHA &200       &\ppu 6558000 & \ppu  9:06:30 &&\ppu 330000 &   0:27:30 & 5.03 \\
&MFGR&200       &\ppu 4932000 & \ppu  6:51:00 &&\ppu 570000 &   0:47:30 &11.56 \\

\rowcolor{gris}
&\multicolumn{1}{c}{$\widehat{\mu}$}  
              & &        & \ppu 7:51:30   &&        &   0:52:06 &11.25 \\
\rowcolor{gris}
&\multicolumn{1}{c}{$\widehat{\sigma}$} 
              & &        & \ppu 0:57:36   &&        &   0:23:00 & 5.13 \\
\midrulec

\multicolumn{6}{l}{\textbf{Grupo PDC}}\\
&CLO &512       &    14499840 & \ppu  7:52:00 &&    2027520 &   1:06:00 &13.98 \\
&RLO &512       &    12994560 & \ppu  7:03:00 &&    1520640 &   0:49:30 &11.70 \\
&RRU &200       &\ppu 2484000 & \ppu  3:27:00 &&\ppu 228000 &   0:19:00 & 9.18 \\
&JGZ &512       &    18539520 &      10:03:30 &&\ppu 506880 &   0:16:30 & 2.73 \\
&AEFP &512       &    14699520 &       7:58:30 &&\ppu 629760 &   0:20:30 & 4.28 \\

\rowcolor{gris}
&\multicolumn{1}{c}{$\widehat{\mu}$}  
              & &        & \ppu 7:16:48   &&        &   0:34:18 &8.38 \\
\rowcolor{gris}
&\multicolumn{1}{c}{$\widehat{\sigma}$} 
              & &        & \ppu 2:24:43   &&        &   0:22:14 &4.79 \\
\bottomrulec
\end{tabular}\\
*Dado que el sueño MOR aparece fragmentado, se reporta la suma de tales tiempos
}
\label{frecuencias}
\end{table}

%%%%%%%%%%%%%%%%%%%%%%%%%%%%%%%%%%%%%%%%%%%%%%%%%%%%%%%%%%%%%%%%%%%%%%%%%%%%%%%%%%%%%%%%%%%%%%%%%%%
%%%%%%%%%%%%%%%%%%%%%%%%%%%%%%%%%%%%%%%%%%%%%%%%%%%%%%%%%%%%%%%%%%%%%%%%%%%%%%%%%%%%%%%%%%%%%%%%%%%

\section{Análisis de las características de los participantes}

Previo al análisis de la estacionariedad, se corroboró la hipótesis de que las variables 
independientes son estadísticamente iguales entre los grupos CTL y PDC. Las comparaciones
usando la prueba $t$ de Welch (cuadro \ref{var_ind}) indican que efectivamente la hipótesis se cumple salvo
para los puntajes de Neuropsi.

Se evalúo si pudieran existir relaciones entre las variables que se presumen independientes.
Usando la prueba de correlación de Spearman (cuadro \ref{cor_ind}) se encontró sólo hay 
correlaciones monotónicas entre los siguientes pares de variables:
\begin{itemize}
\item Edad y Escolaridad
\item Puntaje en Neuropsi y Puntaje en Mini Mental-State Examination (MMSE)
\item Tiempo de MOR (en segundos) y Tiempo en MOR (porcentaje)
\end{itemize}

La primera relación, no muy fuerte, puede explicarse como un \textit{efecto generacional}: la educación 
superior ha aumentado su cobertura durante las últimas décadas, y entonces los grupos poblacionales 
más jóvenes tienen en promedio más años de escolaridad. 
%
Algunos autores han sugerido que un bajo nivel de escolaridad es un factor de riesgo para padecer
deterioro cognitivo, en base a estudios horizontales de larga escala \cite{Mejia_Arango2007}.
%
En el presente trabajo se ignora este dato, en vista de que no se pudieron relacionar el nivel de 
escolaridad de los participantes con su desempeño en pruebas neuropsicológicas.

La relación entre los puntajes en Neuropsi y en MMSE era de esperarse, ya que ambas pruebas miden
parámetros similares y tienen contenidos independientes. Cabe mencionar el curioso fenómeno en que (1) 
los puntajes de MMSE
tienen estadísticamente las mismas medias grupales, (2) los puntajes de MMSE están 
fuertemente correlacionados con los puntajes de Neuropsi, y (3) los puntajes de Neuropsi
tienen estadísticamente medias grupales diferentes. Se confirma que la prueba MMSE
tiene menor sensibilidad que la prueba Neuropsi para detectar deterioro cognitivo.

Era por demás obvia la relación entre la cantidad total de sueño MOR, con su proporción respecto a 
todo el sueño. Sin embargo, conviene mencionar que la cantidad de sueño MOR no es afectada por
ninguna de las otras variables independientes; luego entonces las cantidades que fueron estudiadas
(estacionariedad, espectro de potencias) no tienen correlaciones sesgadas con las demás variables.

\begin{table}
\centering
\caption{Variables independientes entre grupos}
\begin{tabular}{lrlcrlcccc}
\toprule
 & \multicolumn{2}{l}{Grupo CTL} & \phantom{.} & \multicolumn{2}{l}{Grupo PDC} 
 & \phantom{.} & \multicolumn{3}{l}{t de Welch}
 \\
\cmidrule{2-3} \cmidrule{5-6} \cmidrule{8-10}
& Media & (DE) & & Media & (DE) & & $p$ & $t$ & $\nu$ \\
\midrule
Edad              &  68.2   & (7.2)     & &    67.6 & (3.4)     & & 0.8746 & 0.16 & 6.11 \\
Escolaridad       &   9.2   & (2.7)     & &     8.4 & (2.2)     & & 0.6201 & 0.52 & 7.69 \\
Neuropsi          & 108.9   & (5.2)     & &    87.8 & (5.6)     & & \bf 0.0003 & 6.17 & 7.94 \\
MMSE              &  29.4   & (0.9)     & &    27.4 & (1.4)     & & 0.0706 & 0.16 & 6.11 \\
Sueño [s]         & 7:51:30 & (0:57:36) & & 7:16:48 & (2:24:43) & & 0.6836 & 0.50 & 5.24 \\
MOR [s]           & 0:52:06 & (0:23:00) & & 0:34:18 & (0:22:14) & & 0.2486 & 1.24 & 7.99 \\
MOR [\%]          &  11.3   & (5.1)     & &     8.4 & (4.8)     & & 0.3871 & 0.91 & 7.96 \\
\bottomrule 
\multicolumn{5}{l}{DE=Desviación Estándar}
\end{tabular} 
\label{var_ind}
\end{table}

\begin{table}
\centering
\caption{Correlaciones entre variables independientes}
\begin{tabular}{llllllll}
\toprule
  &             & 1 & 2 & 3 & 4 & 5 & 6 \\
\midrule
1 & Edad        & $\bullet$  &          &           &          &          & \\
2 & Escolaridad & -0.7134* &  $\bullet$ &           &          &          & \\
3 & Neuropsi    & -0.2432  & \phm 0.3776  & $\bullet$   &          &          & \\
4 & MMSE        & -0.1063  & \phm 0.1812  & 0.8477*** & $\bullet$  &          & \\
5 & Sueño [s]   & \phm  0.0486  & -0.0944  & 0.0545    & 0.0374   & $\bullet$  & \\
6 & MOR [s]     & \phm 0.2796  & -0.5035  & 0.1879    & 0.2618   & -0.1515  & $\bullet$ \\
7 & MOR [\%]    & \phm 0.3709  & -0.5287  & 0.0182    & 0.0748   & -0.3578  & 1**** \\
\bottomrule
\multicolumn{7}{l}{Niveles de significancia: *$<$.05 , **$<$.01 , ***$<$.005 , ****$<$.001}
\end{tabular}
\label{cor_ind}
\end{table}

%%%%%%%%%%%%%%%%%%%%%%%%%%%%%%%%%%%%%%%%%%%%%%%%%%%%%%%%%%%%%%%%%%%%%%%%%%%%%%%%%%%%%%%%%%%%%%%%%%%
%%%%%%%%%%%%%%%%%%%%%%%%%%%%%%%%%%%%%%%%%%%%%%%%%%%%%%%%%%%%%%%%%%%%%%%%%%%%%%%%%%%%%%%%%%%%%%%%%%%
%%%%%%%%%%%%%%%%%%%%%%%%%%%%%%%%%%%%%%%%%%%%%%%%%%%%%%%%%%%%%%%%%%%%%%%%%%%%%%%%%%%%%%%%%%%%%%%%%%%