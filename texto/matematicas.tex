%%%%%%%%%%%%%%%%%%%%%%%%%%%%%%%%%%%%%%%%%%%%%%%%%%%%%%%%%%%%%%%%%%%%%%%%%%%%%%%%%%%%%%%%%%%%%%%%%%%
%%%%%%%%%%%%%%%%%%%%%%%%%%%%%%%%%%%%%%%%%%%%%%%%%%%%%%%%%%%%%%%%%%%%%%%%%%%%%%%%%%%%%%%%%%%%%%%%%%%
%%%%%%%%%%%%%%%%%%%%%%%%%%%%%%%%%%%%%%%%%%%%%%%%%%%%%%%%%%%%%%%%%%%%%%%%%%%%%%%%%%%%%%%%%%%%%%%%%%%
%%%%%%%%%%%%%%%%%%%%%%%%%%%%%%%%%%%%%%%%%%%%%%%%%%%%%%%%%%%%%%%%%%%%%%%%%%%%%%%%%%%%%%%%%%%%%%%%%%%

\subsection{Matem\'aticas}

%Se describen algunos conceptos sobre estimaci\'on espectral para procesos estoc\'asticos 
%d\'ebilmente estacionarios, as\'i como la generalizaci\'on 
%propuesta por Priestley para procesos no-estacionarios. Esta descripci\'on se basa principalmente 
%en libro \textit{Spectral Analysis and Time Series} de M. B. Priestley \cite{Priestley81}.

%Como se mencion\'o, el objeto de estudio en este trabajo son registros de se\~nales
%electrofisiol\'ogicas (en particular, de PSG) que son modeladas como procesos estoc\'asticos; (por 
%simplicidad, se escribir\'a simplemente 'procesos'); antes de continuar, conviene presentar la 
%definici\'on de este concepto. 

%Si bien el de variable aleatoria es b\'asico para la probabilidad y la estad\'istica,
%en el presente trabajo s\'olo se expone de forma m\'as bien pragm\'atica en un ap\'endice 
%(definici\'on \ref{variable_aleatoria}). 
%En cambio se empieza definiendo un proceso estoc\'astico.

Debido a las condiciones particulares del presente trabajo, se expondrá de forma pragmática los
objetos abstractos usados y se darán los detalles formales pertinentes como apéndice. En particular,
se definirá lo que es un proceso estocástico dando por sentado que es conocido
el concepto de variable aleatoria (definición \ref{variable_aleatoria}).

\begin{defn}[Proceso estoc\'astico]
Un proceso estoc\'astico \xt es una familia de variables aleatorias real, 
indexadas por $t \in T$.
\label{proc_estocastico}
\end{defn}

Respecto al conjunto $T$ que indexa a un proceso estoc\'astico, y que ser\'a referido como 
\textit{tiempo}, conviene introducir dos grandes grupos para los mismos
\begin{itemize}
\item \textit{A tiempo continuo} si $T$ es un intervalo cerrado
\item \textit{A tiempo discreto} si $T$ es de la forma 
$\{ t_0 + n \delta \lvert n \in U \subseteq \mathbb{Z} \}$
\end{itemize}

Los procesos a tiempo discreto contemplan conjuntos finitos e infinitos de puntos en el tiempo.
%, as\'i como 'tiempos ausentes' si la cantidad de estos es negligible.
No se manejan discutir\'a sobre otros tipos de tiempo en este trabajo.

Como notaci\'on, se usar\'a \xt  para el proceso estoc\'astico y $X(t)$ para una de las variables
aleatorias que lo componen; de la misma manera $x(t)$ es una realizaci\'on de $X(t)$ y $F_{X(t)}$ 
es la funci\'on de probabilidad acumulada para $X(t)$.
%Cabe enfatizar que para cada valor de $t$, $X(t)$ es una 
%variable aleatoria y que una realizaci\'on del proceso {X(t)}.

%La definici\'on \ref{proc_estocastico} es aplicable tanto para procesos en tiempo continuo como 
%para tiempo discreto\footnote{Un proceso se dice \textbf{a tiempo discreto} cuando el \'indice $t$ 
%(tradicionalmente asociado al tiempo) pertenece a un conjunto a lo m\'as numerable, mientras que se 
%dice \textbf{a tiempo continuo} si este conjunto es un intervalo; en este trabajo s\'olo se 
%consideran estos dos tipos de 'tiempos'}.
Como modelo se asume que las se\~nales que componen el PSG constituyen un fen\'omeno a tiempo 
continuo que no es registrable sino en un conjunto finito de puntos en el tiempo; 
esta forma de entender a los datos 
permitir\'a asumir algunas propiedades que ser\'an descritas m\'as adelante.
%en otras palabras, se espera que 
%tenga caracter\'isticas propias de un proceso a tiempo continuo aunque en la pr\'actica se trate 
%con un proceso a tiempo discreto.

%\subsection{Estacionariedad d\'ebil}

%Una vez definidos los procesos estoc\'asticos, es posible definir una caracter\'istica de los 
%mismos y que se investiga en el presente trabajo: la estacionariedad.
%Informalmente, esta propiedad se refiere a que la 'estructura' del proceso sea invariante en el
%tiempo; para un proceso estacionario, calcular el promedio sobre cualquier subconjunto de datos 
%deber\'ia arrojar resultados estad\'isticamente id\'enticos, mientras que no hay tal garant\'ia
%para proceso no-estacionarios.
%La definici\'on \ref{est_fuerte}, referida como estacionariedad fuerte o estricta, induce fielmente 
%estas propiedades.

\begin{defn}[Estacionariedad fuerte]
Un proceso estoc\'astico \xt es fuertemente estacionario si, para cualquier conjunto de 
tiempos admisibles\footnote{El t\'ermino \textit{tiempos admisibles} indica que para tiempo 
discreto o continuo la afirmaci\'on es la misma, bajo las restricciones pertinentes} 
$t_1,t_2,\dots,t_n$ y cualquier $\tau$ tal que los tiempos $t_i+\tau$ son admisibles; se cumple que
\begin{equation*}
F_{\left(X(t_1),X(t_2),\dots,X(t_n)\right) }
\equiv
F_{\left(X(t_1+\tau),X(t_2+\tau),\dots,X(t_n+\tau)\right)}
\end{equation*}

Donde $F_{\left(X(t_1),X(t_2),\dots,X(t_n)\right) }$ es la funci\'on de distribuci\'on de 
probabilidad conjunta para el vector $\left(X(t_1),X(t_2),\dots,X(t_n)\right)$
\label{est_fuerte}
\end{defn}

%Sin embargo, la definici\'on \ref{est_fuerte} resulta in\'util si se pretende verificar que un 
%registro dado pudiera ser modelado como la realizaci\'on de un proceso es fuertemente estacionario 
%(un objetivo descartado para este trabajo). 
%En ese escenario, para cada tiempo registrado $t$ se 
%tiene s\'olo una observaci\'on para cada variable aleatoria $X(t)$, lo cual es muy poca 
%informaci\'on como para estimar las funciones de probabilidad acumulada mencionadas en la 
%definici\'on.
%Estas limitaciones motivan una definici\'on m\'as d\'ebil de estacionariedad, pero que pueda ser 
%detectada en realizaciones dadas de procesos; se exhibe la definici\'on \ref{est_orden_m} como 
%alternativa.

%\begin{defn}[Estacionariedad de orden $\boldsymbol{m}$]
%Un proceso estoc\'astico $\{ X(t) \}$ se dice estacionario de orden $m$ si, para cualquier conjunto 
%de tiempos admisibles $t_1,t_2,\dots,t_n$ y cualquier $\tau \in \R$ se cumple que
%\begin{equation*}
%\E{ X^{m_1}(t_1)X^{m_2}(t_2)\cdots X^{m_n}(t_n) }
%=
%\E{ X^{m_1}(t_1+\tau)X^{m_2}(t_2+\tau)\cdots X^{m_n}(t_n+\tau) }
%\end{equation*}
%Para cualesquiera enteros $m_1,m_2,\dots,m_n$ tales que $m_1+m_2+\dots+m_n \leq m$
%\label{est_orden_m}
%\end{defn}

%Para entender mejor la definici\'on \ref{est_orden_m} y sus limitaciones frente a la 
%estacionariedad fuerte, consid\'erense tres procesos: $\{X(t)\}$ fuertemente estacionario, 
%$\{Y_1(t)\}$ estacionario de orden 1, y $\{Y_2(t)\}$ estacionario de orden 2. Luego
%\begin{itemize}
%\item Las medias\footnote{La media de una variable aleatoria $V$ se define como $ \mu_V := \E{V}$} 
%$ \mu_{X(t)}$, $ \mu_{Y_1(t)}$ y $ \mu_{Y_2(t)}$ no dependen de $t$
%
%\item Las varianzas\footnote{La varianza de una variable aleatoria $V$ se define como 
%$ \Var{V} := \E{\left(V - \mu_V \right)^{2}}$} $ \Var{Y_1(t)}$ y $ \Var{Y_2(t)}$ no dependen de 
%$t$, pero no se puede garantizar lo mismo para $\Var{X(t)}$
%
%\item El coeficiente de asimetr\'ia\footnote{El coeficiente de asimetr\'ia para una variable 
%aleatoria $V$ se define como 
%$\gamma_V = \frac{\E{\left(V-\mu_V\right)^{3}}}{\Var{V}^{\nicefrac{3}{2}}}$}
%$ \gamma_{X(t)}$ no depende de $t$, pero no se puede garantizar lo mismo para $ \gamma_{Y_1(t)}$ ni 
%para $ \gamma_{Y_2(t)}$
%\end{itemize}
%
%Cabe mencionar que hay una relaci\'on de contenci\'on clara en familia de los conjuntos de procesos 
%estacionarios de orden finito (si un proceso es estacionario de orden $m$, entonces es estacionario 
%de orden $n$ para todo $n \leq m$); es posible definir procesos estacionarios de orden 'infinito' 
%seg\'un \ref{est_orden_m}, que intuitivamente ser\'ian fuertemente estacionarios. 
%De manera pragm\'atica, en este trabajo no se discuten tales interrogantes, sino que se usar\'a 
%\'unicamente la definici\'on correspondiente al caso $m=2$, referida como estacionariedad d\'ebil o 
%de orden 2, y repetida en la definici\'on \ref{est_orden_2}.

\begin{defn}[Estacionariedad d\'ebil]
Un proceso estoc\'astico $\{ X(t) \}$ es d\'ebilmente estacionario si, para cualesquiera tiempos 
admisibles $t, s, t+\tau, s+\tau$, se cumple que
\begin{center}
\begin{tabular}{ccc}
$\E{X(t)} = \E{X(t+\tau)}$
& y &
$\E{X(t)X(s)} = \E{X(t+\tau)X(s+\tau)}$
\end{tabular}
\end{center}
\label{est_orden_2}
\end{defn}

M\'as a\'un, parece conveniente exhibir una caracterizaci\'on equivalente para los procesos 
d\'ebilmente estacionarios, pero que tiene una interpretaci\'on m\'as sencilla.
\begin{thrm}
Un proceso estoc\'astico es d\'ebilmente estacionario si y s\'olo si para cualesquiera tiempos 
admisibles $t$, $s$ se tiene que
\begin{itemize}
\item $\E{X(t)} = \mu_X$
\item $\Var{X(t)} = \sigma^{2}_X$
\item $\Cov{X(t),X(s)} = \rho_X (s-t)$
\end{itemize}
Donde $\mu_X$, $\sigma^{2}_X$ son constantes, $\rho_X(\tau)$ es una funci\'on que \'unicamente 
depende de $\tau$
\label{est_usual}
\end{thrm}

%Cabe comentar  sobre la existencia de procesos que son fuertemente estacionarios pero que no son 
%estacionarios de ning\'un orden: por ejemplo, un proceso de variables aleatorias independientes con 
%distribuci\'on de Cauchy\footnote{Una variable aleatoria tiene distribuci\'on de Cauchy si su 
%funci\'on de probabilidad acumulada es de la forma 
%$\displaystyle F(x) = \frac{1}{\pi} \int_{-\infty}^{x} \frac{1}{1+y^{2}} dy$}.
%Una condici\'on suficiente para que un proceso fuertemente estacionario sea estacionario de orden 
%$m$ es que tenga sus primeros $m$ momentos bien definidos.
%Con respecto a las se\~nales registradas en el EEG, entendidas como procesos estoc\'asticos, se 
%espera que tengan (cuando menos) segundos momentos bien definidos; m\'as adelante se presentan 
%argumentos, desde una interpretaci\'on f\'isica, sobre por qu\'e se espera que ocurra lo anterior.

A continuaci\'on (definici\'on \ref{cont_est}) se presenta una tipo de regularidad que se supone
para las se\~nales registradas en el EEG: continuidad de alg\'un tipo.

\begin{defn}[Continuidad estoc\'astica en media cuadr\'atica]
Un proceso estoc\'astico a tiempo continuo $\{ X(t) \}$ es estoc\'asticamente continuo, en el 
sentido de media cuadr\'atica, en un tiempo admisible $t_0$ si y s\'olo si
\begin{equation*}
\lim_{t \rightarrow t_0} \E{\left( X(t) - X(t_0) \right)^{2}} = 0
\end{equation*}
\label{cont_est}
\end{defn}

Una forma natural de pensar en la definici\'on \ref{cont_est} es que, si $\abso{t-t_0}$ es muy 
peque\~no, entonces $X(t)$ y $X(t_0)$ difieren muy poco entre s\'i (como variables aleatorias).
Es destacable que si un proceso es estoc\'asticamente continuo en un intervalo, sus realizaciones 
solamente se pueden garantizar continuas casi en todas partes \footnote{Una propiedad se cumple 
\textbf{casi en todas partes} si se cumple en un conjunto cuyo complemento tiene medida cero} en 
ese intervalo.
%Como ejemplos, un proceso ruido blanco (definici\'on \ref{r_blanco}) no es estoc\'asticamente 
%continuo, mientras que un proceso de Wiener (definici\'on \ref{r_wiener}) s\'i lo es.
%
%\begin{defn}[Proceso ruido blanco]
%Se dice de un proceso estoc\'astico $\{ R(t) \}$ que cumple, para cualesquiera tiempos admisibles
%$t$ y $s$, las siguientes propiedades:
%\begin{itemize}
%\item $\E{R(t)}=0$
%\item $\Cov{R(t),R(s)}=0 \Leftrightarrow t=s$ 
%\end{itemize}
%\label{r_blanco}
%\end{defn}
%
%\begin{defn}[Proceso de Wiener]
%Se dice de un proceso estoc\'astico $\{ W(t) \}$ que cumple, para cualesquiera tiempos admisibles
%$t$ y $s$ (con $s>t$) las siguientes propiedades:
%\begin{itemize}
%\item $W(0) = 0$ ($W(0)$ es constante)
%\item $W(s)-W(t)$ es independiente de $W(u)$, para todo $u<t$ admisible
%\item $W(s)-W(t) \sim N(0,\abso{t-s})$  (los incrementos tienen distribuci\'on normal)
%\end{itemize}
%\label{r_wiener}
%\end{defn}

%%%%%%%%%%%%%%%%%%%%%%%%%%%%%%%%%%%%%%%%%%%%%%%%%%%%%%%%%%%%%%%%%%%%%%%%%%%%%%%%%%%%%%%%%%%%%%%%%%%
%%%%%%%%%%%%%%%%%%%%%%%%%%%%%%%%%%%%%%%%%%%%%%%%%%%%%%%%%%%%%%%%%%%%%%%%%%%%%%%%%%%%%%%%%%%%%%%%%%%

\subsubsection{Espectro de potencias para procesos estoc\'asticos}

%Existe una larga tradici\'on en ciencias biom\'edicas para entender/modelar las se\~nales
%electrofisiol\'ogicos en t\'erminos de ondas y frecuencias, ya que fundamentalmente son fen\'omenos 
%el\'ectricos \cite{Kaiser00}.  
%Estos modelos est\'an t\'ipicamente asociados con las series de Fourier y la transformada de 
%Fourier (definiciones \ref{FourierClasico} y \ref{trFourier}, respectivamente).

\begin{defn}[Serie de Fourier]
Sea $k$ una funci\'on real peri\'odica (con periodo $2T$) tal que 
$\simint{T} \abso{k(t)} dt < \infty$. 
Se le llamar\'a 'serie de Fourier para la funci\'on $k$' a la sucesi\'on $\left( A_n \right)$, 
calculada de la siguiente manera
\begin{equation*}
A_n = \frac{1}{2 T} \simint{T} k(t) e^{-\nicefrac{ i n t}{2T}} dt
\end{equation*}
\label{FourierClasico}
\end{defn}

\begin{defn}[Transformada de Fourier]
Sea $P_T$ el espacio de las funciones peri\'odicas con periodo $2T$ que tienen una serie de Fourier 
bien definida. Existe una funci\'on $\mathfrak{F}$ que mapea cada funci\'on a su respectiva serie 
de Fourier, y \'esta ser\'a referida como la transformada de Fourier
\label{trFourier}
\end{defn}

Como se mencion\'o, en este trabajo se suponen conocidas las propiedades de las series de Fourier 
(seg\'un la definici\'on \ref{FourierClasico}), de entre las cuales se destacan las siguientes:
\begin{itemize}
\item Las series de Fourier  son cuadrado-sumables\footnote{Una sucesi\'on $\left( S_n \right)$ se 
dice cuadrado-integrable si cumple que $\sum_{n\in \mathbb{Z}} \abso{S_n}^{2} < \infty$. El 
conjunto de estas series es denotado por $\ell^{2}$}

\item La transformada de Fourier, $\mathfrak{F}$, no es invertible en general. Es com\'un definir
una pseudoinversa como 
$\mathfrak{F}_{\text{inv}} 
: \left( A_n \right) \mapsto \sum_{n \in \mathbb{Z}} A_n e^{\nicefrac{i n t}{2 T}} $

\item Los conjuntos $P_T$ y $\ell^{2}$, con la suma y producto usuales, tienen la estructura de 
espacio vectorial. M\'as a\'un, usando las respectivas normas $\Vert k \Vert = \intR k(t) dt$ y 
$\left\Vert \left( A_n \right) \right\Vert_2 = \sum_{n\in \mathbb{Z}} \abso{S_n}^{2} < \infty$,
ambos son espacios de Hilbert.
\end{itemize}

La transformada de Fourier goza de una interpretaci\'on f\'isica muy extendida, seg\'un la cual 
toda se\~nal peri\'odica puede verse como la superposici\'on (suma) de se\~nales senoidales 
ortogonales de diferentes frecuencias\footnote{De manera pragm\'atica, en el presente trabajo la 
palabra  'frecuencia' se usar\'a para referirse a la cantidad $q$ en expresiones del tipo 
$e^{i q t}$}; suponiendo previamente que las se\~nales se suman y multiplican por escalares de la 
manera usual. 
Esta afirmaci\'on es equivalente a que el conjunto de funciones con una serie de Fourier bien 
definida, tiene estructura de espacio vectorial y admite una base ortonormal de funciones 
senoidales (referida como la base de Fourier). 
La interpretaci\'on f\'isica adquiere mayor relevancia cuando se exhibe el concepto de energ\'ia
disipada (definici\'on \ref{energia}) en conjunto con el el teorema \ref{parseval_serie}, ya que 
permite formular la siguiente interpretaci\'on: la energ\'ia disipada por una se\~nal peri\'odica 
puede verse como la suma de la energ\'ia disipada por sus componentes en la base de Fourier 
(usualmente referidos como \textbf{componentes de frecuencias}).
La funci\'on que 'desglosa' estos componentes se conoce como espectro de potencias (definici\'on 
\ref{espec}).

\begin{defn}[Energ\'ia de una se\~nal]
Sea $k$ una funci\'on real que modela una se\~nal. La energ\'ia disipada por $k$ en el intervalo de
tiempo $[a,b]$ est\'a dada por
\begin{equation*}
\text{Energ\'ia}_{[a,b]} = \int_{a}^{b} \abso{k\left(t\right)}^{2} dt
\end{equation*}
\label{energia}
\end{defn}

\begin{thrm}[Relaci\'on de Parseval]
Sea $k$ una funci\'on peri\'odica (de periodo $2T$) que admite una serie de Fourier bien definida,
$(A_n)$. Se cumple que
\begin{equation*}
\int_T^{T} k^{2}(t) dt = \sum_{n=-\infty}^{\infty} \abso{A_n}^{2}
\end{equation*}
\label{parseval_serie}
\end{thrm}

%\newpage

\begin{defn}[Espectro de potencias]
Sea $P_T$ el espacio de las funciones peri\'odicas con periodo $2T$ que tienen una serie de Fourier 
bien definida. Se le llama espectro de potencias a la funci\'on 
$h: P_T \times \mathbb{R} \mapsto \mathbb{R}$ definida como
\begin{equation*}
h_k(\omega) = 
\begin{cases}
\abso{A_i}^{2} & \text{ , si } \omega = \frac{n}{2T} \text{   con } n\in \mathbb{Z} \\
0 & \text{ ,  otro caso}
\end{cases}
\end{equation*}
donde $\left( A_n \right)$ es la serie de Fourier de $k$
\label{espec}
\end{defn}

Es importante mencionar que la energ\'ia, entendida como la integral de una forma cuadr\'atica, es 
un concepto com\'un a varias ramas de la f\'isica y las ingenier\'ias; en cambio, en econom\'ia o 
en epidemiolog\'ia, por ejemplo, no hay una motivaci\'on clara para usar el concepto de energ\'ia 
seg\'un \ref{energia}.
%En el presente trabajo no s\'olo se manejar\'a este concepto de energ\'ia, sino que se supondr\'a
%que los registros de PSG contienen energ\'ia finita para cualquier intervalo finito de tiempo.

%%%%%%%%%%%%%%%%%%%%%%%%%%%%%%%%%%%%%%%%%%%%%%%%%%%%%%%%%%%%%%%%%%%%%%%%%%%%%%%%%%%%%%%%%%%%%%%%%%%
%%%%%%%%%%%%%%%%%%%%%%%%%%%%%%%%%%%%%%%%%%%%%%%%%%%%%%%%%%%%%%%%%%%%%%%%%%%%%%%%%%%%%%%%%%%%%%%%%%%

%\subsubsection{Espectro de potencias para funciones no-peri\'odicas}

%A continuaci\'on se presentan dos generalizaciones de las series de Fourier previas, a aquella que
%ser\'a usada para procesos estoc\'asticos. La primera de ellas es las integrales de Fourier
%(definici\'on \ref{fourier_int}), que operan de la misma forma que las series de Fourier pero
%sobre funciones absolutamente sumables; se pueden extender sencillamente la transformada de Fourier 
%y el espectro de potencias para esta funci\'on.

\begin{defn}[Integral de Fourier]
Sea $k$ una funci\'on tal que $\intR \abso{k(t)} dt < \infty$. Se le llamar\'a 'integral de 
Fourier para la funci\'on $k$' a la funci\'on $A:\mathbb{R}\rightarrow \mathbb{C}$ calculada de la 
siguiente manera
\begin{equation*}
A(\omega) = \frac{1}{2 \pi} \intR k(t) e^{- i \omega t} dt
\end{equation*}
\label{fourier_int}
\end{defn}

%La integral de Fourier  se encuentra bien definida para cualquier $\omega \in \mathbb{R}$, e 
%incluso se puede probar que es una funci\'on continua y que converge a 0 en $\pm \infty$.
%Sin embargo, por brevedad s\'olo se usar\'a como motivaci\'on para la transformada de 
%Fourier-Stieltjes.

\begin{defn}[Transformada de Fourier-Stieltjes]
Se dice que una funci\'on real $A$ es la transformada de Fourier-Stieltjes de una funci\'on $k$ si 
\'esta puede expresarse casi en todas partes como
\begin{equation*}
k(x) = \intR e^{ i \omega t} dF(\omega)
\end{equation*}
siendo que la integral est\'a definida en el sentido de Stieltjes
\label{fourier_stieltjes}
\end{defn}

%En la definici\'on \ref{fourier_stieltjes}, si $F$ es derivable en todas partes entonces $F\prima$ 
%cumple el mismo papel que la integral de Fourier; en cambio, si $k$ es una funci\'on peri\'odica 
%entonces $F$ toma una forma escalonada cuyos aumentos coinciden con la serie de Fourier para $k$. 
%M\'as a\'un, existen funciones que no son ni peri\'odicas ni absolutamente sumables pero poseen 
%una transformada de Fourier-Stieltjes, como $k(x)=\SEN{x}+\SEN{\sqrt{2}x}$, 
%$k(x)=\COS{x} + (1+x^{2})^{-1}$.
%
%\begin{comment}
%\begin{thrm}[Descomposici\'on de Lebesgue]
%Sea $f:I\rightarrow \R$ una funci\'on de variaci\'on acotada, con $I$ un intervalo. Entonces pueden 
%hallarse funciones $f_j, f_c, f_a :I\rightarrow \R$ tales que
%\begin{itemize}
%\item $f = f_j+ f_c+ f_a$
%\item $f_j = \sum_{y \leq x} f(x-0) + f(x+0)$
%\item $f_a$ es absolutamente continua\footnote{Para que una funci\'on sea absolutamente continua,
%basta que sea de variaci\'on acotada y que mapee conjuntos de medida cero en conjuntos de medida
%cero} en $I$
%\item $f_c$ es una funci\'on singular\footnote{Una funci\'on es singular si es continua, de 
%variaci\'on acotada y no-constante, y se cumple que tiene derivada cero casi en todas partes} en 
%$I$
%\end{itemize}
%Estas funciones son \'unicas excepto por constantes, y en conjunto son llamados la 
%\textit{descomposici\'on de Lebesgue} de $f$
%\label{Lebesgue_decomp}
%\end{thrm}
%\end{comment}

%%%%%%%%%%%%%%%%%%%%%%%%%%%%%%%%%%%%%%%%%%%%%%%%%%%%%%%%%%%%%%%%%%%%%%%%%%%%%%%%%%%%%%%%%%%%%%%%%%%
%%%%%%%%%%%%%%%%%%%%%%%%%%%%%%%%%%%%%%%%%%%%%%%%%%%%%%%%%%%%%%%%%%%%%%%%%%%%%%%%%%%%%%%%%%%%%%%%%%%

%\subsubsection{Espectro de potencias para un proceso d\'ebilmente estacionario}

Por simplicidad, el enfoque la definici\'on que se aborda considera un proceso d\'ebilmente 
estacionario, $\{X(t)\}$, y posteriormente una realizaci\'on particular, $x(t)$; como no hay 
garant\'ia que $x$ tenga una transformada de Fourier bien definida (o alguna de las 
generalizaciones presentadas previamente), se procede a aproximarla por una serie de funciones que 
s\'i las tienen.
As\'i entonces, para cada $T>0$ se define a $x_T$ como
\begin{equation*}
x_T(t) = 
\begin{cases}
x(t) & \text{ , } -T\leq t \leq T \\
0 & \text{ , otro caso}
\end{cases}
\end{equation*}
Posteriormente se define a $G_T$; bien puede entenderse como una transformada de Fourier para una 
funci\'on extendida peri\'odicamente, y de modo que el espacio entre frecuencias se vuelven 
infinitesimal, o bien puede verse como una integral de Fourier (claramente bien definida) para una 
serie convergente de funciones.

\begin{equation*}
G_T (\omega) = \frac{1}{\sqrt{2 \pi}} \intR x_T(t) e^{-i \omega t} dt
= \frac{1}{\sqrt{2 \pi}} \int_{-T}^{T} x(t) e^{-i \omega t} dt
\end{equation*}

Sin la garant\'ia de que $x(t)$ tenga una integral de Fourier bien definida, no hay garant\'ia que 
$G_T$ converja cuando $T\rightarrow \infty$. Recuperando el concepto de energ\'ia disipada, ligado 
a $\left| G_T(\omega) \right|^{2}$, se puede construir una explicaci\'on f\'isica para la 
divergencia de $G_T$: un sistema que disipa 'niveles constantes' de energ\'ia (porque $\Var{X(t)}$ 
es constante) puede disipar una cantidad infinita de energ\'ia durante un tiempo infinito. 
Conviene, entonces, usar el tama\~no de los intervalos sobre los cuales se define $G_T$, de modo 
que se converja a una 'densidad' de energ\'ia disipada:
$\lim_{T\rightarrow{\infty}} \frac{ \left| G_T(\omega) \right|^{2}}{2 T}$.
Tal expresi\'on se puede adaptar al proceso per se como en la definici\'on \ref{FDE}.

\begin{defn}[Funci\'on de densidad espectral (FDE)]
Sea $\{X(t)\}$ un proceso estoc\'astico en tiempo continuo, d\'ebilmente estacionario. Se define la 
funci\'on de densidad espectral (FDE) para $\{X(t)\}$ como
%\begin{equation*}
%h(\omega) = \lim_{T\rightarrow \infty} \E{ \frac{ \left| G_T(\omega) \right|^{2}}{2 T} }
%\end{equation*}
\begin{equation*}
h(\omega) = \lim_{T\rightarrow \infty} \E{ \frac{1}{2T} \frac{1}{2 \pi}
\abso{ \int_{-T}^{T} X(t) e^{-i \omega t} dt}^{2} }
\end{equation*}
%Donde $G_T (\omega) = \frac{1}{\sqrt{2 \pi}} \int_{-T}^{T} X(t) e^{-i \omega t} dt$
\label{FDE}
\end{defn}

Conviene mencionar que la convergencia de la FDE est\'a garantizada casi en todas partes, pero la
funci\'on resultante puede no ser suficiente para explicar adecuadamente al proceso; por ejemplo, 
un proceso peri\'odico, con periodo $T$, tendr\'a una FDE que es cero en todos sus puntos salvo en 
los m\'ultiplos enteros de $T$.
Este fen\'omeno es similar al caso de variables aleatorias discretas, que no poseen una funci\'on 
de densidad de probabilidad bien definida en todos sus puntos.
Conviene definir una versi\'on 'integral' de la FDE (definici\'on\ref{FDE_integrado}).

\begin{defn}[Funci\'on de espectro integrado]
Sea $\{X(t)\}$ un proceso estoc\'astico a tiempo continuo, d\'ebilmente estacionario. Se define la 
funci\'on de espectro integrado para $\{X(t)\}$ como
\begin{equation*}
H(\omega) = \int_{-\infty}^{\omega} h(\lambda) d\lambda
\end{equation*}
Donde $h$ es la funci\'on de densidad espectral para $\{X(t)\}$
\label{FDE_integrado}
\end{defn}

Si la FDE, $h$, est\'a bien definida en todos sus puntos, entonces la funci\'on de espectro 
integrado ($H$) satisface que $H\prima= h$ y se dir\'a que el proceso tiene un \textbf{espectro 
puramente continuo}; si $H$ tiene una forma escalonada, con escalones rectos, se dir\'a que es un 
\textbf{espectro puramente discreto}.
Como es de esperarse, cada tipo de proceso tiene caracter\'istica diferentes y se puede estudiar 
mejor con herramientas diferentes; para el caso de procesos con un espectro mixto (ninguno de los 
anteriores), se exhiben herramientas que los reducen a estos casos 'puros'.

Cabe destacar que, por como se defini\'o la FDE integrada, \'esta es una funci\'on positiva, 
no-decreciente, y que en $-\infty$ vale 0; esta observaci\'on ser\'a importante.

%%%%%%%%%%%%%%%%%%%%%%%%%%%%%%%%%%%%%%%%%%%%%%%%%%%%%%%%%%%%%%%%%%%%%%%%%%%%%%%%%%%%%%%%%%%%%%%%%%%
%%%%%%%%%%%%%%%%%%%%%%%%%%%%%%%%%%%%%%%%%%%%%%%%%%%%%%%%%%%%%%%%%%%%%%%%%%%%%%%%%%%%%%%%%%%%%%%%%%%

\subsubsection{Estimaci\'on de la funci\'on de densidad espectral}

Anteriormente se defini\'o espectro de potencias para procesos d\'ebilmente estacionarios con 
segundos momentos finitos (definici\'on \ref{FDE}); sin embargo, aquella definici\'on es sumamente 
ineficiente dentro de la estimaci\'on pues requiere valores esperados sobre todo del proceso.
A continuaci\'on se muestran dos teoremas respecto a la FDE facilitar\'an su estimaci\'on.

\begin{thrm}[Wiener-Khinchin]
Una condici\'on suficiente y necesaria para que $\rho$ sea una funci\'on de autocorrelaci\'on de 
alg\'un proceso estoc\'astico a tiempo continuo $\{X(t)\}$ d\'ebilmente estacionario y 
estoc\'asticamente continuo, es que exista una funci\'on $F$ que tenga las siguientes propiedades
\begin{itemize}
\item Mon\'otonamente creciente
\item $F(-\infty) = 0$
\item $F(+\infty) = 1$
\end{itemize}
y tal que para todo $\tau \in \R$ se cumple que
\begin{equation*}
\rho(\tau) = \intR e^{i \omega \tau} dF(\omega)
\end{equation*}
\label{t_wienerkhinchin}
\end{thrm}

\begin{thrm}[Wold]
Una condici\'on suficiente y necesaria para que $\rho$ sea una funci\'on de autocorrelaci\'on de 
alg\'un proceso estoc\'astico a tiempo discreto $\{X(t)\}$ d\'ebilmente estacionario es que exista 
una funci\'on $F$ con las siguientes propiedades
\begin{itemize}
\item Mon\'otonamente creciente
\item $F(-\pi) = 0$
\item $F(+\pi) = 1$
\end{itemize}
y tal que para todo $\tau \in \R$ se cumple que
\begin{equation*}
\rho(\tau) = \intPI e^{i \omega \tau} dF(\omega)
\end{equation*}
\label{t_wold}
\end{thrm}

Una observaci\'on interesante sobre estos teoremas es el caso $\tau = 0$
\begin{equation*}
\rho(0) = \int_{-A}^{+A} dF(\omega) = F(A) - F(-A)
\end{equation*}
donde $A$ vale $\infty$ o $\pi$ seg\'un sea el caso discreto o continuo. Si $R$ es la funci\'on de
autocovarianza del proceso, entonces la ecuaci\'on anterior se traduce en que
\begin{equation*}
R(0) = \sigma^{2} \left( F(A) - F(-A) \right) = \sigma^{2} F(A)
\end{equation*}
donde $\sigma^{2}$ es la varianza del proceso. 
Esta observaci\'on adquiere importancia porque la FDE integrada ($H$), por definici\'on, satisface 
el papel de $F$ salvo por la condici\'on $F(\infty)=1$; si se puede garantizar que 
$H(\infty)<\infty$ entonces puede ser normalizada para satisfacer tal condici\'on y, m\'as a\'un,
si tal fuera el caso entonces $H(\infty)=\sigma^{2}$. Una consecuencia muy fuerte de este 
comentario es que, como se ha establecido previamente que s\'olo se considerar\'an procesos con
segundos momentos finitos, entonces la FDE de los procesos considerados siempre es acotada.

Como corolario, se puede afirmar que la FDE de un proceso (d\'ebilmente estacionario y de varianza
finita) es la transformada de Fourier-Stieltjes de la funci\'on de autocovarianza.
Esto implica que el estimador m\'as 'natural' para la FDE es la transformada de Fourier (discreta) 
de la funci\'on de autocorrelaci\'on (estimada); esta funci\'on se conoce como 
\textit{periodograma}. 

Conviene introducir estimadores para la funci\'on de autocovarianza de un proceso d\'ebilmente 
estacionario, $\{ X(t) \}$, a partir de un conjunto de $N$ observaciones equiespaciadas en el 
tiempo con separaci\'on $\Delta t$; se denotar\'a a estas observaciones como 
$x_1, x_2 , \dots, x_N$. Como se cumple la siguiente propiedad para la funci\'on de autocovarianza, 
$R$, por definici\'on
\begin{equation*}
R(\tau) = \E{X(n\Delta t)X(n\Delta t + \tau)} \text{  ,  } n = 0, 1, 2,  3,\dots, N
\end{equation*}
el estimados est\'andar para $R$ est\'a dado por la siguiente expresi\'on
\begin{equation*}
\widehat{R}(\tau) = \frac{1}{N-\abso{\tau}} 
\sum_{t = 1}^{N-\abso{\tau}} x_t x_{t+\abso{\tau}}
\label{estimador_R}
\end{equation*}

Se puede demostrar que $\widehat{R}$ es un estimador insesgado\footnote{Un estimador para el 
par\'ametro $\theta$, $\widehat{\theta}$, se dice \textbf{insesgado} si 
$\E{\widehat{\theta}}=\theta$} y consistente\footnote{Un estimador para el par\'ametro $\theta$ que 
depende de $N$ observaciones, 
$\widehat{\theta}_N$, se dice \textbf{consistente} si 
$\lim_{N\rightarrow \infty} \Var{\widehat{\theta}_N} = 0$} 
para $R$; sin embargo conviene introducir un estimador diferente para $R$
\begin{equation*}
\aste{R}(\tau) = \frac{1}{N} 
\sum_{t = 1}^{N-\abso{\tau}} x_t x_{t+\abso{\tau}}
\label{estimador_R_ast}
\end{equation*}

Se puede demostrar que $\aste{R}$ tiene las siguientes propiedades:
\begin{itemize}
\item $\E{\aste{R}(\tau)} = \left(1 - \frac{\abso{\tau}}{N} \right) R(\omega)$
\item $\Var{\aste{R}(\tau)} \approx \frac{1}{N} 
\sum_{r=-\infty}^{\infty} \left( R^{2}(r) + R(r-\tau)R(r+\tau) \right)$
\item $\Cov{\aste{\rho}(\tau),\aste{\rho}(\tau+\nu)} \approx \frac{1}{N} 
\sum_{r=-\infty}^{\infty} \left( \rho(r)\rho(r+\nu) + \rho(r-\tau)\rho(r+\tau+\nu) \right)$
\end{itemize}
Las aproximaciones para la varianza y covarianza se vuelven exactas si el proceso sigue una 
distribuci\'on normal en todos los tiempos.

Bajo estas consideraciones, se exhibe formalmente el periodograma (definici\'on 
\ref{periodograma}); si bien la expresi\'on con que se escribe es f\'acil de interpretar (la 
transformada de Fourier discreta para la se\~nal) y es eficiente computacionalmente, es conveniente
ligarla con la funci\'on de autocovarianza (teorema \ref{periodograma_rho}).

\begin{defn}[Periodograma]
Sean $x_1, x_2 , \dots, x_N$ observaciones de un proceso estoc\'astico de media cero y varianza
finita. Se llama periodograma a la funci\'on $I_N: [-\pi,\pi] \rightarrow \R$ calculada como
\begin{equation*}
I_N(\omega) = \frac{2}{N} \abso{ \sum_{t=0}^{N} e^{i \omega t} x_t }^{2}
\end{equation*}
\label{periodograma}
\end{defn}

\begin{thrm}
Sean $x_1, x_2 , \dots, x_N$ observaciones de un proceso estoc\'astico de media cero y varianza
finita. Se puede calcular el periodograma para estos datos como
\begin{equation*}
I_N(\omega) = 2 \sum_{r = -(N-1)}^{N-1} \aste{R}(r) \COS{r \omega}
\end{equation*}
Donde $\aste{R}$ es el estimador para la funci\'on de autocovarianza del proceso, calculado como
$\widehat{R}(\tau) = \frac{1}{N-\abso{\tau}} \sum_{t = 1}^{N-\abso{\tau}} x_t x_{t+\abso{\tau}}$
\label{periodograma_rho}
\end{thrm}

Se puede demostrar que el periodograma es un estimador insesgado de la FDE para los proceso 
considerados; sin embargo, si el proceso tuviera un espectro puramente continuo, ocurre que 
$\lim_{N\rightarrow \infty} \Var{I_N(\omega)} = h^{2}(\omega)$, con $h$ la FDE del proceso: el 
periodograma, en general, no es consistente.
En parte esto ocurre porque el periodograma depende de los estimadores para la funci\'on de 
autocovarianza, $\est{R}$, evaluada en todos los puntos posibles: para calcular $\est{R}$ en 
valores muy altos se requieren puntos muy alejados, los cuales son menos abundantes e implican 
una mayor varianza.

Si efectivamente el periodograma aumenta su varianza cuando incluye las 'colas' de la funci\'on de 
autocovarianza, entonces una soluci\'on es evitarlas, multiplicando por una funci\'on de pesos. 
Tales consideraciones dan origen a estimadores de la forma
\begin{equation*}
\est{h}(\omega) = \frac{1}{2\pi} \sum_{s = -(N-1)}^{N-1} 
\lambda(s) \aste{R}(s) e^{i \omega t}
\label{ventaneando}
\end{equation*}
donde la funci\'on de pesos, $\lambda$, es referida como \textbf{ventana de retrasos}. Para 
estudiar las propiedades estos estimadores, conviene reescribirlos en funci\'on del periodograma

\begin{equation*}
\est{h}(\omega) = \intPI I_N(\theta) W(\omega-\theta) d\theta
\end{equation*}
donde $W$ es la transformada de Fourier finita de $\lambda$
\begin{equation*}
W(\theta) = \frac{1}{2\pi} \sum_{s = -(N-1)}^{N-1} \lambda(s) e^{-is\theta}
\end{equation*}

Cabe destacar la forma que adopta $\est{h}$ como la convoluci\'on $I_N \ast W$, que bien puede 
entenderse como que $W$ es una funci\'on de pesos en el 'dominio de las frecuencias'; por ello, $W$ 
es referida como \textbf{ventana de retrasos}.
En la tabla \ref{ventanas} hay una lista corta de algunas funciones tipo ventana. Estos estimadores 
son consistentes y sesgados, aunque son asint\'oticamente insesgados.

\begin{SidewaysTable}
\centering
\bordes{1.5}
\begin{tabular}{c}
\textbf{Algunas funciones tipo ventana}
\vspace{1em}
\end{tabular}
\begin{tabular}{lll}
\toprule
 & \textbf{Ventana de retrasos} & \textbf{Ventana en las frecuencias} \\
\midrule
\textbf{P. truncado} & 
$\displaystyle
\lambda(s) = \begin{cases}
1 &\text{, si } \abso{s} \leq M \\
0 &\text{, otro caso}
\end{cases}$ &
$\displaystyle
W(\theta) = \frac{1}{2\pi} \frac{\SEN{(M+\frac{1}{2})\theta}}{\SEN{\nicefrac{\theta}{2}}}
=: D_M(\theta)$
\\
\rowcolor{gris}
\textbf{Bartlet} &
$\displaystyle
\lambda(s) = \begin{cases}
1-\nicefrac{\abso{s}}{M} &\text{, si } \abso{s} \leq M \\
0 &\text{, otro caso}
\end{cases}$ &
$\displaystyle
W(\theta) = \frac{1}{2\pi M} 
\left( \frac{\SEN{ \nicefrac{M\theta}{2}}}{\SEN{\nicefrac{\theta}{2}}} \right)^{2}
=: F_M(\theta)$
\\
\textbf{Daniell} &
$\displaystyle
\lambda(s) = \frac{\SEN{\nicefrac{\pi s}{M}}}{\nicefrac{\pi s}{M}}$ &
$\displaystyle
W(\theta) = \begin{cases}
\nicefrac{M}{2\pi} &\text{, si } \abso{\theta} \leq \nicefrac{\pi}{M} \\
0 &\text{, otro caso}
\end{cases}$
\\
\rowcolor{gris}
\textbf{Tukey-Hanning} &
$\displaystyle
\lambda(s) = \begin{cases}
\nicefrac{1}{2}\left( 1+ \COS{\nicefrac{\pi s}{M}} \right) &\text{, si } \abso{s} \leq M \\
0 &\text{, otro caso}
\end{cases}$ &
$\displaystyle
W(\theta) = \frac{1}{4} D_M\left(\theta - \frac{\pi}{M} \right) 
+\frac{1}{2} D_M\left(\theta \right)
\frac{1}{4} D_M\left(\theta + \frac{\pi}{M} \right)$
\\
\textbf{Parzen} &
$\displaystyle
\lambda(s) = \begin{cases}
1-6\left( \nicefrac{s}{M} \right)^{2} + 6\left( \nicefrac{\abso{s}}{M} \right)^{3} 
&\text{, si } \abso{s} \leq \nicefrac{M}{2} \\
2\left( 1 - \nicefrac{\abso{s}}{M} \right)^{3}
 &\text{, si } \nicefrac{M}{2} \leq \abso{s} \leq M \\
0 &\text{, otro caso}
\end{cases}$ &
$\displaystyle
W(\theta) = \frac{3}{8 \pi M^{3}} \left( 
\frac{\SEN{\nicefrac{M\theta}{4}}}{\nicefrac{1}{2}\SEN{\nicefrac{\theta}{2}}} 
\right)^{4}
\left( 1- \nicefrac{2}{3} \SEN{\nicefrac{\theta}{2}}^{2} \right)$
\\
\rowcolor{gris}
\textbf{Bartlet-Priestley} &
$\displaystyle
\lambda(s) = \frac{3M^{2}}{\left(\pi s\right)^{2} }
\left( \frac{\SEN{\nicefrac{\pi s}{M}}}{\nicefrac{\pi s}{M}} - \COS{\nicefrac{\pi s}{M}}
\right)
$ 
&
$\displaystyle
W(\theta) = \begin{cases}
\frac{3M}{4\pi} \left( 1- \left(\frac{M \theta}{\pi} \right)^{2} \right)
&\text{, si } \abso{\theta} \leq \nicefrac{\pi}{M} \\
0 &\text{, otro caso}
\end{cases}$
\\
\bottomrule
\end{tabular}
\caption{Ejemplos de algunas ventanas que suavizan el periodograma, formando estimadores 
consistente de la FDE para el caso de espectro puramente continuo.
Las funciones $F_M$ y $D_M$ toman, respectivamente, los nombres de \textit{n\'ucleo de Fejer} y
\textit{N\'ucleo de Dirichlet} de orden $M$}
\label{ventanas}
\end{SidewaysTable}

%%%%%%%%%%%%%%%%%%%%%%%%%%%%%%%%%%%%%%%%%%%%%%%%%%%%%%%%%%%%%%%%%%%%%%%%%%%%%%%%%%%%%%%%%%%%%%%%%%%
%%%%%%%%%%%%%%%%%%%%%%%%%%%%%%%%%%%%%%%%%%%%%%%%%%%%%%%%%%%%%%%%%%%%%%%%%%%%%%%%%%%%%%%%%%%%%%%%%%%

\subsubsection{Filtros lineales}

Como herramienta auxiliar en la estimaci\'on de la FDE, conviene considerar procesos del tipo 
$Y(t) = \intR g(u) X(t-u) du$, referidos como 'filtrados'.
Hist\'oricamente los filtros se originan como circuitos f\'isicos que, por dise\~no, son de ayuda 
en la 'cancelaci\'on' anal\'ogica de componentes de frecuencia (ver m\'as adelante).
%
Los filtros, aplicados a proceso d\'ebilmente estacionarios, afectan su FDE de manera 'simple' 
(teorema \ref{filtro}); por otro lado, permiten interpretar a procesos autorregresivos (AR) y de 
medias m\'oviles (MA) como versiones filtradas de procesos ruido blanco; m\'as a\'un, si los 
procesos filtrados pueden tener la forma\footnote{La integral est\'a definida en el sentido de 
Stieltjes} $Y(t) = \intR X(t-u) dG(u)$, entonces es posible manejar el operador retraso 
($B[Y](t) = Y(t)-Y(t-1)$).
%

\begin{thrm}
Sea $\{ X(t) \}$ un proceso d\'ebilmente estacionario que admite una representaci\'on de 
Wold-Cram\'er. Sea $g$ una funci\'on real con $\intR \abso{g(u)} du < \infty$, y sea $\{ Y(t) \}$ 
un proceso definido de la siguiente forma:
\begin{equation*}
Y(t) = \intR g(u) X(t-u) du
\end{equation*}
Entonces, se cumple la siguiente relaci\'on
\begin{equation*}
h_X(\omega) = h_Y(\omega) \abso{\Gamma(\omega)}^{2}
\end{equation*}
donde $h_X$ y $h_Y$ son las respectivas FDE de $\{X(t)\}$ y $\{Y(t)\}$, y 
$\Gamma(\omega) = \intR g(u) e^{i \omega u} du$.
\label{filtro}
\end{thrm}

En el teorema \ref{filtro}, la funci\'on $g$ es referida como \textbf{funci\'on de respuesta}; tal 
nombre tiene sentido si $\{X(t)\}$ no fuera un proceso, sino un 'impulso unitario' (una funci\'on 
tipo \dirac \footnote{La funci\'on $\delta_x:\R\rightarrow \R$ es una funci\'on $\delta$ de Dirac 
si puede verse como la funci\'on de distribuci\'on de masa para una medida finita que es cero para 
todo conjunto que no contenga a $x$}).
La funci\'on $\Gamma$ es referida como \textbf{funci\'on de transferencia}, nombre motivado de 
manera similar al considerar a $\{X(t)\}$ una 'se\~nal' senoidal ($X(t) = e^{i \omega t}$); la 
conexi\'on se vuelve m\'as clara si se interpreta esta segunda se\~nal como funci\'on tipo \dirac 
en el dominio de las frecuencias.
Respecto a la 'cancelaci\'on' de frecuencias basta decir que, en teor\'ia, se puede construir a $g$ 
de modo que $\Gamma$ sea 0 fuera de un intervalo cerrado y 1 dentro del mismo; este efecto es 
referido como \textbf{filtro de banda}.

%%%%%%%%%%%%%%%%%%%%%%%%%%%%%%%%%%%%%%%%%%%%%%%%%%%%%%%%%%%%%%%%%%%%%%%%%%%%%%%%%%%%%%%%%%%%%%%%%%%
%%%%%%%%%%%%%%%%%%%%%%%%%%%%%%%%%%%%%%%%%%%%%%%%%%%%%%%%%%%%%%%%%%%%%%%%%%%%%%%%%%%%%%%%%%%%%%%%%%%

\subsubsection{Representaci\'on de Wold-Cram\'er}

Como consecuencia de los teoremas \ref{t_wienerkhinchin} y \ref{t_wold}, los procesos d\'ebilmente 
estacionarios pueden ser caracterizados usando directamente la FDE (sin usar la funci\'on de
autocorrelaci\'on).
Esta representaci\'on existe en virtud del teorema \ref{rep_espectral}, cuya demostraci\'on no 
ser\'a incluida en este trabajo; el lector interesado en tan imponente teorema puede referirse a 
\cite{Priestley81}.

\newpage

\begin{thrm}
Sea $\{X(t)\}$ un proceso estoc\'astico a tiempo continuo d\'ebilmente estacionario de media 0 y 
estoc\'asticamente continuo en el sentido de media cuadr\'atica. Entonces, existe un proceso 
ortogonal $\{Z(\omega)\}$ tal que, para todo tiempo $\omega$ admisible, se puede 
escribir\footnote{La integral se encuentra definida en el sentido de media cuadr\'atica.}
\begin{equation*}
X(t) = \intR e^{i t \omega} dZ(\omega)
\end{equation*}
Donde el proceso $\{Z(t)\}$ tiene las siguientes propiedades para todo $\omega$
\begin{itemize}
\item $\E{dZ(\omega)} = 0$
\item $\E{\abso{dZ(\omega)}^{2}} = dH(\omega)$
\item $\Cov{dZ(\omega),dZ(\lambda)} = 0 \Leftrightarrow \omega \neq \lambda$
\end{itemize}
Donde $dH(\omega)$ la FDE integrada de $\{X(t)\}$
\label{rep_espectral}
\end{thrm}

En virtud del teorema de Wold, se puede tener una variante del teorema \ref{rep_espectral}
para procesos a tiempo discreto, raz\'on por la cual  
tal representaci\'on es referida como \textbf{representaci\'on de Wold-Cram\'er}.

%%%%%%%%%%%%%%%%%%%%%%%%%%%%%%%%%%%%%%%%%%%%%%%%%%%%%%%%%%%%%%%%%%%%%%%%%%%%%%%%%%%%%%%%%%%%%%%%%%%
%%%%%%%%%%%%%%%%%%%%%%%%%%%%%%%%%%%%%%%%%%%%%%%%%%%%%%%%%%%%%%%%%%%%%%%%%%%%%%%%%%%%%%%%%%%%%%%%%%%

\subsubsection{Prueba de Priestley-Subba Rao}

Esta t\'ecnica fue presentada por Priestley y Subba Rao en 1969 \cite{Priestley69}, y consiste en 
estimar el espectro local del proceso len varios puntos en el tiempo y las frecuencias, para luego 
probar la hip\'otesis de que el espectro cambia en el tiempo; la explicaci\'on presente es m\'as 
bien somera, los detalles formales que se omiten en el presente texto (la existencia del proceso 
evolutivo para cierto tipo de procesos, las propiedades de los estimadores usados) pueden 
encontrarse son tomadas de \cite{Priestley65} y \cite{Priestley66}.

Se pre-supone que los datos pueden entenderse como una cantidad finita de 
observaciones provenientes de un proceso estoc\'astico a tiempo continuo que, para todos los 
tiempos, tiene media cero y varianza finita, adem\'as de ser estoc\'asticamente continuo y tener 
un espectro puramente continuo.
Si bien son numerosas las condiciones, no se ha supuesto que
el proceso sea estacionario; adem\'as, se ha argumentado porqu\'e los que conviene ver a los 
registros de PSG  como observaciones de procesos, estoc\'asticamente continuos, con valor 
esperado y varianzas finitas. El tener media cero y espectro puramente continuo ser\'an 'forzadas' 
num\'ericamente, estimando del proceso su media y componente peri\'odica mediante regresi\'on
lineal adaptativa; en particular, se utiliza el algoritmo \textit{Seasonal-Trend decomposition
using Loess} (STL), propuesto por Cleveland y colaboradores \cite{Cleveland1990}.

Por brevedad, se mostrar\'a s\'olo someramente una condici\'on extra para los procesos a considerar:
tener una representaci\'on de la forma
\begin{equation*}
X(t) = \intPI A(t,\omega) e^{i t \omega} dZ(\omega)
\end{equation*}
donde $\{ dZ(\omega) \}$ es un proceso ortogonal 
($\Cov{dZ(\omega),dZ(\lambda)} = 0 \Leftrightarrow \omega \neq \lambda$).
Este objeto, que bien puede entenderse como una representaci\'on de Wold-Cram\'er que puede cambiar
en el tiempo, fue introducida por Priestley \cite{Priestley65} como el
\textbf{espectro evolutivo}; es amplia e interesante
la discusi\'on sobre qu\'e procesos admiten un espectro evolutivo, pero se omite en
el presente trabajo por motivos de tiempo.

Respecto a la estimaci\'on del espectro local se usa el \textbf{estimador de doble ventana}, 
t\'ecnica introducida por Priestley \cite{Priestley69} y que requiere dos funciones, $w_\tau$ y 
$g$, que funcionan como ventana de retrasos y como filtro lineal, respectivamente.
%
En cuando a $g$, se define a $\Gamma(u) = \intR g(u) e^{i u \omega} du$ y se les pide que
\begin{equation*}
2\pi \int_{-\infty}^{\infty} \lvert g(u) \lvert^{2} du 
= 
\int_{-\infty}^{\infty} \lvert \Gamma(\omega) \lvert^{2} d\omega
= 1
\end{equation*}

Cabe mencionar que las ventanas espectrales mostradas en la tabla \ref{ventanas} bien 
pueden cumplir las propiedades requeridas para ser filtros.
Posteriormente se define el estimador $U$ con el objetivo de asignar pesos en el tiempo para estimar
a la FDE
% en el tiempo dado; m\'as a\'un, $U$ sirve 
%como una aproximaci\'on de la representaci\'on de Wold-Cram\'er para 
%el proceso.
\begin{equation*}
U(t,\omega) = \int_{t-T}^{t} g(u) X({t-u}) e^{i \omega (t-u)} du
\end{equation*}

Bajo el entendido que la funci\'on $\Gamma$ converge a una funci\'on tipo \dirac, puede 
considerarse que 
$\E{\abso{U(t,\omega)}^{2}} \approx f_t(\omega)$; sin embargo, se demuestra en \cite{Priestley66} 
que $\Var{\abso{U(t,\omega)}^{2}} \nrightarrow 0$.
%
Debido a ello se usa una segunda funci\'on tipo ventana,
%, para 'suavizar' el estimador y hacerlo consistente (
de forma similar al periodograma.
Se considera la funci\'on $W_\tau$, ventana de retrasos, y su respectiva ventana espectral 
$w_\tau$; deben satisfacer las siguientes propiedades:
\begin{itemize}
\item $w_{\tau}(t) \geq 0$ para cualesquiera $t$, $\tau$
\item $w_{\tau}(t) \rightarrow 0$ cuando $\lvert t \lvert \rightarrow \infty$, para todo $\tau$
\item $\displaystyle \int_{-\infty}^{\infty} w_{\tau}(t) dt = 1$ para todo $\tau$
\item $\displaystyle \int_{-\infty}^{\infty} \left( w_{\tau}(t) \right)^{2} dt < \infty$ para todo $\tau$
\item $\exists C$ tal que  
$\displaystyle \lim_{\tau\rightarrow\infty} \tau \int_{-\infty}^{t} \abso{ W_{\tau}(\lambda) }^{2} d\lambda = C$
\end{itemize}

%Por ejemplo, la ventana de Daniell satisface estas propiedades; para ello, conviene calcular que
%$\lim_{\tau\rightarrow\infty} \tau \int_{t-T}^{t} \lvert W_{\tau}(\lambda) \lvert^{2} d\lambda = 2\pi$;
%m\'as a\'un, 
Cabe mencionar que todas las ventanas mostradas en \ref{ventanas} satisfacen las propiedades 
anteriores.
Finalmente, se define el estimador $\est{f}$ para las FDE normalizada, $f_t$, como
\begin{equation*}
\widehat{f}(t,\omega) = \int_{t-T}^{t} w_{T'}(u) \lvert U(t-u,\omega) \lvert^{2} du
\label{estimador_doble_ventana}
\end{equation*}

Fue demostrado por Priestley \cite{Priestley65} que los estimadores de doble ventana son 
asint\'oticamente insesgados y consistentes, y propone las siguientes aproximaciones:
%conviene exhibir las siguientes expresiones aproximadas propuestas en aqu\'el trabajo
\begin{itemize}
\item $\displaystyle
\E{\est{f}(t,\omega)} \approx 
\intR \widetilde{f}(t,\omega+\theta) \abso{\Gamma(\theta)}^{2} d\theta$
\item $\displaystyle
\Var{\est{f}(t,\omega)} \approx \frac{C}{\tau} \left( \overline{f}^{2}(\omega) \right)
\intR \abso{\Gamma(\theta)}^{4} d\theta $
\end{itemize}

donde las funciones $\widetilde{f}$ y $\overline{f}$ son versiones 'suavizadas' de la FDE 
normalizada, $f$, y est\'an definidas de la siguiente manera
\begin{equation*}
\widetilde{f}(t,\omega+\theta) = 
\intR W_{\tau}(u) f(t-u,\omega+\theta) du
\end{equation*}
\begin{equation*}
\overline{f}^{2} (t,\omega) =
\frac{\intR f^{2}\left(t-u,W_{\tau}^{2}(u)\right) du}
{\intR \left( W_{\tau}(u) \right)^{2} du}
\end{equation*}

Como $W_{\tau}$ funciona como ventana espectral, converge a una 
funci\'on tipo \dirac; luego $\widetilde{f}$ es aproximadamente la convoluci\'on 
$\widetilde{f}(t,\omega+\theta) \approx \delta_t \ast f(\bullet,\omega+\theta)$. 
Una aproximaci\'on muy similar 
puede hacerse respecto al segundo t\'ermino, de modo que $\widetilde{f}\approx f$ y 
$\overline{f}^{2}\approx f^{2}$.
Tales aproximaciones ser\'an mejores en tanto las ventanas $w_{\tau}$ y $W_{\tau}$ sean m\'as 
cercanas a funciones tipo \dirac.
%; m\'as a\'un, una condici\'on adecuada es que estas funciones 
%tengan una forma 'm\'as delgada' que el espacio entre los tiempos y frecuencias donde se estimar\'a 
%$f$.
Dicho esto, se pueden hacer las siguientes aproximaciones, un poco m\'as arriesgadas:
\begin{itemize}
\item $\displaystyle \E{\est{f}(t,\omega)} \approx f(t,\omega)$
\item $\displaystyle \Var{\est{f}(t,\omega)} \approx 
\frac{C}{\tau} f^{2}(t,\omega) \intR \abso{\Gamma (\theta)}^{4} d\theta$
\end{itemize}

%Por otro lado, es importante mostrar expresiones para la covarianza, si bien s\'olo se
%se reescriben 
%aqu\'i unas simplificaciones hechas en el caso que el proceso, adem\'as de cumplir las hip\'otesis 
%de semi-estacionariedad, sea 'normal'
A continuaci\'on se exhiben expresiones para la covarianza del estimador, en el caso particular
de un proceso aproximadamente normal
\begin{equation*}
\Cov{\est{f}(t_1,\omega_1) , \est{f}(t_2,\omega_2)} \approx \intR \intR
w_\tau (u) w_\tau(v) \Cov{ \abso{U(t_1-u,\omega_1)}^{2} , \abso{U(t_2-u,\omega_2)}^{2} }
du dv
\end{equation*}
la covarianza ser\'a negligible en tanto $w_\tau$ se comporte como una 
funci\'on \dirac, con un pico m\'as delgado que $\abso{\omega_1-\omega_2}$.
El mismo efecto se logra si $\abso{U(t_1-u,\omega_1)}^{2}$ y $\abso{U(t_2-u,\omega_2)}^{2}$ son 
no-correlacionados, pero por brevedad s\'olo se citar\'a de \cite{Priestley65} que basta que 
$\Gamma$ sea una funci\'on \dirac, cuyo ancho sea menor a $\abso{t_1-t_2}$.

Un dato sumamente importante para la estimaci\'on de la FDE es c\'omo la varianza del estimador 
$\widehat{f}$ depende 'multiplicativamente' de la verdadera FDE.
Una interpretaci\'on sobre este hecho, difundida dentro de las ingenier\'ias, es sobre la 
'modulaci\'on' de ondas, que puede verse como una 'multiplicaci\'on de ondas', y que motiva el uso 
de la 'transformaci\'on logar\'itmica'.
Formalmente, se introduce el siguiente estimador
\begin{equation*}
Y(t,\omega) = \log{\left( \est{f}(t,\omega)\right)}
\end{equation*}

Se puede demostrar que $Y$ tiene las siguientes propiedades:
\begin{itemize}
\item $\displaystyle 
\E{ Y(t,\omega) } \approx \log \left( f(t,\omega) \right)$
\item $\displaystyle 
\Var{ Y(T,\omega) } 
\approx \frac{C}{\tau} \intR \abso{\Gamma (\theta)}^{4} d\theta $
\end{itemize}
m\'as a\'un, el estimador $Y$ puede ser representado de la siguiente forma
\begin{equation*}
Y(t,\omega) = \log \left( f(t,\omega) \right) + \varepsilon(t,\omega)
\end{equation*}
donde las variables $\varepsilon(t,\omega)$ satisfacen que
\begin{itemize}
\item $\displaystyle \E{\varepsilon(t,\omega)} = 0$
\item $\displaystyle \Var{\varepsilon(t,\omega)}
\approx \frac{C}{\tau} \intR \abso{\Gamma (\theta)}^{4} d\theta$
\end{itemize}

Priestley \cite{Priestley81} destaca que la transformaci\'on logar\'itmica tiene la propiedad de 
hacer que el estimador $Y$ tenga una distribuci\'on 'm\'as cercana' a la normal, y que en la 
pr\'actica bien puede considerarse que las variables $\varepsilon$'s tienen distribuci\'on normal 
con media 0 y varianza $\sigma^{2}$.
Es destacable que las variables $\varepsilon$'s comparten la misma media y varianza, adem\'as de 
que son aproximadamente no-correlacionadas.

La prueba de Priestley-Subba Rao, como se mencion\'o, funciona calculando el estad\'istico $Y$ 
sobre varios puntos en el tiempo y la frecuencia, y luego revisando si se puede afirmar que el 
vector  $\left( Y(t,\omega_1), Y(t,\omega_2), \dots, Y(t,\omega_N) \right)$ es constante en el 
tiempo; de forma concreta  se maneja la siguiente aproximaci\'on
\begin{equation*}
\sum_{i = 1 }^{N} \left( Y(t,\omega_i) - \overline{Y}(\bullet,\omega_i) \right)^{2} 
\sim \sigma^{2} \chi^{2}(N)
\end{equation*}
donde $\sigma^{2} = \Var{\varepsilon(t,\omega)}$, y
$\overline{Y}(\bullet,\omega) = \frac{1}{M} \sum_{j=1}^{M} Y(t_j,\omega)$.
Con tal caracterizaci\'on se puede usar una prueba ANOVA de manera relativamente f\'acil.

%%%%%%%%%%%%%%%%%%%%%%%%%%%%%%%%%%%%%%%%%%%%%%%%%%%%%%%%%%%%%%%%%%%%%%%%%%%%%%%%%%%%%%%%%%%%%%%%%%%
%%%%%%%%%%%%%%%%%%%%%%%%%%%%%%%%%%%%%%%%%%%%%%%%%%%%%%%%%%%%%%%%%%%%%%%%%%%%%%%%%%%%%%%%%%%%%%%%%%%
%%%%%%%%%%%%%%%%%%%%%%%%%%%%%%%%%%%%%%%%%%%%%%%%%%%%%%%%%%%%%%%%%%%%%%%%%%%%%%%%%%%%%%%%%%%%%%%%%%%
%%%%%%%%%%%%%%%%%%%%%%%%%%%%%%%%%%%%%%%%%%%%%%%%%%%%%%%%%%%%%%%%%%%%%%%%%%%%%%%%%%%%%%%%%%%%%%%%%%%
