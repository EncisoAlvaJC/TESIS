%%%%%%%%%%%%%%%%%%%%%%%%%%%%%%%%%%%%%%%%%%%%%%%%%%%%%%%%%%%%%%%%%%%%%%%%%%%%%%%%%%%%%%%%%%%%%%%%%%%
%%%%%%%%%%%%%%%%%%%%%%%%%%%%%%%%%%%%%%%%%%%%%%%%%%%%%%%%%%%%%%%%%%%%%%%%%%%%%%%%%%%%%%%%%%%%%%%%%%%
\documentclass{beamer}
\usepackage[utf8]{inputenc}
\usepackage[spanish]{babel}
\usepackage{amsmath}
\usepackage{amsfonts}
\usepackage{amssymb}
\usepackage{graphicx}

\usepackage{bibentry}

\usepackage{movie15}

\usepackage{verbatim}
%\usepackage{authblk}

\usepackage{listings}
\usepackage{color}

%%%%%%%%%%%%%%%%%%%%%%%%%%%%%%%%%%%%%%%%%%%%%%%%%%%%%%%%%%%%%%%%%%%%%%%%%%%%%%%%%%%%%%%%%%%%%%%%%%%
%%%%%%%%%%%%%%%%%%%%%%%%%%%%%%%%%%%%%%%%%%%%%%%%%%%%%%%%%%%%%%%%%%%%%%%%%%%%%%%%%%%%%%%%%%%%%%%%%%%

%\usetheme{Luebeck}
\usetheme{CambridgeUS}
\usecolortheme{beaver}
{
%\usetheme{Luebeck}
%\setbeamercolor{palette secondary}{use=structure,fg=white,bg=red}
%\setbeamercolor{palette tertiary}{use=structure,fg=white,bg=green}
}

%
%
%
\definecolor{rojo}{rgb}{0.62353,0.10588,0.13725}
\definecolor{naranja}{rgb}{0.91765,0.23137,0.16078}
\definecolor{rosa}{rgb}{0.95294,0.53333,0.32157}
\definecolor{cafe}{rgb}{0.48627,0.14118,0.10196}
\definecolor{cafe2}{rgb}{0.243135,0.07059,0.05098}
\definecolor{cafe3}{rgb}{0.1215675,0.035295,0.02549}


\definecolor{UniBlue}{RGB}{83,121,170}
\setbeamercolor{title}{fg=white,bg=rojo}
\setbeamercolor{frametitle}{fg=white,bg=rojo}
\setbeamercolor{structure}{fg=cafe,bg=rosa}

\setbeamercolor{alerted text}{fg=rosa,bg=cafe2}

\setbeamercolor{palette primary}{bg=rojo,fg=white}
\setbeamercolor{palette tertiary}{bg=cafe2,fg=white}
\setbeamercolor{palette quaternary}{use=structure,bg=cafe3,fg=white}
\setbeamercolor{itemize item}{bg=naranja,fg=white}
\setbeamercolor{section number projected }{bg=rosa,fg=white}

\setbeamertemplate{itemize item}{\color{naranja}$\bullet$}

%%%%%%%%%%%%%%%%%%%%%%%%%%%%%%%%%%%%%%%%%%%%%%%%%%%%%%%%%%%%%%%%%%%%%%%%%%%%%%%%%%%%%%%%%%%%%%%%%%%
%%%%%%%%%%%%%%%%%%%%%%%%%%%%%%%%%%%%%%%%%%%%%%%%%%%%%%%%%%%%%%%%%%%%%%%%%%%%%%%%%%%%%%%%%%%%%%%%%%%

%\DeclareUnicodeCharacter{00A0}{ }

\lstset{ %
%  language=R,                     % the language of the code
  basicstyle=\footnotesize,       % the size of the fonts that are used for the code
}


\theoremstyle{definition}
\newtheorem{defn}{Definici\'on}[section]

%%%%%%%%%%%%%%%%%%%%%%%%%%%%%%%%%%%%%%%%%%%%%%%%%%%%%%%%%%%%%%%%%%%%%%%%%%%%%%%%%%%%%%%%%%%%%%%%%%%
%%%%%%%%%%%%%%%%%%%%%%%%%%%%%%%%%%%%%%%%%%%%%%%%%%%%%%%%%%%%%%%%%%%%%%%%%%%%%%%%%%%%%%%%%%%%%%%%%%%

\title[Estacionariedad en PSG de adultos mayores]
{Estacionariedad d\'ebil en registros de polisomnogr\'aficaos de adultos mayores,
como posible marcador de deterioro cognitivo}

\author[Enciso Alva]
{Julio Cesar Enciso Alva}

\institute[LIMA]
{Licenciatura en Matem\'aticas Aplicadas}

\date[Mayo 2017]
{Seminario de investigaci\'on\\ Mayo de 2017}

\titlegraphic{
\hspace*{8cm}~
\includegraphics[width=9em]{./material1/SIGLAS_UAEH_02.png}
}

%%%%%%%%%%%%%%%%%%%%%%%%%%%%%%%%%%%%%%%%%%%%%%%%%%%%%%%%%%%%%%%%%%%%%%%%%%%%%%%%%%%%%%%%%%%%%%%%%%%
%%%%%%%%%%%%%%%%%%%%%%%%%%%%%%%%%%%%%%%%%%%%%%%%%%%%%%%%%%%%%%%%%%%%%%%%%%%%%%%%%%%%%%%%%%%%%%%%%%%
%%%%%%%%%%%%%%%%%%%%%%%%%%%%%%%%%%%%%%%%%%%%%%%%%%%%%%%%%%%%%%%%%%%%%%%%%%%%%%%%%%%%%%%%%%%%%%%%%%%
%%%%%%%%%%%%%%%%%%%%%%%%%%%%%%%%%%%%%%%%%%%%%%%%%%%%%%%%%%%%%%%%%%%%%%%%%%%%%%%%%%%%%%%%%%%%%%%%%%%

\begin{document}

\frame{\titlepage}

\begin{frame}
\tableofcontents
\end{frame}

%%%%%%%%%%%%%%%%%%%%%%%%%%%%%%%%%%%%%%%%%%%%%%%%%%%%%%%%%%%%%%%%%%%%%%%%%%%%%%%%%%%%%%%%%%%%%%%%%%%
%%%%%%%%%%%%%%%%%%%%%%%%%%%%%%%%%%%%%%%%%%%%%%%%%%%%%%%%%%%%%%%%%%%%%%%%%%%%%%%%%%%%%%%%%%%%%%%%%%%

\section{Introducci\'on}

%%%%%%%%%%%%%%%%%%%%%%%%%%%%%%%%%%%%%%%%%%%%%%%%%%%%%%%%%%%%%%%%%%%%%%%%%%%%%%%%%%%%%%%%%%%%%%%%%%%

\subsection{Antecedentes}

\begin{frame}\frametitle{Antecedentes}
\begin{itemize}
\item Encuesta Nacional de Salud y Nutrici\'on (ENSANUT, M\'exico 2002): 800,000 adultos 
mayores\cite{PlanAlzheimer04}

\item Estudios estad\'isticos \cite{Amer13,Miyata13,Potvin12} sugieren una relaci\'on entre 
trastornos del sue\~no y DC durante la vejez

\item Se estudia la epidemiolog\'ia del DC en Hidalgo \cite{VazquezTagle16}: se registra PSG, se 
encuentran diferencias en el sue\~no MOR

\item Estudio de fractalidad en registros de PSG [Valeria]: diferencias significativas en sujetos 
con y sin DC 

\item Se buscan marcadores cl\'inicos para el diagn\'ostico de DC
\end{itemize}
\end{frame}

%%%%%%%%%%%%%%%%%%%%%%%%%%%%%%%%%%%%%%%%%%%%%%%%%%%%%%%%%%%%%%%%%%%%%%%%%%%%%%%%%%%%%%%%%%%%%%%%%%%

\subsection{Objetivos}

\begin{frame}\frametitle{Pregunta de investigaci\'on}
\textbf{
¿Es posible que la caracterizaci\'on de registros de PSG como series de tiempo d\'ebilmente 
estacionarias, pueda ser usada como un marcador en el diagn\'ostico cl\'inico de PDC en adultos 
mayores?
}

\end{frame}

\begin{frame}\frametitle{Objetivos}
{\small
\begin{description}
%\textbf{General.} 
\item[General:]
Detectar, a partir de pruebas formales, las presencia de estacionariedad d\'ebil 
en registros de PSG para adultos mayores con y sin PDC

%\textbf{Espec\'ificos} 
\item[Espec\'ificos:]
\begin{itemize}
\item Estudiar la definici\'on de estacionariedad y sus consecuencias en un modelo

\item Investigar c\'omo detectar si una serie de tiempo dada proviene de un proceso
d\'ebilmente estacionario

\item Usando los an\'alisis hallados, determinar si los datos considerados provienen de 
procesos débilmente estacionarios.
Revisar si esta informaci\'on muestra diferencias entre sujetos con y sin PDC
\end{itemize}
\end{description}
}
\end{frame}

%%%%%%%%%%%%%%%%%%%%%%%%%%%%%%%%%%%%%%%%%%%%%%%%%%%%%%%%%%%%%%%%%%%%%%%%%%%%%%%%%%%%%%%%%%%%%%%%%%%

\subsection{Conceptos}

%%%%%%%%%%%%%%%%%%%%%%%%%%%%%%%%%%%%%%%%%%%%%%%%%%%%%%%%%%%%%%%%%%%%%%%%%%%%%%%%%%%%%%%%%%%%%%%%%%%

\subsubsection{Fisiolog\'ia}

\begin{frame}
\begin{description}
\item[Adulto Mayor.] Individuo de 60 a\~nos o m\'as que habite un pa\'is en v\'ias de desarrollo, o 
65 a\~nos en pa\'ises desarrollados \cite{Hita14}.
\end{description}
\end{frame}

%%%%%%%%%%%%%%%%%%%%%%%%%%%%%%%%%%%%%%%%%%%%%%%%%%%%%%%%%%%%%%%%%%%%%%%%%%%%%%%%%%%%%%%%%%%%%%%%%%%

\subsubsection{Matem\'aticas}

%%%%%%%%%%%%%%%%%%%%%%%%%%%%%%%%%%%%%%%%%%%%%%%%%%%%%%%%%%%%%%%%%%%%%%%%%%%%%%%%%%%%%%%%%%%%%%%%%%%

\section{Metodolog\'ia}

%%%%%%%%%%%%%%%%%%%%%%%%%%%%%%%%%%%%%%%%%%%%%%%%%%%%%%%%%%%%%%%%%%%%%%%%%%%%%%%%%%%%%%%%%%%%%%%%%%%

\section{Resultados}

%%%%%%%%%%%%%%%%%%%%%%%%%%%%%%%%%%%%%%%%%%%%%%%%%%%%%%%%%%%%%%%%%%%%%%%%%%%%%%%%%%%%%%%%%%%%%%%%%%%

\subsection{Discusi\'on}

%%%%%%%%%%%%%%%%%%%%%%%%%%%%%%%%%%%%%%%%%%%%%%%%%%%%%%%%%%%%%%%%%%%%%%%%%%%%%%%%%%%%%%%%%%%%%%%%%%%

\subsection{Conclusiones}

%%%%%%%%%%%%%%%%%%%%%%%%%%%%%%%%%%%%%%%%%%%%%%%%%%%%%%%%%%%%%%%%%%%%%%%%%%%%%%%%%%%%%%%%%%%%%%%%%%%

\subsection{Trabajo a futuro}

%%%%%%%%%%%%%%%%%%%%%%%%%%%%%%%%%%%%%%%%%%%%%%%%%%%%%%%%%%%%%%%%%%%%%%%%%%%%%%%%%%%%%%%%%%%%%%%%%%%

\begin{frame}[allowframebreaks]
\frametitle{Bibliograf\'ia}
%\nocite{*}
\footnotesize{
%\bibliography{referencias}
\bibliography{referencias_estacionariedad,referencias_fisiologia,referencias_otros,referencias_mixto}{}
%\bibliographystyle{apalike-es}
\bibliographystyle{abbrv}
}
\end{frame}

%%%%%%%%%%%%%%%%%%%%%%%%%%%%%%%%%%%%%%%%%%%%%%%%%%%%%%%%%%%%%%%%%%%%%%%%%%%%%%%%%%%%%%%%%%%%%%%%%%%
%%%%%%%%%%%%%%%%%%%%%%%%%%%%%%%%%%%%%%%%%%%%%%%%%%%%%%%%%%%%%%%%%%%%%%%%%%%%%%%%%%%%%%%%%%%%%%%%%%%

\end{document}

%%%%%%%%%%%%%%%%%%%%%%%%%%%%%%%%%%%%%%%%%%%%%%%%%%%%%%%%%%%%%%%%%%%%%%%%%%%%%%%%%%%%%%%%%%%%%%%%%%%
%%%%%%%%%%%%%%%%%%%%%%%%%%%%%%%%%%%%%%%%%%%%%%%%%%%%%%%%%%%%%%%%%%%%%%%%%%%%%%%%%%%%%%%%%%%%%%%%%%%
%%%%%%%%%%%%%%%%%%%%%%%%%%%%%%%%%%%%%%%%%%%%%%%%%%%%%%%%%%%%%%%%%%%%%%%%%%%%%%%%%%%%%%%%%%%%%%%%%%%
%%%%%%%%%%%%%%%%%%%%%%%%%%%%%%%%%%%%%%%%%%%%%%%%%%%%%%%%%%%%%%%%%%%%%%%%%%%%%%%%%%%%%%%%%%%%%%%%%%%

%\begin{frame}\frametitle{Motivaci\'on}
%El estudio y diagnóstico de una gran cantidad de enfermedades depende de nuestra habilidad para
%registrar y analizar se\~nales electrofisiol\'ogicas. 
%
%\vspace{3em}
%
%Se suele asumir que estas se\~nales son complejas: no lineales, no estacionarias y sin equilibrio 
%por naturaleza. Pero usualmente no se comprueban formalmente estas propiedades.
%\end{frame}

%%%%%%%%%%%%%%%%%%%%%%%%%%%%%%%%%%%%%%%%%%%
%%%%%%%%%%%%%%%%%%%%%%%%%%%%%%%%%%%%%%%%%%%
%
%\begin{frame}\frametitle{El cuarteto de Anscombe}
%\begin{figure}[h]
%\centering
%\includegraphics[width=0.8\linewidth]{./material1/anscombe.pdf} 
%\end{figure}
%\end{frame}
%
%%%%%%%%%%%%%%%%%%%%%%%%%%%%%%%%%%%%%%%%%%%
%%%%%%%%%%%%%%%%%%%%%%%%%%%%%%%%%%%%%%%%%%%

%\begin{frame}%\frametitle{Caracter\'istica buscada}
%\begin{defn}[Estacionariedad d\'ebil]
%Un proceso estoc\'astico $\{ X_t \}$, que satisface $E\left[ X_t^2 \right] < \infty$, 
%es \textbf{d\'ebilmente estacionario} si para todo tiempo $t$ se satisface que
%\begin{itemize}
%\item $E[X_t] = \mu < \infty$
%\item $Var\left[ X_t\right] =\sigma^2$
%\item Para todo $\tau\geq 1$, $Cov(X_t,X_{t+\tau}) = C(\tau)$
%\end{itemize}
%\end{defn}
%\end{frame}
%
%%%%%%%%%%%%%%%%%%%%%%%%%%%%%%%%%%%%%%%%%%%
%%%%%%%%%%%%%%%%%%%%%%%%%%%%%%%%%%%%%%%%%%%
%
%\begin{frame}\frametitle{Caso de estudio}
%Correlaci\'on inter-hemisf\'erica durante el sueño MOR del Adulto Mayor con Deterioro Cognitivo
%% sue\~no: atonia muscular, movimiento ocular r\'apido
%% parte frontal: toma de desiciones
%\begin{figure}[h]
%%\centering
%\includegraphics[width=0.5\linewidth]{./trabajo2/graficaintro.pdf}
%%\end{figure}
%%\begin{figure}[h]
%%\centering
%\includegraphics[width=0.5\linewidth]{./material1/cerebro.jpg}
%\end{figure}
%Adaptado de V\'azquez-Tagle y colaboradores (2016)
%\end{frame}
%
%%%%%%%%%%%%%%%%%%%%%%%%%%%%%%%%%%%%%%%%%%%
%%%%%%%%%%%%%%%%%%%%%%%%%%%%%%%%%%%%%%%%%%%
%
%\begin{frame}\frametitle{Registros de polisomnograma}
%\begin{figure}[h]
%\centering
%\includegraphics[width=\linewidth]{./trabajo2/estudio.pdf}
%\end{figure}
%\end{frame}
%
%%%%%%%%%%%%%%%%%%%%%%%%%%%%%%%%%%%%%%%%%%%
%%%%%%%%%%%%%%%%%%%%%%%%%%%%%%%%%%%%%%%%%%%
%
%\begin{frame}\frametitle{Prerrequisitos}
%Eliminar los efectos de:
%\begin{itemize}
%\item Tendencia
%\item Estacionalidad (\textit{seasonality})
%\end{itemize}
%\end{frame}
%
%%%%%%%%%%%%%%%%%%%%%%%%%%%%%%%%%%%%%%%%%%%
%%%%%%%%%%%%%%%%%%%%%%%%%%%%%%%%%%%%%%%%%%%
%
%\section[STL]{Seasonal Trend decomposition using Loess}
%
%\begin{frame}
%\frametitle{Descomposición cl\'asica usando loess}
%
%Filtro no-param\'etrico para generar las series de tiempo
%
%\begin{equation*}
%X_t = T_t + S_t + R_t
%\end{equation*}\\
%
%Tales que:\\
%\begin{description}
%\item[$S_t$] Función peri\'odica suave (frecuencia dada)
%\\ \textbf{Componente estacional}
%\item[$T_t$] Función suave, no necesariamente peri\'odica
%\\ \textbf{Tendencia}
%\item[$R_t$] Residuo
%\end{description}
%
%Comando en R: \textcolor{red}{\texttt{stl()}}
%\end{frame}
%
%%%%%%%%%%%%%%%%%%%%%%%%%%%%%%%%%%%%%%%%%%%
%%%%%%%%%%%%%%%%%%%%%%%%%%%%%%%%%%%%%%%%%%%
%
%\begin{frame}[fragile]
%\begin{figure}[h]
%\includegraphics[width=\linewidth]{./trabajo2/stl_demo2_1.pdf} 
%\end{figure}
%\begin{center}
%\includemovie{3em}{3em}{./trabajo2/stl_demo2_gif.gif}
%\end{center}
%\end{frame}
%
%%%%%%%%%%%%%%%%%%%%%%%%%%%%%%%%%%%%%%%%%%%
%%%%%%%%%%%%%%%%%%%%%%%%%%%%%%%%%%%%%%%%%%%
%
%%\begin{frame}
%%\begin{figure}[h]
%%\includegraphics[width=0.5\linewidth]{./trabajo2/stl_1.pdf} 
%%\includegraphics[width=0.5\linewidth]{./trabajo2/stl_2.pdf} 
%%\end{figure}
%%\end{frame}
%
%
%\begin{frame}\frametitle{Casos de estudio: sin deterioro}
%\begin{figure}[h]
%\includegraphics[width=0.5\linewidth]{./trabajo2/stl_n_f7.pdf} 
%\includegraphics[width=0.5\linewidth]{./trabajo2/stl_n_f8.pdf} 
%\end{figure}
%\end{frame}
%
%%estacionalidad: frecuencia de muestreo
%
%\begin{frame}\frametitle{Casos de estudio: posible deterioro}
%\begin{figure}[h]
%\includegraphics[width=0.5\linewidth]{./trabajo2/stl_d_f7.pdf} 
%\includegraphics[width=0.5\linewidth]{./trabajo2/stl_d_f8.pdf} 
%\end{figure}
%\end{frame}
%
%%%%%%%%%%%%%%%%%%%%%%%%%%%%%%%%%%%%%%%%%%%
%%%%%%%%%%%%%%%%%%%%%%%%%%%%%%%%%%%%%%%%%%%
%
%\section{Priestley-Subba Rao}
%
%\begin{frame}\frametitle{Proceso arm\'onico de Priestley}
%Al trabajar con se\~nales el\'ectricas, parece natural una representaci\'on del tipo
%
%\begin{equation*}
%X_t = \sum_{i=1}^{K} A_i \cos \left( \omega_i t + \phi_i \right)
%\end{equation*}
%
%Donde
%\begin{description}
%\item[$K$] $\in \mathbb{N}$
%\item[$A_i$] $\in \mathbb{R}$, {amplitudes}
%\item[$\omega_i$] $\in \mathbb{R}$, {frecuencias}
%\item[$ \phi_i $] va iid con distribuci\'on uniforme en $[-\pi,\pi]$
%\end{description}
%\end{frame}
%
%%%%%%%%%%%%%%%%%%%%%%%%%%%%%%%%%%%%%%%%%%%
%%%%%%%%%%%%%%%%%%%%%%%%%%%%%%%%%%%%%%%%%%%
%
%\begin{frame}\frametitle{Proceso arm\'onico de Priestley (ver. continua)}
%%Considerado \textit{todas} las frecuencias en $[-\pi,\pi]$
%Al trabajar con se\~nales el\'ectricas, parece natural una representaci\'on del tipo
%
%\begin{equation*}
%X_t = \int_{-\pi}^{\pi} A(\omega) e^{i\omega t} \, d\xi(\omega)
%\end{equation*}
%
%Donde
%\begin{description}
%\item[$\omega$] Frecuencia puntual
%\item[$\xi(\omega)$] \textit{va. infinitesimal}, asociada con $\omega$
%\item[$A(\omega)$] Amplitud en $\omega$
%\end{description}
%\end{frame}
%
%%%%%%%%%%%%%%%%%%%%%%%%%%%%%%%%%%%%%%%%%%%
%%%%%%%%%%%%%%%%%%%%%%%%%%%%%%%%%%%%%%%%%%%
%
%\begin{frame}\frametitle{Representaci\'on espectral}
%Se tiene un proceso $X_t$, con $E[X_t]<\infty$, $E[T_t^{2}]<\infty$,
%%y que consid\'erese la representaci\'on 
%%Sup\'ongase que es posible expresar
%
%\begin{equation*}
%X_t = \int_{-\pi}^{\pi} A_t(\omega) e^{i\omega t} \, d\xi(\omega)
%\end{equation*}
%
%\begin{description}
%\item[$\{ \xi(\omega) \}$] Familia de procesos ortogonales con {{ $E \lvert d \xi(\omega) \lvert^{2} = \mu(\omega)$ }}
%\item[$A_t(\omega)$] Su tr. de Fourier en $t$ tiene un m\'aximo absoluto en 0
%\end{description}
%
%\pause
%Se define el \textbf{espectro evolutivo} de $X_t$, con respecto a la la familia, como
%\begin{equation*}
%f_t(\omega) = \lvert A_t(\omega) \lvert^{2}
%\end{equation*}
%\end{frame}
%
%%%%%%%%%%%%%%%%%%%%%%%%%%%%%%%%%%%%%%%%%%%
%%%%%%%%%%%%%%%%%%%%%%%%%%%%%%%%%%%%%%%%%%%
%
%\begin{frame}\frametitle{Test de Priestley-Subba Rao}
%Sup\'ongase que puede expresarse
%\begin{equation*}
%X_t = \int_{-\pi}^{\pi} A_t(\omega) e^{i\omega t} \, d\xi(\omega)
%\end{equation*}
%
%\pause
%\begin{center}
%$X_t$ es estacionaria $\Rightarrow A_t(\omega)$ es constante $\Rightarrow f_t(\omega)$ es constante
%\end{center}
%
%\pause
%La contrapositiva
%\begin{center}
%$f_t(\omega)$ no constante $\Rightarrow X_t$ \textbf{no} es estacionaria
%\end{center}
%
%\underline{Por hacer:} Encontrar un \textit{buen} estimador para $f_t$
%\end{frame}
%
%%%%%%%%%%%%%%%%%%%%%%%%%%%%%%%%%%%%%%%%%%%
%%%%%%%%%%%%%%%%%%%%%%%%%%%%%%%%%%%%%%%%%%%
%
%\begin{frame}[fragile]\frametitle{Estimador de doble ventana (sin detalles)}
%
%\begin{center}
%\begin{tabular}{cc}
%\includemovie{5em}{5em}{./trabajo2/priestley_spectra_full.gif}
%\includemovie{5em}{5em}{./trabajo2/priestley_spectra_small.gif}
%\end{tabular} 
%\end{center}
%
%%Por hacer: reescribir las ecuaciones del paper priestley68
%\end{frame}
%
%%%%%%%%%%%%%%%%%%%%%%%%%%%%%%%%%%%%%%%%%%%
%%%%%%%%%%%%%%%%%%%%%%%%%%%%%%%%%%%%%%%%%%%
%
%\begin{frame}
%\begin{figure}[h]
%%\includegraphics[width=\linewidth]{./trabajo2/ejemplo_espectro.pdf} 
%\includegraphics[width=\linewidth]{./trabajo2/n7f.pdf}
%\end{figure}
%\end{frame}
%
%%%%%%%%%%%%%%%%%%%%%%%%%%%%%%%%%%%%%%%%%%%
%%%%%%%%%%%%%%%%%%%%%%%%%%%%%%%%%%%%%%%%%%%
%
%\begin{frame}%\frametitle{Estad\'istico $Y$}
%Se define $Y_{i,j} = \log \left( \widehat{f_{t_i}}(\omega_j) \right)$, con las siguientes propiedades
%
%\begin{equation*}
%E\left[ Y_{i,j} \right] \thicksim \log \left( f_{t_i}(\omega_j) \right)
%\hspace{4em}
%\text{Var}\left( {Y\left(t,\omega\right)}\right) \thicksim \sigma^{2}
%\end{equation*}
%
%Luego, puede escribirse $Y_{i,j} = \log \left( f_{t_i}(\omega_j) \right) + \varepsilon_{i,j}$,
%con $\varepsilon_{i,j}$ va iid
%\vspace{2em}
%
%\pause
%Usando un test ANOVA --de varianza conocida-- se puede saber si $\varepsilon$
%\vspace{-1em}
%\begin{itemize}
%\item Tiene marginales
%\item Constante sobre el tiempo
%\item Constante sobre las frecuencias
%\end{itemize}
%\end{frame}
%
%%%%%%%%%%%%%%%%%%%%%%%%%%%%%%%%%%%%%%%%%%%
%%%%%%%%%%%%%%%%%%%%%%%%%%%%%%%%%%%%%%%%%%%
%
%\begin{frame}[fragile]
%\begin{lstlisting}
%Priestley-Subba Rao stationarity Test for datos
%-----------------------------------------------
%Samples used              : 3072 
%Samples available         : 3069 
%Sampling interval         : 1 
%SDF estimator             : Multitaper 
%  Number of (sine) tapers : 5 
%  Centered                : TRUE 
%  Recentered              : FALSE 
%Number of blocks          : 11 
%Block size                : 279 
%Number of blocks          : 11 
%p-value for T             : 0.4130131 
%p-value for I+R           : 0.1787949 
%p-value for T+I+R         : 0.1801353 
%\end{lstlisting}
%\end{frame}
%
%%%%%%%%%%%%%%%%%%%%%%%%%%%%%%%%%%%%%%%%%%%
%%%%%%%%%%%%%%%%%%%%%%%%%%%%%%%%%%%%%%%%%%%
%
%\begin{frame}[fragile]
%\begin{tabular}{cc}
%\includegraphics[width=0.5\linewidth]{./trabajo2/n7f.pdf} 
%&
%\includegraphics[width=0.5\linewidth]{./trabajo2/d7f.pdf} 
%\\
%\begin{lstlisting}
%T      : 0 
%I+R    : 0 
%T+I+R  : 0 
%\end{lstlisting}
%&
%\begin{lstlisting}
%T      : 4.36434e-07 
%I+R    : 0.0001575723 
%T+I+R  : 6.98897e-06 
%\end{lstlisting}
%\end{tabular}
%\end{frame}
%
%%%%%%%%%%%%%%%%%%%%%%%%%%%%%%%%%%%%%%%%%%%
%%%%%%%%%%%%%%%%%%%%%%%%%%%%%%%%%%%%%%%%%%%
%
%\begin{frame}[fragile]
%\begin{tabular}{cc}
%\includegraphics[width=0.5\linewidth]{./trabajo2/n8f.pdf} 
%&
%\includegraphics[width=0.5\linewidth]{./trabajo2/d8f.pdf} 
%\\
%\begin{lstlisting}
%T      : 0 
%I+R    : 5.787895e-09 
%T+I+R  : 0 
%\end{lstlisting}
%&
%\begin{lstlisting}
%T      : 0.00332259 
%I+R    : 0.03502537 
%T+I+R  : 0.01598073 
%\end{lstlisting}
%\end{tabular}
%\end{frame}
%
%%%%%%%%%%%%%%%%%%%%%%%%%%%%%%%%%%%%%%%%%%%
%%%%%%%%%%%%%%%%%%%%%%%%%%%%%%%%%%%%%%%%%%%
%
%\begin{frame}\frametitle{Conclusi\'on}
%El an\'alisis de la estacionariedad d\'ebil en registros de sue\~no MOR
%muestra diferencias para el grupo con Deterioro Cognitivo (DC)
%en comparaci\'on con el grupo control, lo que indica que el an\'alisis de estacionariedad es
%eficiente como marcator para detectar el DC durante el sue\~no MOR en adultos mayores.
%\end{frame}
%
%%%%%%%%%%%%%%%%%%%%%%%%%%%%%%%%%%%%%%%%%%%
%%%%%%%%%%%%%%%%%%%%%%%%%%%%%%%%%%%%%%%%%%%
%
%\begin{frame}\frametitle{Agradecimientos}
%\begin{figure}
%\includegraphics[width=0.6\linewidth]{cerebro.jpg}
%\end{figure}
%\begin{itemize}
%\item A la Dra. Erika Rodr\'guez Torres, por el \textbf{gran} apoyo prestado
%\item A la Dra. G\'enesis Roc\'io V\'azquez-Tagle Gallegos, por las bases de datos
%\end{itemize}
%%[Tambi\'en deber\'ia mencionarse los datos obtenidos]
%\end{frame}

%%%%%%%%%%%%%%%%%%%%%%%%%%%%%%%%%%%%%%%%%%
%%%%%%%%%%%%%%%%%%%%%%%%%%%%%%%%%%%%%%%%%%

%\begin{frame}[allowframebreaks]
%\frametitle{Bibliograf\'ia}
%%\nocite{*}
%\footnotesize{
%%\bibliography{referencias}
%\bibliography{referencias_estacionariedad,referencias_fisiologia,referencias_otros,referencias_mixto}{}
%\bibliographystyle{apalike-es}
%%\bibliographystyle{abbrv}
%}
%\end{frame}

%\end{document}