%%%%%%%%%%%%%%%%%%%%%%%%%%%%%%%%%%%%%%%%%%%%%%%%%%%%%%%%%%%%%%%%%%%%%%%%%%%%%%%%%%%%%%%%%%%%%%%%%%%
%%%%%%%%%%%%%%%%%%%%%%%%%%%%%%%%%%%%%%%%%%%%%%%%%%%%%%%%%%%%%%%%%%%%%%%%%%%%%%%%%%%%%%%%%%%%%%%%%%%
\chapter{Resultados}

En cada canal que conforma el PSG (EEG, EOG y EMG), 
cada una de las \'epocas consideradas fue clasificada como 
''Posiblemente Estacionaria'' (PE) si no pudo ser rechazado la hip\'otesis de 
estacionariedad usando el test PSR ($\alpha = 0.05$), mientras que 
%el rechazo de esta misma 
%hip\'otesis es argumento ''suficiente'' para afirmar que el segmento de registro en cuenti\'on
%puede ser clasificado omo no.estacionario.
fue clasificado como ''No-estacionaria'' en caso contrario. Variar el valor cr\'itico
para la clasificaci\'on, siempre menor a 0.05, no pareci\'o generar diferencias significativas.
%La cantidad de \'epocas que no fueron para las cuales no es posible rechazar la hip\'otesis de
%estacionariedad ($\alpha=0.05$)
%clasificadas como ''Posiblemente Estacionarias'' (PE). 
La cantidad de \'epocas PE en cada individuo 
%con respecto a la cantidad total de \'epocas
durante el sue\~no MOR y no-MOR
muestra en las tablas \ref{total_gpos_mor}, \ref{total_gpos_nmor} y
\ref{total_gpos_total}; debido a que entre los sujetos hubo una gran variabilidad entre el tiempo 
que permanecieron en sue\~no MOR, se sugiri\'o comparar no el total de \'epocas PE sino
la proporci\'on --con respecto a la duraci\'on, medida en \'epocas-- de estas etapas, 
mustr\'andose estos resultados en las tablas \ref{gpos_mor}, \ref{gpos_nmor} y
\ref{gpos_total}. En estas \'ultimas tablas, se han calculado promedios y desviaciones
est\'andar muestrales entre los dos grupos que son comparados.


\begin{SidewaysFigure}
\centering
\begin{tabular}{c||ccccc||cccc||ccc}
&VCR&MJH&JAE&GHA&MFGR&CLO&RLO&RRU&JGZ&FGH&MGG&EMT \\
\hline
C3&**& &*&**& & &**&*& & & &  \\
C4&*& &***&*& & &***& & & &*&  \\
CZ&***& & & & & & & & & &***&  \\
F3&**& & &**& & &***& & & &*&** \\
F4& & & &*& & &***& & & &***&  \\
F7&*& &***&***& & & &*& & &***&*** \\
F8&**& & &**& & &*&*& & &***&  \\
FP1&***& & &*&*& & &*& & &***&  \\
FP2&***& & &*& & & & & & &***&  \\
FZ& & & &***& & &***&*& & &*&  \\
O1&*& & &***& & & & & &*& &  \\
O2& & &*&***& & &***& & & &***& \\ 
P3&*& &*&***& & &**& & & &*&  \\
P4&***& &***&*& & & & & &*&***&  \\
PZ&**& &***&*& & & &*& & &***&  \\
T3& & &**& & & &***& & & & &** \\
T4& & & &*& & &***& & & &*&  \\
T5&*& & &***& & &***& & & & &  \\
T6& & &*& & & & & & & &***&  \\
LOG&***& &***&***&**&***&**&***&*& &***&  \\
ROG&***& &***&***&*&*&***&*&*& &***&  \\
EMG& & & & & &***& &*& & & &  \\
\hline
General& & & & & &***& &*& & & & 
\end{tabular}
\caption{Diferencias significativas para la comparaci\'on entre la proporci\'on
de \'epocas PE en sue\~no MOR (fase R) y no-MOR (fases W y N).
Los asteriscos representan el pvalor con el cual se rechaza la hip\'otesis de
que las diferencias son significativas: *=0.05 , **=0.01 , ***=0.005}
\label{comparacion_mor_vs_total}
\end{SidewaysFigure}

Como un primer an\'alisis, para cada sujeto y en cada canal 
se compar\'o la proporci\'on de \'epocas PE en sue\~no MOR contra el registro completo;
el fin de ello es verificar si el sue\~no MOR --entendido como muestra no-aleatoria
del sue\~no-- tiene propiedades similares o no, y si \'esta similaridad pudiera estar
relacionada con el PDC del paciente. 
Las comparaciones se llevaron a cabo usando la prueba $\chi^{2}$ para 
proporciones\footnote{Implementada en R como la funci\'on \texttt{prop.test()}},
los resultados se muestran en la tabla \ref{comparacion_mor_vs_total}.
Se encontr\'o que no hay una relaci\'on clara entre el estado de salud del sujeto y
la aparici\'on de diferencias significativas entre estas proporciones, se hallaron
consistentemente diferencias en los canales LOG y ROG --que pueden explicarse como la actividad
ocular caracter\'istica del sue\~no MOR.
%; en la secci\'on
%de discusi\'on se mencionan algunos datos que se ''recuperaron'' de este an\'alisis.


Posteriormente se procedi\'o a comparar, por cada canal, si la proporci\'on de \'epocas PE
presenta diferencias significativas entre los grupos. Esta comparaci\'on se realiz\'o tomando
en cuenta las \'epocas de sue\~no MOR, de sue\~no no-MOR, y el registro completo 
--ver, respectivamente, tablas \ref{gpos_mor}, \ref{gpos_nmor}, \ref{gpos_total}.
Para una mejor visualizaci\'on de estos, se han graficado
% que representan
%las proporciones respectivas de cada sujeto --para ambos grupos.
en la figura \ref{comparacion_graf}
los datos de las tablas \ref{gpos_mor}, \ref{gpos_nmor} y
\ref{gpos_total}.

\begin{figure}
\centering
\subfloat[Comparaci\'on entre \'epocas MOR (fase R)]{
\includegraphics[width=0.95\linewidth]
{./muypreeliminar170408/Comparacion_gpos_MOR.png} 
}\\
\subfloat[Comparaci\'on entre \'epocas no-MOR (fases W y N)]{
\includegraphics[width=0.95\linewidth]
{./muypreeliminar170408/Comparacion_gpos_NMOR.png} 
}\\
%\subfloat[Comparaci\'on entre el total de \'epocas registradas]{
%\includegraphics[width=0.95\linewidth]
%{./muypreeliminar170408/Comparacion_gpos_TOT.png} 
%}\\
\caption{Comparaci\'on sobre las proporciones de \'epocas PE entre los grupos, para diferentes
etapas de sue\~no. Se han graficado las proporciones de PE en todos los sujetos de ambos grupos,
as\'i como sus respectivos promedios, para cada etapa de sue\~no.}
\label{comparacion_graf}
\end{figure}

La comparaci\'on per se se llev\'o a cabo usando la prueba %no param\'etrica
%$t$ de Student y 
$U$ de Mann-Whitney\footnote{Implementada en R como la funci\'on \texttt{wilcox.test()}}.
%El primer test arroja diferencias significativas para los canales LOG, ROG y EMG, mientras que
%el segundo indica que no hay diferencias significativas. 
No se encontraron diferencias significativas para ninguno de los canales.

%Debido a que los canales
%donde se hallaron diferencias significativas no corresponden a registros de actividad
%cerebral, se considera que \textbf{esta caracter\'istica no proporciona evidencias
%suficientes sobre diferencias significativas entre adultos mayores con y sin PDC diagnosticado}.
%Este resultado es clave en el desarrollo de este trabajo.

Una tercera prueba efectuada sobre los datos fue una comparaci\'on para la proporci\'on de \'epocas
PE en cada canal, para revisar diferencias grupales entre sue\~no MOR y NMOR.
Esta prueba tiene una interpretaci\'on m\'as bien complicada, aunque su motivaci\'on es clara:
una vez hecha la comparaci\'on individial de las proporciones de PE al 
''transitar'' los sujetos entre etapas de sue\~no, y viendo que 
no hay diferencias claras entre grupos, cabe preguntarse si conviene considerar a los grupos 
como unidades.
Se encuentra que hay diferencias significativas ($\alpha<.1$) para el grupo normal
en los canales C3, C4, F7, F8, FP1, FP2, O2, P4, LOG, ROG; en el grupo PDC no se encontr\'o
ninguna diferencia.
Las diferencias encontradas pueden ser relevantes fisiol\'ogicamente, ya que 
abarcan gran parte de los l\'obulos frontal y parietal, y una regi\'on occipital-parietal derecha.
Se resta importancia a las diferencias halladas en LOG y ROG, ya que en el grupo PDC puede
aceptarse esta diferencia ''con no mucha probabilidad'' ($\alpha<.15$) y porque se esperaba
esta diferencia t\'ipica del sue\~no MOR

\begin{figure}
\centering
\subfloat[Comparaci\'on para el grupo control]{
\includegraphics[width=0.95\linewidth]
{./muypreeliminar170408/comp_etapas_gpos_NORMALMOR_vs_TOTAL.png} 
}\\
\subfloat[Comparaci\'on para el grupo PDC]{
\includegraphics[width=0.95\linewidth]
{./muypreeliminar170408/comp_etapas_gpos_PDCMOR_vs_TOTAL.png} 
}\\
\subfloat[Comparaci\'on de los p-valores para aceptar diferencias]{
\includegraphics[width=0.95\linewidth]
{./muypreeliminar170408/Comparacion_pvals_gpos_MOR_vs_TOTAL.png} 
}\\
\caption{Comparaci\'on sobre las proporciones de \'epocas PE entre las etapas de sue\~no, 
para ambos grupos por separado. 
Se han graficado las proporciones de PE en todos los sujetos de cada grupo,
para todo el sue\~no y la etapa MOR.}
\label{comparacion_verde}
\end{figure}

%%%%%%%%%%%%%%%%%%%%%%%%%%%%%%%%%%%%%%%%%%%%%%%%%%%%%%%%%%%%%%%%%%%%%%%%%%%%%%%%%%%%%%%%%%%%%%%%%%%
%%%%%%%%%%%%%%%%%%%%%%%%%%%%%%%%%%%%%%%%%%%%%%%%%%%%%%%%%%%%%%%%%%%%%%%%%%%%%%%%%%%%%%%%%%%%%%%%%%%