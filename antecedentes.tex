%%%%%%%%%%%%%%%%%%%%%%%%%%%%%%%%%%%%%%%%%%%%%%%%%%%%%%%%%%%%%%%%%%%%%%%%%%%%%%%%%%%%%%%%%%%%%%%%%%%

\chapter{Antecedentes/Justificaci\'on}

Esta parte se separa conceptualmente de ambas componentes (matem\'atica y fisiol\'ogica)
cuando se centra en la combinaci\'on de ambas: el estudio de las propiedades de EEG
usando herramientas matem\'aticas, una revisi\'on hist\'orica.

Me he tomado la libertad --considerando un tiempo m\'as bien escaso-- de incluir en la parte
matem\'atica las consideraciones formales del modelo considerado: estacionariedad en series
de tiempo a tiempo continuo. Siguiendo la misma l\'ogica he puesto en la parte fisiol\'ogica
las consideraciones t\'ecnicas del modelo: registros polisomnogr\'aficos de sue\~no MOR en
el adulto mayor --con y sin deterioro cognitivo. Esta secci\'on, en contraste, trata
sobre la historia de por qu\'e tiene sientido este modelo, quien lo ha trabajado y bajo qu\'e
consideraciones.

A grosso modo, esta secci\'on dice: yo no me invent\'e de la nada que mi tema de investigaci\'on
era una buena idea, 
m\'as bien hice lo que hice porque
el trabajo de varios acad\'emicos a lo largo de la historia me han hecho
pensar que podr\'ia investigar un rato algunas propiedades que podr\'ian surgir en un tipo de 
datos --y que las conclusiones pudieran ser significativas.

%%%%%%%%%%%%%%%%%%%%%%%%%%%%%%%%%%%%%%%%%%%%%%%%%%%%%%%%%%%%%%%%%%%%%%%%%%%%%%%%%%%%%%%%%%%%%%%%%%%