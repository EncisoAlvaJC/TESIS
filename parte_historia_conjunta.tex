%%%%%%%%%%%%%%%%%%%%%%%%%%%%%%%%%%%%%%%%%%%%%%%%%%%%%%%%%%%%%%%%%%%%%%%%%%%%%%%%%%%%%%%%%%%%%%%%%%%

\chapter{Propiedades estad\'isticas del EEG}

Este secci\'on contiene [va a contener] un bosquejo hist\'orico sobre los supuestos que se
han hecho sobre la estacionariedad y otras series fisiol\'ogicas. Y es que en el resumen del
congreso se incluyo la frase 

\begin{quotation}
Usualmente se asume que las series fisiol\'ogicas son complejas:
no-estacionarias, no-lineales y no en equilibrio por naturaleza. Sin embargo, estas propiedades
no se suelen probar formalmente.
\end{quotation}

Debido a que mi trabajo pretende tomar una posici\'on opesta en alg\'un momento, en alg\'un grado,
ser\'ia incompatible que yo \textbf{simplemente suponga} que esa posici\'on es verdadera.
[Me he dado a la tarea de investigar un poco sobre el papel hist\'rico que han jugado las
hip\'otesis de regularidad en las series electrofisiol\'ogicas.]

Esta secci\'on debiera partir de los comentarios expresados 
en 'Everything you wanted to ask about EEG (..)'
(Klonowski, 2009) sobre c\'omo el concepto de ondas se acu\~na en el estudio de EEG, especilmente
de c\'omo se entienden las frecuencias en este contexto --visi\'on que es reforzada al citar
el manual de la IAAC 2007 para detectar las etaas de sue\~no.

En esta visi\'on, cabe destacar los muchos trabajos de Harmony y de Corsi-Cabrera
sobre la caracterizaci\'on y localizaci\'on de diferentes tipos de actividad cerebral 
durante diversas actividades y condiciones Adem\'as de otros autores. 
Sinceramente, son bastantes trabajos y
son la gu\'ia sobre los an\'alisis de composici\'on espectral que a\'un est\'an por hacerse.

Peor tambi\'en hay una discusi\'on sobre el balance entre los estudios espectrales contra
los avances en teor\'ia espectral: se puede citar a Cohen, quien asume que las series
cortas son b\'asicamente estacionarias. Por supuesto que cada punto en el tiempo es
t\'ecnicamente estacionario, y es completamente plausible --en el contexto de las series
electrofisiol\'ogicas-- suponer que para cada punto existe un abierto
en el tiempo tal que el subproceso definido all\'i es etsacionario para todo fin pr\'actico.
Como comenta Melard, la suposici\'on de estacionariedad para series cortas se 
consider\'o v\'alida por mucho tiempo debido a ala escasa capacidad de c\'omputo; a
modo de sintesis, Adak muestra un resultado negativo sobre la suposicion de estacionariedad
local pero muestra una prueba para detectar y medir la as\'i llamada 'estacionariedad local'.
[escribir\'e tal demostraci\'on]

En este punto, es conveniente hablar sobre los modelos ARMA como la forma mas natural de
estacionariedad a tiempo discreto, y como se usa en los modelos de estacionaeriedad 
local (citar a Adak). Se han posido generalizar estos modelos a parametros que
dependen del tiempo como los modelos ARCH (quia citar a Chatfield y a Subba Rao).

Hay una historia extensa al establecer el concepto de espectro en series no-estacionarias,
y para ello me servire de las revisiones de Loynes, Melard, Adak, Brillinger. En ella,
brillan la funcion de autocorrelacion que depende del tiempo y el que su transformada
de Fourier sea la funci\'on de densidad espectral en caso de existir. Muchas
definiciones de espectro basadas en su forma de ser calculadas.

Me gustar\'ia escribir un segundo resumen sobre el espectro de Wold-Cram\'er (el que se maneja
en el test PSR) en contraposici\'on al espectro de Wigner-Ville, el espectro de
ondeletas de Gabor y el espectro de ondeletas de Haar. Me apoyaria mucho de una discusion
hecha por Nason, de la cuakl resaltare una estimacion sobre los ordenes de tiempo de
computo para los estimadores de estos espectros.

----------------------------------------

Esta secci\'on debiera terminar nuevamente citando a Klonowski y, 
si bien no voy a usar en esta tesis, los
nuevos enfoques que consideran al EEG fundamentalmente como un sistema sujeto a ruido pero
fundamentalmente ca\'otico.

%%%%%%%%%%%%%%%%%%%%%%%%%%%%%%%%%%%%%%%%%%%%%%%%%%%%%%%%%%%%%%%%%%%%%%%%%%%%%%%%%%%%%%%%%%%%%%%%%%%