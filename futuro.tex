%%%%%%%%%%%%%%%%%%%%%%%%%%%%%%%%%%%%%%%%%%%%%%%%%%%%%%%%%%%%%%%%%%%%%%%%%%%%%%%%%%%%%%%%%%%%%%%%%%%

\section{Trabajo a futuro}

Como se ha sugerido, los bloques de estacionariedad pueden tener un uso como 
caracter\'isticas auxiliares
para la detecci\'on autom\'atica de \'epocas MOR en registros de PSG: el hecho que la proporci\'on
de \'epocas PE no se vea afectada --estad\'isticamente-- por el PDC del paciente, sugiere que es
posible obtener resultados independientes de ello. Para ello cabe recordar, como se mencion\'o 
en la secci\'on de discusi\'on, que sujetos fuera del rango de los grupos considerados puede
que fallen respecto a esta conclusi\'on: hace falta m\'as indagaci\'on al respecto. 

Por otro lado, el uso de estimadores espectrales de ventana pueden explorarse de manera m\'as
puntual para detectar estacionariedad sobre componentes de frecuencia espec\'ificas, de modo
que es en principio posible separar las ondas cerebrales 

%%%%%%%%%%%%%%%%%%%%%%%%%%%%%%%%%%%

%%%%%%%%%%%%%%%%%%%%%%%%%%%%%%%%%%%%%%%%%%%%%%%%%%%%%%%%%%%%%%%%%%%%%%%%%%%%%%%%%%%%%%%%%%%%%%%%%%%