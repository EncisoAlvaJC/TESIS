%%%%%%%%%%%%%%%%%%%%%%%%%%%%%%%%%%%%%%%%%%%%%%%%%%%%%%%%%%%%%%%%%%%%%%%%%%%%%%%%%%%%%%%%%%%%%%%%%%%
\chapter{Planteamiento del problema}

La poblaci\'on mundial est\'a envejeciendo aceleradamente, lo que se debe en gran parte a la 
mejor\'ia en la atenci\'on de la salud durante el \'ultimo siglo, traducida en vidas m\'as 
largas y saludables. Sin embargo, este logro tambi\'en ha tenido como resultado un aumento en el 
n\'umero de personas con enfermedades no transmisibles, incluida la demencia.

Por otro lado, todav\'ia son incipientes las investigaciones para identificar 
factores de riesgo modificables de la demencia.

La demencia es un síndrome de naturaleza cr\'onica y progresiva, caracterizado 
por el deterioro de las funciones cognoscitivas y de la conducta, lo que ocasiona 
discapacidad y dependencia.
\cite{PlanAlzheimer04}



Tanto el deterioro cognitivo como las alteraciones del sueño son frecuentes en las personas de edad avanzada y, aunque interfieren con la calidad de vida del individuo, no suelen ser consideradas como patológicas sino como problemas asociados al propio envejecimiento.
Ya que el estudio del deterioro cognitivo en México ha sido medido transversal y longitudinalmente en periodos de años, no se puede comprobar si ser adulto mayor anteriormente, representaba complicaciones objetivas antes de la evaluación en este trabajo se propone un análisis de las evaluaciones neuropsicológicas de la atención y memoria y su relación con el sueño por medio de un electroencefalograma en dos momentos distintos.
El primer momento del presente trabajo es evaluar y diagnosticar al Adulto Mayor hidalguense en los aspectos cognitivos de la atención y memoria por medio de la prueba estandarizada Neuropsi. En un segundo momento, se pretenden medir aspectos cerebrales en el momento del sueño por medio de un encefalograma para que con los resultados obtenidos de ambas fases y se determine la diferencia entre los grupos de adulto mayor en evaluación cognitiva y su encefalografía.
Dada la relación entre el deterioro cognitivo y la calidad de sueño en adultos mayores, parece necesario reflexionar sobre las prevalencias del deterioro cognitivo, el rumbo que han tomado y su evaluación psicológica, las funciones mentales evaluadas, las metodologías y los resultados en México con énfasis particular en su relación con salud mental, así como considerar alternativas para valorar la función mental a nivel estatal y nacional con el apoyo de métodos de análisis biológicos. 


%%%%%%%%%%%%%%%%%%%%%%%%%%%%%%%%%%%%%%%%%%%%%%%%%%%%%%%%%%%%%%%%%%%%%%%%%%%%%%%%%%%%%%%%%%%%%%%%%%%

\section{Justificaci\'on}

La poblaci\'on mundial est\'a envejeciendo aceleradamente, lo que se debe en gran parte a la 
mejor\'ia en la atenci\'on de la salud durante el \'ultimo siglo, traducida en vidas m\'as 
largas y saludables. Sin embargo, este logro tambi\'en ha tenido como resultado un aumento en el 
n\'umero de personas con enfermedades no transmisibles, incluida la demencia.
Por otro lado, todav\'ia son incipientes las investigaciones para identificar 
factores de riesgo modificables de la demencia.\cite{PlanAlzheimer04}

Justificación 
Dentro de la teoría de los ciclos de vida, la última etapa se refiere a la población envejecida, en quienes lo característico son las pérdidas físicas, mentales, sociales y económicas asociadas a la edad avanzada, con una vuelta a la dependencia sobre el grupo intermedio. En la caracterización del grupo envejecido el deterioro de la salud es un elemento principal. El mayor impacto social y probablemente económico del envejecimiento, se desprende de los cambios en el estado de salud que conlleva.
El deterioro cognitivo, que se define como “un estado de transición entre el envejecimiento normal y la demencia” (3) es uno de los problemas de salud más importantes en los países desarrollados y sub desarrollados que afecta directamente la calidad de vida de los adultos mayores lo que genera una mayor demanda de servicios de salud en el país.
Dada su relación con la edad, en la última década se ha dado un continuo incremento tanto en su incidencia como en su prevalencia, secundariamente al aumento progresivo de la longevidad en la población en general.
El estado de salud de la población de edad avanzada en su conjunto, tiene un peso específico que recae en el sistema de salud en mayor o menor grado en función de la eficiencia de éste. Por ello, existe la necesidad de que en el sistema de salud, realice una valoración integral del adulto mayor por parte del equipo de salud, para la detección temprana del deterioro cognitivo ya que  de nuestra población adulta mayor vive con algún tipo de deterioro, y al no ser diagnosticado adecuadamente incrementa el riesgo de evolucionar a demencia y por consiguiente largos y pesados periodos de cuidado y una atención especializada para recibir el tratamiento más apropiado al que no todos tienen acceso, por los gastos que representa en todos los niveles. Por otro lado, existe una gran demanda de datos fiables sobre la salud mental de los adultos mayores del estado de Hidalgo, dadas las limitaciones de los estudios epidemiológicos realizados hasta la fecha y los problemas metodológicos derivados de trabajar con diferentes localidades.

%%%%%%%%%%%%%%%%%%%%%%%%%%%%%%%%%%%%%%%%%%%%%%%%%%%%%%%%%%%%%%%%%%%%%%%%%%%%%%%%%%%%%%%%%%%%%%%%%%%

\section{Pregunta de investigaci\'on}

Considerando los registros de polisomnograma como series de tiempo,
¿es plausible usar sus propiedades estad\'isticas dependientes del tiempo, espec\'ificamente
la estacionariedad d\'ebil,
como marcadores para el diagn\'ostico cl\'inico del deterioro cognitivo --en alguna de 
sus fases-- en adultos mayores?

%%%%%%%%%%%%%%%%%%%%%%%%%%%%%%%%%%%%%%%%%%%%%%%%%%%%%%%%%%%%%%%%%%%%%%%%%%%%%%%%%%%%%%%%%%%%%%%%%%%

\section{Hip\'otesis}

Existen diferencias cuantificables estad\'isticamente significativas entre las propiedades
estad\'isticas dependientes del tiempo, de los registros poliomnogr\'aficos
durante etapas espec\'ificas de sue\~no, en adultos
mayores con deterioro cognitivo con respecto a adultos mayores normales.

%%%%%%%%%%%%%%%%%%%%%%%%%%%%%%%%%%%%%%%%%%%%%%%%%%%%%%%%%%%%%%%%%%%%%%%%%%%%%%%%%%%%%%%%%%%%%%%%%%%

\section{Objetivo general}

Deducir, a partir de pruebas estad\'isticas formales (en el sentido matem\'atico), las propiedades
estad\'isticas de partes 

\subsection{Objetivos espec\'ificos}

\begin{itemize}
\item Estudiar las consecias te\'oricas y posiblemente pr\'acticas de que una serie de tiempo
sea o no d\'ebilmente estacionaria

\item Investigar en la literatura c\'omo detectar si una serie dada es d\'ebilmente estacionaria
a partir de observaciones, y bajo qu\'e supuestos es v\'alida la detecci\'on

\item Usando los an\'alisis hallados en la literatura para determinar si las series de tiempo,
obtenidas a partir de los registros polisomnogr\'aficos en adultos mayores, son o no
d\'ebilmente estacionarias.
Revisar si la informaci\'on obtenida en los diferentes sujetos muestra diferencias entre
sujetos con y sin deterioro cognitivo.

\item Determinar, para el caso de estudio presente, la validez de los \textbf{supuestos de 
estacionariedad d\'ebil}: su presencia o ausencia, en intervalos cortos o arbitrarios de tiempo
\end{itemize}

%%%%%%%%%%%%%%%%%%%%%%%%%%%%%%%%%%%%%%%%%%%%%%%%%%%%%%%%%%%%%%%%%%%%%%%%%%%%%%%%%%%%%%%%%%%%%%%%%%%