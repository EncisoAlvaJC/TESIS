%%%%%%%%%%%%%%%%%%%%%%%%%%%%%%%%%%%%%%%%%%%%%%%%%%%%%%%%%%%%%%%%%%%%%%%%%%%%%%%%%%%%%%%%%%%%%%%%%%%
%\chapter{Planteamiento del problema}

%%%%%%%%%%%%%%%%%%%%%%%%%%%%%%%%%%%%%%%%%%%%%%%%%%%%%%%%%%%%%%%%%%%%%%%%%%%%%%%%%%%%%%%%%%%%%%%%%%%

\section{Justificaci\'on}

El MCI se define como ''un s\'indrome caracterizado por una alteraci\'on adquirida y prolongada de
de una o varias funciones cognitivas, que no corresponde a un s\'indrome focal y no cumple
criterios suficientes de gravedad para ser calificada como demencia'' \cite{Robles02}.
De acuerdo a la Encuesta Nacional de Salud y Nutrici\'on (ENSANUT) efectuada en M\'exico 2002,
se estima que existen 800,000 adultos mayores \cite{Sosa12}.

%El deterioro cognitivo, se define como ''un estado de transici\'on entre el envejecimiento 
%normal y la demencia'' {Garrido09} es uno de los problemas de salud m\'as importantes en los 
%pa\'ises desarrollados y sub-desarrollados que afecta directamente la calidad de vida de los 
%adultos mayores lo que genera una mayor demanda de servicios de salud en el pa\'is.
%Por otro lado, todav\'ia son incipientes las investigaciones para identificar 
%factores de riesgo modificables de la demencia.
%\cite{PlanAlzheimer04}

El cuidado de enfermedades cr\'onicas 
en la poblaci\'on de edad avanzada representa un gran peso
econ\'onomico y de recursos humanos, que
recae sobre el sistema de salud y los familiares de aquellos afectados. 
La mejor o menor calidad 
de vida del adulto mayor depende tanto de la calidad de los servicios de salud a los que tenga
acceso, como de una valoraci\'on adecuada de su cuadro cl\'inico, para un tratamiento acorde.
Por ello, cobra importancia un diagn\'ostico temprano del deterioro cognitivo que disminuya
el risgo de su avance irreversible a demencia.


En este trabajo se retoman los datos adquirido por [citar]; en aqu\'el estudio se analizaron
posibles cambios en la estructura funcional del cerebro para adultos mayores con PDC, con 
respecto a individuos sanos; se reporta que estos cambios son manifiestos durante el sue\~no
profundo --etapa denominada sue\~no MOR o fase R-- 
a trav\'es de la actividad el\'ectrica del cerebro registrada desde el cuero 
cabelludo\footnote{Ver parte de Conceptos fisiol\'ogicos para m\'as informaci\'on}. 
En su primera etapa,
los individuos se sometieron a una bater\'ia de pruebas
neuropsicol\'ogicas para diagnosticar PDC y depresi\'on geri\'atrica\footnote{Para
m\'as detalles, ver la parte de Metodolog\'ia}, que a su vez fungieron como criterios de
inclusi\'on para una segunda fase del estudio.
En la etapa posterior, los individuos se sometieron a un estudio de la
actividad el\'ectrica cerebral durante el
sue\~no; se realizaron
registros de EEG en 22 sitios de muestreo, adicionalmente se midi\'o
actividad ocular y muscular a trav\'es de EOG y EMG --respectivamente-- con
el fin de detectar adecuadamente las etapas cl\'inicas del sue\~no\cite{AASM07}.
El registro simult\'aneo de estas se\~nales durante el sue\~no recibe el nombre de PSG.

En este trabajo se modelan matem\'aticamente los registros de PSG como series de 
tiempo\footnote{Ver la parte de Conceptos matem\'aticos para m\'as detalles}, y se
investigan algunas de propiedades estad\'isticas din\'amicas (dependientes del tiempo)
en cuanto puedan existir diferencias entre los registros obtenidos de sujetos con PSG,
con respecto a individuos sanos. 
Con respecto al anterior estudio, este trabajo busca un mejor
entendimiento --desde las matem\'aticas-- de las diferencias encontradas entre 
individuos sanos y con PSG; adem\'as, se ha buscado generar una metodolog\'ia 
conceptualmente accesible y computacionalmente r\'apida que replique los resultados encontrados.

%%%%%%%%%%%%%%%%%%%%%%%%%%%%%%%%%%%%%%%%%%%%%%%%%%%%%%%%%%%%%%%%%%%%%%%%%%%%%%%%%%%%%%%%%%%%%%%%%%%

\section{Pregunta de investigaci\'on}

Considerando los registros de PSG como series de tiempo,
¿es plausible usar sus propiedades estad\'isticas dependientes del tiempo, espec\'ificamente
la estacionariedad d\'ebil,
como marcadores para el diagn\'ostico cl\'inico del DC --en alguna de 
sus fases-- en adultos mayores?

%%%%%%%%%%%%%%%%%%%%%%%%%%%%%%%%%%%%%%%%%%%%%%%%%%%%%%%%%%%%%%%%%%%%%%%%%%%%%%%%%%%%%%%%%%%%%%%%%%%

\subsection{Hip\'otesis}

Existen diferencias cuantificables estad\'isticamente significativas entre las propiedades
estad\'isticas dependientes del tiempo, de los registros de PSG
durante etapas espec\'ificas de sue\~no, en adultos
mayores con PDC respecto a individuos control.

%%%%%%%%%%%%%%%%%%%%%%%%%%%%%%%%%%%%%%%%%%%%%%%%%%%%%%%%%%%%%%%%%%%%%%%%%%%%%%%%%%%%%%%%%%%%%%%%%%%

\subsection{Objetivo general}

%A modo de estudio de casos retrospectivo,
Deducir a partir de pruebas estad\'isticas formales las propiedades
estad\'isticas de registros de PSG en adultos mayores con PDC, as\'i como individuos control.

%%%%%%%%%%%%%%%%%%%%%%%%%%%%%%%%%%%%%%%%%%%%%%%%%%%%%%%%%%%%%%%%%%%%%%%%%%%%%%%%%%%%%%%%%%%%%%%%%%%

\subsection{Objetivos espec\'ificos}

\begin{itemize}
\item Investigar las definiciones de estacionariedad\footnote{Estacionariedad fuerte, 
estacionariedad de orden finito, estacionariedad local, cuasi-estacionariedad,
ciclo-estacionariedad, procesos estoc\'asticos erg\'odicos} y sus posibles consecuencias dentro
de un modelo para los datos considerados

\item Investigar en la literatura c\'omo detectar si es plausible que una serie de tiempo 
dada sea una realizaci\'on de un proceso d\'ebilmente estacionario, 
y bajo qu\'e supuestos es v\'alida esta caracterizaci\'on

\item Usando los an\'alisis hallados en la literatura para determinar si las series de tiempo,
obtenidas a partir de los datos considerados, provienen de procesos
debilmente estacionarios.
Revisar si la informaci\'on obtenida en los diferentes sujetos muestra diferencias entre
sujetos con y sin PDC

%\item Determinar, para el caso de estudio presente, la validez de los \textbf{supuestos de 
%estacionariedad d\'ebil}: su presencia o ausencia, en intervalos cortos o arbitrarios de tiempo

\end{itemize}

%%%%%%%%%%%%%%%%%%%%%%%%%%%%%%%%%%%%%%%%%%%%%%%%%%%%%%%%%%%%%%%%%%%%%%%%%%%%%%%%%%%%%%%%%%%%%%%%%%%