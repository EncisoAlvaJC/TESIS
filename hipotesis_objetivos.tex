%%%%%%%%%%%%%%%%%%%%%%%%%%%%%%%%%%%%%%%%%%%%%%%%%%%%%%%%%%%%%%%%%%%%%%%%%%%%%%%%%%%%%%%%%%%%%%%%%%%
\chapter{Planteamiento del problema}

%Dentro de la teor\'ia de los ciclos de vida, la \'ultima etapa se refiere a la poblaci\'on 
%envejecida, en quienes lo caracter\'istico son las p\'erdidas f\'isicas, mentales, sociales y 
%econ\'omicas asociadas a la edad avanzada, con una vuelta a la dependencia sobre el grupo 
%intermedio. 
%En la caracterizaci\'on del grupo envejecido el deterioro de la salud es 
%un elemento principal. 
%El mayor impacto social y probablemente económico del envejecimiento, se desprende de los cambios en el estado de salud que conlleva.



%Tanto el deterioro cognitivo como las alteraciones del sueño son frecuentes en las personas de 
%edad avanzada y, aunque interfieren con la calidad de vida del individuo, no suelen ser 
%consideradas como patológicas sino como problemas asociados al propio envejecimiento.
%
%En un segundo momento, se pretenden medir aspectos cerebrales en el momento del sueño por medio 
%de un encefalograma para que con los resultados obtenidos de ambas fases y se determine la 
%diferencia entre los grupos de adulto mayor en evaluación cognitiva y su encefalografía.
%

%Visualmente, el sue\~no MOR se caracteriza por movimientos oculares r\'apidos, aton\'ia muscular y 
%una actividad electroencefalogr\'afica desincronizada \cite{RosalesLagarde09}.

%%%%%%%%%%%%%%%%%%%%%%%%%%%%%%%%%%%%%%%%%%%%%%%%%%%%%%%%%%%%%%%%%%%%%%%%%%%%%%%%%%%%%%%%%%%%%%%%%%%

\section{Justificaci\'on}

El deterioro cognitivo, que se define como ''un estado de transici\'on entre el envejecimiento 
normal y la demencia'' {Garrido09} es uno de los problemas de salud m\'as importantes en los 
pa\'ises desarrollados y sub-desarrollados que afecta directamente la calidad de vida de los 
adultos mayores lo que genera una mayor demanda de servicios de salud en el pa\'is.
Por otro lado, todav\'ia son incipientes las investigaciones para identificar 
factores de riesgo modificables de la demencia.
\cite{PlanAlzheimer04}

%Dada su relación con la edad, en la última década se ha dado un continuo incremento tanto en su incidencia como en su prevalencia, secundariamente al aumento progresivo de la longevidad en la población en general.

El estado de salud de la poblaci\'on de edad avanzada en su conjunto, tiene un peso espec\'ifico 
que recae en el sistema de salud en mayor o menor grado en funci\'on de la eficiencia de \'este. 
Por ello, existe la necesidad de que en el sistema de salud, realice una valoración integral del 
adulto mayor por parte del equipo de salud, para la detecci\'on temprana del deterioro cognitivo ya 
que  
%de nuestra población adulta mayor vive con algún tipo de deterioro, y 
al no ser diagnosticado adecuadamente incrementa el riesgo de evolucionar a demencia y por 
consiguiente largos y pesados periodos de cuidado y una atenci\'on especializada para recibir el 
tratamiento m\'as apropiado al que no todos tienen acceso, por los gastos que representa en todos 
los niveles. 
%Por otro lado, existe una gran demanda de datos fiables sobre la salud mental de los adultos mayores del estado de Hidalgo, dadas las limitaciones de los estudios epidemiológicos realizados hasta la fecha y los problemas metodológicos derivados de trabajar con diferentes localidades.

%Dada la relación entre el deterioro cognitivo y la calidad de sueño en adultos mayores, parece necesario reflexionar sobre las prevalencias del deterioro cognitivo, el rumbo que han tomado y su evaluación psicológica, las funciones mentales evaluadas, las metodologías y los resultados en México con énfasis particular en su relación con salud mental, así como considerar alternativas para valorar la función mental a nivel estatal y nacional con el apoyo de métodos de análisis biológicos. 

%%%%%%%%%%%%%%%%%%%%%%%%%%%%%%%%%%%%%%%%%%%%%%%%%%%%%%%%%%%%%%%%%%%%%%%%%%%%%%%%%%%%%%%%%%%%%%%%%%%

\section{Pregunta de investigaci\'on}

Considerando los registros de PSG como series de tiempo,
¿es plausible usar sus propiedades estad\'isticas dependientes del tiempo, espec\'ificamente
la estacionariedad d\'ebil,
como marcadores para el diagn\'ostico cl\'inico del DC --en alguna de 
sus fases-- en adultos mayores?

%%%%%%%%%%%%%%%%%%%%%%%%%%%%%%%%%%%%%%%%%%%%%%%%%%%%%%%%%%%%%%%%%%%%%%%%%%%%%%%%%%%%%%%%%%%%%%%%%%%

\section{Hip\'otesis}

Existen diferencias cuantificables estad\'isticamente significativas entre las propiedades
estad\'isticas dependientes del tiempo, de los registros de PSG
durante etapas espec\'ificas de sue\~no, en adultos
mayores con DC respecto a individuos control.

%%%%%%%%%%%%%%%%%%%%%%%%%%%%%%%%%%%%%%%%%%%%%%%%%%%%%%%%%%%%%%%%%%%%%%%%%%%%%%%%%%%%%%%%%%%%%%%%%%%

\section{Objetivo general}

A modo de estudio de casos retrospectivo,
deducir a partir de pruebas estad\'isticas formales (en el sentido matem\'atico) las propiedades
estad\'isticas de registros de PSG en adultos mayores con DC.

%%%%%%%%%%%%%%%%%%%%%%%%%%%%%%%%%%%%%%%%%%%%%%%%%%%%%%%%%%%%%%%%%%%%%%%%%%%%%%%%%%%%%%%%%%%%%%%%%%%

\subsection{Objetivos espec\'ificos}

\begin{itemize}
\item Investigar las definiciones de estacionariedad y sus consecuencias te\'oricas

\item Investigar en la literatura c\'omo detectar si es plausible que una serie de tiempo 
dada sea una realizaci\'on de un proceso d\'ebilmente estacionario, 
y bajo qu\'e supuestos es v\'alida esta caracterizaci\'on

\item Usando los an\'alisis hallados en la literatura para determinar si las series de tiempo,
obtenidas a partir de los registros PSG considerados, provienen de procesos
debilmente estacionarios.
Revisar si la informaci\'on obtenida en los diferentes sujetos muestra diferencias entre
sujetos con y sin DC

\item Determinar, para el caso de estudio presente, la validez de los \textbf{supuestos de 
estacionariedad d\'ebil}: su presencia o ausencia, en intervalos cortos o arbitrarios de tiempo
\end{itemize}

%%%%%%%%%%%%%%%%%%%%%%%%%%%%%%%%%%%%%%%%%%%%%%%%%%%%%%%%%%%%%%%%%%%%%%%%%%%%%%%%%%%%%%%%%%%%%%%%%%%