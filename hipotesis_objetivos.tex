%%%%%%%%%%%%%%%%%%%%%%%%%%%%%%%%%%%%%%%%%%%%%%%%%%%%%%%%%%%%%%%%%%%%%%%%%%%%%%%%%%%%%%%%%%%%%%%%%%%
%\chapter{Planteamiento del problema}

La idea que una serie de tiempo puede no ser estacionaria, ni a\'un en un sentido d\'ebil, 
se puede rastrear en el tiempo a los a\~nos 50’s \cite{Page52,Silverman57}. 
Sin embargo, estas interrogantes en el contexto de series electrofisiol\'ogicas --en particular 
EEG-- no se ven claramente reflejados sino hasta los años 70’s 
\cite{Kawabata73,McEwen75,Cohen77,Sugimoto78}.
Esta brecha temporal se debe, quiz\'a, a la aparici\'on de computadoras digitales de bajo costo, 
gracias a las cuales es posible analizar mayores volúmenes de datos; m\'as a\'un, la escasa 
capacidad de c\'omputo promovi\'o la hip\'otesis de que las series de tiempo 
''cortas'' son estacionarias --al menos d\'ebilmente--, un hecho que ha sido 
rebatido \cite{Melard89,Adak98,Klonowski09}.

%Recientemente rige un dogma según el cual las series biológicas son –casi definición– complejas: no lineales, no estacionarias y sin equilibrio por naturaleza. Una vez se ha aceptado que las señales biológicas son naturalmente complejas, este trabajo retoma una pregunta en voga: ¿Qué implicaciones tiene haber captado una señal biológica que no es compleja? En particular, se sospecha que 

%%%%%%%%%%%%%%%%%%%%%%%%%%%%%%%%%%

\begin{quotation}
Usualmente se asume que las series fisiol\'ogicas son complejas:
no-estacionarias, no-lineales y no en equilibrio por naturaleza. Sin embargo, estas propiedades
no se suelen probar formalmente.
\end{quotation}

%Debido a que mi trabajo pretende tomar una posici\'on opesta en alg\'un momento, en alg\'un grado,
%ser\'ia incompatible que yo \textbf{simplemente suponga} que esa posici\'on es verdadera.
%[Me he dado a la tarea de investigar un poco sobre el papel hist\'rico que han jugado las
%hip\'otesis de regularidad en las series electrofisiol\'ogicas.]

%Esta secci\'on debiera partir de los comentarios expresados 
%en 'Everything you wanted to ask about EEG (..)'
%(Klonowski, 2009) sobre c\'omo el concepto de ondas se acu\~na en el estudio de EEG, especilmente
%de c\'omo se entienden las frecuencias en este contexto --visi\'on que es reforzada al citar
%el manual de la IAAC 2007 para detectar las etaas de sue\~no.
%
%En esta visi\'on, cabe destacar los muchos trabajos de Harmony y de Corsi-Cabrera
%sobre la caracterizaci\'on y localizaci\'on de diferentes tipos de actividad cerebral 
%durante diversas actividades y condiciones Adem\'as de otros autores. 
%Sinceramente, son bastantes trabajos y
%son la gu\'ia sobre los an\'alisis de composici\'on espectral que a\'un est\'an por hacerse.
%
%Peor tambi\'en hay una discusi\'on sobre el balance entre los estudios espectrales contra
%los avances en teor\'ia espectral: se puede citar a Cohen, quien asume que las series
%cortas son b\'asicamente estacionarias. Por supuesto que cada punto en el tiempo es
%t\'ecnicamente estacionario, y es completamente plausible --en el contexto de las series
%electrofisiol\'ogicas-- suponer que para cada punto existe un abierto
%en el tiempo tal que el subproceso definido all\'i es etsacionario para todo fin pr\'actico.
%Como comenta Melard, la suposici\'on de estacionariedad para series cortas se 
%consider\'o v\'alida por mucho tiempo debido a ala escasa capacidad de c\'omputo; a
%modo de sintesis, Adak muestra un resultado negativo sobre la suposicion de estacionariedad
%local pero muestra una prueba para detectar y medir la as\'i llamada 'estacionariedad local'.
%[escribir\'e tal demostraci\'on]
%
%En este punto, es conveniente hablar sobre los modelos ARMA como la forma mas natural de
%estacionariedad a tiempo discreto, y como se usa en los modelos de estacionaeriedad 
%local (citar a Adak). Se han posido generalizar estos modelos a parametros que
%dependen del tiempo como los modelos ARCH (quia citar a Chatfield y a Subba Rao).
%
%Hay una historia extensa al establecer el concepto de espectro en series no-estacionarias,
%y para ello me servire de las revisiones de Loynes, Melard, Adak, Brillinger. En ella,
%brillan la funcion de autocorrelacion que depende del tiempo y el que su transformada
%de Fourier sea la funci\'on de densidad espectral en caso de existir. Muchas
%definiciones de espectro basadas en su forma de ser calculadas.
%
%Me gustar\'ia escribir un segundo resumen sobre el espectro de Wold-Cram\'er (el que se maneja
%en el test PSR) en contraposici\'on al espectro de Wigner-Ville, el espectro de
%ondeletas de Gabor y el espectro de ondeletas de Haar. Me apoyaria mucho de una discusion
%hecha por Nason, de la cuakl resaltare una estimacion sobre los ordenes de tiempo de
%computo para los estimadores de estos espectros.

%----------------------------------------
%
%Esta secci\'on debiera terminar nuevamente citando a Klonowski y, 
%si bien no voy a usar en esta tesis, los
%nuevos enfoques que consideran al EEG fundamentalmente como un sistema sujeto a ruido pero
%fundamentalmente ca\'otico.

%%%%%%%%%%%%%%%%%%%%%%%%%%%%%%%%%%%%%%%%%%%%%%%%%%%%%%%%%%%%%%%%%%%%%%%%%%%%%%%%%%%%%%%%%%%%%%%%%%%

%\section{EEG estacionario}

En los ochentas, antes de que las computadoras personales fueran usadas en la medicina, las
de\~nales de EEG eran registradas en una tira de papel. El registro se llevaba a cabo
en l\'ineas separadas entre s\'i por 3 cm aproximadamente, mientras que el papel se mov\'ia a 
raz\'on de 1.5, 3 o 6 cm/s seg\'un el aparato. Un m\'edico dedicado a interpetar el EEG pod\'ia
observar f\'acilmente la frecuancia de las ondas al contar el n\'umero de espigas dibujadas
en un segundo, si hab\'ia entre 2 y 30 de ellas entre dos l\'ineas verticales. De estos registros
en papel vienen los nombres cl\'asicos de las bandas, en especial las $\alpha$ y $\beta$; si las
ondas ten\'ian una frecuencia muy baja no podr\'ian distinguirse por el ojo, 
mientras que si su frecuencia era muy alta el aparato las registrar\'ia como un bloque 
indistinguible\cite{Klonowski09}.

%%%%%%%%%%%%%%%%%%%%%%%%%%%%%%%%%%%%%%%%%%%%%%%%%%%%%%%%%%%%%%%%%%%%%%%%%%%%%%%%%%%%%%%%%%%%%%%%%%%

%%%%%%%%%%%%%%%%%%%%%%%%%%%%%%%%%%%%%%%%%%%%%%%%%%%%%%%%%%%%%%%%%%%%%%%%%%%%%%%%%%%%%%%%%%%%%%%%%%%

\section{Justificaci\'on}

El MCI se define como ''un s\'indrome caracterizado por una alteraci\'on adquirida y prolongada de
de una o varias funciones cognitivas, que no corresponde a un s\'indrome focal y no cumple
criterios suficientes de gravedad para ser calificada como demencia'' \cite{Robles02}.
De acuerdo a la Encuesta Nacional de Salud y Nutrici\'on (ENSANUT) efectuada en M\'exico 2002,
se estima que existen 800,000 adultos mayores \cite{Sosa12}.

%El deterioro cognitivo, se define como ''un estado de transici\'on entre el envejecimiento 
%normal y la demencia'' {Garrido09} es uno de los problemas de salud m\'as importantes en los 
%pa\'ises desarrollados y sub-desarrollados que afecta directamente la calidad de vida de los 
%adultos mayores lo que genera una mayor demanda de servicios de salud en el pa\'is.
%Por otro lado, todav\'ia son incipientes las investigaciones para identificar 
%factores de riesgo modificables de la demencia.
%\cite{PlanAlzheimer04}

El cuidado de enfermedades cr\'onicas 
en la poblaci\'on de edad avanzada representa un gran peso
econ\'onomico y de recursos humanos, que
recae sobre el sistema de salud y los familiares de aquellos afectados. 
La mejor o menor calidad 
de vida del adulto mayor depende tanto de la calidad de los servicios de salud a los que tenga
acceso, como de una valoraci\'on adecuada de su cuadro cl\'inico, para un tratamiento acorde.
Por ello, cobra importancia un diagn\'ostico temprano del deterioro cognitivo que disminuya
el risgo de su avance irreversible a demencia.


En este trabajo se retoman los datos adquirido por [citar]; en aqu\'el estudio se analizaron
posibles cambios en la estructura funcional del cerebro para adultos mayores con PDC, con 
respecto a individuos sanos; se reporta que estos cambios son manifiestos durante el sue\~no
profundo --etapa denominada sue\~no MOR o fase R-- 
a trav\'es de la actividad el\'ectrica del cerebro registrada desde el cuero 
cabelludo\footnote{Ver parte de Conceptos fisiol\'ogicos para m\'as informaci\'on}. 
En su primera etapa,
los individuos se sometieron a una bater\'ia de pruebas
neuropsicol\'ogicas para diagnosticar PDC y depresi\'on geri\'atrica\footnote{Para
m\'as detalles, ver la parte de Metodolog\'ia}, que a su vez fungieron como criterios de
inclusi\'on para una segunda fase del estudio.
En la etapa posterior, los individuos se sometieron a un estudio de la
actividad el\'ectrica cerebral durante el
sue\~no; se realizaron
registros de EEG en 22 sitios de muestreo, adicionalmente se midi\'o
actividad ocular y muscular a trav\'es de EOG y EMG --respectivamente-- con
el fin de detectar adecuadamente las etapas cl\'inicas del sue\~no\cite{AASM07}.
El registro simult\'aneo de estas se\~nales durante el sue\~no recibe el nombre de PSG.

En este trabajo se modelan matem\'aticamente los registros de PSG como series de 
tiempo\footnote{Ver la parte de Conceptos matem\'aticos para m\'as detalles}, y se
investigan algunas de propiedades estad\'isticas din\'amicas (dependientes del tiempo)
en cuanto puedan existir diferencias entre los registros obtenidos de sujetos con PSG,
con respecto a individuos sanos. 
Con respecto al anterior estudio, este trabajo busca un mejor
entendimiento --desde las matem\'aticas-- de las diferencias encontradas entre 
individuos sanos y con PSG; adem\'as, se ha buscado generar una metodolog\'ia 
conceptualmente accesible y computacionalmente r\'apida que replique los resultados encontrados.

%%%%%%%%%%%%%%%%%%%%%%%%%%%%%%%%%%%%%%%%%%%%%%%%%%%%%%%%%%%%%%%%%%%%%%%%%%%%%%%%%%%%%%%%%%%%%%%%%%%

\section{Pregunta de investigaci\'on}

Considerando los registros de PSG como series de tiempo,
¿es plausible usar sus propiedades estad\'isticas dependientes del tiempo, espec\'ificamente
la estacionariedad d\'ebil,
como marcadores para el diagn\'ostico cl\'inico del DC --en alguna de 
sus fases-- en adultos mayores?

%%%%%%%%%%%%%%%%%%%%%%%%%%%%%%%%%%%%%%%%%%%%%%%%%%%%%%%%%%%%%%%%%%%%%%%%%%%%%%%%%%%%%%%%%%%%%%%%%%%

\subsection{Hip\'otesis}

Existen diferencias cuantificables estad\'isticamente significativas entre las propiedades
estad\'isticas dependientes del tiempo, de los registros de PSG
durante etapas espec\'ificas de sue\~no, en adultos
mayores con PDC respecto a individuos control.

%%%%%%%%%%%%%%%%%%%%%%%%%%%%%%%%%%%%%%%%%%%%%%%%%%%%%%%%%%%%%%%%%%%%%%%%%%%%%%%%%%%%%%%%%%%%%%%%%%%

\subsection{Objetivo general}

%A modo de estudio de casos retrospectivo,
Deducir a partir de pruebas estad\'isticas formales las propiedades
estad\'isticas de registros de PSG en adultos mayores con PDC, as\'i como individuos control.

%%%%%%%%%%%%%%%%%%%%%%%%%%%%%%%%%%%%%%%%%%%%%%%%%%%%%%%%%%%%%%%%%%%%%%%%%%%%%%%%%%%%%%%%%%%%%%%%%%%

\subsection{Objetivos espec\'ificos}

\begin{itemize}
\item Investigar las definiciones de estacionariedad\footnote{Estacionariedad fuerte, 
estacionariedad de orden finito, estacionariedad local, cuasi-estacionariedad,
ciclo-estacionariedad, procesos estoc\'asticos erg\'odicos} y sus posibles consecuencias dentro
de un modelo para los datos considerados

\item Investigar en la literatura c\'omo detectar si es plausible que una serie de tiempo 
dada sea una realizaci\'on de un proceso d\'ebilmente estacionario, 
y bajo qu\'e supuestos es v\'alida esta caracterizaci\'on

\item Usando los an\'alisis hallados en la literatura para determinar si las series de tiempo,
obtenidas a partir de los datos considerados, provienen de procesos
debilmente estacionarios.
Revisar si la informaci\'on obtenida en los diferentes sujetos muestra diferencias entre
sujetos con y sin PDC

%\item Determinar, para el caso de estudio presente, la validez de los \textbf{supuestos de 
%estacionariedad d\'ebil}: su presencia o ausencia, en intervalos cortos o arbitrarios de tiempo

\end{itemize}

%%%%%%%%%%%%%%%%%%%%%%%%%%%%%%%%%%%%%%%%%%%%%%%%%%%%%%%%%%%%%%%%%%%%%%%%%%%%%%%%%%%%%%%%%%%%%%%%%%%