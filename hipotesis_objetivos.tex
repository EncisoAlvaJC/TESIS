%%%%%%%%%%%%%%%%%%%%%%%%%%%%%%%%%%%%%%%%%%%%%%%%%%%%%%%%%%%%%%%%%%%%%%%%%%%%%%%%%%%%%%%%%%%%%%%%%%%
\chapter{Antecedentes}

%\chapter{Planteamiento del problema}


%El registro sistem\'atico de la actividad el\'ectrica del cerebro se considera 
%iniciado por Hans Berger \cite{Berger29}, y desde entonces la electrofisiolog\'ia 
%ha logrado cuantiosos avances en cuanto al entendimiento del sistema nervioso.
%tiene una larga 
%tradici\'on hist\'orica.
%La velocidad de los descubrimientos, por otro lado, no ha sido del todo uniforme a trav\'es de 
%la historia.




%Recientemente rige un dogma según el cual las series biológicas son –casi definición– complejas: no lineales, no estacionarias y sin equilibrio por naturaleza. Una vez se ha aceptado que las señales biológicas son naturalmente complejas, este trabajo retoma una pregunta en voga: ¿Qué implicaciones tiene haber captado una señal biológica que no es compleja? En particular, se sospecha que 

%%%%%%%%%%%%%%%%%%%%%%%%%%%%%%%%%%

%%\begin{quotation}
%%Usualmente se asume que las series fisiol\'ogicas son complejas:
%%no-estacionarias, no-lineales y no en equilibrio por naturaleza. Sin embargo, estas propiedades
%%no se suelen probar formalmente.
%%\end{quotation}


El registro sistem\'atico de la actividad el\'ectrica del cerebro se considera 
iniciado por Hans Berger \cite{Berger29}, quien a su vez acu\~n\'o el t\'ermino 
electroencefalograma (EEG); desde entonces la electrofisiolog\'ia 
ha logrado cuantiosos avances en cuanto al entendimiento del sistema nervioso.
%Durante esta introducci\'on se destaca el papel de la modelaci\'on matem\'atica, casi omnipresente
%en cualquier estudio cuantitativo,
%dentro del avance el avance de las ciencias neurofisiol\'ogicas

Dentro de la modelaci\'on matem\'atica, las se\~nales electrofisiol\'ogicas 
usualmente son entendidas como
%son entendidas
como series de tiempo o como sistemas din\'amicos, cual si fueran polos opuestos: 
conjuntos de datos esparcidos en el tiempo que pueden (o no) ser en parte aleatorios, que 
bajo condiciones iniciales muy parecidas tienden a estados similares (o completamente diferentes).
En el presente trabajo se modelan 
%parte de modelar 
los registros de EEG durante el sue\~no (PSG)
% en adultos mayores
como generados por un proceso estoc\'astico, y se intenta verificar si
\'estos pueden modelarse como
procesos estoc\'asticos \textbf{d\'ebilmente estacionarios}
% pueden tienen la propiedad de estacionariedad.
--un supuesto b\'asico en series de tiempo, y que 
usualmente se acepta o rechaza como propiedades int\'insecas del fen\'omeno.
%\footnote{Eventualmente
%s\'i se revisar\'an todas los supuestos involucrados}.
La idea que una se\~nal puede no ser regular --en particular, 
estacionaria--
%, ni a\'un en un sentido d\'ebil, 
se puede rastrear en el tiempo a los a\~nos 50’s \cite{Page52,Silverman57}. 
Sin embargo, estas interrogantes sobre la regularidad de se\~nales electrofisiol\'ogicas
no se presentan claramente sino hasta los a\~nos 70's
%Sin embargo, estas interrogantes en el contexto de se\~nales electrofisiol\'ogicas --en particular 
%EEG-- no se ven claramente reflejados sino hasta los años 70’s 
\cite{Kawabata73,McEwen75,Cohen77,Sugimoto78}.
Esta brecha temporal se debe, quiz\'a, a la aparici\'on de computadoras digitales de bajo costo, 
gracias a las cuales es posible analizar mayores vol\'umenes de datos.


Se cuenta en an\'ecdotas que, antes de la era digital, las
se\~nales de EEG eran registradas en una tira de papel; el registro se llevaba a cabo
sobre una tira de papel que se mov\'ia, a trav\'es de agujas similares a las de un sism\'grafo.
%en l\'ineas separadas entre s\'i por 3 cm aproximadamente, mientras que el papel se mov\'ia a 
%raz\'on de 1.5, 3 o 6 cm/s seg\'un el aparato. 
El encargado de interpetar el EEG med\'ia
la frecuancia de las ondas al contar el n\'umero de espigas dibujadas
en un segundo, 
%De estos registros
%en papel vienen los nombres cl\'asicos de las bandas, en especial las $\alpha$ y $\beta$; si las
%ondas ten\'ian una frecuencia muy baja no podr\'ian distinguirse por el ojo, 
%mientras que si su frecuencia era muy alta el aparato las registrar\'ia como un bloque 
%indistinguible\cite{Klonowski09}.
motivando la caracterizaci\'on de las ondas cerebrales.
%En la era digital, 
%esta forma particular de entender las ondas y las frecuencias se extiende en la electrofisiolog\'ia
%gracias a la transformada r\'apida de Fourier (FFT) 
\cite{Klonowski09}.
Sin embargo, un peligro para estos estudios es el sustento te\'orico que justifica las
t\'ecnicas utilizadas. Conviene citar, por ejemplo, que
%M\'as a\'un, 
la escasa capacidad de c\'omputo 
en los ochentas
%previa a estos estudios 
promovi\'o, indirectamente,
la hip\'otesis de que {toda} serie de tiempo suficientemente corta pod\'ia entenderse
como generada por un proceso estoc\'astico --al menos-- d\'ebilmente estacionario; 
%m\'as a\'un
e incluso
%en estudios de esos a\~nos
pueden encontrarse
cotas 
tentativas
%informales 
sobre la duraci\'on m\'axima de estacionariedad garantizada. 
Actualmente, la estacionariedad d\'ebil ''garantizada'' para series cortas se considera 
rebatida\cite{Melard89,Adak98,Klonowski09}.
%aunque este efecto
%se considera rebatido 
Hoy en d\'ia se considera que las series de tiempo de origen biol\'ogico son, en general,
no-estacionarias, no-lineales y no en equilibrio por naturaleza.


La motivaci\'on principal para este trabajo proviene de un trabajo previo [de Valeria],
en el cual se describen diferencias significaivas entre registros PSG de
adultos mayores con y sin PDC; las diferencias observadas son en cuanto al exponente de Hurst 
estimado, $H_\alpha$.
La cantidad $H_\alpha$, a veces referida como el ''color'' de la se\~nal, 
mide la fractalidad\footnote{Este concepto no se
describir\'a en este trabajo, para m\'as informaci\'on ver el trabajo de Valeria} 
de un proceso estoc\'astico, y es una caracter\'istica generalizada desde sistemas ca\'oticos.
En aqu\'el trabajo, se reporta que
para registros de PSG en adultos mayores con PSG la cantidad $H_\alpha$ es menos, m\'as
cercana a la misma cantidad calculada para el proceso de Wiener.
Luego entonces, cabe preguntarse m\'as sobre la naturaleza de las diferencias detectadas en 
el trabajo de Valeria: los registros de PSG en adultos mayores con PDC ¿son ''menos complejos'' o 
s\'olo ''tienen otro color''?
De manera concreta, en este trabajo se ha hipotetizado sobre la primera opci\'on. 

Formalmente se inicia
este trabajo bajo el supuesto de que durante el PDC la actividad cerebral
del sujeto tiene caracter\'isticas diferentes; 
dada la evidencia del estudio referido, se plantea la hip\'otesis de que este cambio puede ser
una ''menor complejidad'' en cuanto a sus propiedades estad\'isticas 
dependientes del tiempo, y que quiz\'a el proceso es m\'as parecido a un proceso ruido blanco.
La forma que se ha elegido para comprobar esta hip\'otesis es midiendo la ''presencia'' de
estacionariedad d\'ebil en estos registros --como ya se hab\'ia mencionado.

\subsubsection{Sobre la brevedad de este trabajo}

Debido a su naturaleza de reporte, en este documento no es extensivo en cuanto a teor\'ia sino que
describe los conceptos clave para entender e interpretar los resultados obtenidos.
Este enfoque tiene su origen en la intenci\'on de que pueda 
ser accesible a un p\'ublico ''diverso'' (no s\'olo de fisiolog\'ia, no s\'olo de matem\'aticas)
y que haga su peque\~no aporte en una larga tradici\'on colaborativa entre estas \'areas.
Se presenta una gran variedad de teoremas, siendo citadas publicaciones que contienen
sus respectivas demostraciones. As\'imismo, conviene advertir que se expone material respecto a
la ''representaci\'on espectral'' de procesos estoc\'asticos d\'ebilmente estacionarios y
estoc\'asticamente continuos, pero inmediatamente despu\'es 
se exhiben estimadores espectrales para procesos 
que admiten un ''espectro evolutivo'' en el sentido de Priestley; se han omitido una gran cantidad
de detalles en aras de obtener brevedad y mostrar s\'olo los detalles b\'asicos
para interpretar los resultados obtenidos. 
El lector interesado puede ser dirigido al libro ''Spectral Analysis and Time Series'' de
M. Priestley \cite{Priestley81}, el cual resume en gran parte la teor\'ia espectral ''cl\'asica''.


%%\subsubsection{Antecedentes hist\'oricos}
%%
%%El registro sistem\'atico de la actividad el\'ectrica del cerebro se considera 
%%iniciado por Hans Berger \cite{Berger29}, y entonces la electrofisiolog\'ia tiene una larga 
%%tradici\'on hist\'orica.
%%Por otro lado, la idea que una se\~nal ''ruido'' puede no ser regular --en particular, 
%%estacionaria--
%%%, ni a\'un en un sentido d\'ebil, 
%%se puede rastrear en el tiempo a los a\~nos 50’s \cite{Page52,Silverman57}. 
%%Sin embargo, las interrogantes sobre la regularidad de las se\~nales electrofisiol\'ogicas
%%no aparecen claramente sino hasta los a\~nos 70's
%%%Sin embargo, estas interrogantes en el contexto de se\~nales electrofisiol\'ogicas --en particular 
%%%EEG-- no se ven claramente reflejados sino hasta los años 70’s 
%%\cite{Kawabata73,McEwen75,Cohen77,Sugimoto78}.
%%Esta brecha temporal se debe, quiz\'a, a la aparici\'on de computadoras digitales de bajo costo, 
%%gracias a las cuales es posible analizar mayores vol\'umenes de datos.
%%%; m\'as a\'un, la escasa 
%%%capacidad de c\'omputo promovi\'o la hip\'otesis de que las series de tiempo 
%%%''cortas'' son estacionarias --al menos d\'ebilmente--, un hecho que ha sido 
%%%rebatido \cite{Melard89,Adak98,Klonowski09}.
%%M\'as a\'un, la escasa capacidad de c\'omputo previa a estos estudios promovi\'o indirectamente
%%la hip\'otesis de que \textbf{toda} serie de tiempo suficientemente corta pod\'ia entenderse
%%como generada por un proceso estoc\'astico al menos debilmente estacionario; 
%%en estudios de esos a\~nos
%%pueden encontrarse
%%cotas informales sobre la duraci\'on m\'axima de estacionariedad garantizada, aunque este efecto
%%se considera rebatido \cite{Melard89,Adak98,Klonowski09}.

%Recientemente rige un dogma según el cual las series biológicas son –casi definición– complejas: no lineales, no estacionarias y sin equilibrio por naturaleza. Una vez se ha aceptado que las señales biológicas son naturalmente complejas, este trabajo retoma una pregunta en voga: ¿Qué implicaciones tiene haber captado una señal biológica que no es compleja? En particular, se sospecha que 

%%%%%%%%%%%%%%%%%%%%%%%%%%%%%%%%%%

%%\begin{quotation}
%%Usualmente se asume que las series fisiol\'ogicas son complejas:
%%no-estacionarias, no-lineales y no en equilibrio por naturaleza. Sin embargo, estas propiedades
%%no se suelen probar formalmente.
%%\end{quotation}

%%%En los ochentas, antes de que las computadoras personales fueran usadas en la medicina, las
%%%de\~nales de EEG eran registradas en una tira de papel. El registro se llevaba a cabo
%%%en l\'ineas separadas entre s\'i por 3 cm aproximadamente, mientras que el papel se mov\'ia a 
%%%raz\'on de 1.5, 3 o 6 cm/s seg\'un el aparato. Un m\'edico dedicado a interpetar el EEG pod\'ia
%%%observar f\'acilmente la frecuancia de las ondas al contar el n\'umero de espigas dibujadas
%%%en un segundo, si hab\'ia entre 2 y 30 de ellas entre dos l\'ineas verticales. De estos registros
%%%en papel vienen los nombres cl\'asicos de las bandas, en especial las $\alpha$ y $\beta$; si las
%%%ondas ten\'ian una frecuencia muy baja no podr\'ian distinguirse por el ojo, 
%%%mientras que si su frecuencia era muy alta el aparato las registrar\'ia como un bloque 
%%%indistinguible\cite{Klonowski09}.

%Esta secci\'on debiera partir de los comentarios expresados 
%en 'Everything you wanted to ask about EEG (..)'
%(Klonowski, 2009) sobre c\'omo el concepto de ondas se acu\~na en el estudio de EEG, especilmente
%de c\'omo se entienden las frecuencias en este contexto --visi\'on que es reforzada al citar
%el manual de la IAAC 2007 para detectar las etaas de sue\~no.
%
%En esta visi\'on, cabe destacar los muchos trabajos de Harmony y de Corsi-Cabrera
%sobre la caracterizaci\'on y localizaci\'on de diferentes tipos de actividad cerebral 
%durante diversas actividades y condiciones Adem\'as de otros autores. 
%Sinceramente, son bastantes trabajos y
%son la gu\'ia sobre los an\'alisis de composici\'on espectral que a\'un est\'an por hacerse.
%
%Peor tambi\'en hay una discusi\'on sobre el balance entre los estudios espectrales contra
%los avances en teor\'ia espectral: se puede citar a Cohen, quien asume que las series
%cortas son b\'asicamente estacionarias. Por supuesto que cada punto en el tiempo es
%t\'ecnicamente estacionario, y es completamente plausible --en el contexto de las series
%electrofisiol\'ogicas-- suponer que para cada punto existe un abierto
%en el tiempo tal que el subproceso definido all\'i es etsacionario para todo fin pr\'actico.
%Como comenta Melard, la suposici\'on de estacionariedad para series cortas se 
%consider\'o v\'alida por mucho tiempo debido a ala escasa capacidad de c\'omputo; a
%modo de sintesis, Adak muestra un resultado negativo sobre la suposicion de estacionariedad
%local pero muestra una prueba para detectar y medir la as\'i llamada 'estacionariedad local'.
%[escribir\'e tal demostraci\'on]
%
%En este punto, es conveniente hablar sobre los modelos ARMA como la forma mas natural de
%estacionariedad a tiempo discreto, y como se usa en los modelos de estacionaeriedad 
%local (citar a Adak). Se han posido generalizar estos modelos a parametros que
%dependen del tiempo como los modelos ARCH (quia citar a Chatfield y a Subba Rao).
%
%Hay una historia extensa al establecer el concepto de espectro en series no-estacionarias,
%y para ello me servire de las revisiones de Loynes, Melard, Adak, Brillinger. En ella,
%brillan la funcion de autocorrelacion que depende del tiempo y el que su transformada
%de Fourier sea la funci\'on de densidad espectral en caso de existir. Muchas
%definiciones de espectro basadas en su forma de ser calculadas.
%
%Me gustar\'ia escribir un segundo resumen sobre el espectro de Wold-Cram\'er (el que se maneja
%en el test PSR) en contraposici\'on al espectro de Wigner-Ville, el espectro de
%ondeletas de Gabor y el espectro de ondeletas de Haar. Me apoyaria mucho de una discusion
%hecha por Nason, de la cuakl resaltare una estimacion sobre los ordenes de tiempo de
%computo para los estimadores de estos espectros.


%%%%%%%%%%%%%%%%%%%%%%%%%%%%%%%%%%%%%%%%%%%%%%%%%%%%%%%%%%%%%%%%%%%%%%%%%%%%%%%%%%%%%%%%%%%%%%%%%%%
%%%%%%%%%%%%%%%%%%%%%%%%%%%%%%%%%%%%%%%%%%%%%%%%%%%%%%%%%%%%%%%%%%%%%%%%%%%%%%%%%%%%%%%%%%%%%%%%%%%

\section{Justificaci\'on}

El MCI se define como ''un s\'indrome caracterizado por una alteraci\'on adquirida y prolongada de
de una o varias funciones cognitivas, que no corresponde a un s\'indrome focal y no cumple
criterios suficientes de gravedad para ser calificada como demencia'' \cite{Robles02}.
De acuerdo a la Encuesta Nacional de Salud y Nutrici\'on (ENSANUT) efectuada en M\'exico 2002,
se estima que existen 800,000 adultos mayores \cite{Sosa12}.

%El deterioro cognitivo, se define como ''un estado de transici\'on entre el envejecimiento 
%normal y la demencia'' {Garrido09} es uno de los problemas de salud m\'as importantes en los 
%pa\'ises desarrollados y sub-desarrollados que afecta directamente la calidad de vida de los 
%adultos mayores lo que genera una mayor demanda de servicios de salud en el pa\'is.
%Por otro lado, todav\'ia son incipientes las investigaciones para identificar 
%factores de riesgo modificables de la demencia.
%\cite{PlanAlzheimer04}

El cuidado de enfermedades cr\'onicas 
en la poblaci\'on de edad avanzada representa un gran peso
econ\'onomico y de recursos humanos, que
recae sobre el sistema de salud y los familiares de aquellos afectados. 
La mejor o menor calidad 
de vida del adulto mayor depende tanto de la calidad de los servicios de salud a los que tenga
acceso, como de una valoraci\'on adecuada de su cuadro cl\'inico, para un tratamiento acorde.
Por ello, cobra importancia un diagn\'ostico temprano del deterioro cognitivo que disminuya
el risgo de su avance irreversible a demencia.


En este trabajo se retoman %los datos adquiridos por [citar]; 
la l\'inea de investigaci\'on trazada por \cite{VazquezTagle16}.
En aqu\'el estudio se analizaron
posibles cambios en la estructura funcional\footnote{Se suele hablar de 
\textit{conectividad funcional} cuando hay una ''buena'' relaci\'on entre la informaci\'on
que se maneja en las partes involucradas; este t\'ermino se contrapone al de
\textit{conectividad anat\'omica}, que se refiere a conexiones f\'isicas}
del cerebro para adultos mayores con PDC, con 
respecto a individuos sanos; se report\'o que estos cambios son manifiestos durante el sue\~no
profundo --etapa denominada sue\~no MOR o fase R-- 
a trav\'es de la actividad el\'ectrica del cerebro registrada desde el cuero 
cabelludo\footnote{Ver parte de Conceptos fisiol\'ogicos para m\'as informaci\'on}. 

Se hubo estudiado espec\'ificamente la etapa MOR del sue\~no en adultos mayores
%particularmente el sue\~no MOR en adultos mayores 
con el fin de discriminar
cambios permanentes en la funci\'on cerebral, con respecto a cambios transitorios asociados
a actividades espec\'ificas.
A grosso modo, en la literatura se ha reportado cambios en la conectividad anat\'omica
ante da\~nos en tejido nervioso y que tienen como resultado la recuperaci\'on parcial
de la conectividad funcional global; si estos cambios ocurren en el cerebro, se espera que las
funciones cognitivas no reflejen fielmente la gravedad del deterioro/da\~no en el tejido cerebral
--el deterioro cognitivo ser\'ia in\'util para diagnosticar el deterioro cerebral.
As\'i pues, se ha buscado aminorar el efecto de la capacidad cognitiva compensada
registrando una fase de sue\~no en la que no hay presente actividad cerebral 
dirigida\footnote{Ver [conceptos fisiol\'ogicos] para myor informaci\'on}.

%%%%%%%%%%%%%%%%%%%%%%%%%%%%%%%%%%%%%%%%%%%%%%%%%%%%%%%%%%%%%%%%%%%%%%%%%%%%%%%%%%%%%%%%%%%%%%%%%%%

%Por otro lado, hallazgos recientes han puesto de manifiesto la posible relación entre las alteraciones del sueño y el deterioro cognitivo asociado al envejecimiento 77–79. Concretamente, una duración más corta del sueño nocturno y una mala eficiencia de sueño en personas mayores se relaciona con una peor ejecución en tareas de memoria, sugiriendo que la pérdida de calidad de sueño durante el envejecimiento podría incidir negativamente sobre los mecanismos cerebrales que subyacen a la memoria 72. Por lo tanto, el insomnio en personas mayores podría ser más problemático que en otros grupos de edad, dado que cursa con deterioro cognitivo y empeora con la edad (79.
%Un estudio cuantitativo realizado en 2015 por Brayet, Petit, Frauscher, Gagnon, Gosselin, Gagnon, Rouleau y Montplaisir80,tenía por objetivo comparar grupos de superioridad del sueño MOR por medio de una electroencefalografía como una herramienta de detección para enfermedad de Alzheimer preclínica. Su muestra fue de 22 sujetos con deterioro cognitivo leve amnésico (entre 63 y 70 años), 10 sujetos sin deterioro cognitivo (de 64 a 68 años) y 32 controles (de 63 a 69 años). El grupo “amnésico” mostró desaceleración encefalografica en las regiones laterales y frontales en comparación tanto con el “no amnésico” y el grupo control. Esta desaceleración encefalografíca estaba presente en la vigilia (en comparación con los controles), pero fue mucho más prominente en el sueño MOR. En cuanto a la comparación entre grupo “amnésico” y grupo “no amnésico” se encontró significativa sólo para el encefalogama del sueño MOR. Con estos hallazgos se la superioridad de la encefalografía del sueño MOR en la distinción del grupo “amnésico” y el grupo “no amnésico” y los sujetos control.
%
%Otro estudio determinó si la estructura del sueño fisiológico y su calidad subjetiva sufrían alteraciones en personas mayores con deterioro cognitivo. Para lograr este objetivo se realizó un estudio encefalografico nocturno y se tomaron medidas subjetivas del sueño a un grupo de 25 adultos mayores cognitivamente intactos y a otro 25 con deterioro. La calidad subjetiva del sueño se evaluó mediante un cuestionario de 5 preguntas centradas en las alteraciones del sueño más frecuentes en pacientes con deterioro cognitivo. Los resultados confirmaron que la arquitectura del sueño está alterada en personas mayores con deterioro cognitivo mostrando una mayor fragmentación del sueño de ondas alfa y una duración más corta del sueño MOR4.

%%%%%%%%%%%%%%%%%%%%%%%%%%%%%%%%%%%%%%%%%%%%%%%%%%%%%%%%%%%%%%%%%%%%%%%%%%%%%%%%%%%%%%%%%%%%%%%%%%%

%En la primera etapa del tarbajo de \cite{VazquezTagle16},
%los individuos se sometieron voluntariamente a una bater\'ia de pruebas
%neuropsicol\'ogicas para diagnosticar PDC y depresi\'on geri\'atrica\footnote{Para
%m\'as detalles, ver la parte de Metodolog\'ia}, que a su vez fungieron como criterios de
%inclusi\'on para una segunda fase del estudio.
%En la etapa posterior, los individuos se sometieron voluntariamente a un estudio de la
%actividad el\'ectrica cerebral durante el
%sue\~no; se realizaron
%registros de EEG en 22 sitios de muestreo, adicionalmente se midi\'o
%actividad ocular y muscular a trav\'es de EOG y EMG --respectivamente-- con
%el fin de detectar adecuadamente las etapas cl\'inicas del sue\~no\cite{AASM07}.
%El registro simult\'aneo de estas se\~nales durante el sue\~no recibe el nombre de PSG.
%
%En este trabajo se modelan matem\'aticamente los registros de PSG como series de 
%tiempo\footnote{Ver la parte de Conceptos matem\'aticos para m\'as detalles}, y se
%investigan algunas de propiedades estad\'isticas din\'amicas (dependientes del tiempo)
%en cuanto puedan existir diferencias entre los registros obtenidos de sujetos con PSG,
%con respecto a individuos sanos. Se ha elegido una caracter\'istica
%llamada ''estacionariedad d\'ebil'', que refleja la invarianza en el tiempo de las propiedes
%estad\'isticas de un sistema, y que est\'a asociado a una complejidad baja del mismo; 
%debido a que se ha reportado una ''p\'erdida de complejidad'' de la actividad cerebral
%en individuos con deterioro cognitivo, se consider\'o que ser\'ia un medidor oportuno. 
%
%Con respecto al anterior estudio, este trabajo busca un mejor
%entendimiento --desde las matem\'aticas-- de las diferencias encontradas entre 
%individuos sanos y con PSG; adem\'as, se ha buscado generar una metodolog\'ia 
%conceptualmente accesible y computacionalmente r\'apida que replique los resultados encontrados.

%%%%%%%%%%%%%%%%%%%%%%%%%%%%%%%%%%%%%%%%%%%%%%%%%%%%%%%%%%%%%%%%%%%%%%%%%%%%%%%%%%%%%%%%%%%%%%%%%%%

\section{Pregunta de investigaci\'on}

Considerando los registros de PSG como series de tiempo,
¿es plausible usar sus propiedades estad\'isticas dependientes del tiempo, espec\'ificamente
la estacionariedad d\'ebil,
como marcadores para el diagn\'ostico cl\'inico del DC --en alguna de 
sus fases-- en adultos mayores?

%%%%%%%%%%%%%%%%%%%%%%%%%%%%%%%%%%%%%%%%%%%%%%%%%%%%%%%%%%%%%%%%%%%%%%%%%%%%%%%%%%%%%%%%%%%%%%%%%%%

\subsection{Hip\'otesis}

Existen diferencias estad\'isticamente significativas entre 
%las propiedades
%estad\'isticas dependientes del tiempo, 
la ''presencia'' de estacionariedad d\'ebil entre
los registros de PSG,
durante etapas espec\'ificas de sue\~no, en adultos
mayores con PDC (respecto a individuos control).

%%%%%%%%%%%%%%%%%%%%%%%%%%%%%%%%%%%%%%%%%%%%%%%%%%%%%%%%%%%%%%%%%%%%%%%%%%%%%%%%%%%%%%%%%%%%%%%%%%%

\subsection{Objetivo general}

%A modo de estudio de casos retrospectivo,
Deducir a partir de pruebas estad\'isticas formales las propiedades
estad\'isticas de registros de PSG en adultos mayores con PDC, as\'i como individuos control.

%%%%%%%%%%%%%%%%%%%%%%%%%%%%%%%%%%%%%%%%%%%%%%%%%%%%%%%%%%%%%%%%%%%%%%%%%%%%%%%%%%%%%%%%%%%%%%%%%%%

\subsection{Objetivos espec\'ificos}

\begin{itemize}
\item Investigar la definici\'on de estacionariedad para series de tiempo
%\footnote{Estacionariedad fuerte, 
%estacionariedad de orden finito, estacionariedad local, cuasi-estacionariedad,
%ciclo-estacionariedad, procesos estoc\'asticos erg\'odicos} 
y sus posibles consecuencias dentro
de un modelo para los datos considerados

\item Investigar en la literatura c\'omo detectar si es plausible que una serie de tiempo 
dada sea una realizaci\'on de un proceso d\'ebilmente estacionario, 
y bajo qu\'e supuestos es v\'alida esta caracterizaci\'on

\item Usando los an\'alisis hallados en la literatura para determinar si las series de tiempo,
obtenidas a partir de los datos considerados, provienen de procesos
debilmente estacionarios.
Revisar si la informaci\'on obtenida en los diferentes sujetos muestra diferencias entre
sujetos con y sin PDC

%\item Determinar, para el caso de estudio presente, la validez de los \textbf{supuestos de 
%estacionariedad d\'ebil}: su presencia o ausencia, en intervalos cortos o arbitrarios de tiempo

\end{itemize}

%%%%%%%%%%%%%%%%%%%%%%%%%%%%%%%%%%%%%%%%%%%%%%%%%%%%%%%%%%%%%%%%%%%%%%%%%%%%%%%%%%%%%%%%%%%%%%%%%%%