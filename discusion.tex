%%%%%%%%%%%%%%%%%%%%%%%%%%%%%%%%%%%%%%%%%%%%%%%%%%%%%%%%%%%%%%%%%%%%%%%%%%%%%%%%%%%%%%%%%%%%%%%%%%%
%%%%%%%%%%%%%%%%%%%%%%%%%%%%%%%%%%%%%%%%%%%%%%%%%%%%%%%%%%%%%%%%%%%%%%%%%%%%%%%%%%%%%%%%%%%%%%%%%%%
\section{Discusi\'on}

Como se mencion\'o en la secci\'on de hip\'otesis, este trabajo pare del supuesto en que los
sujetos con PDC presentan con mayor probabilidad estacionariedad d\'ebil en sus registros de EEG.
Esta idea fue sugerida por Cohen \cite{Cohen77}, quien a su vez se refiere a trabajos anteriores
sobre regularidad estad\'stica --estacionariedad y normalidad-- sobre registros de 
EEG \cite{McEwen75,Sugimoto78,Kawabata73}. 
Si bien en estos primeros estudios se palpa la posibilidad de que los registros de EEG fueran
ruido de alg\'un tipo, esta idea se ha probado \'erronea en estudios m\'as recientes 
\cite{Klonowski09}.

Cabe entonces mencionar una segunda justificaci\'on, un poco m\'as arbitraria y personal, sobre
las hip\'otesis de este trabajo: en el trabajo de Valeria [no se como citarlo] se describen
diferencias significativas entre los registros de PSG en adultos mayores con y sin PDC,
refiri\'endose al exponente de Hurst ($H_\alpha$) estimado.
La cantidad $H_\alpha$, tambi\'en referido como el ''color'' de la se\~nal,
mide la ''fractalidad''\footnote{Este concepto no se
describir\'a en este trabajo, para m\'as informaci\'on ver el trabajo de Valeria} 
de un proceso estoc\'astico y es estimado a trav\'es del
algoritmo Detrended Fluctuation Analysis (DFA); se reporta
que el exponente $H_\alpha$ es menor para registros de PSG en adultos mayores con PSG, y que es
cercano a aqu\'el en el movimiento browniano. 
Luego entonces, cabe preguntarse sobre la naturaleza exacta de las diferencias detectadas en 
el trabajo de Valeria: ¿la se\~nal es ''menos compleja'' o 
s\'olo ''tiene otro color''?
De manera concreta, en este trabajo se ha hipotetizado sobre la primera opci\'on.

En cierto modo, se ha aportado evidencias suficientes para decir que no hay cambios significativos
en la porci\'on de tiempo durante la cual el registro de PSG se comporta de manera ''simple''
--es PE. Esto puede interpretarse como que --quiz\'a-- los mecanismos afectados durante el PDC no 
provocan que la se\~nal se vuelva m\'as simple desde el punto de vista estad\'istico

Cabe un comentario sobre c\'omo la evidencia exhibe al PSG como se\~nales no-estacionarias
por una porci\'on muy prque\~na de tiempo; luego, no es adecuado analizarla con m\'etodos que
supongan estacionariedad. M\'as a\'un este comentario aplica para individuos con y sin PDC, y
se acent\'ua m\'as en individuos con problemas adicionales.

\subsubsection{La inclusi\'on de sujetos}

%Con respecto a los sujetos con problemas adicionales, cabe mencionar el caso de FGH

Durante el trabajo se menciona constantemente a tres sujetos (FGH,MGG,EMT) que fueron considerados
pero que no son considerados dentro de las estad\'isticas; 
como se mencion\'o anteriormente,
cada uno de ellos fue exclu\'ido del
trabajo original por diversos motivos, pero dieron su consentimiento informado para la etapa
de registro de sue\~no debido a lo cual se decidi\'o analizar el efecto de su inclusi\'on dentro 
de los estad\'isticas.

El caso m\'as notorio es el sujeto FGH, quien padece de par\'alisis facial, problemas 
no especificados en la 
hipotiroides, en la columna y tiene cataratas. Seg\'un se reporta en el trabajo original,
el sujeto no inform\'o de la par\'alsis facial sino hasta despu\'es del registro de PSG, por lo
que su exlusi\'on se efectu\'o a posteriori.
Si bien la metodolog\'ia presentada aqu\'i no tiene como objetivo el diagn\'ostico
de tal padecimiento --y bajo el etendido que hay m\'etodos menos invasivos para ello--, los
registros confirman picos inusuales 


uashdflhasjkdfhlkasjdhflkajdhlkadfkasld+


\begin{comment}
\subsubsection{Otros estimadores espectrales}

Cabe mencionar que una motivaci\'on muy fuerte para utilizar el test PSR para detectar 
estacionariedad d\'ebil, tiene su origen en el objetivo informal de
''usar un m\'etodo previamente validado, f\'acil de
usar e interppretar, y que se encuentre implementado en software de f\'acil acceso''.
Este anhelo parece cumplido usando la functi\'on \texttt{stationarity} del paquete
\texttt{fractal}, en el software estad\'istico multiplataforma, gratuito y de c\'odigo abierto 
\texttt{R} --al menos en el \'ultimo punto.

Sin embargo, dado que la prueba PSR fue mostrada por primera vez en 1969 \cite{Priestley69}, es
intuitivo que debieran existir enfoques ''m\'as actuales''. En ese sentido
se puede hablar, por ejemplo, de la estimaci\'on del espectro que en este trabajo se realiz\'o a
trav\'es del estimador de doble ventana; un enfoque m\'as moderno que
cabe destacar
con mucho \'enfasis es
la familia de estimadores que satisfacen una serie depropiedades descritas por Cohen 
\cite{Cohen89} y que son referidos
como \textbf{la clase de Cohen}.
Esta clase puede ser interpretada como una ''suavizaci\'on'' del espectrograma, de forma similar al
uso de la ventana espectral; su uso parece m\'as adecuado para 
funciones 
deterministas cuyo espectro cambia en el tiempo, y se ha generalizado su uso para procesos 
d\'ebilmente estacionarios. 
El autor desconoce si existe alguna generalizaci\'on
para espectros de procesos estoc\'asticos.
%, pero se pueden exhibir trabajos donde se usan para
%calcular el espectro de realizaciones de procesos que presuponene como aleatorios [citar].
M\'as a\'un, un enfoque m\'as reciente se basa en la definici\'on de estacionariedad local

\end{comment}

\subsubsection{Otros usos para las t\'ecnicas utilizadas}

Como una etapa exploratoria de este trabajo, se dio un
%Un primer 
tratamiento cualitativo a los resultados obtenidos del test PSR
% es su
%disposici\'on gr\'afica.
grafic\'andolos de varias formas distintas; cabe destacar una de ellas que pudiera resultar
\'util pero que no hubo aportado suficiente informaci\'on clara sobre la hip\'otesis
principal de este trabajo.
%Una vez se hubo realizado el test para todas las \'epocas consideradas, se dispuso de los 
%resultados de manera gr\'afica  
%como se muestra en la figura \ref{ejemplo1}.
En esta disposici\'on gr\'afica,
se coloc\'o en l\'inea horizontal un cuadro blanco por cada
\'epoca PE (negro para \'epocas no-estacionarias);
% seg\'un el 
%el segmento en cuesti\'on halla sido clasi
%segmento referido haya sido clasificado como
%no-estacionario o posiblemente estacionario; 
posteriormente se colocaron verticalmente las
l\'ineas as\'i obtenidas.
% de todos los canales.
%Esta disposici\'on gr\'afica pretende ser consistente con las representaciones gr\'aficas
%usuales de EEG.
%, tomando en cuanta una escala m\'as amplia de tiempo gracias a que por
%cada \'epoca s\'olo se ha obtenido un dato.
Puede verse en la figura \ref{ejemplo1} un ejemplo de esta disposici\'on gr\'afica, mientras
que el resto de estos gr\'aficos se incluye como anexo.
%Los gr\'aficos as\'i obtenidos se incluyen como anexo.
%como se muestra en la figura \ref{ejemplo1}.

\begin{figure}
\includegraphics[width=\textwidth]{MJNNVIGILOS_127_mor127_tot1032_esttotal.pdf} 
\caption{Disposici\'on gr\'afica para los resultados del test PSR en el sujeto MJH, 
para 1032 \'epocas de sue\~no y 22 canales. 
En el eje horizontal se muestra el tiempo desde el inicio de registro, en el eje vertical se muestra al 
nombre del canal. 
Se han resaltado con color verde las \'epocas clasificadas como sue\~no MOR (ver texto), que son 127.
Para este gr\'afico se consider\'o con un p-valor cr\'itico de 0.01 para la hip\'otesis
de estacionariedad. Ver texto para m\'as detalles.}
\label{ejemplo1}
\end{figure}

Una debilidad importante de los gr\'aficos as\'i obtenidos es que, si bien muestran patrones 
claros en el tiempo, \'estos no se pueden cuantificar de una manera obvia y se dificulta la
comparaci\'on entre sujetos, raz\'on por la cual se omiti\'on del cuerpo principal del trabajo. 
%Se han inclu\'ido estos resultados porque sus caracter\'isticas 
%sugieren una posible utilizaci\'on para otros fines --en alg\'un trabajo futuro.
Sin embargo, dentro de un mismo sujeto, parecen visibles diferencias cualitativas
entre el sue\~no MOR y el resto del sue\~no nocturno.

Conviene mencionar que el origen de esta representaci\'on gr\'afica es un intento preeliminar de
fragmentar los an\'alsis de sue\~no MOR en grupos de \'epocas consecutivas en sue\~no MOR ya que,
en general, el sue\~no MOR aparece fragmetnado durante el sue\~no nocturno \cite{CarrilloMora}.
La hip\'otesis de que se podr\'ian definir diferencias que involucraran la componente espacial,
sin embargo, se vio opacada por la dificultad de definir formalmente tales diferencias a modo
que pudieran compararse entre sujetos.
Una sugerencia recibida consiste en seguir explorando estos patrones en el tiempo, pero quiz\'a
no con la intenci\'on de detectar deterioro cognitivo sino como apoyo para la identificaci\'on de
diferentes etapas de sue\~no.

%\begin{figure}
%\includegraphics[width=\textwidth]{est02.png} 
%\caption{En este gráfico sólo se ilustran épocas MOR. Las líneas punteadas separan bloques continuos.
%Total de épocas: 1032 , Épocas MOR: 127}
%\label{ejemplo2}
%\end{figure}

%Me siento particularmente orgulloso
%de haber dise\~nado este tipo de gr\'aficos, ya que  organizan datos que ya se ten\'ian
%y dejan la sensaci\'on de portar nueva informaci\'on.

%\includepdf[pages={1-},scale=.85]{reporte_de_estacionariedad_170120.pdf}
%
%\afterpage{%
%    \clearpage% Flush earlier floats (otherwise order might not be correct)
%    \thispagestyle{empty}% empty page style (?)
%    \begin{landscape}% Landscape page
%        \centering % Center table
%        \begin{figure}
%            \includegraphics[width=\textwidth]{MJNNVIGILOS_127_mor127_tot1032_esttotal.pdf} 
%            \caption{Total de \'epocas: 1032, \'epocas MOR: 127}
%            %\label{ejemplo1}
%        \end{figure}
%    \end{landscape}
%    \clearpage% Flush page
%}

%%%%%%%%%%%%%%%%%%%%%%%%%%%%%%%%%%%%%%%%%%%%%%%%%%%%%%%%%%%%%%%%%%%%%%%%%%%%%%%%%%%%%%%%%%%%%%%%%%%
%%%%%%%%%%%%%%%%%%%%%%%%%%%%%%%%%%%%%%%%%%%%%%%%%%%%%%%%%%%%%%%%%%%%%%%%%%%%%%%%%%%%%%%%%%%%%%%%%%%

\section{Conclusiones}

Se aportan evidencias sobre que la presencia proporcional de estacionariedad d\'ebil en registros 
de PSG para adultos mayores, 
no presenta diferencias significativas entre sujetos con y sin PDC diagnosticado.
Luego entonces, esta caracter\'istica no es un indicador fiable 

%%%%%%%%%%%%%%%%%%%%%%%%%%%%%%%%%%%%%%%%%%%%%%%%%%%%%%%%%%%%%%%%%%%%%%%%%%%%%%%%%%%%%%%%%%%%%%%%%%%
%%%%%%%%%%%%%%%%%%%%%%%%%%%%%%%%%%%%%%%%%%%%%%%%%%%%%%%%%%%%%%%%%%%%%%%%%%%%%%%%%%%%%%%%%%%%%%%%%%%

\section{Trabajo a futuro}

Como se ha sugerido, los bloques de estacionariedad pueden tener un uso como 
caracter\'isticas auxiliares
para la detecci\'on autom\'atica de \'epocas MOR en registros de PSG: el hecho que la proporci\'on
de \'epocas PE no se vea afectada --estad\'isticamente-- por el PDC del paciente, sugiere que es
posible obtener resultados independientes de ello. Para ello cabe recordar, como se mencion\'o 
en la secci\'on de discusi\'on, que sujetos fuera del rango de los grupos considerados puede
que fallen respecto a esta conclusi\'on: hace falta m\'as indagaci\'on al respecto. 

Por otro lado, el uso de estimadores espectrales de ventana pueden explorarse de manera m\'as
puntual para detectar estacionariedad sobre componentes de frecuencia espec\'ificas, de modo
que es en principio posible separar las ondas cerebrales.

%%%%%%%%%%%%%%%%%%%%%%%%%%%%%%%%%%%%%%%%%%%%%%%%%%%%%%%%%%%%%%%%%%%%%%%%%%%%%%%%%%%%%%%%%%%%%%%%%%%
%%%%%%%%%%%%%%%%%%%%%%%%%%%%%%%%%%%%%%%%%%%%%%%%%%%%%%%%%%%%%%%%%%%%%%%%%%%%%%%%%%%%%%%%%%%%%%%%%%%