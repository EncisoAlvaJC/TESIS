% $Log: abstract.tex,v $
% Revision 1.1  93/05/14  14:56:25  starflt
% Initial revision
% 
% Revision 1.1  90/05/04  10:41:01  lwvanels
% Initial revision
% 
%
%% The text of your abstract and nothing else (other than comments) goes here.
%% It will be single-spaced and the rest of the text that is supposed to go on
%% the abstract page will be generated by the abstractpage environment.  This
%% file should be \input (not \include 'd) from cover.tex.

El estudio y diagnóstico de una gran cantidad de enfermedades depende de nuestra habilidad para
registrar y analizar se\~nales electrofisiol\'ogicas. Por ejemplo, electrocardiogramas
de pacientes con riesgo de infarto, electroencefalogramas de pacientes con epilepsia,
electroespinogramas que muestran lesiones en la m\'edula espinal.
Se suele asumir que estas se\~nales son complejas: no lineales, no estacionarias y
sin equilibrio por naturaleza. Pero usualmente no se comprueban formalmente estas propiedades.

El objetivo principal de este trabajo es investigar la estacionariedad d\'ebil
%una serie electrofisiol\'ogica particular, 
en registros de la actividad espont\'anea en el dorso de la m\'edula espinal del gato anestesiado.
Para lo cual se aplican la siguientes pruebas formales: %para
%detectar estacionariedad d\'ebil
%Se utilizan las siguientes
%; est\'a dirigido a todo aqu\'el interesado en el estudio de series
%de tiempo complejas, pero sin una preparaci\'on orientada a las matem\'aticas.
%Se describe la estacionaridad d\'ebil
%como una propiedad de las 
%en series de tiempo: %, 
%y son revisados algunos m\'etodos para detectarla
%en registros: 
%para lo cual se emplean los m\'etodos siguientes:
Dickey-Fuller Aumentada, Priestley-Subba Rao y
de Espectro de Ondeletas de Haar (HWTOS, por sus siglas en ingl\'es).
 Estos m\'etodos primero son probados con series artificiales cuyo comportamiento es conocido.
% y posteriormente a la serie electrofisiol\'ogica de inter\'es.
Posteriormente
se aplican al an\'alisis de la estacionariedad en la serie de inter\'es.

Los resultados de las pruebas de Priestley-Subba Rao y de HWTOS indican que los registros de la m\'edula espinal son no-estacionarios. Aunque la prueba de Dickey-Fuller indica que la serie es estacionaria, 
se muestra  %la hip\'otesis puede ser aceptada, pero deja de ser \'util como modelo.
porqu\'e no se puede aceptar ese resultado.
%Como resultado de esta \'utima se concluye que la hip\'otesis de estacionariedad, si bien puede ser aceptada,
%no es \'util como modelo.
Todos los an\'alisis son realizados con el software estad\'istico R, accesible para el p\'ublico de manera gratuita.
%Todas las rutinas mencionadas se encuentran accesibles para el p\'ublico, de manera gratuita, 
%en el software estad\'istico R.
