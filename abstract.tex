% $Log: abstract.tex,v $
% Revision 1.1  93/05/14  14:56:25  starflt
% Initial revision
% 
% Revision 1.1  90/05/04  10:41:01  lwvanels
% Initial revision
% 
%
%% The text of your abstract and nothing else (other than comments) goes here.
%% It will be single-spaced and the rest of the text that is supposed to go on
%% the abstract page will be generated by the abstractpage environment.  This
%% file should be \input (not \include 'd) from cover.tex.

%%%%%%%%%%%%%%%%%%%%%%%%%%%%%%%%%%%%%%%%%%%%%%%%%%%%%%%%%%%%%%%%%%%%%%%%%%%%%%%%%%%%%%%%%%%%%%%%%%%

Los avances m\'edicos del \'ultimo siglo se han traducido en un incremento tanto en la esperanza
de vida como en la calidad de la misma. Sin embargo, tambi\'en se ve incrementada la presencia
de enfermedades no-transmisibles tipificadas esterotipadas como \textit{propias de la edad},
entre ellas la demencia.
%, que por su naturaleza cr\'onica y progresiva representa un peso
%econ\'omico y humano sobre sus familiares y el sistema de saul p\'ublica. 
Por otro lado, los trastornos del sue\~no 
%--otro padecimiento com\'un a la
%vejez y supuestamente \textit{causado por la edad}-- 
han sido se\~nalados recientemente como posiblemente 
relacionados con el deterioro cognitivo.
Todav\'ia son incipientes las investigaciones para identificar los factores de riesgo modificables
asociados a la demencia. \cite{PlanAlzheimer04}

%%%%%%%%%%%%%%%%%%%%%%%%%%%%%%%%%%%%%%%%%%%%%%%%%%%%%%%%%%%%%%%%%%%%%%%%%%%%%%%%%%%%%%%%%%%%%%%%%%%

En este trabajo se busca identificar patrones espec\'ificos en la actividad cerebral del adulto
mayor con PDC, y que pudieran servir como marcadores neurol\'ogicos para un diagn\'ostico
temprano del mismo.
A trav\'es de un estudio de casos \textit{a posteriori}
se investigan registros de actividad cerebral durante el sue\~no (registros PSG)
en busca de carater\'isticas estad\'isticas espec\'ificas (estacionariedad d\'ebil) para adultos 
mayores con PDC diagn\'osticado --a trav\'es de una bater\'ia de test neuropsicol\'ogicos--
adem\'as de individuos control.

%%%%%%%%%%%%%%%%%%%%%%%%%%%%%%%%%%%%%%%%%%%%%%%%%%%%%%%%%%%%%%%%%%%%%%%%%%%%%%%%%%%%%%%%%%%%%%%%%%%